\documentclass[format=acmlarge,review,natbib]{acmart}

\usepackage{fdsymbol}
\usepackage{bbold}
\usepackage{proof}

%% \usepackage{hyphenat}
%% \usepackage{color}
%% \usepackage{url}
%% \usepackage{comment}
%% \usepackage{amssymb}
%% \usepackage{amsmath}
%% \usepackage{mdwtab}
%% \usepackage{mdwlist}
%% \usepackage{listings}
%% \lstMakeShortInline[columns=fullflexible]|
%% \lstnewenvironment{code}{\lstset{basicstyle={\sffamily\footnotesize}}}{}

\newcommand{\alt}{~\mid~}
\newcommand{\invc}[1]{!#1}
\newcommand{\compc}{\circ}
\newcommand{\evalone}[2]{eval(#1,#2)}
\newcommand{\evalbone}[2]{evalB(#1,#2)}
\newcommand{\refl}{\mathtt{refl}}
\newcommand{\idc}{\mathtt{id}}
\newcommand{\swapc}{\mathtt{swap}}
\newcommand{\iso}{\leftrightarrow}
\newcommand{\zt}{\mathbb{0}}
\newcommand{\ot}{\mathbb{1}}
\newcommand{\fc}{\mathtt{false}}
\newcommand{\tc}{\mathtt{true}}
\newcommand{\boolt}{\mathbb{B}}
\newcommand{\uzero}{\mathcal{U}_0}
\newcommand{\uone}{\mathcal{U}_1}
\newcommand{\Rule}[2]{
\makebox{
$\displaystyle
\frac{\begin{array}{l}#1\\\end{array}}
{\begin{array}{l}#2\\\end{array}}$}}
\newcommand{\proves}{\vdash}
\newcommand{\jdg}[3]{#1 \proves #2 : #3}
%% codes
%% denotations

\newcommand{\amr}[1]{\fbox{\begin{minipage}{0.8\textwidth}\color{red}{Amr says: {#1}}\end{minipage}}}


%%%%%%%%%%%%%%%%%%%%%%%%%%%%%%%%%%%%%%%%%%%%%%%%%%%%%%%%%%%%%%%%%%%%%%%%%%%%%%
\begin{document}

\title{Featherweight HoTT}

\author{X}
\affiliation{
  \institution{Y}
  \country{Z}}
\email{A@B.C}

\begin{abstract}
\end{abstract}

\maketitle

%%%%%%%%%%%%%%%%%%%%%%%%%%%%%%%%%%%%%%%%%%%%%%%%%%%%%%%%%%%%%%%%%%%%%%%%%%%%%%
\section{Introduction}

HoTT development very specialized: a smaller collection of types and only
reversible functions.

%%%%%%%%%%%%%%%%%%%%%%%%%%%%%%%%%%%%%%%%%%%%%%%%%%%%%%%%%%%%%%%%%%%%%%%%%%%%%%
\section{Setup}

%%%%%
\subsection{Judgments}

We consider two kinds of judgments. The $=$ is \emph{judgmental equality}.

\[
\jdg{}{a}{A} \qquad\qquad\qquad \jdg{}{a = b}{A}
\]

%%%%%
\subsection{Universes}

We postulate two universes $\uzero$ and $\uone$:

\[
\Rule{}{\jdg{}{\uzero}{\uone}}
\]

%%%%%
\subsection{(Reversible) Functions}

\[\begin{array}{c}
\Rule{\jdg{}{c}{A \iso B}
         \quad\jdg{}{a}{A}}
        {\jdg{}{\evalone{c}{a}}{B}}
\quad
\Rule{\jdg{}{c}{A \iso B}
         \quad\jdg{}{b}{B}}
        {\jdg{}{\evalbone{c}{b}}{A}}
\end{array}\]

%%%%%
\subsection{Booleans}

\[\begin{array}{c}
\Rule{}{\jdg{}{\boolt}{\uzero}}
\quad
\Rule{}{\jdg{}{\fc}{\boolt}}
\quad
\Rule{}{\jdg{}{\tc}{\boolt}}
\\
\\
\Rule{\jdg{}{b}{\boolt}}
        {\jdg{}{\evalone{\idc_{\boolt}}{b} = b}{\boolt}}
\quad
\Rule{}
        {\jdg{}{\evalone{\swapc_{\boolt}}{\fc} = \tc}{\boolt}}
\quad
\Rule{}
        {\jdg{}{\evalone{\swapc_{\boolt}}{\tc} = \fc}{\boolt}}
\end{array}\]

%%%%%
\subsection{Identity Types}

\[\begin{array}{c}
\Rule{\jdg{}{A}{\uzero}
         \quad\jdg{}{a}{A}
         \quad\jdg{}{b}{A}}
        {\jdg{}{a \iso b}{\uzero}}
\quad
\Rule{\jdg{}{A}{\uzero}
          \quad\jdg{}{a}{A}}
         {\jdg{}{\idc_{a}}{a \iso a}}
\\
\\
\Rule{\jdg{}{A}{\uone}
         \quad\jdg{}{a}{A}
         \quad\jdg{}{b}{A}}
        {\jdg{}{a \iso b}{\uone}}
\quad
\Rule{\jdg{}{A}{\uone}
          \quad\jdg{}{a}{A}}
         {\jdg{}{\idc_{a}}{a \iso a}}
\\
\Rule{\jdg{}{A}{\uzero}
         \quad\jdg{}{B}{\uzero}
         \quad\jdg{}{c}{A \iso B}}
         {\jdg{}{\idc_{c:A\iso B}}{c \iso c}}
\quad
\Rule{}{\jdg{}{\swapc_{\boolt}}{\boolt\iso\boolt}}
\\
\\
\Rule{\jdg{}{c_1}{A \iso B}
         \quad\jdg{}{c_2}{A \iso B}}
         {}
\\
\\
\mbox{Need to specialize the rules for } ind_{=A}
\end{array}\]

%%%%%
\subsection{Function Extensionality and Univalence}

\[\begin{array}{c}
\mbox{Specialize the rule of univalence}
\end{array}\]

%%%%%
\subsection{Higher-Inductive Types (Fractionals)}

\[\begin{array}{c}
\mbox{To think about}
\end{array}\]

%%%%%%%%%%%%%%%%%%%%%%%%%%%%%%%%%%%%%%%%%%%%%%%%%%%%%%%%%%%%%%%%%%%%%%%%%%%%%%
\bibliographystyle{acm}
{\footnotesize
\bibliography{cites}
}
\end{document}
