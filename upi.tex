\documentclass[format=acmlarge,review,natbib]{acmart}

% \usepackage{fdsymbol}
\usepackage{bbold}
\usepackage{bussproofs}
\usepackage{keystroke}
\usepackage{comment}

%% \usepackage{hyphenat}
%% \usepackage{color}
%% \usepackage{url}
%% \usepackage{amssymb}
%% \usepackage{amsmath}
%% \usepackage{mdwtab}
%% \usepackage{mdwlist}
%% \usepackage{listings}
%% \lstMakeShortInline[columns=fullflexible]|
%% \lstnewenvironment{code}{\lstset{basicstyle={\sffamily\footnotesize}}}{}

\newcommand{\unitepl}{\texttt{unitepl}}
\newcommand{\unitipl}{\texttt{unitipl}}
\newcommand{\unitepr}{\texttt{unitepr}}
\newcommand{\unitipr}{\texttt{unitipr}}
\newcommand{\swap}{\texttt{swap}}
\newcommand{\swapp}{\texttt{swapp}}
\newcommand{\assoclp}{\texttt{assoclp}}
\newcommand{\assocrp}{\texttt{assocrp}}
\newcommand{\unitetl}{\texttt{unitetl}}
\newcommand{\unititl}{\texttt{unititl}}
\newcommand{\unitetr}{\texttt{unitetr}}
\newcommand{\unititr}{\texttt{unititr}}
\newcommand{\swapt}{\texttt{swapt}}
\newcommand{\assoclt}{\texttt{assoclt}}
\newcommand{\assocrt}{\texttt{assocrt}}
\newcommand{\absorbr}{\texttt{absorbr}}
\newcommand{\absorbl}{\texttt{absorbl}}
\newcommand{\factorzr}{\texttt{factorzr}}
\newcommand{\factorzl}{\texttt{factorzl}}
\newcommand{\factor}{\texttt{factor}}
\newcommand{\distl}{\texttt{distl}}
\newcommand{\dist}{\texttt{dist}}
\newcommand{\factorl}{\texttt{factorl}}
\newcommand{\id}{\texttt{id}}
\newcommand{\compc}[2]{#1 \circ #2}
\newcommand{\compcc}[2]{#1 \bullet #2}
\newcommand{\respcomp}[2]{#1 \odot #2}

\newcommand{\alt}{~\mid~}
\newcommand{\patht}[1]{\textsc{PATHS}(#1,#1)}
\newcommand{\fpatht}[1]{\textsc{FREEPATHS}(#1,\Box)}
\newcommand{\fpathp}[2]{\textsc{freepath}~#1~#2}
\newcommand{\pathind}[2]{\textsc{pathind}~#1~#2}
\newcommand{\invc}[1]{!\;#1}
\newcommand{\evalone}[2]{eval(#1,#2)}
\newcommand{\evalbone}[2]{evalB(#1,#2)}
\newcommand{\reflp}{\textsc{refl}}
\newcommand{\reflh}{\mathit{refl}_{\sim}}
\newcommand{\symh}[1]{\mathit{sym}_{\sim}~#1}
\newcommand{\transh}[2]{\mathit{trans}_{\sim}~#1~#2}
\newcommand{\reflq}{\mathit{refl}_{\simeq}}
\newcommand{\symq}[1]{\mathit{sym}_{\simeq}~#1}
\newcommand{\transq}[2]{\mathit{trans}_{\simeq}~#1~#2}
\newcommand{\isequiv}[1]{\mathit{isequiv}(#1)}
\newcommand{\idc}{\mathit{id}_{\boolt}}
\newcommand{\swapc}{\mathit{swap}_{\boolt}}
\newcommand{\assocc}{\mathit{assoc}}
\newcommand{\invl}{\mathit{invl}}
\newcommand{\invr}{\mathit{invr}}
\newcommand{\invinv}{\mathit{inv}^2}
\newcommand{\idlc}{\mathit{idl}}
\newcommand{\idrc}{\mathit{idr}}
\newcommand{\swapswap}{\swapc^2}
\newcommand{\compsim}{\compc_{\isotwo}}
\newcommand{\iso}{\leftrightarrow}
\newcommand{\isotwo}{\Leftrightarrow}
\newcommand{\piso}{\multimapdotbothB~~}
\newcommand{\zt}{\mathbb{0}}
\newcommand{\ot}{\mathbb{1}}
\newcommand{\fc}{\mathit{false}}
\newcommand{\tc}{\mathit{true}}
\newcommand{\boolt}{\mathbb{B}}
\newcommand{\univ}{\mathcal{U}}
\newcommand{\uzero}{\mathcal{U}_0}
\newcommand{\uone}{\mathcal{U}_1}
\newcommand{\Rule}[2]{
\makebox{
$\displaystyle
\frac{\begin{array}{l}#1\\\end{array}}
{\begin{array}{l}#2\\\end{array}}$}}
\newcommand{\proves}{\vdash}
\newcommand{\jdgg}[3]{#1 \proves #2 : #3}
\newcommand{\jdg}[2]{\proves #1 : #2}
\newcommand{\jdge}[3]{\proves #1 = #2 : #3}
%% codes
%% denotations

\newcommand{\amr}[1]{\fbox{\begin{minipage}{0.8\textwidth}\color{red}{Amr says: {#1}}\end{minipage}}}


%%%%%%%%%%%%%%%%%%%%%%%%%%%%%%%%%%%%%%%%%%%%%%%%%%%%%%%%%%%%%%%%%%%%%%%%%%%%%%
\begin{document}

\title{Univalent Universes and Completness of Reversible Programs}

\author{X}
\affiliation{
  \institution{Y}
  \country{Z}}
\email{A@B.C}

\begin{abstract}
\end{abstract}

\maketitle

%%%%%%%%%%%%%%%%%%%%%%%%%%%%%%%%%%%%%%%%%%%%%%%%%%%%%%%%%%%%%%%%%%%%%%%%%%%%%%
\section{Story and Conjecture}

There is a cottage industry of reversible programming languages, reversible logic, programming applications of type isomorphisms, etc. This work seems it should be connected to HoTT and univalence but there aren't any precise connections or theorems.

Our conjecture:

\begin{itemize}
\item Let $\mathcal{U}$ be the univalent subuniverse generated by $0$, $1$, $+$, and $*$ suitably 1-truncated.
\item Let $\Pi$ be the previously  described reversible programming with 1-paths and 2-paths
\item We conjecture that $\Pi$ includes codes for \emph{all} paths in $\mathcal{U}$
\end{itemize}

%%%%%%%%%%%%%%%%%%%%%%%%%%%%%%%%%%%%%%%%%%%%%%%%%%%%%%%%%%%%%%%%%%%%%%%%%%%%%%
\section{The Bool Case}

To get started let's look at the case of $\mathbb{2}$ instead of the entire set of finite types defined using $0$, $1$, $+$, and $*$.

%%%%%
\subsection{$\Pi$}

The restriction of $\Pi$ is the following:

\[\begin{array}{rcl}
\tau &::=& \mathbb{2} \\
\\
v &::=& \begin{array}[t]{lrcl}
                    & \fc &:& \mathbb{2} \\
              \alt & \tc &:& \mathbb{2}
               \end{array} \\
\\
c &::=& \begin{array}[t]{lrcl}
              & \id &:& \tau \iso \tau \\
               \alt & \swap &:& \mathbb{2} \iso \mathbb{2} \\
               \alt & \circ &:& (\tau_1 \iso \tau_2) \rightarrow (\tau_2 \iso \tau_3)
                              \rightarrow (\tau_1 \iso \tau_3)
               \end{array} \\
\\
! &:& (\tau_1 \iso \tau_2) \rightarrow (\tau_2 \iso \tau_1) \\
\invc{\id} &=& \id \\
\invc{\swap} &=& \swap \\
\invc{(\compc{c_1}{c_2})} &=& \compc{\invc{c_2}\;}{\;\invc{c_1}} \\
\\
\alpha &::=& \begin{array}[t]{lrcl}
               & \id &:& c \isotwo c \\
               \alt & \assocc &:& \compc{c_1}{(\compc{c_2}{c_3})} \isotwo
                                              \compc{(\compc{c_1}{c_2})}{c_3} \\
               \alt & \idlc &:& \compc{\id}{c} \isotwo c \\
               \alt & \idrc &:& \compc{c}{\id} \isotwo c \\
               \alt & \invl &:& \compc{c\;}{\;\invc{c}} \isotwo \id \\
               \alt & \invr &:& \compc{\invc{c}}{c} \isotwo \id \\
               \alt & \bullet &:& (c_1 \isotwo c_2) \rightarrow (c_2 \isotwo c_3)
                                            \rightarrow (c_1 \isotwo c_3) \\
               \alt & \odot &:& (c_1 \isotwo c_1') \rightarrow (c_2 \isotwo c_2')
                                            \rightarrow (\compc{c_1}{c_2} \isotwo \compc{c_1'}{c_2'}) \\
             \end{array}
\end{array}\]
The 2-combinators $\alpha$ also have inverses \ldots

%%%%%
\subsection{$\mathcal{U}$}

The univalent universe is $\Sigma_{X:\mathbf{Type}} \| X \equiv \mathbb{2} \|_1$ where \ldots

%%%%%
\subsection{Equivalence}

%%%%%%%%%%%%%%%%%%%%%%%%%%%%%%%%%%%%%%%%%%%%%%%%%%%%%%%%%%%%%%%%%%%%%
%%%%%%%%%
\section{Generalizations}


%%%%%%%%%%%%%%%%%%%%%%%%%%%%%%%%%%%%%%%%%%%%%%%%%%%%%%%%%%%%%%%%%%%%%%%%%%%%%%
\bibliographystyle{acm}
{\footnotesize
\bibliography{cites}
}
\end{document}
