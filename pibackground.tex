%%%%%%%%%%%%%%%%%%%%%%%%%%%%%%%%%%%%%%%%%%%%%%%%%%%%%%%%%%%%%%%%%%%%%%%%%%%%%%
\section{Programming with Equivalences} 

The main syntactic vehicle for the technical developments in this
paper is the language $\Pi$ whose only computations are isomorphisms
between finite types and equivalences between these
isomorphisms~\cite{James:2012:IE:2103656.2103667,Carette2016}. We
present the syntax and operational semantics of the parts of the
language relevant to our development.

\begin{figure*}[ht]
\[\begin{array}{cc}
\begin{array}{rrcll}
\unitepl :&  \zt \oplus \tau & \iso & \tau &: \unitipl \\
\unitepr :&  \tau \oplus \zt & \iso & \tau &: \unitipr \\
\swapp :&  \tau_1 \oplus \tau_2 & \iso & \tau_2 \oplus \tau_1 &: \swapp \\
\assoclp :&  \tau_1 \oplus (\tau_2 \oplus \tau_3) & \iso & (\tau_1 \oplus \tau_2) \oplus \tau_3 
  &: \assocrp \\
\\
\unitetl :&  \ot \otimes \tau & \iso & \tau &: \unititl \\
\unitetr :&  \tau \otimes \ot & \iso & \tau &: \unititr \\
\swapt :&  \tau_1 \otimes \tau_2 & \iso & \tau_2 \otimes \tau_1 &: \swapt \\
\assoclt :&  \tau_1 \otimes (\tau_2 \otimes \tau_3) & \iso & (\tau_1 \otimes \tau_2) \otimes \tau_3 
  &: \assocrt \\
\\
\absorbr :&~ \zt \otimes \tau & \iso & \zt &: \factorzl \\
\absorbl :&~ \tau \otimes \zt & \iso & \zt &: \factorzr \\
\dist :&~ (\tau_1 \oplus \tau_2) \otimes \tau_3 & 
  \iso & (\tau_1 \otimes \tau_3) \oplus (\tau_2 \otimes \tau_3)~ &: \factor \\
\distl :&~ \tau_1 \otimes (\tau_2 \oplus \tau_3) & 
  \iso & (\tau_1 \otimes \tau_2) \oplus (\tau_1 \otimes \tau_3)~ &: \factorl 
\end{array}
& 
\begin{minipage}{0.4\textwidth}
\[\begin{array}{c}
\Rule{}
{}
{\jdg{}{}{\idiso : \tau \iso \tau}}
{}
\\
\\
\Rule{}
{\jdg{}{}{c_1 : \tau_1 \iso \tau_2} \quad \vdash c_2 : \tau_2 \iso \tau_3}
{\jdg{}{}{c_1 \odot c_2 : \tau_1 \iso \tau_3}}
{}
\\
\\
\Rule{}
{\jdg{}{}{c_1 : \tau_1 \iso \tau_2} \quad \vdash c_2 : \tau_3 \iso \tau_4}
{\jdg{}{}{c_1 \oplus c_2 : \tau_1 \oplus \tau_3 \iso \tau_2 \oplus \tau_4}}
{}
\\
\\
\Rule{}
{\jdg{}{}{c_1 : \tau_1 \iso \tau_2} \quad \vdash c_2 : \tau_3 \iso \tau_4}
{\jdg{}{}{c_1 \otimes c_2 : \tau_1 \otimes \tau_3 \iso \tau_2 \otimes \tau_4}}
{}
\end{array}\]
\end{minipage}
\end{array}\]
Each 1-combinator $c$ has an inverse $!~c$, e.g, $!~\unitepl=\unitipl$,
$!(c_1 \odot c_2) = ~!c_2 \odot~ !c_1$, etc.
\caption{$\Pi$ 1-combinators~\cite{James:2012:IE:2103656.2103667}
\label{pi-combinators}}
\end{figure*}

\begin{figure*}[ht]
\[\begin{array}{c}
\Rule{}
{c : \tau_1 \iso \tau_2}
{\jdg{}{}{\idisotwo : c \isotwo c}}
{}
~
\Rule{}
{c_1,c_2,c_3 : \tau_1 \iso \tau_2 \quad \alpha_1 : c_1 \isotwo c_2 \quad \alpha_2 : c_2 \isotwo c_3}
{\jdg{}{}{\transtwo ~\alpha_1 ~\alpha_2 : c_1 \isotwo c_3}}
{}
~
\Rule{}
{c_1 : \tau_1 \iso \tau_2 \quad c_2 : \tau_2 \iso \tau_3 \quad c_3 : \tau_3 \iso \tau_4}
{\jdg{}{}{\assocdl : c_1 \odot (c_2 \odot c_3) \isotwo (c_1 \odot c_2) \odot c_3 : \assocdr}}
{}
\\
\\
\Rule{}
{c : \tau_1 \iso \tau_2}
{\jdg{}{}{\idldl : \idiso \odot c \isotwo c : \idldr}}
{}
~
\Rule{}
{c : \tau_1 \iso \tau_2}
{\jdg{}{}{\idrdl : c \odot \idiso \isotwo c : \idrdr}}
{}
~
\Rule{}
{c : \tau_1 \iso \tau_2}
{\jdg{}{}{\rinvdl : ~! c \odot c \isotwo \idiso : \rinvdr}}
{}
~
\Rule{}
{c : \tau_1 \iso \tau_2}
{\jdg{}{}{\linvdl : c ~\odot ~! c \isotwo \idiso : \linvdr}}
{}
\\
\\
\Rule{}
{c_1,c_3 : \tau_1 \iso \tau_2 \quad c_2,c_4 : \tau_2 \iso \tau_3 \quad
  \alpha_1 : c_1 \isotwo c_3 \quad \alpha_2 : c_2 \isotwo c_4}
{\jdg{}{}{\alpha_1 ~\respstwo~ \alpha_2 : c_1 \odot c_2 \isotwo c_3 \odot c_4}}
{}
\\
\\
\Rule{}
{c_1,c_3 : \tau_1 \iso \tau_2 \quad c_2,c_4 : \tau_2 \iso \tau_3 \quad
  \alpha_1 : c_1 \isotwo c_3 \quad \alpha_2 : c_2 \isotwo c_4}
{\jdg{}{}{\respptwo ~\alpha_1 ~\alpha_2 : c_1 \oplus c_2 \isotwo c_3 \oplus c_4}}
{}
~
\Rule{}
{c_1,c_3 : \tau_1 \iso \tau_2 \quad c_2,c_4 : \tau_2 \iso \tau_3 \quad
  \alpha_1 : c_1 \isotwo c_3 \quad \alpha_2 : c_2 \isotwo c_4}
{\jdg{}{}{\respttwo ~\alpha_1 ~\alpha_2 : c_1 \otimes c_2 \isotwo c_3 \otimes c_4}}
{}
\end{array}\]

Each 2-combinator $\alpha$ has an inverse $2!~\alpha$, e.g, $2!~\assocdl=\assocdr$,
$2!(\transtwo~\alpha_1~\alpha_2) = \transtwo~(2!~\alpha_2)~(2!~\alpha_1)$, etc. 

\caption{$\Pi$ 2-combinators (excerpt)~\cite{Carette2016}
\label{pi-combinators2}}
\end{figure*}

\begin{figure*}[ht]
\begin{multicols}{2}
\[\begin{array}{r@{\!}lcl}
%% \evalone{\unitepl}{&(\inl{()})} \\
\evalone{\unitepl}{&(\inr{v})} &=& v \\
\evalone{\unitipl}{&v} &=& \inr{v} \\
\evalone{\unitepr}{&(\inl{v})} &=& v \\
%% \evalone{\unitepr}{&(\inr{()})} \\
\evalone{\unitipr}{&v} &=& \inl{v} \\
\evalone{\swapp}{&(\inl{v})} &=& \inr{v} \\
\evalone{\swapp}{&(\inr{v})} &=& \inl{v} \\
\evalone{\assoclp}{&(\inl{v})} &=& \inl{(\inl{v})} \\
\evalone{\assoclp}{&(\inr{(\inl{v})})} &=& \inl{(\inr{v})} \\
\evalone{\assoclp}{&(\inr{(\inr{v})})} &=& \inr{v} \\
\evalone{\assocrp}{&(\inl{(\inl{v})})} &=& \inl{v} \\
\evalone{\assocrp}{&(\inl{(\inr{v})})} &=& \inr{(\inl{v})} \\
\evalone{\assocrp}{&(\inr{v})} &=& \inr{(\inr{v})} \\
\\
\evalone{\unitetl}{&(\unitv,v)} &=& v \\
\evalone{\unititl}{&v} &=& (\unitv,v) \\
\evalone{\unitetr}{&(v,\unitv)} &=& v \\
\evalone{\unititr}{&v} &=& (v,\unitv) \\
\evalone{\swapt}{&(v_1,v_2)} &=& (v_2,v_1) \\
\evalone{\assoclt}{&(v_1,(v_2,v_3))} &=& ((v_1,v_2),v_3) \\
\evalone{\assocrt}{&((v_1,v_2),v_3)} &=& (v_1,(v_2,v_3)) \\
\end{array}\]
\[\begin{array}{r@{\!}lcl}
\evalone{\absorbr}{&(v,\_)} &=& v \\
\evalone{\absorbl}{&(\_,v)} &=& v \\
%% \evalone{\factorzl}{&()} \\
%% \evalone{\factorzr}{&()} \\
\evalone{\dist}{&(\inl{v_1},v_3)} &=& \inl{(v_1,v_3)} \\
\evalone{\dist}{&(\inr{v_2},v_3)} &=& \inr{(v_2,v_3)} \\
\evalone{\factor}{&\inl{(v_1,v_3)}} &=& (\inl{v_1},v_3) \\
\evalone{\factor}{&\inr{(v_2,v_3)}} &=& (\inr{v_2},v_3) \\
\evalone{\distl}{&(v_1,\inl{v_3})} &=& \inl{(v_1,v_3)} \\
\evalone{\distl}{&(v_2,\inr{v_3})} &=& \inr{(v_2,v_3)} \\
\evalone{\factorl}{&\inl{(v_1,v_3)}} &=& (v_1,\inl{v_3}) \\
\evalone{\factorl}{&\inr{(v_2,v_3)}} &=& (v_2,\inr{v_3}) \\
\\
\evalone{\idiso}{&v} &=& v \\
\evalone{(c_1 \odot c_2)}{&v} &=& 
  \evalone{c_2}{(\evalone{c_1}{v})} \\
\evalone{(c_1 \oplus c_2)}{&(\inl{v})} &=& 
  \inl{(\evalone{c_1}{v})} \\
\evalone{(c_1 \oplus c_2)}{&(\inr{v})} &=& 
  \inr{(\evalone{c_2}{v})} \\
\evalone{(c_1 \otimes c_2)}{&(v_1,v_2)} &=& 
  (\evalone{c_1}v_1, \evalone{c_2}v_2) 
\end{array}\]
\end{multicols}
\caption{\label{opsem}$\Pi$ operational semantics}
\end{figure*}

%%%%%%%%%%%%%%%%%%%%%%%
\subsection{Syntax of $\Pi$}
\label{opsempi}

The $\Pi$ family of languages is based on type isomorphisms. In the
variant we consider, the set of types $\tau$ includes the empty
type~$\zt$, the unit type $\ot$, and sum $\oplus$ and product
$\otimes$ types. The values classified by these types are the
conventional ones: $\unitv$ of type~$\ot$, $\inl{v}$ and $\inr{v}$ for
injections into sum types, and $(v_1,v_2)$ for product types. The
language has two other syntactic categories of programs to be
described in detail:
\[\begin{array}{lrcl}
(\textrm{Types}) & 
  \tau &::=& \zt \alt \ot \alt \tau_1 \oplus \tau_2 \alt \tau_1 \otimes \tau_2 \\
(\textrm{Values}) & 
  v &::=& \unitv \alt \inl{v} \alt \inr{v} \alt (v_1,v_2) \\
(\textrm{1-combinators}) & 
  c &:& \tau_1 \iso \tau_2 ~ [\textit{see Fig.~\ref{pi-combinators}}] \\
(\textrm{2-combinators}) &
  \alpha &:& c_1 \isotwo c_2 \mbox{~where~} c_1, c_2 : \tau_1 \iso \tau_2 \\
  & && [\textit{see Fig.~\ref{pi-combinators2}}]
\end{array}\]

Both classes of programs, 1-combinators $c$ and
2-combinators~$\alpha$, denote \emph{equivalences} in the HoTT
sense. The elements $c$ of 1-combinators denote type isomorphisms. The
elements $\alpha$ of 2-combinators denote equivalences between these
type isomorphisms. Using the 1-combinators, it is possible to write
any reversible boolean function and hence encode arbitrary boolean
functions by a technique that goes back to \citet{Toffoli:1980}. The
2-combinators provide a layer of programs that computes semantic
preserving transformations of 1-combinators. As a small example, let
us abbreviate $\ot \oplus \ot$ as $\mathsf{Bool}$ and examine two
possible implementations of boolean negation. The first directly uses
the primitive combinator
$\swapp : \tau_1 \oplus \tau_2 \iso \tau_2 \oplus \tau_1$ to swap the
two values of type $\mathsf{Bool}$; the second use three consecutive
copies of $\swapp$ to achieve the same effect:
\[\begin{array}{rcl}
\mathsf{not_1} &=& \swapp \\
\mathsf{not_2} &=& (\swapp \odot \swapp) \odot \swapp 
\end{array}\]
We can write a 2-combinator whose \emph{type} is $\mathsf{not_2}
\isotwo \mathsf{not_1}$:
\[
\transtwo~(\idldl ~\respstwo~ \idisotwo)~\idldl
\]
which not only shows the equivalence of the two implementations of
negation but also shows \emph{how} to transform one to the other.

Fig.~\ref{pi-combinators} lists all the 1-combinators which consist of
base combinators (on the left) and compositions (on the right). Each
line of the base combinators introduces a pair of dual
constants\footnote{where $\swapp$ and $\swapt$ are self-dual.} that
witness the type isomorphism in the middle. This set of isomorphisms
is known to be sound and
complete~\cite{Fiore:2004,fiore-remarks}. Fig.~\ref{pi-combinators2}
lists a few of the 2-combinators that we use in this paper. Each
2-combinator relates two 1-combinators of the same type and witnesses
their equivalence. The 2-equivalences behave as expected with respect
to inverses of 1-combinators as shown below. 

\begin{proposition}
For any $c : \tau_1 \iso \tau_2$, we have $c \isotwo ~!~(!~c)$.
\end{proposition} 

\begin{proposition}
For any $c_1,c_2 : \tau_1 \iso \tau_2$, we have $c_1 \isotwo c_2$ implies
$!c_1 \isotwo ~!c_2$.
\end{proposition} 

%%%%%%%%%%%%%%%%%%%%%%%
\subsection{Operational Semantics and the Order of 1-Combinators}
\label{sec:pisem}

We give an operational semantics for the 1-combinators of $\Pi$ which
represent ``conventional'' programs.  Operationally, the semantics
consists of a pair of mutually recursive evaluators that take a
combinator and a value and propagate the value in the forward
direction~$\triangleright$ or in the backward
direction~$\triangleleft$. We show the complete forward evaluator in
Fig.~\ref{opsem}; the backward evaluator is easy to infer.

As an example, let $\mathbb{3}$ abbreviate the type
$(\ot \oplus \ot) \oplus \ot$. There are three values of type
$\mathbb{3}$ which are $\inl{\inl{\unitv}}$, $\inl{\inr{\unitv}}$, and
$\inr{\unitv}$. There are six combinators of type
$\mathbb{3} \iso \mathbb{3}$, each representing a different
permutation of three elements, which could be written as follows:
\[\begin{array}{rcl}
a_1 &=& \idiso \\
a_2 &=& \swapp \oplus \idiso \\
a_3 &=& \assocrp \odot (\idiso \oplus \swapp) \odot \assoclp \\
a_4 &=& a_2 \odot a_3 \\
a_5 &=& a_3 \odot a_2 \\
a_6 &=& a_4 \odot a_2
\end{array}\]
Tracing the evaluation of $a_2$ on each of the possible inputs yields:
\[\begin{array}{rcl}
\evalone{(\swapp\oplus\idiso)}{\inl{\inl{\unitv}}} &=& \inl{\evalone{\swapp}{\inl{\unitv}}} \\
&=& \inl{\inr{\unitv}} \\
\\
\evalone{(\swapp\oplus\idiso)}{\inl{\inr{\unitv}}} &=& \inl{\evalone{\swapp}{\inr{\unitv}}} \\
&=& \inl{\inl{\unitv}} \\
\\
\evalone{(\swapp\oplus\idiso)}{\inr{\unitv}} &=& \inr{\evalone{\idiso}{\unitv}} \\
&=& \inr{\unitv}
\end{array}\]
Thus the effect of combinator $a_2$ is to swap the values
$\inl{\inl{\unitv}}$ and $\inl{\inr{\unitv}}$ leaving the value
$\inr{\unitv}$ intact. Iterating $a_2$ again is therefore equivalent
to the identity permutation, which can be verified using the full set of
2-combinators~\cite{Carette2016}:
\[\begin{array}{rcl}
a_2 \odot a_2 &=& (\swapp \oplus \idiso) \odot (\swapp \oplus \idiso) \\
&\isotwo& (\swapp \odot \swapp) \oplus (\idiso \odot \idiso) \\
&\isotwo& \idiso \oplus \idiso \\
&=& \idiso
\end{array}\]

A similar calculation shows that the combinator $a_1$ is the identity
permutation; the combinators $a_3$ and $a_6$ also swap two of the
three elements leaving the third intact; and the combinators $a_4$ and
$a_5$ rotate the three elements. We therefore have
$\mathit{order}(a_1)=1$,
$\mathit{order}(a_2)=\mathit{order}(a_3)=\mathit{order}(a_6)=2$, and
$\mathit{order}(a_4)=\mathit{order}(a_5)=3$. More generally, we define
the power and order of 1-combinators as follows.

\begin{definition}[Power of a 1-combinator]
  The $k^{th}$ power of a 1-combinator $p : \tau \iso \tau$, for
  $k \in \Z$ is
  \[
    p^k =
  \begin{cases}
    \idiso & k = 0 \\
    p \odot p^{k - 1} & k > 0 \\
    (!~p) \odot p^{k + 1} & k < 0 \\
  \end{cases}
  \]
\end{definition}

\begin{definition}[Order of a 1-combinator]
  The order of a 1-combinator $p : \tau \iso \tau$, $\ord{p}$, is the
  least postitive natural number $k \in \N^+$ such that
  $p^k \isotwo \idiso$.
\end{definition}

% \begin{lemma}
%   Every $p : \tau \iso \tau$ has an order.
% \end{lemma}

% \begin{proof}
%   By cases.
%   \begin{enumerate}
%   \item $\ord{\idiso} = 1$
%   \item $\ord{\swapp} = \ord{\swapt} = 2$
%   \item $\ord{p_1 \odot p_2} = ???$
%   \item $\ord{p_1 \oplus p_2} = \ord{p_1 \otimes p_2} = \mathsf{lcm}(\ord{p_1}, \ord{p_2})$
%   \end{enumerate}
% \end{proof}

\begin{lemma}
  For $p : \tau \iso \tau$, $n \in \Z$, $p^{k + n} \isotwo p^n$ where
  $k = \ord{p}$.
\end{lemma}

% \begin{proof}
%   Trivial.
% \end{proof}

\begin{lemma}
\label{lem:distiterplus}
  For $p : \tau\iso\tau$, $m,n\in\Z$, we have a 2-combinator
  $\distiterplus{p}{m}{n} : (p^m \odot p^n) \isotwo p ^{m + n}$.
\end{lemma}


%%%%%%%%%%%%%%%%%%%%%%%%%%%%%%%%%%%%%%%%%%%%%%%%%%%%%%%%%%%%%%%%%%%%%%%%%%%%%%

