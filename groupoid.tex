\section{Groupoids}
\label{sec:groupoids}

%%%%%%%%%%%%%%%%%%%%%%%
\subsection{Groupoid Model} 

For $c_1, c_2 : \tau_1\iso\tau_2$, we have level-2 combinators $\alpha
: c_1 \isotwo c_2$ which are equivalences of isomorphisms, and which
happen to correspond to the coherence conditions for rig
groupoids. Many of the level 2 combinators not used. We only present
the ones we use in this paper.

 From the perspective of category theory, the language $\Pi$ models what is
called a \emph{symmetric bimonoidal category} or a \emph{commutative rig
category}. These are categories with two binary operations and satisfying the
axioms of a commutative rig (i.e., a commutative ring without negative
elements also known as a commutative semiring) up to coherent
isomorphisms. And indeed the types of the $\Pi$-combinators are precisely the
commutative semiring axioms. A formal way of saying this is that $\Pi$ is the
\emph{categorification}~\cite{math/9802029} of the natural numbers. A simple
(slightly degenerate) example of such categories is the category of finite
sets and permutations in which we interpret every $\Pi$-type as a finite set,
interpret the values as elements in these finite sets, and interpret the
combinators as permutations. 

Explain that we only reach a trivial class of groupoids

introduce discreteC and discreteG

discrete category: give a set $S$ objects are $S$, morphisms are only
identity morphisms; each object has exactly one morphism id. That is a
groupoid, trivially, because id is symmetric

Instead of modeling the types of $\Pi$ using sets and the combinators using
permutations we use a semantics that identifies $\Pi$-combinators with
\emph{paths}. More precisely, we model the universe of $\Pi$-types as a space
\AgdaFunction{U} whose points are the individual $\Pi$-types (which are
themselves spaces \AgdaBound{t} containing points). We then postulate that
there is path between the spaces \AgdaBound{t₁} and \AgdaBound{t₂} if there
is a $\Pi$-combinator $c : t_1 \iso t_2$. Our postulate is similar in spirit
to the univalence axiom but, unlike the latter, it has a simple computational
interpretation. A path directly corresponds to a type isomorphism with a
clear operational semantics as presented in the previous section. As we will
explain in more detail below, this approach replaces the datatype
\AgdaSymbol{≡} modeling propositional equality with the datatype
\AgdaSymbol{⟷} modeling type isomorphisms. With this switch, the
$\Pi$-combinators of Fig.~\ref{pi-combinators} become \emph{syntax} for the
paths in the space $U$. Put differently, instead of having exactly one
constructor \AgdaInductiveConstructor{refl} for paths with all other paths
discovered by proofs (see Secs. 2.5--2.12 of the HoTT
book~\citeyearpar{hottbook}) or postulated by the univalence axiom, we have
an \emph{inductive definition} that completely specifies all the paths in the
space $U$.

We begin with the datatype definition of the universe \AgdaFunction{U} of finite
types which are constructed using \AgdaInductiveConstructor{ZERO},
\AgdaInductiveConstructor{ONE}, \AgdaInductiveConstructor{PLUS}, and
\AgdaInductiveConstructor{TIMES}. Each of these finite types will correspond
to a set of points with paths connecting some of the points. The underlying
set of points is computed by \AgdaSymbol{⟦}\_\AgdaSymbol{⟧} as follows:
\AgdaInductiveConstructor{ZERO} maps to the empty set~\AgdaSymbol{⊥},
\AgdaInductiveConstructor{ONE} maps to the singleton set \AgdaSymbol{⊤},
\AgdaInductiveConstructor{PLUS} maps to the disjoint union \AgdaSymbol{⊎},
and \AgdaInductiveConstructor{TIMES} maps to the cartesian product
\AgdaSymbol{×}.

We note that the refinement of the $\Pi$-combinators to combinators on
pointed spaces is given by an inductive family for
\emph{heterogeneous} equality that generalizes the usual inductive
family for propositional equality. Put differently, what used to be
the only constructor for paths \AgdaInductiveConstructor{refl} is now
just one of the many constructors (named
\AgdaInductiveConstructor{id⟷} in the figure). Among the new
constructors we have \AgdaInductiveConstructor{◎} that constructs path
compositions. By construction, every combinator has an inverse
calculated as shown in Fig.~\ref{sym}. These constructions are
sufficient to guarantee that the universe~\AgdaFunction{U} is a
groupoid. Additionally, we have paths that connect values in different
but isomorphic spaces like \ldots

The example \AgdaFunction{notpath} which earlier required the use of the
univalence axiom can now be directly defined using
\AgdaInductiveConstructor{swap1₊} and \AgdaInductiveConstructor{swap2₊}. To
see this, note that \AgdaPrimitiveType{Bool} can be viewed as a shorthand for
\AgdaInductiveConstructor{PLUS} \AgdaInductiveConstructor{ONE}
\AgdaInductiveConstructor{ONE} with \AgdaInductiveConstructor{true} and
\AgdaInductiveConstructor{false} as shorthands for
\AgdaInductiveConstructor{inj₁} \AgdaInductiveConstructor{tt} and
\AgdaInductiveConstructor{inj₂} \AgdaInductiveConstructor{tt}. With this in
mind, the path corresponding to boolean negation consists of two ``fibers'',
one for each boolean value as shown below:

\noindent In other words, a path between spaces is really a collection of
paths connecting the various points. Note however that we never need to
``collect'' these paths using a universal quantification.

Shouldn't we also show that \AgdaPrimitiveType{Bool} contains exactly
$2$ things, and that TRUE and FALSE are ``different''? The other thing
is, whereas not used to be a path between Bool and Bool, we no longer
have that.  Shouldn't we show that, somehow, BOOL and BOOL.F 'union'
BOOL.T are somehow ``equivalent''?  And there there is a coherent
notpath built the same way?  Especially since I am sure it is quite
easy to build incoherent sets of paths!

%%%%%
\subsection{A Universe of Groupoids} 

% \begin{code}
% -- Conjecture:  p ⇔ q   implies  order p = order q
% -- Corollary:   p ⇔ !q  implies  order p = order (! q)

% -- The opposite is not true.

% -- Example
% -- p = (1 2 3 4)

% -- compose p 0 = compose !p 0 = compose p 4 = compose !p 4
% -- 1 -> 1
% -- 2 -> 2
% -- 3 -> 3
% -- 4 -> 4

% -- compose p 1  ***     compose !p 1
% -- 1 -> 2       ***     1 -> 4
% -- 2 -> 3       ***     2 -> 1
% -- 3 -> 4       ***     3 -> 2
% -- 4 -> 1       ***     4 -> 3

% -- compose p 2  ***     compose !p 2
% -- 1 -> 3       ***     1 -> 3
% -- 2 -> 4       ***     2 -> 4
% -- 3 -> 1       ***     3 -> 1
% -- 4 -> 2       ***     4 -> 2

% -- compose p 3  ***     compose !p 3
% -- 1 -> 4       ***     1 -> 2
% -- 2 -> 1       ***     2 -> 3
% -- 3 -> 2       ***     3 -> 4
% -- 4 -> 3       ***     4 -> 1

% -- there is a morphism 1 -> 2 using
% -- (compose p 1) and (compose !p 3)
% -- p¹ is the same as !p³
% -- p² is the same as !p²
% -- p³ is the same as !p¹

% data FT/ : Set where
%   ⇑    : FT → FT/
%   #    : {τ : FT} → (p : τ ⟷ τ) → FT/ 
%   1/#  : {τ : FT} → (p : τ ⟷ τ) → FT/
%   _⊞_  : FT/ → FT/ → FT/              
%   _⊠_  : FT/ → FT/ → FT/

% UG : Universe l0 (lsuc l0)
% UG = record {
%     U = FT/
%  ;  El = λ T → Σ[ ℂ ∈ Category l0 l0 l0 ] (Groupoid ℂ)
%  }

% card : FT/ → ℚ
% card (⇑ τ)      = mkRational ∣ τ ∣ 1 {tt}
% card (# p)      = mkRational (order p) 1 {tt}
% card (1/# p)    = mkRational 1 (order p) {order-nz}
% card (T₁ ⊞ T₂)  = (card T₁) ℚ+ (card T₂)
% card (T₁ ⊠ T₂)  = (card T₁) ℚ* (card T₂)
% \end{code}

% %%%%%
% \subsection{Groupoids from $\Pi$-Combinators} 

% The goal is to define a function that takes a $T$ in $FT/$ and
% produces something of type $Universe.El~UG~T$, i.e., a particular
% groupoid.

% \begin{code}

% -- First each p is an Agda type
% -- Perm p i is the type that contains the i^th iterate
% -- of p, i.e p^i up to <=>.
% -- the parens in the definition of ^ need to be there!

% _^_ : {τ : FT} → (p : τ ⟷ τ) → (k : ℤ) → (τ ⟷ τ)
% p ^ (+ 0) = id⟷
% p ^ (+ (suc k)) = p ◎ (p ^ (+ k))
% p ^ -[1+ 0 ] = ! p
% p ^ (-[1+ (suc k) ]) = (! p) ◎ (p ^ -[1+ k ])

% -- i.e. Perm is: for all i, any p' such that
% -- p' ⇔ p ^ i

% record Perm {τ : FT} (p : τ ⟷ τ) : Set where
%   constructor perm
%   field
%     iter : ℤ
%     p' : τ ⟷ τ
%     p'⇔p^i : p' ⇔ p ^ iter
    
% cong^ : {τ : FT} → {p q : τ ⟷ τ} → (k : ℤ) → (eq : p ⇔ q) →
%   p ^ k ⇔ q ^ k
% cong^ (+_ ℕ.zero) eq = id⇔
% cong^ (+_ (suc n)) eq = eq ⊡ cong^ (+ n) eq
% cong^ (-[1+_] ℕ.zero) eq = ⇔! eq
% cong^ (-[1+_] (suc n)) eq = (⇔! eq) ⊡ cong^ (-[1+ n ]) eq

% -- this should go into PiLevel1

% !!⇔id : {t₁ t₂ : FT} → (p : t₁ ⟷ t₂) → p ⇔ ! (! p)
% !!⇔id _⟷_.unite₊l = id⇔
% !!⇔id _⟷_.uniti₊l = id⇔
% !!⇔id _⟷_.unite₊r = id⇔
% !!⇔id _⟷_.uniti₊r = id⇔
% !!⇔id _⟷_.swap₊ = id⇔
% !!⇔id _⟷_.assocl₊ = id⇔
% !!⇔id _⟷_.assocr₊ = id⇔
% !!⇔id _⟷_.unite⋆l = id⇔
% !!⇔id _⟷_.uniti⋆l = id⇔
% !!⇔id _⟷_.unite⋆r = id⇔
% !!⇔id _⟷_.uniti⋆r = id⇔
% !!⇔id _⟷_.swap⋆ = id⇔
% !!⇔id _⟷_.assocl⋆ = id⇔
% !!⇔id _⟷_.assocr⋆ = id⇔
% !!⇔id _⟷_.absorbr = id⇔
% !!⇔id _⟷_.absorbl = id⇔
% !!⇔id _⟷_.factorzr = id⇔
% !!⇔id _⟷_.factorzl = id⇔
% !!⇔id _⟷_.dist = id⇔
% !!⇔id _⟷_.factor = id⇔
% !!⇔id _⟷_.distl = id⇔
% !!⇔id _⟷_.factorl = id⇔
% !!⇔id id⟷ = id⇔
% !!⇔id (p ◎ q) = !!⇔id p ⊡ !!⇔id q
% !!⇔id (p _⟷_.⊕ q) = resp⊕⇔ (!!⇔id p) (!!⇔id q)
% !!⇔id (p _⟷_.⊗ q) = resp⊗⇔ (!!⇔id p) (!!⇔id q)

% -- because ^ is iterated composition of the same thing,
% -- then by associativity, we can hive off compositions
% -- from left or right

% assoc1 : {τ : FT} → {p : τ ⟷ τ} → (m : ℕ) →
%   (p ◎ (p ^ (+ m))) ⇔ ((p ^ (+ m)) ◎ p)
% assoc1 ℕ.zero = trans⇔ idr◎l idl◎r
% assoc1 (suc m) = trans⇔ (id⇔ ⊡ assoc1 m) assoc◎l

% assoc1- : {τ : FT} → {p : τ ⟷ τ} → (m : ℕ) →
%   ((! p) ◎ (p ^ -[1+ m ])) ⇔ ((p ^ -[1+ m ]) ◎ (! p))
% assoc1- ℕ.zero = id⇔
% assoc1- (suc m) = trans⇔ (id⇔ ⊡ assoc1- m) assoc◎l

% -- Property of ^: negating exponent is same as
% -- composing in the other direction, then reversing.
% ^⇔! : {τ : FT} → {p : τ ⟷ τ} → (k : ℤ) →
%   (p ^ (ℤ- k)) ⇔ ! (p ^ k)
% ^⇔! (+_ ℕ.zero) = id⇔
% -- need to dig deeper, as we end up negating
% ^⇔! (+_ (suc ℕ.zero)) = idl◎r
% ^⇔! (+_ (suc (suc n))) = trans⇔ (assoc1- n) (^⇔! (+ suc n) ⊡ id⇔)
% ^⇔! {p = p} (-[1+_] ℕ.zero) = trans⇔ idr◎l (!!⇔id p)
% ^⇔! {p = p} (-[1+_] (suc n)) =
%   trans⇔ (assoc1 (suc n)) ((^⇔! -[1+ n ]) ⊡ (!!⇔id p))

% -- first match on m, n, then proof is purely PiLevel1
% lower : {τ : FT} {p : τ ⟷ τ} (m n : ℤ) →
%   p ^ (m ℤ+ n) ⇔ ((p ^ m) ◎ (p ^ n))
% lower (+_ ℕ.zero) (+_ n) = idl◎r
% lower (+_ ℕ.zero) (-[1+_] n) = idl◎r
% lower (+_ (suc m)) (+_ n) =
%   trans⇔ (id⇔ ⊡ lower (+ m) (+ n)) assoc◎l
% lower {p = p} (+_ (suc m)) (-[1+_] ℕ.zero) = 
%   trans⇔ idr◎r (trans⇔ (id⇔ ⊡ linv◎r) (
%   trans⇔ assoc◎l (2! (assoc1 m) ⊡ id⇔)))  -- p ^ ((m + 1) -1)
% lower (+_ (suc m)) (-[1+_] (suc n)) = -- p ^ ((m + 1) -(1+1+n)
%   trans⇔ (lower (+ m) (-[1+ n ])) (
%   trans⇔ ((trans⇔ idr◎r (id⇔ ⊡ linv◎r))  ⊡ id⇔) (
%   trans⇔ assoc◎r (trans⇔ (id⇔ ⊡ assoc◎r) (
%   trans⇔ assoc◎l (2! (assoc1 m) ⊡ id⇔))))) 
% lower (-[1+_] m) (+_ ℕ.zero) = idr◎r
% lower (-[1+_] ℕ.zero) (+_ (suc n)) = 2! (trans⇔ assoc◎l (
%   trans⇔ (rinv◎l ⊡ id⇔) idl◎l))
% lower (-[1+_] (suc m)) (+_ (suc n)) = -- p ^ (-(1+m) + (n+1))
%   trans⇔ (lower (-[1+ m ]) (+ n)) (
%     trans⇔ ((trans⇔ idr◎r (id⇔ ⊡ rinv◎r))  ⊡ id⇔) (
%   trans⇔ assoc◎r (trans⇔ (id⇔ ⊡ assoc◎r) (
%   trans⇔ assoc◎l ((2! (assoc1- m)) ⊡ id⇔)))))
% lower (-[1+_] ℕ.zero) (-[1+_] n) = id⇔
% lower (-[1+_] (suc m)) (-[1+_] n) = -- p ^ (-(1+1+m) - (1+n))
%   trans⇔ (id⇔ ⊡ lower (-[1+ m ]) (-[1+ n ])) assoc◎l


% -- orderC is the groupoid with objects p^i

% orderC : {τ : FT} → (p : τ ⟷ τ) → Category _ _ _
% orderC {τ} p = record {
%      Obj = Perm p
%    ; _⇒_ = λ { (perm i p₁ _) (perm j p₂ _) → p₁ ⇔ p₂ } 
%    ; _≡_ = λ _ _ → ⊤ 
%    ; id = id⇔ 
%    ; _∘_ = λ α β → trans⇔ β α
%    ; assoc = tt
%    ; identityˡ = tt
%    ; identityʳ = tt 
%    ; equiv = record { refl = tt; sym = λ _ → tt; trans = λ _ _ → tt }
%    ; ∘-resp-≡ = λ _ _ → tt  
%    }
%    where open Perm
   
% orderG : {τ : FT} → (p : τ ⟷ τ) → Groupoid (orderC p)
% orderG {τ} p = record {
%     _⁻¹ = 2!
%   ; iso = record {
%         isoˡ = tt
%       ; isoʳ = tt
%       }
%   }

% -- discrete groupoids corresponding to plain pi types

% discreteC : Set → Category _ _ _
% discreteC S = record {
%      Obj = S
%     ; _⇒_ = _≡_
%     ; _≡_ = λ _ _ → ⊤ 
%     ; id = refl 
%     ; _∘_ = λ { {A} {.A} {.A} refl refl → refl }
%     ; assoc = tt 
%     ; identityˡ = tt 
%     ; identityʳ = tt 
%     ; equiv = record { refl = tt; sym = λ _ → tt; trans = λ _ _ → tt }  
%     ; ∘-resp-≡ = λ _ _ → tt 
%     }

% discreteG : (S : Set) → Groupoid (discreteC S)
% discreteG S = record
%   { _⁻¹ = λ { {A} {.A} refl → refl }
%   ; iso = record { isoˡ = tt; isoʳ = tt }
%   }

% -- fractional groupoid

% 1/orderC : {τ : FT} (p : τ ⟷ τ) → Category _ _ _
% 1/orderC {τ} pp = record {
%      Obj = ⊤
%     ; _⇒_ = λ _ _ → Perm pp
%     ; _≡_ = λ { (perm m p _) (perm n q  _) → p ⇔ q } -- pp ^ m ⇔ pp ^ n 
%     ; id = perm (+ 0) id⟷ id⇔
%     ; _∘_ = λ { (perm m p α) (perm n q β) →
%         perm (m ℤ+ n) (p ◎ q) (trans⇔ (α ⊡ β) (2! (lower m n))) }
%     ; assoc = assoc◎r
%     ; identityˡ = idl◎l
%     ; identityʳ =  idr◎l
%     ; equiv = record { refl = id⇔; sym = 2!; trans = trans⇔ }
%     ; ∘-resp-≡ = _⊡_
%     }

% 1/orderG : {τ : FT} (p : τ ⟷ τ) → Groupoid (1/orderC p)
% 1/orderG p = record {
%       _⁻¹ = λ { (perm i q eq) →
%               perm (ℤ- i) (! q) (trans⇔ (⇔! eq) (2! (^⇔! {p = p} i)))}
%     ; iso = record { isoˡ = rinv◎l ; isoʳ = linv◎l }
%     }
% \end{code}

% %% _//_ : (τ : FT) → (p : τ ⟷ τ) → Category _ _ _
% %% τ // p = Product (discreteC (El τ)) (1/orderC p)
% %%   where open Universe.Universe UFT
% %% 
% %% quotientG : (τ : FT) → (p : τ ⟷ τ) → Groupoid (τ // p)
% %% quotientG = {!!} 

% So now we can finally define our denotations:

% \begin{code}
% ⟦_⟧/ : (T : FT/) → Universe.El UG T
% ⟦ ⇑ S ⟧/ = , discreteG (Universe.El UFT S)
% ⟦ # p ⟧/ = , orderG p
% ⟦ 1/# p ⟧/ = , 1/orderG p
% ⟦ T₁ ⊞ T₂ ⟧/ with ⟦ T₁ ⟧/ | ⟦ T₂ ⟧/
% ... | (_ , G₁) | (_ , G₂) = , GSum G₁ G₂
% ⟦ T₁ ⊠ T₂ ⟧/ with ⟦ T₁ ⟧/ | ⟦ T₂ ⟧/
% ... | (_ , G₁) | (_ , G₂) = , GProduct G₁ G₂
% \end{code}

%%%%%
\subsection{Information Equivalence}
 
We need to show coherence of the definition of cardinalities on the
universe syntax with the Euler characteric of the category which in
our case also corresponds to the groupoid cardinality. There are
several formulations and explanations. The following is quite simple
to implement: first collapse all the isomorphic objects. Then fix a
particular order of the objects and write a matrix whose $ij$'s entry
is the number of morphisms from $i$ to $j$. Invert the matrix. The
cardinality is the sum of the elements in the matrix.

Our notion of information equivalence is coarser than the conventional
notion of equivalence of categories (groupoids). This is fine as there
are several competing notions of equivalence of groupoids that are
coarser than strict categorical equivalence. 

There are however other notions of equivalence of groupoids like
Morita equivalence and weak equivalence that we explore later. The
intuition of these weaker notions of equivalence is that two groupoids
can be considered equivalent if it is not possible to distinguish them
using certain observations. This informally corresponds to the notion
of ``observational equivalence'' in programming language
semantics. Note that negative entropy can only make sense locally in
an open system but that in a closed system, i.e., in a complete
computation, entropy cannot be negative. Thus we restrict
observational contexts to those in which fractional types do not
occur. Note that two categories can have the same cardinality but not
be equivalent or even Morita equivalent but the converse is
guaranteed. So it is necessary to have a separate notion of
equivalence and check that whenever we have the same cardinality, the
particular categories in question are equivalent. 

Define action groupoids $\ag{\tau}{p}$ and show they are equivalent to
uparrow tau times 1 over hash p

Talk about monad/comonad ????

%%%%%%%%%%%%%%%%%%%%%%%%%%%%%%%%%%%%%%%%%%%%%%%%%%%%%%%%%%%%%%%%%%%%%%%%%%%%%%

% -- p is a monad on (order p)

% ^suc : {τ : FT} {p : τ ⟷ τ} {i : ℤ} → p ^ ℤsuc i ⇔ p ◎ p ^ i
% ^suc = {!!}

% ^resp : {τ : FT} {p q : τ ⟷ τ} {i : ℤ} → (q ^ i ⇔ p ^ i) → (q ◎ q ^ i ⇔ p ◎ p ^ i)
% ^resp = {!!} 

% orderM : {τ : FT} → (p : τ ⟷ τ) → Monad (orderC p)
% orderM {τ} p = record {
%     F = record {
%           F₀ = λ { (i , (q , α)) →
%                  (ℤsuc i , (q , trans⇔ (^suc {p = q} {i = i})
%                                 (trans⇔ (^resp {p = p} {q = q} {i = i} α)
%                                 (2! (^suc {p = p} {i = i})))))}
%         ; F₁ = {!!}
%         }
%   ; η = record {
%           η = {!!}
%         ; commute = λ _ → tt
%         }
%   ; μ = record {
%           η = {!!}
%         ; commute = λ _ → tt
%         }
%   ; assoc = tt
%   ; identityˡ = tt
%   ; identityʳ = tt
%   }

% -- ! p is a comonad on (order p)

% orderCom : {τ : FT} → (p : τ ⟷ τ) → Comonad (orderC p)
% orderCom {τ} p = record {
%     F = record {
%           F₀ = {!!} 
%         ; F₁ = {!!}
%         }
%   ; η = record {
%           η = {!!}
%         ; commute = λ _ → tt
%         }
%   ; μ = record {
%           η = {!!}
%         ; commute = λ _ → tt
%         }
%   ; assoc = tt
%   ; identityˡ = tt
%   ; identityʳ = tt
%   }

% -- the monad and comonad are inverses
% -- idea regarding the significance of the
% -- monad/comonad construction. Say we have
% -- a combinator c : #p ⟷ #q that maps
% -- p^i to q^j. Then we can use the q-monad
% -- to write a combinator pc : #p ⟷ #q which
% -- maps p^i to q^j using c and then to
% -- q^(suc j) using the monad. We can use
% -- the comonad to map p^i to p^(suc i) and
% -- then to #q using c. So as an effect we can
% -- constuct maps that move around #p and #q
% -- using p and q.
% --
% -- A more general perspective: computations
% -- happen in a context in the following sense:
% -- say we have a collection of values v1, v2, ...
% -- a computation takes vi to wi. In many cases,
% -- the vi's form a structure of some kind and
% -- so do the wi's. A monad focuses on the w's
% -- structure and how to compose computations
% -- on it. The comonad focuses on the v's structure
% -- and how to compose computations on it. Some
% -- people talk about monads expressing how to
% -- affect the context and comonads expressing
% -- what to expect from the context. 

% -- moncom = ?

% -- 1/orderC is the the groupoid with one object
% --   and morphisms p^i

% 1/orderM : {τ : FT} (p : τ ⟷ τ) → Monad (1/orderC p)
% 1/orderM = {!!} 

% 1/orderCom : {τ : FT} (p : τ ⟷ τ) → Comonad (1/orderC p)
% 1/orderCom = {!!} 

% The definition of $p$ will induce three types (groupoids):

% \begin{itemize}

% \item The first is the action groupoid $\ag{C}{p}$ depicted below. The
% objects are the elements of $C$ and there is a morphism between $x$
% and $y$ iff $p^k$ for some $k$ maps $x$ to $y$. We do not draw the
% identity morphisms. Note that all of $p^0$, $p^1$, and $p^2$ map
% \texttt{sat} to \texttt{sat} which explains the two non-trivial
% morphisms on \texttt{sat}:

% \begin{center}
% \begin{tikzpicture}[scale=0.4,every node/.style={scale=0.4}]
%   \draw (0,0) ellipse (8cm and 1.6cm);
%   \node[below] at (-6,0) {\texttt{sun}};
%   \node[below] at (-4,0) {\texttt{mon}};
%   \node[below] at (-2,0) {\texttt{tue}};
%   \node[below] at (0,0) {\texttt{wed}};
%   \node[below] at (2,0) {\texttt{thu}};
%   \node[below] at (4,0) {\texttt{fri}};
%   \node[below] (B) at (6,0) {\texttt{sat}};
%   \draw[fill] (-6,0) circle [radius=0.05];
%   \draw[fill] (-4,0) circle [radius=0.05];
%   \draw[fill] (-2,0) circle [radius=0.05];
%   \draw[fill] (0,0) circle [radius=0.05];
%   \draw[fill] (2,0) circle [radius=0.05];
%   \draw[fill] (4,0) circle [radius=0.05];
%   \draw[fill] (6,0) circle [radius=0.05];
%   \draw (-6,0) -- (-4,0);
%   \draw (-4,0) -- (-2,0);
%   \draw (0,0) -- (2,0);
%   \draw (2,0) -- (4,0);
%   \draw (-6,0) to[bend left] (-2,0) ;
%   \draw (0,0) to[bend left] (4,0) ;
%   \path (B) edge [loop above, looseness=3, in=48, out=132] node[above] {} (B);
%   \path (B) edge [loop above, looseness=5, in=40, out=140] node[above] {} (B);
% \end{tikzpicture}
% \end{center}

% To calculate the cardinality, we first collapse all the isomorphic
% objects (i.e., collapse the two strongly connected components to one
% object each) and write the resulting matrix:
% \[
% \begin{pmatrix}
% 1 & 0 & 0 \\
% 0 & 1 & 0 \\
% 0 & 0 & 3 
% \end{pmatrix}
% \]
% Its inverse is 0 everywhere except on the main diagonal which has
% entries 1, 1, and $\frac{1}{3}$ and hence the cardinality of this
% category is $2\frac{1}{3}$.

% \item The second which we call $1/p$ is depicted below. It has one
% trivial object and a morphism for each iteration of $p$. It has
% cardinality $\frac{1}{3}$ as the connectivity matrix has one entry
% $3$ whose inverse is $\frac{1}{3}$:

% \begin{center}
% \begin{tikzpicture}[scale=0.7,every node/.style={scale=0.8}]
%   \draw (0,1.4) ellipse (2cm and 2cm);
%   \node[below] (B) at (0,0) {\texttt{*}};
% %%  \path (B) edge [loop above] node[above] {$p^0$} (B);
%   \path (B) edge [loop above, looseness=15, in=48, out=132] node[above] {$p$} (B);
%   \path (B) edge [loop above, looseness=25, in=40, out=140] node[above] {$p^2$} (B);
% \end{tikzpicture}
% \end{center}

% \item The third is the order type $\order{p}$ depicted below. It has
% three objects corresponding to each iteration of $p$. It has
% cardinality $3$:
% \begin{center}
% \begin{tikzpicture}[scale=0.7,every node/.style={scale=0.8}]
%   \draw (0,0) ellipse (4cm and 1cm);
%   \node[below] at (-2,0) {$p^0$};
%   \node[below] at (0,0) {$p^1$};
%   \node[below] at (2,0) {$p^2$};
%   \draw[fill] (-2,0) circle [radius=0.05];
%   \draw[fill] (0,0) circle [radius=0.05];
%   \draw[fill] (2,0) circle [radius=0.05];
% \end{tikzpicture}
% \end{center}
% \end{itemize}

% Each combinator $p : \tau ⟷ \tau$ will give rise to two groupoids:
% \begin{itemize}
% \item one groupoid $\mathit{order}~p$ with objects $p^i$ and morphisms $⇔$, and 
% \item another groupoid $\mathit{1/order}~p$ with one object and morphisms $p^i$ under $⇔$
% \end{itemize}
% There is also a third groupoid $\ag{\tau}{p}$ that is equivalent to
% $\tau \boxtimes \mathit{1/order}~p$ and that is a conventional quotient type.

% Is weak equivalence in HoTT related??? Here is one definition: A map
% $f : X \rightarrow Y$ is a weak homotopy equivalence (or just a weak
% equivalence) if for every $x \in X$, and all $n \geq 0$ the map
% $\pi_n(X,x) \rightarrow \pi_n(Y,f(x))$ is a bijection. In our setting
% this might mean something like: two types $T$ and $U$ are equivalent
% if $T \leftrightarrow T$ is equivalent to $U \leftrightarrow U$ are
% equivalent.

% -- These are true, but no longer used
% -- cancel-rinv : {τ : FT} → {p : τ ⟷ τ} → (i : ℤ) →
% --   ((p ^ i) ◎ ((! p) ^ i)) ⇔ id⟷
% -- cancel-rinv (+_ ℕ.zero) = idl◎l
% -- cancel-rinv (+_ (suc n)) = 
% --   trans⇔ (assoc1 n ⊡ id⇔) (trans⇔ assoc◎l (trans⇔ (assoc◎r ⊡ id⇔)
% --   (trans⇔ ((id⇔ ⊡ linv◎l) ⊡ id⇔) (trans⇔ (idr◎l ⊡ id⇔) (
% --   cancel-rinv (+ n))))))
% -- cancel-rinv (-[1+_] ℕ.zero) = linv◎l
% -- cancel-rinv (-[1+_] (suc n)) = 
% --   trans⇔ (assoc1- n ⊡ id⇔) (
% --   trans⇔ assoc◎l (trans⇔ (assoc◎r ⊡ id⇔)
% --   (trans⇔ ((id⇔ ⊡ linv◎l) ⊡ id⇔) (trans⇔ (idr◎l ⊡ id⇔)
% --   (cancel-rinv -[1+ n ])))))

% -- cancel-linv : {τ : FT} → {p : τ ⟷ τ} → (i : ℤ) →
% --   (((! p) ^ i) ◎ (p ^ i)) ⇔ id⟷
% -- cancel-linv (+_ ℕ.zero) = idr◎l
% -- cancel-linv (+_ (suc n)) = trans⇔ (assoc1 n ⊡ id⇔) (
% --    trans⇔ assoc◎l (trans⇔ (assoc◎r ⊡ id⇔) (
% --    trans⇔ ((id⇔ ⊡ rinv◎l) ⊡ id⇔) (trans⇔ (idr◎l ⊡ id⇔)
% --    (cancel-linv (+ n))))))
% -- cancel-linv (-[1+_] ℕ.zero) = rinv◎l
% -- cancel-linv (-[1+_] (suc n)) = trans⇔ (assoc1- n ⊡ id⇔) (
% --   trans⇔  assoc◎l (trans⇔ (assoc◎r ⊡ id⇔) (
% --   trans⇔ ((id⇔ ⊡ rinv◎l) ⊡ id⇔) (trans⇔ (idr◎l ⊡ id⇔) (
% --   cancel-linv -[1+ n ])))))

