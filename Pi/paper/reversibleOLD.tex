\section{Reversible Programming Languages}~\label{sec:reversible}

The practice of programming languages is replete with \emph{ad hoc} instances of reversible computations: database
transactions, mechanisms for data provenance, checkpoints, stack and exception traces, logs, backups, rollback
recoveries, version control systems, reverse engineering, software transactional memories, continuations, backtracking
search, and multiple-level undo features in commercial applications. In the early nineties,
\citet{Baker:1992:LLL,Baker:1992:NFT} argued for a systematic, first-class, treatment of reversibility. But intensive
research in full-fledged reversible models of computations and reversible programming languages was only sparked by the
discovery of deep connections between physics and
computation~\cite{Landauer:1961,PhysRevA.32.3266,Toffoli:1980,bennett1985fundamental,Frank:1999:REC:930275, Hey:1999:FCE:304763,fredkin1982conservative}, and by the
potential for efficient quantum computation~\cite{springerlink:10.1007/BF02650179}.

The early developments of reversible programming languages started
with a conventional programming language, e.g., an extended
$\lambda$-calculus, and either:
\begin{enumerate}
\item extended the language with a history
mechanism~\cite{vanTonder:2004,Kluge:1999:SEMCD,lorenz,danos2004reversible}, or
\item imposed constraints on the control flow constructs to make them
reversible~\cite{Yokoyama:2007:RPL:1244381.1244404}.
\end{enumerate}
More modern approaches recognize that reversible programming languages require a fresh approach and should be designed
from first principles without the detour via conventional irreversible
languages~\cite{Yokoyama:2008:PRP,Mu:2004:ILRC,abramsky2005structural,DiPierro:2006:RCL:1166042.1166047,
  rc2011,James:2012:IE:2103656.2103667,Carette2016}.

%%%%%%%%%%%%%%%%%
\subsection{The $\Pi$ Family of Languages}

A natural candidate for a semantic foundation for reversible programming languages is the notion of type isomorphism
(equivalence). Indeed, the type isomorphisms among finite types represent a universal language for reversible boolean
circuits~\cite{James:2012:IE:2103656.2103667} and the extension with recursive types and trace
operators~\cite{Hasegawa:1997:RCS:645893.671607} is a Turing-complete reversible
language~\cite{James:2012:IE:2103656.2103667,rc2011}.

Focusing on finite types, the building blocks of type theory are: the
empty type ($\mathbb{0}$), the unit type ($\mathbb{1}$) containing
just one value $\Acon{tt}$, the sum type ($+$) containing values of
the form $\inlv{v}$ and $\inrv{v}$, and the product ($\times$) type
containing pairs of values $(v_1,v_2)$. One may view each type $A$
as a collection of physical wires that can transmit $|A|$ distinct
values where $|A|$ is a natural number that indicates the size of a
type, computed as: $| \mathbb{0} | = 0$; $| \mathbb{1} | = 1$;
$| A + B | = | A | + |B |$; and
$| A \times B | = | A | * |B |$.  Thus the type
$\mathbb{B} = \mathbb{1} + \mathbb{1}$ corresponds to a wire that can
transmit one of two values, i.e., bits, with the convention that
$\inlv{\Acon{tt}}$ represents \AgdaFunction{𝔽} and
$\inrv{\Acon{tt}}$ represents \AgdaFunction{𝕋}.  The type
$\mathbb{B} \times \mathbb{B} \times \mathbb{B}$ corresponds to a
collection of wires that can transmit three bits. From that
perspective, a type isomorphism between types $A$ and $B$
(such that $|A|=|B|=n$) models a \emph{reversible}
combinational circuit that \emph{permutes} the $n$ different
values. These type isomorphisms are collected in Fig.~\ref{pi-terms}.
\begin{figure}[t]
{\scalebox{\scalef}{$%
%%\noindent\begin{minipage}{.7\linewidth}
\begin{array}{rrcll}
\idc :& A & \iso & A &: \idc \\
\\
\identlp :&  \mathbb{0} + A & \iso & A &: \identrp \\
\swapp :&  A + B & \iso & B + A &: \swapp \\
\assoclp :&  A + (B + C) & \iso & (A + B) + C &: \assocrp \\ [1.5ex]
\identlt :&  \mathbb{1} \times A & \iso & A &: \identrt \\
\swapt :&  A \times B & \iso & B \times A &: \swapt \\
\assoclt :&  A \times (B \times C) & \iso & (A \times B) \times C &: \assocrt \\ [1.5ex]
\absorbr :&~ \mathbb{0} \times A & \iso & \mathbb{0} ~ &: \factorzl \\
\dist :&~ (A + B) \times C & \iso & (A \times C) + (B \times C)~ &: \factor
  \end{array}$}}

{\scalebox{\scalef}{%
\Rule{}
{\jdg{}{}{c_1 : A \iso B} \quad \vdash c_2 : B \iso C}
{\jdg{}{}{c_1 \circledcirc c_2 : A \iso C}}
{}

\Rule{}
{\jdg{}{}{c_1 : A \iso B} \quad \vdash c_2 : C \iso D}
{\jdg{}{}{c_1 \oplus c_2 : A + C \iso B + D}}
{}

\Rule{}
{\jdg{}{}{c_1 : A \iso B} \quad \vdash c_2 : C \iso D}
{\jdg{}{}{c_1 \otimes c_2 : A \times C \iso B \times D}}
{}
}}
\caption{$\Pi$-terms, combinators, and their types.}
\label{pi-terms}
\end{figure}

Each line in the top part of the figure introduces a pair of dual constants that witness the type isomorphism in the
middle.  These are the \emph{base} (non-reducible) terms of $\Pi$. The isomorphisms are extended to form a congruence
relation by adding the three constructors at the bottom of the figure that witness equivalence and compatible
closure. Although austere, this combinator-based language has the advantage of being amenable to formal analysis for at
least two reasons: (i) it is conceptually simple and small, and (ii) it has direct and evident connections to type
theory and category theory and hence possesses a rich algebraic structure which we exploit in the remainder of the
paper.

% Specifically, the type isomorphisms of~$\Pi$ are sound and complete for all permutations on finite
% types~\cite{Fiore:2004,fiore-remarks} and hence they are \emph{complete} for expressing reversible combinational
% circuits~\cite{fredkin1982conservative, James:2012:IE:2103656.2103667,Toffoli:1980}. Algebraically, these types and
% combinators form a \emph{commutative semiring} (up to type isomorphism). Logically, they form a superstructural logic
% capturing space-time trade-offs~\cite{superstructural}. Categorically, they form a \emph{distributive bimonoidal
%   category}~\cite{laplaza72}.

% For the fragment over finite types, the syntax of the language $\Pi$ consists of several sorts:

%  {\scalebox{\scalef}{$%
% \begin{array}{lrcl}
% \textit{Value types} & A,B,C,D &::=& \mathbb{0} \alt \mathbb{1} \alt A+B \alt A\times B \\
% \textit{Values}      & v,w,x,y &::=& \Acon{tt} \alt \inlv{v} \alt \inrv{v} \alt (v,w) \\
% \textit{Program types} &&& A \leftrightarrow B \\
% \textit{Programs} & c &::=& (\textrm{See Fig.~\ref{pi-terms}})
% \end{array}$}}

\begin{figure}[t]
{\scalebox{\scalef}{%
$\begin{array}{rclcrcl}
\delta(\identlp,~\inrv{v}) & = & v & & \delta(\identrp,~v)        & = & \inrv{v}\\
\delta(\swapp,~\inlv{v})     & = & \inrv{v}\\
\delta(\swapp,~\inrv{v})     & = & \inlv{v}\\
\delta(\assoclp,~\inlv{v})          & = & \inlv{(\inlv{v})} & & \delta(\assocrp,~\inlv{(\inlv{v})}) & = & \inlv{v}         \\
\delta(\assoclp,~\inrv{(\inlv{v})}) & = & \inlv{(\inrv{v})} & & \delta(\assocrp,~\inlv{(\inrv{v})}) & = & \inrv{(\inlv{v})}\\
\delta(\assoclp,~\inrv{(\inrv{v})}) & = & \inrv{v}          & & \delta(\assocrp,~\inrv{v})          & = & \inrv{(\inrv{v})}\\
\delta(\identlt,~(\Acon{tt},v)) & = & v & & \delta(\identrt,~v)             & = & (\Acon{tt},v)\\
\delta(\swapt,~(x,y))         & = & (y,x)    \\
\delta(\assoclt,~(x,(y,z))) & = & ((x,y),z) & & \delta(\assocrt,~((x,y),z)) & = & (x,(y,z))\\
\delta(\dist,~(\inlv{x},z)) & = & \inlv{(x,z)} & & \delta(\factor,~\inlv{(x,z)}) & = & (\inlv{x},z)\\
\delta(\dist,~(\inrv{y},z)) & = & \inrv{(y,z)} & & \delta(\factor,~\inrv{(y,z)}) & = & (\inrv{y},z)
\end{array}$}}
\caption{Semantics of base combinators}\label{fig:delta}
\end{figure}

\begin{figure}[t]
  {\scalebox{\scalef}{%
 $\begin{array}{c}
 \infer{\interp{c}{v} \Downarrow \delta(c,v)}{c ~\mbox{base combinator}}
\qquad
\infer{\interp{c_1\oplus c_2}{\inlv{v}} \Downarrow \inlv{w}}
    {\interp{c_1}{v} \Downarrow w}
\qquad
\infer{\interp{c_1\oplus c_2}{\inrv{v}} \Downarrow \inrv{w}}
    {\interp{c_2}{v} \Downarrow w} \\
\\
 \infer{\interp{\idc}{v} \Downarrow v}{}
\qquad
\infer{\interp{c_1\circledcirc c_2}{v} \Downarrow v_2}
    {\interp{c_1}{v} \Downarrow v_1 \quad \interp{c_2}{v_1} \Downarrow v_2}
\qquad
\infer{\interp{c_1\otimes c_2}{(v_1 , v_2)} \Downarrow (w_1 , w_2)}
    {\interp{c_1}{v_1} \Downarrow w_1 \quad \interp{c_2}{v_2} \Downarrow w_2}
\end{array}$}}
\caption{$\Pi$-interpreter}\label{fig:pi:interp}
\end{figure}

\noindent To formalize the semantics of $\Pi$, we first specify the
semantics of the base combinators in Fig.~\ref{fig:delta} using a
function $\delta$. It is clear that this function has an inverse
$\delta^\dagger$ such that $\delta^\dagger(c,\delta(c,v))=v$. The semantics of full programs is given by the interpreter
in Fig.~\ref{fig:pi:interp}.

%%%%%%%%%
\subsection{Denotational Semantics}

We choose a canonical set of size $n$, called $\mathsf{Fin}~n$, whose elements are natural numbers less than $n$. To
compute the denotation of a type $A$, we first calculate its size $n = \sizet{A}$. We then construct the canonical set
$\mathsf{Fin}~n$ and provide the (trivial) evidence that this set is identical to $(\mathsf{Fin}~n)$:

\[\begin{array}{rcll}
\sem{A} &=& (\mathsf{Fin}~n, [ n , \mathsf{refl} ]) & \mbox{where}~\sizet{A} = n
\end{array}\]

\noindent The denotation $\sem{c}$ of a combinator $c : A \isot B$ is a path between $\sem{A}$ and $\sem{B}$. If the
size of $A$ is $m$ and the size of $B$ is $n$, the desired path is between $(\mathsf{Fin}~m, [ m , \mathsf{refl} ])$ and
$(\mathsf{Fin}~n, [ n , \mathsf{refl} ])$. This path is directly constructed using $\mathit{ap}$ and the fact that $m=n$
since combinators are always between types of the same size.

\noindent Finally, given two combinators $p , q : A \isot_1 B$ and a 2-combinator $\alpha : p \iso_2 q$, the denotation
$\sem{\alpha}$ of $\alpha$ is a path between $\sem{p}$ and $\sem{q}$.





%%%%%%%%%
\subsection{Examples}
\label{sec:langRev-examples}
\label{examples}

\note{Revisit to pick examples relevant to the rest of the paper; also unify notation}

\paragraph*{Booleans}
Let us start with encoding booleans. We use the type \ensuremath{1{\sumtype}1} to
represent booleans with \ensuremath{\mathit{left} ~()} representing \ensuremath{\mathit{true}} and
\ensuremath{\mathit{right}~()} representing \ensuremath{\mathit{false}}.
Boolean negation is straightforward to define:

\ensuremath{\mathit{not} : \mathit{bool} \leftrightarrow \mathit{bool}}

\ensuremath{\mathit{not} = \swapp}

\noindent
It is easy to verify that \ensuremath{\mathit{not}} changes \ensuremath{\mathit{true}} to \ensuremath{\mathit{false}} and
vice versa.

\paragraph*{Bit Vectors.}
We can represent $n$-bit words using an n-ary product of
\ensuremath{\mathit{bool}}s. For example, we can represent a 3-bit word, \ensuremath{\mathit{word}_3},
using the type \ensuremath{\mathit{bool} {\prodtype} (\mathit{bool} {\prodtype}  \mathit{bool})}.  We can perform various
operations on these 3-bit words using combinators in \ensuremath{\Pi }. For
instance the bitwise \ensuremath{\mathit{not}} operation is the parallel composition of
three \ensuremath{\mathit{not}} operations:

\ensuremath{\mathit{not}_{\mathit{word}_3} :: \mathit{word}_3 \leftrightarrow \mathit{word}_3}

\ensuremath{\mathit{not}_{\mathit{word}_3} = \mathit{not}  {\prodtype}  (\mathit{not}  {\prodtype}  \mathit{not})}

\noindent We can express a 3-bit word reversal operation as follows:

\ensuremath{\mathit{reverse} : \mathit{word}_3 \leftrightarrow \mathit{word}_3}

\ensuremath{\mathit{reverse} = \swapt \odot (\swapt  \otimes  \idc)~ \odot \assocrt}

\noindent We can check that \ensuremath{\mathit{reverse}} does the right thing by
applying it to a value \ensuremath{(v_1, (v_2, v_3))} and writing out the full
derivation tree of the reduction.  The combinator \ensuremath{\mathit{reverse}}, like
many others we will see in this paper, is formed by sequentially
composing several simpler combinators. Instead of presenting the
operation of \ensuremath{\mathit{reverse}} as a derivation tree, it is easier (purely
for presentation reasons) to flatten the tree into a sequence of
reductions as caused by each component. Such a sequence of reductions
is given below:
\[\begin{array}{rlr}
 & (v_1, (v_2, v_3)) \\
 \swapt & ((v_2, v_3), v_1) \\
 \swapt \otimes  \idc & ((v_3, v_2), v_1) \\
 \assocrt & (v_3, (v_2, v_1)) \\
 \end{array}\]
%subcode source isomorphisms.tex:979

\noindent On the first line is the initial value. On each subsequent
line is a fragment of the \ensuremath{\mathit{reverse}} combinator and the value that
results from applying this combinator to the value on the previous
line. For example, \ensuremath{\swapt} transforms \ensuremath{(v_1, (v_2, v_3))} to
\ensuremath{((v_2,v_3),v_1)}.  On the last line we see the expected result with
the bits in reverse order.

We can also draw out the graphical representation of the 3-bit reverse
combinator. In the graphical representation, it is clear that the
combinator achieves the required shuffling.

% \inkscape{reverse-3-bit.pdf}

\paragraph*{Conditionals.}
Even though \ensuremath{\Pi } lacks conditional expressions, they are
expressible using the distributivity and factoring laws. The
diagrammatic representation of \ensuremath{\dist} shows that it redirects the flow
of a value \ensuremath{v:b} based on the value of another one of type
\ensuremath{b_1{\sumtype}b_2}. If we choose \ensuremath{1{\sumtype}1} to be
\ensuremath{\mathit{bool}} and apply either \ensuremath{c_1:b_1\leftrightarrow
b_2} or \ensuremath{c_2:b_1\leftrightarrow b_2} to the value \ensuremath{v},
then we essentially have an `if' expression.

\ensuremath{\mathit{if}_{c_1,c_2} : \mathit{bool}  {\prodtype}  b_1 \leftrightarrow \mathit{bool}  {\prodtype}  b_2}

\ensuremath{\mathit{if}_{c_1,c_2} = \dist \odot ((\idc  \otimes\  c_1) {\sumtype} (\idc \otimes\  c_2)) \odot \factor}


% \inkscape{if-c1-c2.pdf}


The diagram above shows the input value of type \ensuremath{(1{\sumtype}1) {\prodtype}  b_1}
processed by the distribute operator \ensuremath{\dist}, which converts it into
a value of type \ensuremath{(1 {\prodtype}  b_1){\sumtype}(1 {\prodtype}  b_1)}. In the
\ensuremath{\mathit{left}} branch, which corresponds to the
case when the boolean is \ensuremath{\mathit{true}} (i.e. the value was
\ensuremath{\mathit{left} ~()}), the combinator~\ensuremath{c_1} is applied to
the value of type~\ensuremath{b_1}. The right
branch which corresponds to the boolean being \ensuremath{\mathit{false}} passes
the value of type \ensuremath{b_1} through the combinator \ensuremath{c_2}.
The inverse of \ensuremath{\dist}, namely \ensuremath{\factor} is applied
to get the final result of type \ensuremath{(1{\sumtype}1) {\prodtype} b_2}.

\paragraph*{Logic Gates}
There are several universal primitives for conventional (irreversible)
hardware circuits, such as \ensuremath{\mathit{nand}} and \ensuremath{\mathit{fanout}}. In the case
of reversible hardware circuits, the canonical universal primitive is
the Toffoli gate~\cite{Toffoli:1980}. The Toffoli gate takes three
boolean inputs: if the first two inputs are \ensuremath{\mathit{true}} then the third
bit is negated. In a traditional language, the Toffoli gate would be
most conveniently expressed as a conditional expression like:

\noindent
\ensuremath{ \mathit{toffoli}(v_1,v_2,v_3) = \mathit{if} ~(v_1 ~\mathit{and} ~v_2) ~\mathit{then} ~(v_1, v_2, \mathit{not}(v_3)) ~\mathit{else} ~(v_1, v_2, v_3)}

We will derive Toffoli gate in \ensuremath{\Pi } by first deriving a simpler
logic gate called \ensuremath{\mathit{cnot}}.  Consider a one-armed version, \ensuremath{\mathit{if}_c},
of the conditional derived above. If the \ensuremath{\mathit{bool}} is
\ensuremath{\mathit{true}}, the value of type \ensuremath{b} is modified by the operator \ensuremath{c}.

\begin{center}
\scalebox{1.5}{
%%subcode-line{pdfimage}[diagrams/if_c.pdf]
% \includegraphics{inkscape/cnot.pdf}
}
\end{center}


By choosing \ensuremath{b} to be \ensuremath{\mathit{bool}} and \ensuremath{c} to be \ensuremath{\mathit{not}}, we have the
combinator \ensuremath{\mathit{if}_{\mathit{not}} : \mathit{bool} {\prodtype}  \mathit{bool}\leftrightarrow \mathit{bool} {\prodtype}  \mathit{bool}} which negates its
second argument if the first argument is \ensuremath{\mathit{true}}. This gate
\ensuremath{\mathit{if}_{\mathit{not}}} is often referred to as the \ensuremath{\mathit{cnot}} gate\cite{Toffoli:1980}.

If we iterate this construction once more, the resulting combinator
\ensuremath{\mathit{if}_{\mathit{cnot}}} has type \ensuremath{\mathit{bool} {\prodtype}  (\mathit{bool} {\prodtype}  \mathit{bool})\leftrightarrow \mathit{bool} {\prodtype}  (\mathit{bool} {\prodtype}  \mathit{bool})}. The
resulting gate checks the first argument and if it is \ensuremath{\mathit{true}},
proceeds to check the second argument. If that is also \ensuremath{\mathit{true}} then
it will negate the third argument. Thus \ensuremath{\mathit{if}_{\mathit{cnot}}} is the required
Toffoli gate.

\begin{center}
\scalebox{1.6}{
% \includegraphics{inkscape/toffoli.pdf}
}
\end{center}

%%%%%%%%%
\subsection{A Language of Equivalences between Type Equivalences}
\label{langeqeq}

In the previous section, we examined equivalences between conventional data structures, i.e., structured trees of
values. We now consider a richer but foundational notion of data: programs themselves. Indeed, universal computation
models crucially rely on the fact that \emph{programs are (or can be encoded as) data}, e.g., a Turing machine can be
encoded as a string that another Turing machine (or even the same machine) can manipulate. Similarly, first-class
functions are the \emph{only} values in the $\lambda$-calculus.  In our setting, we ask whether the programs developed
in the previous section can themselves be subject to (higher-level) equivalences?

We will explain the ideas using two small exaamples. Consider the following two programs mapping between the types
$A + B$ and $C+D$:

\begin{center}
\begin{tikzpicture}[scale=0.7,every node/.style={scale=0.8}]
  \draw[>=latex,<->,double,red,thick] (2.25,-1.2) -- (2.25,-2.9) ;
  \draw[fill] (-2,-1.5) circle [radius=0.025];
  \node[below] at (-2.1,-1.5) {$A$};
  \node[below] at (-2.1,-1.9) {$+$};
  \draw[fill] (-2,-2.5) circle [radius=0.025];
  \node[below] at (-2.1,-2.5) {$B$};

  \draw[fill] (6.5,-1.5) circle [radius=0.025];
  \node[below] at (6.7,-1.5) {$C$};
  \node[below] at (6.7,-1.9) {$+$};
  \draw[fill] (6.5,-2.5) circle [radius=0.025];
  \node[below] at (6.7,-2.5) {$D$};

  \draw[<-] (-2,-1.5) to[bend left] (1,0.5) ;
  \draw[<-] (-2,-2.5) to[bend left] (1,-0.5) ;
  \draw[->] (3.5,0.5) to[bend left] (6.5,-1.45) ;
  \draw[->] (3.5,-0.5) to[bend left] (6.5,-2.45) ;

  \draw[<-] (-2,-1.5) to[bend right] (1,-3.5) ;
  \draw[<-] (-2,-2.5) to[bend right] (1,-4.5) ;
  \draw[->] (3.5,-3.5) to[bend right] (6.5,-1.55) ;
  \draw[->] (3.5,-4.5) to[bend right] (6.5,-2.55) ;


  \draw     (2,0.5)  -- (2.5,0.5)  ;
  \draw     (2,-0.5) -- (2.5,-0.5) ;

  \draw     (2.5,0.5)  -- (3.5,-0.5)  ;
  \draw     (2.5,-0.5) -- (3.5,0.5) ;

  \draw     (1,-3.5)  -- (2,-4.5)    ;
  \draw     (1,-4.5) -- (2,-3.5)   ;

  \draw     (2,-3.5)  -- (2.5,-3.5)    ;
  \draw     (2,-4.5) -- (2.5,-4.5)   ;

  \path (1.5,0.5) node (tc1) [func] {$c_1$};
  \path (1.5,-0.5) node (tc2) [func] {$c_2$};
  \path (3,-4.5) node (bc1) [func] {$c_1$};
  \path (3,-3.5) node (bc2) [func] {$c_2$};
\end{tikzpicture}
\end{center}
The top path is the $\Pi$ program $(c_1~\oplus~c_2)~\odot~\swapp$ which acts on the type $A$ by $c_1$, acts on the type
$B$ by $c_2$, and acts on the resulting value by a twist that exchanges the two injections into the sum type. The bottom
path performs the twist first and then acts on the type $A$ by $c_1$ and on the type $B$ by $c_2$ as before. One could
imagine the paths are physical \emph{elastic} wires in $3$ space, where the programs $c_1$ and $c_2$ as arbitrary
deformations on these wires, and the twists do not touch but are in fact well-separated. Then, holding the points $A$,
$B$, $C$, and $D$ fixed, it is possible to imagine sliding $c_1$ and $c_2$ from the top wire rightward past the twist,
and then using the elasticity of the wires, pull the twist back to line up with that of the bottom --- thus making both
parts of the diagram identical.  Each of these moves can be undone (reversed), and doing so would take the bottom part
of the diagram into the top part.  In other words, there exists an equivalence of the program
$(c_1~\oplus~c_2)~\odot~\swapp$ to the program $\swapp \odot (c_2~\oplus~c_1)$. We can also show that this means that,
as permutations, $(c_1~\oplus~c_2)~\odot~\swapp$ and $\swapp \odot (c_2~\oplus~c_1)$ are equal. And, of course, not all
programs between the same types can be deformed into one another. The simplest example of inequivalent deformations are
the two automorphisms of $1+1$, namely $\idc$ and $\swapp$.

As another example, consider consider a circuit that takes an input type consisting of three values \Tree [ {\small a} [
{\small b} {\small c} ] ]~ and swaps the leftmost value with the rightmost value to produce \Tree [ {\small c} [ {\small
  b} {\small a} ] ]~.  We can implement two such circuits using our Agda library for $\Pi$:

\begin{code}%
\>[0]\AgdaFunction{swap{-}fl1}\AgdaSpace{}%
\AgdaFunction{swap{-}fl2}\AgdaSpace{}%
\AgdaSymbol{:}\AgdaSpace{}%
\AgdaSymbol{\{}\AgdaBound{a}\AgdaSpace{}%
\AgdaBound{b}\AgdaSpace{}%
\AgdaBound{c}\AgdaSpace{}%
\AgdaSymbol{:}\AgdaSpace{}%
\AgdaDatatype{U}\AgdaSymbol{\}}\AgdaSpace{}%
\AgdaSymbol{→}\AgdaSpace{}%
\AgdaInductiveConstructor{PLUS}\AgdaSpace{}%
\AgdaBound{a}\AgdaSpace{}%
\AgdaSymbol{(}\AgdaInductiveConstructor{PLUS}\AgdaSpace{}%
\AgdaBound{b}\AgdaSpace{}%
\AgdaBound{c}\AgdaSymbol{)}\AgdaSpace{}%
\AgdaDatatype{⟷}\AgdaSpace{}%
\AgdaInductiveConstructor{PLUS}\AgdaSpace{}%
\AgdaBound{c}\AgdaSpace{}%
\AgdaSymbol{(}\AgdaInductiveConstructor{PLUS}\AgdaSpace{}%
\AgdaBound{b}\AgdaSpace{}%
\AgdaBound{a}\AgdaSymbol{)}\<%
\\
\>[0]\AgdaFunction{swap{-}fl1}\AgdaSpace{}%
\AgdaSymbol{=}\AgdaSpace{}%
\AgdaInductiveConstructor{assocl₊}\AgdaSpace{}%
\AgdaInductiveConstructor{◎}\AgdaSpace{}%
\AgdaInductiveConstructor{swap₊}\AgdaSpace{}%
\AgdaInductiveConstructor{◎}\AgdaSpace{}%
\AgdaSymbol{(}\AgdaInductiveConstructor{id⟷}\AgdaSpace{}%
\AgdaInductiveConstructor{⊕}\AgdaSpace{}%
\AgdaInductiveConstructor{swap₊}\AgdaSymbol{)}\<%
\\
%
\\[\AgdaEmptyExtraSkip]%
\>[0]\AgdaFunction{swap{-}fl2}\AgdaSpace{}%
\AgdaSymbol{=}%
\>[52I]\AgdaSymbol{(}\AgdaInductiveConstructor{id⟷}\AgdaSpace{}%
\AgdaInductiveConstructor{⊕}\AgdaSpace{}%
\AgdaInductiveConstructor{swap₊}\AgdaSymbol{)}\AgdaSpace{}%
\AgdaInductiveConstructor{◎}\<%
\\
\>[.]\<[52I]%
\>[11]\AgdaInductiveConstructor{assocl₊}\AgdaSpace{}%
\AgdaInductiveConstructor{◎}\<%
\\
%
\>[11]\AgdaSymbol{(}\AgdaInductiveConstructor{swap₊}\AgdaSpace{}%
\AgdaInductiveConstructor{⊕}\AgdaSpace{}%
\AgdaInductiveConstructor{id⟷}\AgdaSymbol{)}\AgdaSpace{}%
\AgdaInductiveConstructor{◎}\<%
\\
%
\>[11]\AgdaInductiveConstructor{assocr₊}\AgdaSpace{}%
\AgdaInductiveConstructor{◎}\<%
\\
%
\>[11]\AgdaSymbol{(}\AgdaInductiveConstructor{id⟷}\AgdaSpace{}%
\AgdaInductiveConstructor{⊕}\AgdaSpace{}%
\AgdaInductiveConstructor{swap₊}\AgdaSymbol{)}\<%
\end{code}

\noindent The first implementation rewrites the incoming values as follows:
\[
\Tree [ {\small a} [ {\small b} {\small c} ] ] ~\to~
\Tree [ [ {\small a} {\small b} ] {\small c} ] ~\to~
\Tree [ {\small c} [ {\small a} {\small b} ] ] ~\to~
\Tree [ {\small c} [ {\small b} {\small a} ] ] ~.
\]
\noindent
The second implementation rewrites the incoming values as follows:
\[
\Tree [ {\small a} [ {\small b} {\small c} ] ] ~\to~
\Tree [ {\small a} [ {\small c} {\small b} ] ] ~\to~
\Tree [ [ {\small a} {\small c} ] {\small b} ] ~\to~
\Tree [ [ {\small c} {\small a} ] {\small b} ] ~\to~
\Tree [ {\small c} [ {\small a} {\small b} ] ] ~\to~
\Tree [ {\small c} [ {\small b} {\small a} ] ] ~.
\]
\noindent The two circuits are extensionally equal. Using the level-2
isomorphisms we can \emph{explicitly} construct a sequence of
rewriting steps that transforms the second circuit to the first.

Recalling that the $\lambda$-calculus arises as the internal language of Cartesian Closed Categories
(Elliott~\cite{Elliott-2017} gives a particularly readable account of this), we can think of $\Pi$ in similar terms, but
for symmetric Rig Groupoids instead. For example, we can ask what does the equivalence above represent? It is actually a
``linear'' representation of a 2-categorial commutative diagram! In fact, it is a painfully verbose version thereof, as
it includes many \emph{refocusing} steps because our language does not build associativity into its syntax. Categorical
diagrams usually do.  Thus if we rewrite the example in diagrammatic form, eliding all uses of associativity, but
keeping explicit uses of identity transformations, we get that \AgdaFunction{swap{-}fl2⇔swap{-}fl1} represents

\vspace*{3mm}
\begin{tikzcd}[column sep=normal, row sep=normal]
 && (a+c)+b \arrow [r, "\swapp \oplus\idd", ""{name=U, below}] & (c+a)+b \arrow [dr, "\assocrp"] && \\
 & a+(c+b) \arrow [ur, "\assoclp"] & & & c+(a+b) \arrow [dr, "\idd\oplus\swapp"] &  \\
a+(b+c) \arrow [ur, "\idd\oplus\swapp"] \arrow [r, "\assoclp"]
  \arrow [dr, "\assoclp"]
  \arrow [ddr, swap, "\assoclp"]
    & (a+b)+c \arrow [r, "\swapp"] &
    c+(a+b) \arrow [r, swap, "\assoclp", ""{name=D, above}]
    & |[alias=Z]| (c+a)+b \arrow [r, "\assocrp"] &c+(a+b) \arrow [r, "\idd\oplus\swapp"] & c+(b+a) \\
 & (a+b)+c \arrow [dr, "\swapp"] &&&& \\
 & (a+b)+c \arrow [dr, swap, "\swapp"] & c+(a+b) \arrow [rr, swap, "\idd", ""{name=DD, above}]
             \arrow [d, Rightarrow, "\idf\, \mathit{idl}\odot{l}"] &&
    c+(a+b) \arrow [ruu, "\idd\oplus\swapp"] & \\
 && c+(a+b) \arrow [rrruuu, bend right = 40, swap, "\idd\oplus\swapp"] && \\
 \arrow[Rightarrow, from=U, to=D, "\mathit{hexagon}\oplus{r}\, \boxdot\, \idf"]
 \arrow[Rightarrow, from=Z, to=DD, swap, "\idf\boxdot\mathit{linv}\odot{l}\,\boxdot\,\idf"]
\end{tikzcd}


%%% Local Variables:
%%% mode: latex
%%% TeX-master: "main"
%%% fill-column: 120
%%% End:
