\section{Applications to Reversible Circuits}
\label{sec:applications}

Using our semantics, we can normalise, synthesise, prove equivalence, and reason about~$\PiLang$ programs. We describe a
few applications in this section, and a more comprehensive list of applications and examples along with their
formalisation is available in our artifact~\cite{choudhuryArtifactSymmetriesReversible2021}.

% \[
%   \begin{tikzpicture}[->,>=stealth',auto,thick,node distance=1em]
%     % nodes
%     \node (A) at (0, 2) {\Afun{reversibleOr1}};
%     \node (B) at (0, 0) {\Afun{reversibleOr2}};
%     \node (C) at (4, 1) {\Afun{reversibleOrNorm}};
%     \node (D) at (9, 1) {\AgdaSymbol{(}\AgdaNumber{0 5 6 7 4 1 2 3}\AgdaSymbol{)}};
%     % arrows
%     \path
%     (A) edge[bend left] node [above left] { $\strut\normt_{1}$ } (C)
%     (B) edge[bend right] node [below left] { $\strut\normt_{1}$ } (C)
%     (D) edge node[above] { $\strut\term{quote}$ } (C);
%   \end{tikzpicture}
% \]

% \[
%   \begin{tabular}{>{$}c<{$} @{\hspace{5em}} >{$}c<{$}}
%     \tikzmarknode{A}{\Afun{reversibleOr1}} & \\
%                                            & \tikzmarknode{C}{\Afun{reversibleOrNorm}} \\
%     \tikzmarknode{B}{\Afun{reversibleOr2}} &
%   \end{tabular}
%   % \tikzmarknode{D}{4}
%   \begin{tikzpicture}[overlay, remember picture, shorten <=1mm,
%     nodes={inner sep=1pt, align=center, font=\footnotesize}]
%     \draw (A.south) -- ++ (-1,-1) node[below] {2 bits};
%     \draw (B.south) -- ++ (-.5,-1) node[below] {2 bits};
%     % \draw (D.south) -- ++ (.5,-1) node[below] {3 bits};
%   \end{tikzpicture}
% \]

\subsection{Normalisation}

Normalisation happens by evaluating a combinator to a permutation and then reifying it back. The evaluation step goes
from $\PiLang$ to $\PiPlusLang$ reducing $\times$ to $+$ and translating the corresponding combinators, then from
$\PiPlusLang$ to $\PiHatLang$ reducing them to natural numbers and adjacent transpositions, then finally to $\UFin$
through words in $\Sn$ and Lehmer codes. Finally, the quotation step goes all the way back until $\PiPlusLang$. We
define this composite $\normt$ function below using our previous evaluation function $\gdenot{\blank}$.

\begin{definition}[Normalisation of $\PiLang$ programs]
  \begin{gather*}
    \begin{aligned}
      \normt_{0} & : U \to \UPlus                                            \\
      \normt_{0} & = \quotep_{0} \comp \quoteh_{0} \comp \gdenot{\blank}_{0}
    \end{aligned}
    \qquad
    \begin{aligned}
      \normt_{1} & : (c : X \iso Y) \to \normt_{0}(X) \isop \normt_{0}(Y)    \\
      \normt_{1} & = \quotep_{1} \comp \quoteh_{1} \comp \gdenot{\blank}_{1}
    \end{aligned}
  \end{gather*}
\end{definition}

\noindent
As an example, consider the \Afun{reversibleOr1} circuit introduced in~\cref{sec:pi}. After applying $\normt_{1}$, the
type $\mathbb{B}\ 3$ normalises to $\mathbb{8}+ \equiv \mathbb{1} + (\mathbb{1} + (\mathbb{1} + (\mathbb{1} +
  (\mathbb{1} + (\mathbb{1} + (\mathbb{1} + (\mathbb{1} + \mathbb{0})))))))$, and the circuit reduces to the following
normal form:

\medskip
\resetnormtwo{}

\subsection{Equivalence}

By normalisation, we also get a decision procedure for program equivalence -- two circuits are equivalent iff they
represent the same permutation, or equivalently, they have the same normal form. Recall that in~\cref{sec:examples} we
introduced two different ways of computing reversible disjunction using two Qiskit circuit implementations, and then
in~\cref{sec:pi} showed the corresponding $\PiLang$ definitions: \Afun{reversibleOr1} and \Afun{reversibleOr2}.

After applying $\normt_{1}$ to each circuit, we can verify that they both reduce to the same normal form
\Afun{reversibleOrNorm}, and there exist 2-combinators between each of them and the normal form. More general,
user-guided, reasoning can be done using the sound and complete level-2 combinators to rewrite $\PiLang$ programs.
% as an example, we can manually produce a proof of equivalence of ?? and ?? -- a 2-combinator presented below: 
% \medskip
% \manualTwoCombinator{}

\subsection{Synthesis and verification}

Instead of manually writing $\PiLang$ programs to implement the reversible disjunction specification, it is possible to
simply write down the permutation directly and then quote it, synthesising a $\PiLang$ program. We show this below,
using a function $\term{lookup}$ to write tabulated permutations for readability.

\medskip
\resetperm{}

\noindent
The permutation uses the canonical encoding of sequences of bits as natural numbers, e.g.,
${(\mathsf{false},\mathsf{true},\mathsf{true})}$ is encoded as 011 or 3. The second entry in the $\term{lookup}$ table
maps index 1 (= 001) to the value 5 (= 101), following the reversible disjunction specification (recall that
$\mathsf{reversibleOr}(h,b_1,b_2) = (h \,\underline{\vee}\, (b_1 \vee b_2), ~b_1, ~b_2)$ where~$\vee$ is disjunction
and~$\underline{\vee}$ is exclusive-disjunction). Quoting this permutation generates the same normalised program
\Afun{reversibleOrNorm}, matching the desired structured type of a 1-combinator on $\mathbb{8}+$.

Similarly, we can also verify whether a circuit is correctly implemented -- by running the equivalence described above
in the other direction, we can evaluate the circuit to a bijection and compare it extensionally with the intended one.

\subsection{Reasoning}

In our development, we described a number of different representations of permutations: $\PiLang$, $\PiPlusLang$ and
$\PiHatLang$ programs, lists of transpositions, Lehmer codes, and automorphisms of finite sets. The equivalences between
these representations allow us to transfer theorems about permutations -- we prove them on the most suitable
representation, and then transport them to a different representation ``for free''.

As an example, consider the following problem. The Toffoli gate on 3 bits can be extended to a reversible gate on 4
bits, by inserting a wire in four possible positions. This gate performs identity on the newly inserted bit, and
controlled-controlled not on the three remaining ones. We show how to write these gates using \Afun{cif}.
\medskip
\extendedToffoli{}

\noindent
On the other hand, it is possible to construct a proper 4-bit toffoli gate.
\medskip
\toffoli{}

\noindent
Now, we can ask: is it possible to implement \Afun{toffoli₄} gate using only \Afun{toffoli₃} gates?
\medskip

Solving this problem in the circuit representation is difficult. However, we can observe that all \Afun{toffoli₃} gates
implement even permutations, that is, permutations that contain an even number of transpositions. On the other hand,
\Afun{toffoli₄} implements an odd permutation. It can be easily proved that any composition of even permutations always
produces an even permutation, which answers the question to the negative. Thus, we formulate the theorem using the
representation of lists of transpositions, and transfer it to circuits using the equivalences proven in the previous
sections.

We define a function $\parity : (X \iso Y) \to \Bool$ to compute the parity of a permutation defined by a combinator
$c$, by first transforming a $\PiLang$ program into a list of transpositions (using $\evalh_{1}$ defined
in~\cref{sec:equivalence}), and then computing the parity of this representation, which amounts to a simple checking of
the parity of the length of the list.

\begin{propositionrep}
  \label{prop:parity-preserved}
  For two $\PiLang$ circuits $c_1, c_2$, if $c_1 \Iso c_2$, then $\parity(c_1) = \term{parity}(c_2)$.
\end{propositionrep}
\begin{proof}
  First, it is easy to check that Coxeter relations preserve parity of the list. Second, if $c_1$ and $c_2$ are related
  by a 2-combinator, then their Lehmer codes are the same, so their list-of-transpositions representations have to be
  related by the Coxeter relation. Composing these two facts completes the proof.
\end{proof}

\noindent
Using~\cref{prop:parity-preserved} and the fact that any composition of even circuits is even, we complete the proof by
computing the $\parity$ of \Afun{toffoli₄}, and all of the \Afun{toffoli₃} combinations.

% \paragraph*{Program synthesis.} The NbE process embodies a quoting mechanism that synthesizes programs from
% permutations. Indeed, instead of writing a program for \Afun{reset 2}, one could simply specify the desired permutation
% as:

%%\begin{minipage}{.65\textwidth}
%%  \PiRESET{}
%%\end{minipage}
%%\begin{minipage}{.30\textwidth}
%%  \begin{center}
%%  \resizebox{0.5\textwidth}{!}{\begin{tikzpicture}
	\begin{pgfonlayer}{nodelayer}
		\node [style=none] (6) at (-9, 7) {};
		\node [style=none] (9) at (-8, 7) {};
		\node [style=none] (14) at (-9, 2) {};
		\node [style=none] (15) at (-8, 2) {};
		\node [style=none] (17) at (-8.5, 4.75) {\dist};
		\node [style=none] (41) at (-8.5, 6) {{\color{red}$\mathbb{F}$}};
		\node [style=none] (44) at (-8.5, 3.25) {{\color{red}$\mathbb{T}$}};
		\node [style=none] (75) at (-6.75, 4) {};
		\node [style=none] (76) at (-6.75, 3.5) {};
		\node [style=none] (77) at (-5.75, 3.5) {};
		\node [style=none] (78) at (-5.75, 4) {};
		\node [style=none] (79) at (-5.75, 3.5) {};
		\node [style=none] (80) at (-6.25, 3.75) {\AgdaFunction{NOT}};
		\node [style=none] (81) at (-7, 7) {};
		\node [style=none] (82) at (-7, 5) {};
		\node [style=none] (83) at (-5.5, 5) {};
		\node [style=none] (84) at (-5.5, 7) {};
		\node [style=none] (95) at (-4.5, 7) {};
		\node [style=none] (96) at (-3.5, 7) {};
		\node [style=none] (98) at (-4.5, 2) {};
		\node [style=none] (99) at (-3.5, 2) {};
		\node [style=none] (101) at (-4, 4.75) {\factor};
		\node [style=none] (102) at (-4, 6) {{\color{red}$\mathbb{F}$}};
		\node [style=none] (103) at (-4, 3.25) {{\color{red}$\mathbb{T}$}};
		\node [style=none] (106) at (-4.5, 4.5) {};
		\node [style=none] (113) at (-6.25, 5.75) {$\AgdaFunction{RESET}_{n-1}$};
		\node [style=none] (127) at (-6.5, 4.5) {};
		\node [style=none] (128) at (-6, 4.5) {};
		\node [style=none] (129) at (-6.25, 4.75) {};
		\node [style=none] (130) at (-6.25, 4.25) {};
		\node [style=none] (131) at (-10.75, 4.75) {};
		\node [style=none] (132) at (-10.75, 5.25) {};
		\node [style=none] (134) at (-10.75, 5.75) {};
		\node [style=none] (135) at (-10.25, 5.75) {};
		\node [style=none] (136) at (-10.75, 5.25) {};
		\node [style=none] (137) at (-10.25, 5.25) {};
		\node [style=none] (138) at (-9.5, 5.25) {};
		\node [style=none] (139) at (-9.5, 5.75) {};
		\node [style=none] (140) at (-10.75, 3.75) {};
		\node [style=none] (141) at (-9, 3.75) {};
		\node [style=none] (142) at (-9.75, 4.25) {$\vdots$};
		\node [style=none] (143) at (-9, 5.75) {};
		\node [style=none] (144) at (-9, 5.25) {};
		\node [style=none] (145) at (-9, 4.75) {};
		\node [style=none] (146) at (-10.75, 4.75) {};
		\node [style=none] (147) at (-3.5, 4.75) {};
		\node [style=none] (148) at (-3.5, 5.25) {};
		\node [style=none] (149) at (-3.5, 5.75) {};
		\node [style=none] (150) at (-3, 5.75) {};
		\node [style=none] (151) at (-3.5, 5.25) {};
		\node [style=none] (152) at (-3, 5.25) {};
		\node [style=none] (153) at (-2.25, 5.25) {};
		\node [style=none] (154) at (-2.25, 5.75) {};
		\node [style=none] (155) at (-3.5, 3.75) {};
		\node [style=none] (156) at (-1.75, 3.75) {};
		\node [style=none] (157) at (-2.75, 4.25) {$\vdots$};
		\node [style=none] (158) at (-1.75, 5.75) {};
		\node [style=none] (159) at (-1.75, 5.25) {};
		\node [style=none] (160) at (-1.75, 4.75) {};
		\node [style=none] (161) at (-3.5, 4.75) {};
		\node [style=none] (162) at (-8, 6.25) {};
		\node [style=none] (163) at (-8, 6.75) {};
		\node [style=none] (164) at (-8, 6.75) {};
		\node [style=none] (165) at (-7, 6.75) {};
		\node [style=none] (167) at (-8, 5.25) {};
		\node [style=none] (168) at (-7, 5.25) {};
		\node [style=none] (169) at (-7.5, 5.75) {$\vdots$};
		\node [style=none] (171) at (-7, 6.25) {};
		\node [style=none] (172) at (-8, 6.25) {};
		\node [style=none] (173) at (-5.5, 6.25) {};
		\node [style=none] (174) at (-5.5, 6.75) {};
		\node [style=none] (175) at (-5.5, 6.75) {};
		\node [style=none] (176) at (-4.5, 6.75) {};
		\node [style=none] (177) at (-5.5, 5.25) {};
		\node [style=none] (178) at (-4.5, 5.25) {};
		\node [style=none] (179) at (-5, 5.75) {$\vdots$};
		\node [style=none] (180) at (-4.5, 6.25) {};
		\node [style=none] (181) at (-5.5, 6.25) {};
		\node [style=none] (182) at (-5.75, 3.25) {};
		\node [style=none] (183) at (-5.75, 3.75) {};
		\node [style=none] (184) at (-5.75, 3.75) {};
		\node [style=none] (185) at (-4.5, 3.75) {};
		\node [style=none] (186) at (-6.75, 2.25) {};
		\node [style=none] (187) at (-4.5, 2.25) {};
		\node [style=none] (188) at (-5.25, 2.75) {$\vdots$};
		\node [style=none] (189) at (-4.5, 3.25) {};
		\node [style=none] (190) at (-5.75, 3.25) {};
		\node [style=none] (191) at (-8, 3.25) {};
		\node [style=none] (192) at (-8, 3.75) {};
		\node [style=none] (193) at (-8, 3.75) {};
		\node [style=none] (194) at (-6.75, 3.75) {};
		\node [style=none] (195) at (-8, 2.25) {};
		\node [style=none] (196) at (-6.75, 2.25) {};
		\node [style=none] (197) at (-7.5, 2.75) {$\vdots$};
		\node [style=none] (198) at (-5.75, 3.25) {};
		\node [style=none] (199) at (-8, 3.25) {};
	\end{pgfonlayer}
	\begin{pgfonlayer}{edgelayer}
		\draw (6.center) to (14.center);
		\draw (14.center) to (15.center);
		\draw (15.center) to (9.center);
		\draw (6.center) to (9.center);
		\draw (75.center) to (76.center);
		\draw (76.center) to (77.center);
		\draw (75.center) to (78.center);
		\draw (78.center) to (77.center);
		\draw (81.center) to (82.center);
		\draw (82.center) to (83.center);
		\draw (83.center) to (84.center);
		\draw (81.center) to (84.center);
		\draw (95.center) to (98.center);
		\draw (98.center) to (99.center);
		\draw (99.center) to (96.center);
		\draw (95.center) to (96.center);
		\draw (127.center) to (128.center);
		\draw (129.center) to (130.center);
		\draw (134.center) to (135.center);
		\draw [style=new edge style 5] (136.center) to (137.center);
		\draw (135.center) to (138.center);
		\draw [style=new edge style 5] (137.center) to (139.center);
		\draw (140.center) to (141.center);
		\draw (138.center) to (144.center);
		\draw [style=new edge style 5] (139.center) to (143.center);
		\draw (146.center) to (145.center);
		\draw [style=new edge style 5] (149.center) to (150.center);
		\draw (151.center) to (152.center);
		\draw [style=new edge style 5] (150.center) to (153.center);
		\draw (152.center) to (154.center);
		\draw (155.center) to (156.center);
		\draw [style=new edge style 5] (153.center) to (159.center);
		\draw (154.center) to (158.center);
		\draw (161.center) to (160.center);
		\draw (164.center) to (165.center);
		\draw (167.center) to (168.center);
		\draw (172.center) to (171.center);
		\draw (175.center) to (176.center);
		\draw (177.center) to (178.center);
		\draw (181.center) to (180.center);
		\draw (184.center) to (185.center);
		\draw (186.center) to (187.center);
		\draw (190.center) to (189.center);
		\draw (193.center) to (194.center);
		\draw (195.center) to (196.center);
		\draw (199.center) to (198.center);
		\draw [style=new edge style 4] (144.center) to (164.center);
		\draw [style=new edge style 4] (144.center) to (193.center);
		\draw [style=new edge style 4] (145.center) to (172.center);
		\draw [style=new edge style 4] (145.center) to (199.center);
		\draw [style=new edge style 4] (141.center) to (167.center);
		\draw [style=new edge style 4] (141.center) to (195.center);
		\draw [style=new edge style 4] (176.center) to (151.center);
		\draw [style=new edge style 4] (151.center) to (185.center);
		\draw [style=new edge style 4] (180.center) to (161.center);
		\draw [style=new edge style 4] (161.center) to (189.center);
		\draw [style=new edge style 4] (178.center) to (155.center);
		\draw [style=new edge style 4] (155.center) to (187.center);
	\end{pgfonlayer}
\end{tikzpicture}
}
%%  \end{center}
%%\end{minipage}

%% \resettwo{}

% \noindent The syntax will be explained in detail in the next section and the full definitions of the helpers are
% provided in the supplementary material. For now, it is sufficient to know that there is some program that implements the
% reversible function of interest and that applying \Afun{reset} to 2 produces \verb|reversibleOr2| from the introduction.

% In principle, the normalised program can be produced following two strategies: (i) by repeatedly applying the rewrite
% rules of our calculus of reversible functions (explained in Sec.~\ref{sec:reversibletwo}), or (ii) in the case above by
% using a NbE process that evaluates the program to a permutation on a finite set of 8
% elements and reifies that permutation back to a program. The key idea of the NbE process is a systematic way to express
% permutations as sequences of adjacent swaps as illustrated in the following small example where the permutation on the
% left is compiled to the sequence of four adjacent transpositions on the right:

% \note{Motivation: There are two reversible circuits which describe the following permutation. They can be shown to be
%   equal using the 2-combinators.}

% \[
%   \begin{tikzpicture}
%     \begin{knot}[clip width=5]
%       \filldraw (0,5) circle (2pt) node[above] {0};
%       \filldraw (1,5) circle (2pt) node[above] {1};
%       \filldraw (2,5) circle (2pt) node[above] {2};
%       \filldraw (3,5) circle (2pt) node[above] {3};
%       \filldraw (4,5) circle (2pt) node[above] {4};
%       \filldraw (0,0) circle (2pt) node[below] {1};
%       \filldraw (1,0) circle (2pt) node[below] {4};
%       \filldraw (2,0) circle (2pt) node[below] {0};
%       \filldraw (3,0) circle (2pt) node[below] {3};
%       \filldraw (4,0) circle (2pt) node[below] {2};
%       \strand (0,5) .. controls (0.5,0.5) and (1.5,3.5) .. (2,0);
%       \strand (1,5) .. controls (0.75,0.5) and (0.25,3.5) .. (0,0);
%       \strand (2,5) .. controls (2.5,2.5) and (3.5,1.5) .. (4,0);
%       \strand (3,5) .. controls (4.5,2.5) and (4,1.5) .. (3,0);
%       \strand (4,5) .. controls (3.5,2.5) and (1.5,2.5) .. (1,0);
%       \flipcrossings{4,5};
%     \end{knot}
%   \end{tikzpicture}
% \]

% \paragraph*{Program equivalence.} The permutation above reveals another way to think about the desired program: it is a
% special addition circuit that keeps 0 and 4 fixed but otherwise adds 4 modulo 8 to its input. From this specification,
% one can use a standard synthesis algorithm for reversible circuits~\cite{10.1145/775832.775915} to generate the
% following program:

% \adder{}

% \noindent The \Afun{adder3} program looks nothing like the original \Afun{reset 2} program and yet they both have the
% same normal form thus establishing their equivalence. The reader can check that this circuit is the same as
% \verb|reversibleOr1| from the introduction.

% Now imagine we want to write the following reversible function:

% 0 -> 0
% 8 -> 8
% n -> n + 8 `mod` 16

%%% Local Variables:
%%% mode: latex
%%% TeX-master: "main"
%%% fill-column: 120
%%% End:
