\section{Discussion \& Related Work}~\label{sec:discussion}

In this paper, we \ldots

Our work lies at the intersection of programming language theory, category theory, group theory, rewriting theory, and
formalised mathematics. We review related work in the literature for each topic.

\paragraph{Algebraic Theories} In universal algebra, algebraic theories are used to describe algebraic structures, such
as groups or rings. A specific group or ring is a model of the appropriate algebraic theory. Algebraic theories are
usually \emph{presented} in terms of logical syntax, that is, as first-order theories whose signatures allow only
functional symbols, and whose axioms are universally quantified equations. In his seminal
thesis~\cite{lawvereFUNCTORIALSEMANTICSALGEBRAIC1963}, Lawvere defined a presentation-free categorical notion of
universal algebraic structure, called a Lawvere theory.

Programming Languages, such as the $\lambda$-calculus, can be viewed as algebraic structures with variable-binding
operators, which can be formalised using second-order algebraic theories~\cite{fioreSecondOrderAlgebraicTheories2010},
or algebraic theories with closed structure~\cite{hylandClassicalLambdaCalculus2017}, called $\lambda$-theories, making
the $\lambda$-calculus the presentation of the initial $\lambda$-theory $\Lambda$.

Our family of reversible languages have been presented as first-order algebraic
2-theories~\cite{cohenCoherenceRewriting2theories2009,bekeCategorificationTermRewriting2011,yanofskySyntaxCoherence2000},
which are a categorification of algebraic theories. The types $\zerot$ and $\onet$ are nullary function symbols, the
type formers $+$ and $\times$ are binary function symbols, the 1-combinators are invertible reduction rules, and the
2-combinators are equations or coherence diagrams of compositions of reduction rules. Just like models of Lawvere
theories are given by algebras of (finitary) monads on $\SetCat$, models of 2-theories are given by algebras of 2-monads
on $\CatCat$. The particular one we're interested in here is the free symmetric monoidal completion 2-monad.

\paragraph{Free Symmetric Monoidal Category} The forgetful functor from $\SymMonCat$, the 2-category of symmetric
monoidal categories, strong symmetric monoidal functors, and symmetric monoidal natural transformations, to the
2-category $\CatCat$, has a left adjoint.

Curry-Howard-Lambek correspondence.

Univalent Subuniverses, Univalent Typoids.

Higher order rewriting theory in constructive type theory.

Free monoids, free commutative monoids, normalisation of monoids.

Coherence problems in type theory, coherence via Well-Foundedness.

Computational group theory (in HoTT?)

Presentation of the free symmetric monoidal groupoid on one generator, proof of Mac Lane's coherence theorem in HoTT.

Alternatively, we can describe them using HITs, prove they have the universal property and hence equivalent.
The full Pi is FSMG 1, and the normalised fragment of Pi is M1.

Applications of FSMG and history of the coherence theorems.

Other proofs of coherence theorems, Joyal-Street.

Pi has other extensions (fractional/negative/recursive types).
What are the free X monoidal structures they're describing?

Using our presentation of $S_{n}$, we can construct the Eilenberg-Maclane space (using a HIT) $K(S_{n},1)$. Then, it
should be true that $\UFin \eqv \sqcup_{n:\Nat} K(S_{n},1)$. This is future work.

Other applications of symmetric groups.

Actions of symmetric groups, permutation groupoids.

%%% Local Variables:
%%% mode: latex
%%% TeX-master: "main"
%%% fill-column: 120
%%% End:
