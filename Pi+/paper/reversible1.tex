\section{A Reversible Programming Language: Set-Theoretic Semantics}~\label{sec:reversibleone}

The practice of programming languages is replete with \emph{ad hoc} instances of reversible computations: database
transactions, mechanisms for data provenance, checkpoints, stack and exception traces, logs, backups, rollback
recoveries, version control systems, reverse engineering, software transactional memories, continuations, backtracking
search, and multiple-level undo features in commercial applications. In the early nineties,
\citet{Baker:1992:LLL,Baker:1992:NFT} argued for a systematic, first-class, treatment of reversibility. But intensive
research in full-fledged reversible models of computations and reversible programming languages was only sparked by the
discovery of deep connections between physics and
computation~\cite{Landauer:1961,PhysRevA.32.3266,Toffoli:1980,bennett1985fundamental,Frank:1999:REC:930275, Hey:1999:FCE:304763,fredkin1982conservative}, and by the
potential for efficient quantum computation~\cite{springerlink:10.1007/BF02650179}.

The early developments of reversible programming languages started
with a conventional programming language, e.g., an extended
$\lambda$-calculus, and either:
\begin{enumerate}
\item extended the language with a history
mechanism~\cite{vanTonder:2004,Kluge:1999:SEMCD,lorenz,danos2004reversible}, or
\item imposed constraints on the control flow constructs to make them
reversible~\cite{Yokoyama:2007:RPL:1244381.1244404}.
\end{enumerate}
More foundational approaches recognize that reversible programming languages require a fresh approach and should be
designed from first principles without the detour via conventional irreversible
languages~\cite{Yokoyama:2008:PRP,Mu:2004:ILRC,abramsky2005structural,DiPierro:2006:RCL:1166042.1166047,
  rc2011,James:2012:IE:2103656.2103667,Carette2016}.

%%%%%%%%%%%%%%%%%
\subsection{The $\Pi$ Family of Languages}

A natural candidate for a semantic foundation for reversible programming languages is the notion of type
isomorphism. Indeed, the type isomorphisms among finite types represent a universal language for reversible boolean
circuits~\cite{James:2012:IE:2103656.2103667} and the extension with recursive types and trace
operators~\cite{Hasegawa:1997:RCS:645893.671607} is a Turing-complete reversible
language~\cite{James:2012:IE:2103656.2103667,rc2011}.

Focusing on finite types, the building blocks of type theory are: the
empty type ($\mathbb{0}$), the unit type ($\mathbb{1}$) containing
just one value $\Acon{tt}$, the sum type ($+$) containing values of
the form $\inlv{v}$ and $\inrv{v}$, and the product ($\times$) type
containing pairs of values $(v_1,v_2)$. One may view each type $A$
as a collection of physical wires that can transmit $|A|$ distinct
values where $|A|$ is a natural number that indicates the size of a
type, computed as: $| \mathbb{0} | = 0$; $| \mathbb{1} | = 1$;
$| A + B | = | A | + |B |$; and
$| A \times B | = | A | * |B |$.  Thus the type
$\mathbb{B} = \mathbb{1} + \mathbb{1}$ corresponds to a wire that can
transmit one of two values, i.e., bits, with the convention that
$\inlv{\Acon{tt}}$ represents \AgdaFunction{𝔽} and
$\inrv{\Acon{tt}}$ represents \AgdaFunction{𝕋}.  The type
$\mathbb{B} \times \mathbb{B} \times \mathbb{B}$ corresponds to a
collection of wires that can transmit three bits. From that
perspective, a type isomorphism between types $A$ and $B$
(such that $|A|=|B|=n$) models a \emph{reversible}
combinational circuit that \emph{permutes} the $n$ different
values. These type isomorphisms are collected in Fig.~\ref{pi-terms}.
\begin{figure}[t]
{\scalebox{\scalef}{$%
%%\noindent\begin{minipage}{.7\linewidth}
\begin{array}{rrcll}
\idc :& A & \iso & A &: \idc \\
\\
\identlp :&  \mathbb{0} + A & \iso & A &: \identrp \\
\swapp :&  A + B & \iso & B + A &: \swapp \\
\assoclp :&  A + (B + C) & \iso & (A + B) + C &: \assocrp \\ [1.5ex]
\identlt :&  \mathbb{1} \times A & \iso & A &: \identrt \\
\swapt :&  A \times B & \iso & B \times A &: \swapt \\
\assoclt :&  A \times (B \times C) & \iso & (A \times B) \times C &: \assocrt \\ [1.5ex]
\absorbr :&~ \mathbb{0} \times A & \iso & \mathbb{0} ~ &: \factorzl \\
\dist :&~ (A + B) \times C & \iso & (A \times C) + (B \times C)~ &: \factor
  \end{array}$}}

{\scalebox{\scalef}{%
\Rule{}
{\jdg{}{}{c_1 : A \iso B} \quad \vdash c_2 : B \iso C}
{\jdg{}{}{c_1 \fatsemi c_2 : A \iso C}}
{}

\Rule{}
{\jdg{}{}{c_1 : A \iso B} \quad \vdash c_2 : C \iso D}
{\jdg{}{}{c_1 \oplus c_2 : A + C \iso B + D}}
{}

\Rule{}
{\jdg{}{}{c_1 : A \iso B} \quad \vdash c_2 : C \iso D}
{\jdg{}{}{c_1 \otimes c_2 : A \times C \iso B \times D}}
{}
}}
\caption{$\Pi$-terms, combinators, and their types.}
\label{pi-terms}
\end{figure}

Each line in the top part of the figure introduces a pair of dual constants that witness the type isomorphism in the
middle.  These are the \emph{base} (non-reducible) terms of $\Pi$. The isomorphisms are extended to form a congruence
relation by adding the three constructors at the bottom of the figure that witness equivalence and compatible
closure. Although austere, this combinator-based language has the advantage of being amenable to formal analysis for at
least two reasons: (i) it is conceptually simple and small, and (ii) it has direct and evident connections to type
theory and category theory and hence possesses a rich algebraic structure which we exploit in the remainder of the
paper.

% Specifically, the type isomorphisms of~$\Pi$ are sound and complete for all permutations on finite
% types~\cite{Fiore:2004,fiore-remarks} and hence they are \emph{complete} for expressing reversible combinational
% circuits~\cite{fredkin1982conservative, James:2012:IE:2103656.2103667,Toffoli:1980}. Algebraically, these types and
% combinators form a \emph{commutative semiring} (up to type isomorphism). Logically, they form a superstructural logic
% capturing space-time trade-offs~\cite{superstructural}. Categorically, they form a \emph{distributive bimonoidal
%   category}~\cite{laplaza72}.

% For the fragment over finite types, the syntax of the language $\Pi$ consists of several sorts:

%  {\scalebox{\scalef}{$%
% \begin{array}{lrcl}
% \textit{Value types} & A,B,C,D &::=& \mathbb{0} \alt \mathbb{1} \alt A+B \alt A\times B \\
% \textit{Values}      & v,w,x,y &::=& \Acon{tt} \alt \inlv{v} \alt \inrv{v} \alt (v,w) \\
% \textit{Program types} &&& A \leftrightarrow B \\
% \textit{Programs} & c &::=& (\textrm{See Fig.~\ref{pi-terms}})
% \end{array}$}}

\begin{figure}[t]
{\scalebox{\scalef}{%
$\begin{array}{rclcrcl}
\delta(\identlp,~\inrv{v}) & = & v & & \delta(\identrp,~v)        & = & \inrv{v}\\
\delta(\swapp,~\inlv{v})     & = & \inrv{v}\\
\delta(\swapp,~\inrv{v})     & = & \inlv{v}\\
\delta(\assoclp,~\inlv{v})          & = & \inlv{(\inlv{v})} & & \delta(\assocrp,~\inlv{(\inlv{v})}) & = & \inlv{v}         \\
\delta(\assoclp,~\inrv{(\inlv{v})}) & = & \inlv{(\inrv{v})} & & \delta(\assocrp,~\inlv{(\inrv{v})}) & = & \inrv{(\inlv{v})}\\
\delta(\assoclp,~\inrv{(\inrv{v})}) & = & \inrv{v}          & & \delta(\assocrp,~\inrv{v})          & = & \inrv{(\inrv{v})}\\
\delta(\identlt,~(\Acon{tt},v)) & = & v & & \delta(\identrt,~v)             & = & (\Acon{tt},v)\\
\delta(\swapt,~(x,y))         & = & (y,x)    \\
\delta(\assoclt,~(x,(y,z))) & = & ((x,y),z) & & \delta(\assocrt,~((x,y),z)) & = & (x,(y,z))\\
\delta(\dist,~(\inlv{x},z)) & = & \inlv{(x,z)} & & \delta(\factor,~\inlv{(x,z)}) & = & (\inlv{x},z)\\
\delta(\dist,~(\inrv{y},z)) & = & \inrv{(y,z)} & & \delta(\factor,~\inrv{(y,z)}) & = & (\inrv{y},z)
\end{array}$}}
\caption{Semantics of base combinators}\label{fig:delta}
\end{figure}

%%%%%%%%%
\subsection{Denotational Semantics}

\noindent A simple denotational semantics of $\Pi$ can be specified by mapping each type to a finite set and each
combinator to a bijection between finite sets, that is, a permutation.

Functions with finite sets as domains are finitely supported, that is, their outputs can be tabulated using the
canonical ordering on $\Fin[n]$. For bijections, this gives a listed permutation. By observing the action of the
combinators on the values of the finite set, we can define a denotational semantics which constructs the bijection.

\note{\(\Fin[n] \to A \eqv \type{Vec_{n}}(A)\)}

An alternative point of view is to work in the category of finite sets and functions, $\SetFin$, which is the category
freely generated by finite coproduct completion of the terminal category. Objects of $\SetFin$ can be identified with
sets of fixed cardinality, that is, $\Fin[n] \defeq \Set{0,1,\ldots,n-1}$. $\SetFin$ has finite coproducts and products,
which lets us interpret the types of $\PiLang$. Combinators are intepreted as morphisms in $\SetFin$, but we have to
restrict to invertible morphisms, that is, isomorphisms. This gives the \emph{groupoid} of finite sets and bijections,
$\BFin \defeq \mathsf{core}(\SetFin)$. The isomorphisms satisfied by coproducts and products in $\SetFin$ lift to
$\BFin$, but they're no longer categorical coproducts and products. They give two symmetric monoidal tensor products on
$\BFin$, the additive and multiplicative ones, with the multiplicative tensor distributing over the additive tensor.

\note{Explain code below as best as you can}

% semT : (t : U) → Set
% semT t = Fin ∣ t ∣

% semC : {t₁ t₂ : U} → (c : t₁ ⟷₁ t₂) → Fin ∣ t₁ ∣ → Fin ∣ t₂ ∣
% semC unite₊l = idf _
% semC uniti₊l = idf _
% semC {I × t₂} {t₂} unite⋆l = transport Fin (N.+-unit-r ∣ t₂ ∣)
% semC {t₁} {I × t₁} uniti⋆l = transport Fin (! (N.+-unit-r ∣ t₁ ∣))
% semC {t₁ + t₂} {t₂ + t₁} swap₊ = (comp+ t₂ t₁) ∘ (swap+Fin {t₁} {t₂}) ∘ (decomp+ t₁ t₂)
% semC {t₁ × t₂} {t₂ × t₁} swap⋆ = (comp* t₂ t₁) ∘ (swap*Fin {t₁} {t₂}) ∘ (decomp* t₁ t₂)
% semC {t₁ + (t₂ + t₃)} {(t₁ + t₂) + t₃} assocl₊ = transport Fin (! (N.+-assoc (∣ t₁ ∣) (∣ t₂ ∣) (∣ t₃ ∣)))
% semC {(t₁ + t₂) + t₃} {t₁ + (t₂ + t₃)} assocr₊ = transport Fin (N.+-assoc (∣ t₁ ∣) (∣ t₂ ∣) (∣ t₃ ∣))
% semC {t₁ × (t₂ × t₃)} {(t₁ × t₂) × t₃} assocl⋆ = transport Fin (! (*-assoc (∣ t₁ ∣) (∣ t₂ ∣) (∣ t₃ ∣)))
% semC {(t₁ × t₂) × t₃} {t₁ × (t₂ × t₃)} assocr⋆ = transport Fin (*-assoc (∣ t₁ ∣) (∣ t₂ ∣) (∣ t₃ ∣))
% semC absorbr ()
% semC {t × O} {O} absorbl = transport Fin (*-unit-r ∣ t ∣)
% semC factorzr ()
% semC factorzl ()
% semC {(t₁ + t₂) × t₃} {(t₁ × t₃) + (t₂ × t₃)} dist =
%   comp+ (t₁ × t₃) (t₂ × t₃) ∘
%   map+ (comp* t₁ t₃) (comp* t₂ t₃) ∘
%   distNative ∘
%   map* (decomp+ t₁ t₂) (idf _) ∘
%   decomp* (t₁ + t₂) t₃
% semC {(t₁ × t₃) + (t₂ × t₃)} {(t₁ + t₂) × t₃} factor =
%   comp* (t₁ + t₂) t₃ ∘
%   map* (comp+ t₁ t₂) (idf _) ∘
%   factorNative ∘
%   map+ (decomp* t₁ t₃) (decomp* t₂ t₃) ∘
%   decomp+ (t₁ × t₃) (t₂ × t₃)
% semC id⟷₁ = idf _
% semC (c₁ ◎ c₂) = semC c₂ ∘ semC c₁
% semC {t₁ + t₂} {t₃ + t₄} (c₁ ⊕ c₂) = (comp+ t₃ t₄) ∘ map+ (semC c₁) (semC c₂) ∘ (decomp+ t₁ t₂)
% semC {t₁ × t₂} {t₃ × t₄} (c₁ ⊗ c₂) = (comp* t₃ t₄) ∘ map* (semC c₁) (semC c₂) ∘ (decomp* t₁ t₂)

%%%%%%%%%
\subsection{Examples}
\label{sec:langRev-examples}
\label{examples}

\note{Revisit to pick examples relevant to the rest of the paper; also unify notation}

\paragraph*{Booleans}
Let us start with encoding booleans. We use the type \ensuremath{1{\sumtype}1} to
represent booleans with \ensuremath{\mathit{left} ~()} representing \ensuremath{\mathit{true}} and
\ensuremath{\mathit{right}~()} representing \ensuremath{\mathit{false}}.
Boolean negation is straightforward to define:

\ensuremath{\mathit{not} : \mathit{bool} \leftrightarrow \mathit{bool}}

\ensuremath{\mathit{not} = \swapp}

\noindent
It is easy to verify that \ensuremath{\mathit{not}} changes \ensuremath{\mathit{true}} to \ensuremath{\mathit{false}} and
vice versa.

\paragraph*{Bit Vectors.}
We can represent $n$-bit words using an n-ary product of
\ensuremath{\mathit{bool}}s. For example, we can represent a 3-bit word, \ensuremath{\mathit{word}_3},
using the type \ensuremath{\mathit{bool} {\prodtype} (\mathit{bool} {\prodtype}  \mathit{bool})}.  We can perform various
operations on these 3-bit words using combinators in \ensuremath{\Pi }. For
instance the bitwise \ensuremath{\mathit{not}} operation is the parallel composition of
three \ensuremath{\mathit{not}} operations:

\ensuremath{\mathit{not}_{\mathit{word}_3} :: \mathit{word}_3 \leftrightarrow \mathit{word}_3}

\ensuremath{\mathit{not}_{\mathit{word}_3} = \mathit{not}  {\prodtype}  (\mathit{not}  {\prodtype}  \mathit{not})}

\noindent We can express a 3-bit word reversal operation as follows:

\ensuremath{\mathit{reverse} : \mathit{word}_3 \leftrightarrow \mathit{word}_3}

\ensuremath{\mathit{reverse} = \swapt \odot (\swapt  \otimes  \idc)~ \odot \assocrt}

\noindent We can check that \ensuremath{\mathit{reverse}} does the right thing by
applying it to a value \ensuremath{(v_1, (v_2, v_3))} and writing out the full
derivation tree of the reduction.  The combinator \ensuremath{\mathit{reverse}}, like
many others we will see in this paper, is formed by sequentially
composing several simpler combinators. Instead of presenting the
operation of \ensuremath{\mathit{reverse}} as a derivation tree, it is easier (purely
for presentation reasons) to flatten the tree into a sequence of
reductions as caused by each component. Such a sequence of reductions
is given below:
\[\begin{array}{rlr}
 & (v_1, (v_2, v_3)) \\
 \swapt & ((v_2, v_3), v_1) \\
 \swapt \otimes  \idc & ((v_3, v_2), v_1) \\
 \assocrt & (v_3, (v_2, v_1)) \\
 \end{array}\]
%subcode source isomorphisms.tex:979

\noindent On the first line is the initial value. On each subsequent
line is a fragment of the \ensuremath{\mathit{reverse}} combinator and the value that
results from applying this combinator to the value on the previous
line. For example, \ensuremath{\swapt} transforms \ensuremath{(v_1, (v_2, v_3))} to
\ensuremath{((v_2,v_3),v_1)}.  On the last line we see the expected result with
the bits in reverse order.

We can also draw out the graphical representation of the 3-bit reverse
combinator. In the graphical representation, it is clear that the
combinator achieves the required shuffling.

% \inkscape{reverse-3-bit.pdf}

\paragraph*{Conditionals.}
Even though \ensuremath{\Pi } lacks conditional expressions, they are
expressible using the distributivity and factoring laws. The
diagrammatic representation of \ensuremath{\dist} shows that it redirects the flow
of a value \ensuremath{v:b} based on the value of another one of type
\ensuremath{b_1{\sumtype}b_2}. If we choose \ensuremath{1{\sumtype}1} to be
\ensuremath{\mathit{bool}} and apply either \ensuremath{c_1:b_1\leftrightarrow
b_2} or \ensuremath{c_2:b_1\leftrightarrow b_2} to the value \ensuremath{v},
then we essentially have an `if' expression.

\ensuremath{\mathit{if}_{c_1,c_2} : \mathit{bool}  {\prodtype}  b_1 \leftrightarrow \mathit{bool}  {\prodtype}  b_2}

\ensuremath{\mathit{if}_{c_1,c_2} = \dist \odot ((\idc  \otimes\  c_1) {\sumtype} (\idc \otimes\  c_2)) \odot \factor}


% \inkscape{if-c1-c2.pdf}


The diagram above shows the input value of type \ensuremath{(1{\sumtype}1) {\prodtype}  b_1}
processed by the distribute operator \ensuremath{\dist}, which converts it into
a value of type \ensuremath{(1 {\prodtype}  b_1){\sumtype}(1 {\prodtype}  b_1)}. In the
\ensuremath{\mathit{left}} branch, which corresponds to the
case when the boolean is \ensuremath{\mathit{true}} (i.e. the value was
\ensuremath{\mathit{left} ~()}), the combinator~\ensuremath{c_1} is applied to
the value of type~\ensuremath{b_1}. The right
branch which corresponds to the boolean being \ensuremath{\mathit{false}} passes
the value of type \ensuremath{b_1} through the combinator \ensuremath{c_2}.
The inverse of \ensuremath{\dist}, namely \ensuremath{\factor} is applied
to get the final result of type \ensuremath{(1{\sumtype}1) {\prodtype} b_2}.

\paragraph*{Logic Gates}
There are several universal primitives for conventional (irreversible)
hardware circuits, such as \ensuremath{\mathit{nand}} and \ensuremath{\mathit{fanout}}. In the case
of reversible hardware circuits, the canonical universal primitive is
the Toffoli gate~\cite{Toffoli:1980}. The Toffoli gate takes three
boolean inputs: if the first two inputs are \ensuremath{\mathit{true}} then the third
bit is negated. In a traditional language, the Toffoli gate would be
most conveniently expressed as a conditional expression like:

\noindent
\ensuremath{ \mathit{toffoli}(v_1,v_2,v_3) = \mathit{if} ~(v_1 ~\mathit{and} ~v_2) ~\mathit{then} ~(v_1, v_2, \mathit{not}(v_3)) ~\mathit{else} ~(v_1, v_2, v_3)}

We will derive Toffoli gate in \ensuremath{\Pi } by first deriving a simpler
logic gate called \ensuremath{\mathit{cnot}}.  Consider a one-armed version, \ensuremath{\mathit{if}_c},
of the conditional derived above. If the \ensuremath{\mathit{bool}} is
\ensuremath{\mathit{true}}, the value of type \ensuremath{b} is modified by the operator \ensuremath{c}.

\begin{center}
\scalebox{1.5}{
%%subcode-line{pdfimage}[diagrams/if_c.pdf]
% \includegraphics{inkscape/cnot.pdf}
}
\end{center}


By choosing \ensuremath{b} to be \ensuremath{\mathit{bool}} and \ensuremath{c} to be \ensuremath{\mathit{not}}, we have the
combinator \ensuremath{\mathit{if}_{\mathit{not}} : \mathit{bool} {\prodtype}  \mathit{bool}\leftrightarrow \mathit{bool} {\prodtype}  \mathit{bool}} which negates its
second argument if the first argument is \ensuremath{\mathit{true}}. This gate
\ensuremath{\mathit{if}_{\mathit{not}}} is often referred to as the \ensuremath{\mathit{cnot}} gate\cite{Toffoli:1980}.

If we iterate this construction once more, the resulting combinator
\ensuremath{\mathit{if}_{\mathit{cnot}}} has type \ensuremath{\mathit{bool} {\prodtype}  (\mathit{bool} {\prodtype}  \mathit{bool})\leftrightarrow \mathit{bool} {\prodtype}  (\mathit{bool} {\prodtype}  \mathit{bool})}. The
resulting gate checks the first argument and if it is \ensuremath{\mathit{true}},
proceeds to check the second argument. If that is also \ensuremath{\mathit{true}} then
it will negate the third argument. Thus \ensuremath{\mathit{if}_{\mathit{cnot}}} is the required
Toffoli gate.

\begin{center}
\scalebox{1.6}{
% \includegraphics{inkscape/toffoli.pdf}
}
\end{center}

%%% Local Variables:
%%% mode: latex
%%% TeX-master: "main"
%%% fill-column: 120
%%% End:
