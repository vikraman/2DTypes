\section{A Reversible Programming Language: Set-Theoretic Semantics}
\label{sec:reversibleone}

The practice of programming languages is replete with \emph{ad hoc} instances of reversible computations: database
transactions, mechanisms for data provenance, checkpoints, stack and exception traces, logs, backups, rollback
recoveries, version control systems, reverse engineering, software transactional memories, continuations, backtracking
search, and multiple-level undo features in commercial applications. In the early nineties,
\citet{Baker:1992:LLL,Baker:1992:NFT} argued for a systematic, first-class, treatment of reversibility. But intensive
research in full-fledged reversible models of computations and reversible programming languages was only sparked by the
discovery of deep connections between physics and
computation~\cite{Landauer:1961,PhysRevA.32.3266,Toffoli:1980,bennett1985fundamental,Frank:1999:REC:930275, Hey:1999:FCE:304763,fredkin1982conservative}, and by the
potential for efficient quantum computation~\cite{springerlink:10.1007/BF02650179}.

The early developments of reversible programming languages started
with a conventional programming language, e.g., an extended
$\lambda$-calculus, and either:
\begin{enumerate}
\item extended the language with a history
mechanism~\cite{vanTonder:2004,Kluge:1999:SEMCD,lorenz,danos2004reversible}, or
\item imposed constraints on the control flow constructs to make them
reversible~\cite{Yokoyama:2007:RPL:1244381.1244404}.
\end{enumerate}
More foundational approaches recognize that reversible programming languages require a fresh approach and should be
designed from first principles without the detour via conventional irreversible
languages~\cite{Yokoyama:2008:PRP,Mu:2004:ILRC,abramsky2005structural,DiPierro:2006:RCL:1166042.1166047,
  rc2011,James:2012:IE:2103656.2103667,Carette2016}.

%%%%%%%%%%%%%%%%%
\subsection{The $\Pi$ Family of Languages}

A natural candidate for a semantic foundation for reversible programming languages is the notion of type
isomorphism. Indeed, the type isomorphisms among finite types are sound and complete for all permutations on finite
types~\cite{Fiore:2004,fiore-remarks} and hence they are \emph{complete} for expressing reversible combinational
circuits~\cite{fredkin1982conservative, James:2012:IE:2103656.2103667,Toffoli:1980} and the extension with recursive
types and trace operators~\cite{Hasegawa:1997:RCS:645893.671607} is a Turing-complete reversible
language~\cite{James:2012:IE:2103656.2103667,rc2011}.

Focusing on finite types, the building blocks of type theory are: the empty type ($\zerot$), the unit type
($\onet$) containing just one value $\Acon{tt}$, the sum type ($+$) containing values of the form $\inlv{v}$ and
$\inrv{v}$, and the product ($\times$) type containing pairs of values $(v_1,v_2)$. For this fragment of types, the
syntax of the language $\Pi$ consists of the following sorts:

{\scalebox{\scalef}{$%
\begin{array}{lrcl}
\textit{Value types} & A,B,C,D &::=& \zerot \alt \onet \alt A+B \alt A\times B \\
\textit{Values}      & v,w,x,y &::=& \Acon{tt} \alt \inlv{v} \alt \inrv{v} \alt (v,w) \\
\textit{Program types} &&& A \iso B \\
\textit{Programs} & c &::=& (\textrm{See Fig.~\ref{pi-terms}})
\end{array}$}}
\begin{figure}[t]
{\scalebox{\scalef}{$%
%%\noindent\begin{minipage}{.7\linewidth}
\begin{array}{rrcll}
\idc :& A & \iso & A &: \idc \\
\\
\identlp :&  \zerot + A & \iso & A &: \identrp \\
\swapp :&  A + B & \iso & B + A &: \swapp \\
\assoclp :&  A + (B + C) & \iso & (A + B) + C &: \assocrp \\ [1.5ex]
\identlt :&  \onet \times A & \iso & A &: \identrt \\
\swapt :&  A \times B & \iso & B \times A &: \swapt \\
\assoclt :&  A \times (B \times C) & \iso & (A \times B) \times C &: \assocrt \\ [1.5ex]
\absorbr :&~ \zerot \times A & \iso & \zerot ~ &: \factorzl \\
\dist :&~ (A + B) \times C & \iso & (A \times C) + (B \times C)~ &: \factor
  \end{array}$}}

{\scalebox{\scalef}{%
\Rule{}
{\jdg{}{}{c_1 : A \iso B} \quad \vdash c_2 : B \iso C}
{\jdg{}{}{c_1 \fatsemi c_2 : A \iso C}}
{}

\Rule{}
{\jdg{}{}{c_1 : A \iso B} \quad \vdash c_2 : C \iso D}
{\jdg{}{}{c_1 \oplus c_2 : A + C \iso B + D}}
{}

\Rule{}
{\jdg{}{}{c_1 : A \iso B} \quad \vdash c_2 : C \iso D}
{\jdg{}{}{c_1 \otimes c_2 : A \times C \iso B \times D}}
{}
}}
\caption{$\Pi$-terms, combinators, and their types.}
\label{pi-terms}
\end{figure}

\noindent Intuitively, one may view each type $A$ as a collection of physical wires that can transmit $\sizet{A}$ distinct
values where $\sizet{A}$ is a natural number that indicates the size of a type, computed as: $\sizet{\zerot} = 0$;
$\sizet{\onet} = 1$; $\sizet{A + B} = \sizet{A} + \sizet{B}$; and $\sizet{A \times B} = \sizet{A} * \sizet{B}$.  Thus the type
$\mathbb{B} = \onet + \onet$ corresponds to a wire that can transmit one of two values, i.e., bits, with the
convention that $\inlv{\Acon{tt}}$ represents \AgdaFunction{𝔽} and $\inrv{\Acon{tt}}$ represents \AgdaFunction{𝕋}.  The
type $\mathbb{B} \times \mathbb{B} \times \mathbb{B}$ corresponds to a collection of wires that can transmit three
bits. From that perspective, a type isomorphism between types $A$ and $B$ (such that $\sizet{A}=\sizet{B}=n$) models a
\emph{reversible} combinational circuit that \emph{permutes} the $n$ different values. These type isomorphisms form a
\emph{commutative semiring} and are collected in Fig.~\ref{pi-terms}. Each line in the top part of the figure
introduces a pair of dual constants that witness the type isomorphism in the middle.  These are the \emph{base}
(non-reducible) terms of $\Pi$. The isomorphisms are extended to form a congruence relation by adding the three
constructors at the bottom of the figure that witness equivalence and compatible closure. Although austere, this
combinator-based language has the advantage of being amenable to formal analysis for at least two reasons: (i) it is
conceptually simple and small, and (ii) it has direct and evident connections to type theory and category theory and
hence possesses a rich algebraic structure which we exploit in the remainder of the paper.

%%%%%%%%%
\subsection{Denotational Semantics}

\noindent A denotational semantics of $\Pi$ can be specified by mapping each type to a finite set and each combinator to
a bijection between finite sets, that is, a permutation. In this section, we outline a simple semantics expressed in the
metalanguage of set theory. This semantics is formalized and extended to the metalanguage of homotopy type theory in
Sec.~\ref{sec:reversibletwo}.

As explained in the previous section, a $\Pi$-type $A$ has $\sizet{A}$-elements and for all combinators $c : A \iso B$
we have that $\sizet{A} = \sizet{B}$. Hence the denotation of a type $A$ with $n$-elements will be a finite set
$\Fin[n]$ and the denotation of a combinator $c : A \iso B$ such that $\sizet{A} = \sizet{B} = n$ will be a function
from $\Fin[n]$ to $\Fin[n]$ that permutes the elements. In more detail, we enumerate the values $v : A$ where
$\sizet{A} = n$ in a canonical order such that each value $v$ is given an index in $\{ 0,\cdots,n-1 \}$. The enumeration
is defined as follows:
\[\begin{array}{rcl}
    \mathit{enum}(\zerot) &=& \emptyset \\
    \mathit{enum}(\onet) &=& \{ \Acon{tt} \} \\
    \mathit{enum}(A + B) &=& \mathit{map}~\Acon{inj₂}~\mathit{enum}(A) ~\textsf{+\!+}~ \mathit{map}~\Acon{inj₁}~\mathit{enum}(B) \\
    \mathit{enum}(A \times B) &=& \mathit{concat}~(\mathit{map}~(\lambda v.\mathit{map}~(\lambda w. (v,w))~\mathit{enum}(B))~\mathit{enum}(A))
\end{array}\]
\noindent For example, the 3 values $\inrv{(\inrv{\Acon{tt}})}$, $\inrv{(\inlv{\Acon{tt}})}$, and $\inlv{\Acon{tt}}$ of
type $\onet + (\onet + \onet)$ are given indices 0, 1, and 2 respectively. Given this canonical order, the combinator
$\swapp : \onet + (\onet + \onet) \iso (\onet + \onet) + \onet$ would correspond to the permutation $(1~2~0)$ while the
combinator $\assoclp : \onet + (\onet + \onet) \iso (\onet + \onet) + \onet$ at the same type would correspond to the
identity permutation.

\[\begin{array}{l}
    \Tree[ {\small a} [ {\small b} {\small c} ] ] \\
    2 \quad 1 \quad 0 \\
    0 \quad 2 \quad 1 \\
    \Tree[ [ {\small b}  {\small c} ] {\small a} ]
\end{array}\]

% Such functions are finitely supported, that is, their outputs can be tabulated using the canonical ordering on
% $\Fin[n]$. For bijections, this gives a listed permutation. By observing the action of the combinators on the values of
% the finite set, we can define a denotational semantics which constructs the bijection.
% \note{\(\Fin[n] \to A \eqv \type{Vec_{n}}(A)\)}

%%%%%%%%%
\subsection{Examples}
\label{sec:langRev-examples}
\label{examples}

\note{Revisit to pick examples relevant to the rest of the paper; also unify notation}

\paragraph*{Booleans}
Let us start with encoding booleans. We use the type \ensuremath{1{\sumtype}1} to
represent booleans with \ensuremath{\mathit{left} ~()} representing \ensuremath{\mathit{true}} and
\ensuremath{\mathit{right}~()} representing \ensuremath{\mathit{false}}.
Boolean negation is straightforward to define:

\ensuremath{\mathit{not} : \mathit{bool} \leftrightarrow \mathit{bool}}

\ensuremath{\mathit{not} = \swapp}

\noindent
It is easy to verify that \ensuremath{\mathit{not}} changes \ensuremath{\mathit{true}} to \ensuremath{\mathit{false}} and
vice versa.

\paragraph*{Bit Vectors.}
We can represent $n$-bit words using an n-ary product of
\ensuremath{\mathit{bool}}s. For example, we can represent a 3-bit word, \ensuremath{\mathit{word}_3},
using the type \ensuremath{\mathit{bool} {\prodtype} (\mathit{bool} {\prodtype}  \mathit{bool})}.  We can perform various
operations on these 3-bit words using combinators in \ensuremath{\Pi }. For
instance the bitwise \ensuremath{\mathit{not}} operation is the parallel composition of
three \ensuremath{\mathit{not}} operations:

\ensuremath{\mathit{not}_{\mathit{word}_3} :: \mathit{word}_3 \leftrightarrow \mathit{word}_3}

\ensuremath{\mathit{not}_{\mathit{word}_3} = \mathit{not}  {\prodtype}  (\mathit{not}  {\prodtype}  \mathit{not})}

\noindent We can express a 3-bit word reversal operation as follows:

\ensuremath{\mathit{reverse} : \mathit{word}_3 \leftrightarrow \mathit{word}_3}

\ensuremath{\mathit{reverse} = \swapt \odot (\swapt  \otimes  \idc)~ \odot \assocrt}

\noindent We can check that \ensuremath{\mathit{reverse}} does the right thing by
applying it to a value \ensuremath{(v_1, (v_2, v_3))} and writing out the full
derivation tree of the reduction.  The combinator \ensuremath{\mathit{reverse}}, like
many others we will see in this paper, is formed by sequentially
composing several simpler combinators. Instead of presenting the
operation of \ensuremath{\mathit{reverse}} as a derivation tree, it is easier (purely
for presentation reasons) to flatten the tree into a sequence of
reductions as caused by each component. Such a sequence of reductions
is given below:
\[\begin{array}{rlr}
 & (v_1, (v_2, v_3)) \\
 \swapt & ((v_2, v_3), v_1) \\
 \swapt \otimes  \idc & ((v_3, v_2), v_1) \\
 \assocrt & (v_3, (v_2, v_1)) \\
 \end{array}\]
%subcode source isomorphisms.tex:979

\noindent On the first line is the initial value. On each subsequent
line is a fragment of the \ensuremath{\mathit{reverse}} combinator and the value that
results from applying this combinator to the value on the previous
line. For example, \ensuremath{\swapt} transforms \ensuremath{(v_1, (v_2, v_3))} to
\ensuremath{((v_2,v_3),v_1)}.  On the last line we see the expected result with
the bits in reverse order.

We can also draw out the graphical representation of the 3-bit reverse
combinator. In the graphical representation, it is clear that the
combinator achieves the required shuffling.

% \inkscape{reverse-3-bit.pdf}

\paragraph*{Conditionals.}
Even though \ensuremath{\Pi } lacks conditional expressions, they are
expressible using the distributivity and factoring laws. The
diagrammatic representation of \ensuremath{\dist} shows that it redirects the flow
of a value \ensuremath{v:b} based on the value of another one of type
\ensuremath{b_1{\sumtype}b_2}. If we choose \ensuremath{1{\sumtype}1} to be
\ensuremath{\mathit{bool}} and apply either \ensuremath{c_1:b_1\leftrightarrow
b_2} or \ensuremath{c_2:b_1\leftrightarrow b_2} to the value \ensuremath{v},
then we essentially have an `if' expression.

\ensuremath{\mathit{if}_{c_1,c_2} : \mathit{bool}  {\prodtype}  b_1 \leftrightarrow \mathit{bool}  {\prodtype}  b_2}

\ensuremath{\mathit{if}_{c_1,c_2} = \dist \odot ((\idc  \otimes\  c_1) {\sumtype} (\idc \otimes\  c_2)) \odot \factor}


% \inkscape{if-c1-c2.pdf}


The diagram above shows the input value of type \ensuremath{(1{\sumtype}1) {\prodtype}  b_1}
processed by the distribute operator \ensuremath{\dist}, which converts it into
a value of type \ensuremath{(1 {\prodtype}  b_1){\sumtype}(1 {\prodtype}  b_1)}. In the
\ensuremath{\mathit{left}} branch, which corresponds to the
case when the boolean is \ensuremath{\mathit{true}} (i.e. the value was
\ensuremath{\mathit{left} ~()}), the combinator~\ensuremath{c_1} is applied to
the value of type~\ensuremath{b_1}. The right
branch which corresponds to the boolean being \ensuremath{\mathit{false}} passes
the value of type \ensuremath{b_1} through the combinator \ensuremath{c_2}.
The inverse of \ensuremath{\dist}, namely \ensuremath{\factor} is applied
to get the final result of type \ensuremath{(1{\sumtype}1) {\prodtype} b_2}.

\paragraph*{Logic Gates}
There are several universal primitives for conventional (irreversible)
hardware circuits, such as \ensuremath{\mathit{nand}} and \ensuremath{\mathit{fanout}}. In the case
of reversible hardware circuits, the canonical universal primitive is
the Toffoli gate~\cite{Toffoli:1980}. The Toffoli gate takes three
boolean inputs: if the first two inputs are \ensuremath{\mathit{true}} then the third
bit is negated. In a traditional language, the Toffoli gate would be
most conveniently expressed as a conditional expression like:

\noindent
\ensuremath{ \mathit{toffoli}(v_1,v_2,v_3) = \mathit{if} ~(v_1 ~\mathit{and} ~v_2) ~\mathit{then} ~(v_1, v_2, \mathit{not}(v_3)) ~\mathit{else} ~(v_1, v_2, v_3)}

We will derive Toffoli gate in \ensuremath{\Pi } by first deriving a simpler
logic gate called \ensuremath{\mathit{cnot}}.  Consider a one-armed version, \ensuremath{\mathit{if}_c},
of the conditional derived above. If the \ensuremath{\mathit{bool}} is
\ensuremath{\mathit{true}}, the value of type \ensuremath{b} is modified by the operator \ensuremath{c}.

\begin{center}
\scalebox{1.5}{
%%subcode-line{pdfimage}[diagrams/if_c.pdf]
% \includegraphics{inkscape/cnot.pdf}
}
\end{center}


By choosing \ensuremath{b} to be \ensuremath{\mathit{bool}} and \ensuremath{c} to be \ensuremath{\mathit{not}}, we have the
combinator \ensuremath{\mathit{if}_{\mathit{not}} : \mathit{bool} {\prodtype}  \mathit{bool}\leftrightarrow \mathit{bool} {\prodtype}  \mathit{bool}} which negates its
second argument if the first argument is \ensuremath{\mathit{true}}. This gate
\ensuremath{\mathit{if}_{\mathit{not}}} is often referred to as the \ensuremath{\mathit{cnot}} gate\cite{Toffoli:1980}.

If we iterate this construction once more, the resulting combinator
\ensuremath{\mathit{if}_{\mathit{cnot}}} has type \ensuremath{\mathit{bool} {\prodtype}  (\mathit{bool} {\prodtype}  \mathit{bool})\leftrightarrow \mathit{bool} {\prodtype}  (\mathit{bool} {\prodtype}  \mathit{bool})}. The
resulting gate checks the first argument and if it is \ensuremath{\mathit{true}},
proceeds to check the second argument. If that is also \ensuremath{\mathit{true}} then
it will negate the third argument. Thus \ensuremath{\mathit{if}_{\mathit{cnot}}} is the required
Toffoli gate.

\begin{center}
\scalebox{1.6}{
% \includegraphics{inkscape/toffoli.pdf}
}
\end{center}

%%% Local Variables:
%%% mode: latex
%%% TeX-master: "main"
%%% fill-column: 120
%%% End:
