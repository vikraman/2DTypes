\documentclass[acmsmall,review,anonymous]{acmart}
\settopmatter{printfolios=true,printccs=false,printacmref=false}

\usepackage{cleveref}
\usepackage{macros}
\usepackage{hott}
\usepackage{quiver}
\usepackage{enumitem}

%% Some recommended packages.
%% \usepackage{booktabs}   %% For formal tables:
                        %% http://ctan.org/pkg/booktabs
%% \usepackage{subcaption} %% For complex figures with subfigures/subcaptions
                        %% http://ctan.org/pkg/subcaption
%% \usepackage{verbatim}
%% \usepackage{amsmath,amsbsy}
%% \usepackage{alltt}
%% \usepackage{fdsymbol}
%% \usepackage{amsthm,proof}
%% \usepackage{bbold,stmaryrd,bbm}
\usepackage{bbold}
%% \usepackage{ucs}
%% \usepackage{wrapfig}
%% \usepackage[utf8x]{inputenc}
%% \usepackage{newunicodechar}
%% \usepackage{microtype}
%% \usepackage{subcaption}
%% \usepackage{agda}
\usepackage{tikz}
\usetikzlibrary{cd}
\usetikzlibrary{quotes}
\usetikzlibrary{decorations.markings}
\usetikzlibrary{quotes,fit,positioning,decorations}
\usetikzlibrary{knots}
\tikzstyle{func}=[rectangle,draw,fill=black!20,minimum size=1.9em,text width=2.4em, text centered]

\usepackage[utf8x]{inputenc}
\usepackage{agda}
\usepackage[nocenter]{qtree}
\usepackage{amsthm,amsmath}
\usepackage{url}
\usepackage{tikz}
\usepackage{tikz-cd}
\usepackage{listings}
\usepackage{multicol}
\usepackage{stmaryrd}
\usepackage{proof}
\usetikzlibrary{quotes,fit,positioning,decorations}
\usepackage{tikzit}
\input{tikzit.tikzstyles}
\tikzset{>=latex}
\usepackage{newunicodechar}

\newtheorem{theorem}{Theorem}
\newtheorem{corollary}[theorem]{Corollary}
\newtheorem{lemma}[theorem]{Lemma}
\newtheorem{definition}[theorem]{Definition}
\newtheorem*{remark}{Remark}

\newcommand{\sem}[1]{\llbracket #1 \rrbracket}
\newcommand{\sizet}[1]{| #1 |}
\newcommand{\idd}{\mathit{id}\leftrightarrow}
\newcommand{\idf}{\mathit{id}\Leftrightarrow}
\newcommand{\Gpd}{\ensuremath{\mathsf{Groupoid}}}
\newcommand{\scalef}{0.85}
\newcommand{\alt}{~|~}
\newcommand{\zerot}{\mathbb{0}}
\newcommand{\onet}{\mathbb{1}}
\newcommand{\sumt}{\mathbin{\mathsf{+}}}
\newcommand{\prodt}{\mathbin{\mathsf{\times}}}
\newcommand{\Acon}[1]{\AgdaInductiveConstructor{#1}}
\newcommand{\inlv}[1]{\ensuremath{\Acon{inj₁} \; #1}}
\newcommand{\inrv}[1]{\ensuremath{\Acon{inj₂} \; #1}}
\newcommand{\idc}{\AgdaInductiveConstructor{id↔}}
\newcommand{\assoc}{\circ}
\newcommand{\transLR}{\AgdaInductiveConstructor{trans⇔}}
\newcommand{\identlp}{\AgdaInductiveConstructor{unite₊l}}
\newcommand{\identrp}{\AgdaInductiveConstructor{uniti₊l}}
\newcommand{\identlsp}{\AgdaInductiveConstructor{unite₊r}}
\newcommand{\identrsp}{\AgdaInductiveConstructor{uniti₊r}}
\newcommand{\swapp}{\AgdaInductiveConstructor{swap₊}}
\newcommand{\assoclp}{\AgdaInductiveConstructor{assocl₊}}
\newcommand{\assocrp}{\AgdaInductiveConstructor{assocr₊}}
\newcommand{\identlt}{\AgdaInductiveConstructor{unite⋆l}}
\newcommand{\identrt}{\AgdaInductiveConstructor{uniti⋆l}}
\newcommand{\identlst}{\AgdaInductiveConstructor{unite⋆r}}
\newcommand{\identrst}{\AgdaInductiveConstructor{uniti⋆r}}
\newcommand{\swapt}{\AgdaInductiveConstructor{swap⋆}}
\newcommand{\assoclt}{\AgdaInductiveConstructor{assocl⋆}}
\newcommand{\assocrt}{\AgdaInductiveConstructor{assocr⋆}}
\newcommand{\absorbr}{\AgdaInductiveConstructor{absorbr}}
\newcommand{\absorbl}{\AgdaInductiveConstructor{absorbl}}
\newcommand{\factorzr}{\AgdaInductiveConstructor{factorzr}}
\newcommand{\factorzl}{\AgdaInductiveConstructor{factorzl}}
\newcommand{\dist}{\AgdaInductiveConstructor{dist}}
\newcommand{\factor}{\AgdaInductiveConstructor{factor}}
\newcommand{\distl}{\AgdaInductiveConstructor{distl}}
\newcommand{\factorl}{\AgdaInductiveConstructor{factorl}}
\newcommand{\isot}{\ensuremath{\leftrightarrow}}
\newcommand{\Rule}[4]{
\makebox{{\rm #1}
$\displaystyle
\frac{\begin{array}{l}#2 \\\end{array}}
{\begin{array}{l}#3      \\\end{array}}$
 #4}}
\newcommand{\jdg}[3]{#2 \proves_{#1} #3}
\newcommand{\proves}{\vdash}
\newcommand{\interp}[2]{\mathit{interp}(#1,#2)}
\newcommand{\sumtype}{+}
\newcommand{\prodtype}{\times}
\newcommand{\inkscape}[2][1.5]{
\begin{center}
\scalebox{#1}{
\includegraphics{inkscape/#2}
}
\end{center}
}

\newunicodechar{⊸}{$\multimap$}
\newunicodechar{𝕍}{$\mathbb{V}$}
\newunicodechar{𝕃}{$\mathbb{L}$}
\newunicodechar{𝕄}{$\mathbb{M}$}
\newunicodechar{ℝ}{$\mathbb{R}$}
\newunicodechar{𝕌}{$\mathbb{U}$}
\newunicodechar{𝔹}{$\mathbb{B}$}
\newunicodechar{𝕐}{$\mathbb{Y}$}
\newunicodechar{𝔼}{$\mathbb{E}$}
\newunicodechar{𝔽}{$\mathbb{F}$}
\newunicodechar{𝕋}{$\mathbb{T}$}
\newunicodechar{𝕚}{$\mathbb{i}$}
\newunicodechar{𝟘}{$\mathbb{0}$}
\newunicodechar{𝟙}{$\mathbb{1}$}
\newunicodechar{𝟚}{$\mathbb{2}$}
\newunicodechar{𝟛}{$\mathbb{3}$}
\DeclareUnicodeCharacter{9678}{\ensuremath{\odot}}
\DeclareUnicodeCharacter{9636}{\ensuremath{\Box}}
\DeclareUnicodeCharacter{10231}{\ensuremath{\leftrightarrow}}

%%
%% \BibTeX command to typeset BibTeX logo in the docs
\AtBeginDocument{%
  \providecommand\BibTeX{{%
    \normalfont B\kern-0.5em{\scshape i\kern-0.25em b}\kern-0.8em\TeX}}}

%% Rights management information.  This information is sent to you
%% when you complete the rights form.  These commands have SAMPLE
%% values in them; it is your responsibility as an author to replace
%% the commands and values with those provided to you when you
%% complete the rights form.
\setcopyright{acmcopyright}
\copyrightyear{2018}
\acmYear{2018}
\acmDOI{10.1145/1122445.1122456}

%% These commands are for a PROCEEDINGS abstract or paper.
\acmConference[Woodstock '18]{Woodstock '18: ACM Symposium on Neural
  Gaze Detection}{June 03--05, 2018}{Woodstock, NY}
\acmBooktitle{Woodstock '18: ACM Symposium on Neural Gaze Detection,
  June 03--05, 2018, Woodstock, NY}
\acmPrice{15.00}
\acmISBN{978-1-4503-XXXX-X/18/06}

%%
%% Submission ID.
%% Use this when submitting an article to a sponsored event. You'll
%% receive a unique submission ID from the organizers
%% of the event, and this ID should be used as the parameter to this command.
%%\acmSubmissionID{123-A56-BU3}

%%
%% The majority of ACM publications use numbered citations and
%% references.  The command \citestyle{authoryear} switches to the
%% "author year" style.
%%
%% If you are preparing content for an event
%% sponsored by ACM SIGGRAPH, you must use the "author year" style of
%% citations and references.
%% Uncommenting
%% the next command will enable that style.
\citestyle{acmauthoryear}

%%
%% end of the preamble, start of the body of the document source.
\begin{document}

%%
%% The "title" command has an optional parameter,
%% allowing the author to define a "short title" to be used in page headers.
\title{Reversible Programming with Univalent Finite Types}

%%
%% The "author" command and its associated commands are used to define
%% the authors and their affiliations.
%% Of note is the shared affiliation of the first two authors, and the
%% "authornote" and "authornotemark" commands
%% used to denote shared contribution to the research.
\author{Anonymous}

%%
%% By default, the full list of authors will be used in the page
%% headers. Often, this list is too long, and will overlap
%% other information printed in the page headers. This command allows
%% the author to define a more concise list
%% of authors' names for this purpose.
\renewcommand{\shortauthors}{Anonymous}

%%
%% The abstract is a short summary of the work to be presented in the
%% article.
\begin{abstract}
Reversible computing is an alternative model of computation motivated by the physical nature of computational processes, where computations are required to preserve information. Several reversible programming languages have been proposed which permit hardware-independent, low-level and high-level descriptions of reversible circuits.

The $\PiLang$ family of reversible programming languages were proposed by~\citet{jamesInformationEffects2012} in which all computations are logically reversible. Formally, it is presented as a language of 1-combinators witnessing type isomorphisms, and an equational theory given by 2-combinators witnessing 1-combinator optimisations.

In this paper, we present a denotational semantics for this language, using the language of groupoids \`{a} la Homotopy Type Theory (HoTT). We establish a correspondence between the syntactic groupoid of the language and a formally presented univalent subuniverse of finite types. The correspondence relates 1-combinators to 1-paths, and 2-combinators to 2-paths in the universe, which is shown to be sound and complete for both levels. The result suggests a Curry-Howard-Lambek/Lawvere correspondence between Reversible Logic, Reversible Programming Languages, and Symmetric Monoidal Groupoids, similar to the well-known correspondence of Intuitionistic Logic, Simply Typed Lambda Calculus, and Cartesian Closed Categories.

Using soundness and completeness, we show how to perform Normalisation by Evaluation (NbE) for boolean reversible circuits, motivated by a few examples of reversible logic gates used in Quantum Computing. We also show how to synthesise normal forms for reversible circuits starting from the extensional view of permutations of bits.
\end{abstract}

%%
%% The code below is generated by the tool at http://dl.acm.org/ccs.cfm.
%% Please copy and paste the code instead of the example below.
%%
\begin{CCSXML}
  <ccs2012>
   <concept>
       <concept_id>10003752.10003790.10011740</concept_id>
       <concept_desc>Theory of computation~Type theory</concept_desc>
       <concept_significance>500</concept_significance>
       </concept>
   <concept>
       <concept_id>10003752.10010124.10010131.10010137</concept_id>
       <concept_desc>Theory of computation~Categorical semantics</concept_desc>
       <concept_significance>500</concept_significance>
       </concept>
   <concept>
       <concept_id>10003752.10010124.10010131.10010133</concept_id>
       <concept_desc>Theory of computation~Denotational semantics</concept_desc>
       <concept_significance>500</concept_significance>
       </concept>
   <concept>
       <concept_id>10011007.10011006.10011008.10011009.10011012</concept_id>
       <concept_desc>Software and its engineering~Functional languages</concept_desc>
       <concept_significance>500</concept_significance>
       </concept>
   <concept>
       <concept_id>10011007.10011006.10011039.10011040</concept_id>
       <concept_desc>Software and its engineering~Syntax</concept_desc>
       <concept_significance>500</concept_significance>
       </concept>
   <concept>
       <concept_id>10011007.10011006.10011039.10011311</concept_id>
       <concept_desc>Software and its engineering~Semantics</concept_desc>
       <concept_significance>500</concept_significance>
       </concept>
 </ccs2012>
\end{CCSXML}

\ccsdesc[500]{Theory of computation~Type theory}
\ccsdesc[500]{Theory of computation~Categorical semantics}
\ccsdesc[500]{Theory of computation~Denotational semantics}
\ccsdesc[500]{Software and its engineering~Functional languages}
\ccsdesc[500]{Software and its engineering~Syntax}
\ccsdesc[500]{Software and its engineering~Semantics}

%%
%% Keywords. The author(s) should pick words that accurately describe
%% the work being presented. Separate the keywords with commas.
\keywords{reversible computing, reversible programming languages, homotopy type theory, denotational semantics}

%%
%% This command processes the author and affiliation and title
%% information and builds the first part of the formatted document.
\maketitle

%%%%%%%%%%%%%%%%%%%%%%%%%%%%%%%%%%%%%%%%%%%%%%%%%%%%%%

\section{Introduction}~\label{sec:introduction}

%%% Local Variables:
%%% mode: latex
%%% TeX-master: "main"
%%% End:

\section{The Underlying Ideas}

\todo{Not the right title.}

\note{This section should explain the main technical parts of the paper
  informally, without using any technology. Use an example, such as, a
  reversible language with $\leq 5$ bits, and examples of permutations and
  transpositions, and when they're equal.}

% Shor's quantum factoring algorithm requires, at its core, a classical reversible function for modular exponentiation. As
% explained in standard accounts of the algorithm (e.g., the Qiskit implementation), such efficient modular exponentiation
% circuits are not straightforward and are the bottleneck in Shor’s algorithm. Typical derivations of the circuit start
% from elementary gates, build a circuit for modular addition, which is used to build a circuit for modular
% multiplication, and then finally a circuit for modular exponentiation taking care at each step to avoid the exponential
% blowup~\cite{shorefficient}. Ultimately however the circuit has a simple specification as a \emph{permutation}. For
% example, consider the function $f(r) = 11^{r} \mod 15$. Using the Toffoli construction~\cite{Toffoli:1980}, we can embed
% this function $f$ into a reversible function $g$, shown below, such that $g(r,0) = (r,f(r))$:
% \[\begin{array}{rcll}
% g(r,h) &=& (r,h+1) & \mbox{when~$r$~even~and~$h$~even} \\
% g(r,h) &=& (r,h-1) & \mbox{when~$r$~even~and~$h$~odd} \\
% g(r,h) &=& (r,11-h) & \mbox{when~$r$~odd~and~$4 > h \geq 0$~or~$12 > h \geq 8$} \\
% g(r,h) &=& (r,19-h) & \mbox{when~$r$~odd~and~$8 > h \geq 4$~or~$16 > h \geq 12$}
% \end{array}\]
% Wouldn't it be simpler to write this permutation and automatically derive a circuit from it?

% \note{Can we do this for the supplementary material??}

\paragraph*{Normalization by evaluation of reversible circuits.} Consider the following reversible function:
$\Afun{RESET}(b,b_1,\ldots,b_n) = (b \; \underline{\vee} \; (\bigvee_{i=1}^n b_i),b_1,\ldots,b_n)$ where $\vee$ is
logical-or and $\underline{\vee}$ is exclusive-or. The function sets or resets the first bit depending on whether any of
the other bits is true or not. The natural definition for this function is recursive: modulo some re-shuffling of the
bits, the strategy is to examine the bits $b_i$ one-by-one: if we encounter a true value, we negate the first bit and
terminate; otherwise we continue making recursive calls until we reach the last bit at which point we return the
incoming value with no change. A modular implementation using 4 helpers is provided in the supplementary
material. Unrolling the definition of $\Afun{reset}~2$ and inlining all the helpers produces the following program:

%%\begin{minipage}{.65\textwidth}
%%  \PiRESET{}
%%\end{minipage}
%%\begin{minipage}{.30\textwidth}
%%  \begin{center}
%%  \resizebox{0.5\textwidth}{!}{\begin{tikzpicture}
	\begin{pgfonlayer}{nodelayer}
		\node [style=none] (6) at (-9, 7) {};
		\node [style=none] (9) at (-8, 7) {};
		\node [style=none] (14) at (-9, 2) {};
		\node [style=none] (15) at (-8, 2) {};
		\node [style=none] (17) at (-8.5, 4.75) {\dist};
		\node [style=none] (41) at (-8.5, 6) {{\color{red}$\mathbb{F}$}};
		\node [style=none] (44) at (-8.5, 3.25) {{\color{red}$\mathbb{T}$}};
		\node [style=none] (75) at (-6.75, 4) {};
		\node [style=none] (76) at (-6.75, 3.5) {};
		\node [style=none] (77) at (-5.75, 3.5) {};
		\node [style=none] (78) at (-5.75, 4) {};
		\node [style=none] (79) at (-5.75, 3.5) {};
		\node [style=none] (80) at (-6.25, 3.75) {\AgdaFunction{NOT}};
		\node [style=none] (81) at (-7, 7) {};
		\node [style=none] (82) at (-7, 5) {};
		\node [style=none] (83) at (-5.5, 5) {};
		\node [style=none] (84) at (-5.5, 7) {};
		\node [style=none] (95) at (-4.5, 7) {};
		\node [style=none] (96) at (-3.5, 7) {};
		\node [style=none] (98) at (-4.5, 2) {};
		\node [style=none] (99) at (-3.5, 2) {};
		\node [style=none] (101) at (-4, 4.75) {\factor};
		\node [style=none] (102) at (-4, 6) {{\color{red}$\mathbb{F}$}};
		\node [style=none] (103) at (-4, 3.25) {{\color{red}$\mathbb{T}$}};
		\node [style=none] (106) at (-4.5, 4.5) {};
		\node [style=none] (113) at (-6.25, 5.75) {$\AgdaFunction{RESET}_{n-1}$};
		\node [style=none] (127) at (-6.5, 4.5) {};
		\node [style=none] (128) at (-6, 4.5) {};
		\node [style=none] (129) at (-6.25, 4.75) {};
		\node [style=none] (130) at (-6.25, 4.25) {};
		\node [style=none] (131) at (-10.75, 4.75) {};
		\node [style=none] (132) at (-10.75, 5.25) {};
		\node [style=none] (134) at (-10.75, 5.75) {};
		\node [style=none] (135) at (-10.25, 5.75) {};
		\node [style=none] (136) at (-10.75, 5.25) {};
		\node [style=none] (137) at (-10.25, 5.25) {};
		\node [style=none] (138) at (-9.5, 5.25) {};
		\node [style=none] (139) at (-9.5, 5.75) {};
		\node [style=none] (140) at (-10.75, 3.75) {};
		\node [style=none] (141) at (-9, 3.75) {};
		\node [style=none] (142) at (-9.75, 4.25) {$\vdots$};
		\node [style=none] (143) at (-9, 5.75) {};
		\node [style=none] (144) at (-9, 5.25) {};
		\node [style=none] (145) at (-9, 4.75) {};
		\node [style=none] (146) at (-10.75, 4.75) {};
		\node [style=none] (147) at (-3.5, 4.75) {};
		\node [style=none] (148) at (-3.5, 5.25) {};
		\node [style=none] (149) at (-3.5, 5.75) {};
		\node [style=none] (150) at (-3, 5.75) {};
		\node [style=none] (151) at (-3.5, 5.25) {};
		\node [style=none] (152) at (-3, 5.25) {};
		\node [style=none] (153) at (-2.25, 5.25) {};
		\node [style=none] (154) at (-2.25, 5.75) {};
		\node [style=none] (155) at (-3.5, 3.75) {};
		\node [style=none] (156) at (-1.75, 3.75) {};
		\node [style=none] (157) at (-2.75, 4.25) {$\vdots$};
		\node [style=none] (158) at (-1.75, 5.75) {};
		\node [style=none] (159) at (-1.75, 5.25) {};
		\node [style=none] (160) at (-1.75, 4.75) {};
		\node [style=none] (161) at (-3.5, 4.75) {};
		\node [style=none] (162) at (-8, 6.25) {};
		\node [style=none] (163) at (-8, 6.75) {};
		\node [style=none] (164) at (-8, 6.75) {};
		\node [style=none] (165) at (-7, 6.75) {};
		\node [style=none] (167) at (-8, 5.25) {};
		\node [style=none] (168) at (-7, 5.25) {};
		\node [style=none] (169) at (-7.5, 5.75) {$\vdots$};
		\node [style=none] (171) at (-7, 6.25) {};
		\node [style=none] (172) at (-8, 6.25) {};
		\node [style=none] (173) at (-5.5, 6.25) {};
		\node [style=none] (174) at (-5.5, 6.75) {};
		\node [style=none] (175) at (-5.5, 6.75) {};
		\node [style=none] (176) at (-4.5, 6.75) {};
		\node [style=none] (177) at (-5.5, 5.25) {};
		\node [style=none] (178) at (-4.5, 5.25) {};
		\node [style=none] (179) at (-5, 5.75) {$\vdots$};
		\node [style=none] (180) at (-4.5, 6.25) {};
		\node [style=none] (181) at (-5.5, 6.25) {};
		\node [style=none] (182) at (-5.75, 3.25) {};
		\node [style=none] (183) at (-5.75, 3.75) {};
		\node [style=none] (184) at (-5.75, 3.75) {};
		\node [style=none] (185) at (-4.5, 3.75) {};
		\node [style=none] (186) at (-6.75, 2.25) {};
		\node [style=none] (187) at (-4.5, 2.25) {};
		\node [style=none] (188) at (-5.25, 2.75) {$\vdots$};
		\node [style=none] (189) at (-4.5, 3.25) {};
		\node [style=none] (190) at (-5.75, 3.25) {};
		\node [style=none] (191) at (-8, 3.25) {};
		\node [style=none] (192) at (-8, 3.75) {};
		\node [style=none] (193) at (-8, 3.75) {};
		\node [style=none] (194) at (-6.75, 3.75) {};
		\node [style=none] (195) at (-8, 2.25) {};
		\node [style=none] (196) at (-6.75, 2.25) {};
		\node [style=none] (197) at (-7.5, 2.75) {$\vdots$};
		\node [style=none] (198) at (-5.75, 3.25) {};
		\node [style=none] (199) at (-8, 3.25) {};
	\end{pgfonlayer}
	\begin{pgfonlayer}{edgelayer}
		\draw (6.center) to (14.center);
		\draw (14.center) to (15.center);
		\draw (15.center) to (9.center);
		\draw (6.center) to (9.center);
		\draw (75.center) to (76.center);
		\draw (76.center) to (77.center);
		\draw (75.center) to (78.center);
		\draw (78.center) to (77.center);
		\draw (81.center) to (82.center);
		\draw (82.center) to (83.center);
		\draw (83.center) to (84.center);
		\draw (81.center) to (84.center);
		\draw (95.center) to (98.center);
		\draw (98.center) to (99.center);
		\draw (99.center) to (96.center);
		\draw (95.center) to (96.center);
		\draw (127.center) to (128.center);
		\draw (129.center) to (130.center);
		\draw (134.center) to (135.center);
		\draw [style=new edge style 5] (136.center) to (137.center);
		\draw (135.center) to (138.center);
		\draw [style=new edge style 5] (137.center) to (139.center);
		\draw (140.center) to (141.center);
		\draw (138.center) to (144.center);
		\draw [style=new edge style 5] (139.center) to (143.center);
		\draw (146.center) to (145.center);
		\draw [style=new edge style 5] (149.center) to (150.center);
		\draw (151.center) to (152.center);
		\draw [style=new edge style 5] (150.center) to (153.center);
		\draw (152.center) to (154.center);
		\draw (155.center) to (156.center);
		\draw [style=new edge style 5] (153.center) to (159.center);
		\draw (154.center) to (158.center);
		\draw (161.center) to (160.center);
		\draw (164.center) to (165.center);
		\draw (167.center) to (168.center);
		\draw (172.center) to (171.center);
		\draw (175.center) to (176.center);
		\draw (177.center) to (178.center);
		\draw (181.center) to (180.center);
		\draw (184.center) to (185.center);
		\draw (186.center) to (187.center);
		\draw (190.center) to (189.center);
		\draw (193.center) to (194.center);
		\draw (195.center) to (196.center);
		\draw (199.center) to (198.center);
		\draw [style=new edge style 4] (144.center) to (164.center);
		\draw [style=new edge style 4] (144.center) to (193.center);
		\draw [style=new edge style 4] (145.center) to (172.center);
		\draw [style=new edge style 4] (145.center) to (199.center);
		\draw [style=new edge style 4] (141.center) to (167.center);
		\draw [style=new edge style 4] (141.center) to (195.center);
		\draw [style=new edge style 4] (176.center) to (151.center);
		\draw [style=new edge style 4] (151.center) to (185.center);
		\draw [style=new edge style 4] (180.center) to (161.center);
		\draw [style=new edge style 4] (161.center) to (189.center);
		\draw [style=new edge style 4] (178.center) to (155.center);
		\draw [style=new edge style 4] (155.center) to (187.center);
	\end{pgfonlayer}
\end{tikzpicture}
}
%%  \end{center}
%%\end{minipage}

  \resettwo{}

The syntax will be explained in detail in the next section. For now, it is sufficient to know that there is some program
that implements the reversible function of interest. Our Agda infrastructure provides tools to normalize all programs to
their normal form. For \Afun{reset2} we get:

  \resetnormtwo{}





Now imagine we want to write the following reversible function:

0 -> 0
8 -> 8
n -> n + 8 `mod` 16

% (assocl⋆ ◎ (swap⋆ ⊗ id⟷₁) ◎ assocr⋆) ◎
% (dist ◎
%  ((id⟷₁ ⊗ (swap₊ ⊗ id⟷₁)) ⊕
%   (id⟷₁ ⊗
%    (swap⋆ ◎ (dist ◎ ((id⟷₁ ⊗ swap₊) ⊕ id⟷₁) ◎ factor) ◎ swap⋆)))
%  ◎ factor)
% ◎ assocl⋆ ◎ (swap⋆ ⊗ id⟷₁) ◎ assocr⋆

% reset+ 2:

% (id⟷₁ ⊕ id⟷₁ ⊕ id⟷₁ ⊕ id⟷₁ ⊕ assocl₊ ◎ (swap₊ ⊕ id⟷₁) ◎ assocr₊) ◎
% (id⟷₁ ⊕ id⟷₁ ⊕ id⟷₁ ⊕ id⟷₁ ⊕ id⟷₁ ⊕ assocl₊ ◎ (swap₊ ⊕ id⟷₁) ◎ assocr₊) ◎
% (id⟷₁ ⊕ id⟷₁ ⊕ id⟷₁ ⊕ id⟷₁ ⊕ assocl₊ ◎ (swap₊ ⊕ id⟷₁) ◎ assocr₊) ◎
% id⟷₁

% reset 3 =

% (assocl⋆ ◎ (swap⋆ ⊗ id⟷₁) ◎ assocr⋆) ◎
% (dist ◎
%  ((id⟷₁ ⊗ (swap₊ ⊗ id⟷₁)) ⊕
%   (id⟷₁ ⊗
%    ((assocl⋆ ◎ (swap⋆ ⊗ id⟷₁) ◎ assocr⋆) ◎
%     (dist ◎
%      ((id⟷₁ ⊗ (swap₊ ⊗ id⟷₁)) ⊕
%       (id⟷₁ ⊗
%        (swap⋆ ◎ (dist ◎ ((id⟷₁ ⊗ swap₊) ⊕ id⟷₁) ◎ factor) ◎ swap⋆)))
%      ◎ factor)
%     ◎ assocl⋆ ◎ (swap⋆ ⊗ id⟷₁) ◎ assocr⋆)))
%  ◎ factor)
% ◎ assocl⋆ ◎ (swap⋆ ⊗ id⟷₁) ◎ assocr⋆

% reset+ 3

% (id⟷₁ ⊕ id⟷₁ ⊕ id⟷₁ ⊕ id⟷₁ ⊕ id⟷₁ ⊕ id⟷₁ ⊕ id⟷₁ ⊕ id⟷₁ ⊕ id⟷₁ ⊕ id⟷₁ ⊕ id⟷₁ ⊕ id⟷₁ ⊕ assocl₊ ◎ (swap₊ ⊕ id⟷₁) ◎ assocr₊) ◎
% (id⟷₁ ⊕ id⟷₁ ⊕ id⟷₁ ⊕ id⟷₁ ⊕ id⟷₁ ⊕ id⟷₁ ⊕ id⟷₁ ⊕ id⟷₁ ⊕ id⟷₁ ⊕ id⟷₁ ⊕ id⟷₁ ⊕ id⟷₁ ⊕ id⟷₁ ⊕ assocl₊ ◎ (swap₊ ⊕ id⟷₁) ◎ assocr₊) ◎
% (id⟷₁ ⊕ id⟷₁ ⊕ id⟷₁ ⊕ id⟷₁ ⊕ id⟷₁ ⊕ id⟷₁ ⊕ id⟷₁ ⊕ id⟷₁ ⊕ id⟷₁ ⊕ id⟷₁ ⊕ id⟷₁ ⊕ id⟷₁ ⊕ assocl₊ ◎ (swap₊ ⊕ id⟷₁) ◎ assocr₊)◎
% id⟷₁


\note{Motivation: There are two reversible circuits which describe the following permutation. They can be shown to be
  equal using the 2-combinators.}

\[
  \begin{tikzpicture}
    \begin{knot}[clip width=5]
      \filldraw (0,5) circle (2pt) node[above] {0};
      \filldraw (1,5) circle (2pt) node[above] {1};
      \filldraw (2,5) circle (2pt) node[above] {2};
      \filldraw (3,5) circle (2pt) node[above] {3};
      \filldraw (4,5) circle (2pt) node[above] {4};
      \filldraw (0,0) circle (2pt) node[below] {1};
      \filldraw (1,0) circle (2pt) node[below] {4};
      \filldraw (2,0) circle (2pt) node[below] {0};
      \filldraw (3,0) circle (2pt) node[below] {3};
      \filldraw (4,0) circle (2pt) node[below] {2};
      \strand (0,5) .. controls (0.5,0.5) and (1.5,3.5) .. (2,0);
      \strand (1,5) .. controls (0.75,0.5) and (0.25,3.5) .. (0,0);
      \strand (2,5) .. controls (2.5,2.5) and (3.5,1.5) .. (4,0);
      \strand (3,5) .. controls (4.5,2.5) and (4,1.5) .. (3,0);
      \strand (4,5) .. controls (3.5,2.5) and (1.5,2.5) .. (1,0);
      \flipcrossings{4,5};
    \end{knot}
  \end{tikzpicture}
\]

\note{Example: We reduce $\mathsf{swap} : 2 + 2 \leftrightarrow 2 + 2$ to a sequence of adjacent swaps. This is an
  example of a translation from $\PiPlusLang$ to $\PiHatLang$.}

\[
  \begin{tikzpicture}
    \begin{knot}[clip width=4]
      \filldraw (0,4) circle (2pt) node[above] {0};
      \filldraw (1,4) circle (2pt) node[above] {1};
      \filldraw (2,4) circle (2pt) node[above] {2};
      \filldraw (3,4) circle (2pt) node[above] {3};
      \filldraw (0,0) circle (2pt) node[below] {2};
      \filldraw (1,0) circle (2pt) node[below] {3};
      \filldraw (2,0) circle (2pt) node[below] {0};
      \filldraw (3,0) circle (2pt) node[below] {1};
      \strand (0,4) .. controls (0.5,1.5) and (1.5,2.5) .. (2,0);
      \strand (1,4) .. controls (1.5,1.5) and (2.5,2.5) .. (3,0);
      \strand (2,4) .. controls (1.5,1.5) and (1.5,2.5) .. (0,0);
      \strand (3,4) .. controls (2.5,1.5) and (2.5,2.5) .. (1,0);
    \end{knot}
  \end{tikzpicture}
\]

\begin{align*}
  \begin{tikzpicture}
    \begin{knot}[clip width=4]
      \filldraw (0,4) circle (2pt) node[above] {0};
      \filldraw (1,4) circle (2pt) node[above] {1};
      \filldraw (2,4) circle (2pt) node[above] {2};
      \filldraw (3,4) circle (2pt) node[above] {3};
      \filldraw (0,0) circle (2pt) node[below] {0};
      \filldraw (1,0) circle (2pt) node[below] {2};
      \filldraw (2,0) circle (2pt) node[below] {1};
      \filldraw (3,0) circle (2pt) node[below] {3};
      \strand (0,4) to (0,0);
      \strand (1,4) .. controls (0.5,2) and (2.5,2) .. (2,0);
      \strand (2,4) .. controls (2.5,2) and (0.5,2) .. (1,0);
      \strand (3,4) to (3,0);
    \end{knot}
  \end{tikzpicture}
  &&
    \begin{tikzpicture}
      \begin{knot}[clip width=4]
        \filldraw (0,4) circle (2pt) node[above] {0};
        \filldraw (1,4) circle (2pt) node[above] {2};
        \filldraw (2,4) circle (2pt) node[above] {1};
        \filldraw (3,4) circle (2pt) node[above] {3};
        \filldraw (0,0) circle (2pt) node[below] {2};
        \filldraw (1,0) circle (2pt) node[below] {0};
        \filldraw (2,0) circle (2pt) node[below] {1};
        \filldraw (3,0) circle (2pt) node[below] {3};
        \strand (0,4) .. controls (-0.5,2) and (1.5,2) .. (1,0);
        \strand (1,4) .. controls (1.5,2) and (-0.5,2) .. (0,0);
        \strand (2,4) to (2,0);
        \strand (3,4) to (3,0);
      \end{knot}
    \end{tikzpicture}
  \\
  \begin{tikzpicture}
    \begin{knot}[clip width=4]
      \filldraw (0,4) circle (2pt) node[above] {2};
      \filldraw (1,4) circle (2pt) node[above] {0};
      \filldraw (2,4) circle (2pt) node[above] {1};
      \filldraw (3,4) circle (2pt) node[above] {3};
      \filldraw (0,0) circle (2pt) node[below] {2};
      \filldraw (1,0) circle (2pt) node[below] {0};
      \filldraw (2,0) circle (2pt) node[below] {3};
      \filldraw (3,0) circle (2pt) node[below] {1};
      \strand (0,4) to (0,0);
      \strand (1,4) to (1,0);
      \strand (2,4) .. controls (1.5,2) and (3.5,2) .. (3,0);
      \strand (3,4) .. controls (3.5,2) and (1.5,2) .. (2,0);
    \end{knot}
  \end{tikzpicture}
  &&
    \begin{tikzpicture}
      \begin{knot}[clip width=4]
        \filldraw (0,4) circle (2pt) node[above] {2};
        \filldraw (1,4) circle (2pt) node[above] {0};
        \filldraw (2,4) circle (2pt) node[above] {3};
        \filldraw (3,4) circle (2pt) node[above] {1};
        \filldraw (0,0) circle (2pt) node[below] {2};
        \filldraw (1,0) circle (2pt) node[below] {3};
        \filldraw (2,0) circle (2pt) node[below] {0};
        \filldraw (3,0) circle (2pt) node[below] {1};
        \strand (0,4) to (0,0);
        \strand (1,4) .. controls (0.5,2) and (2.5,2) .. (2,0);
        \strand (2,4) .. controls (2.5,2) and (0.5,2) .. (1,0);
        \strand (3,4) to (3,0);
      \end{knot}
    \end{tikzpicture}
\end{align*}

\note{This might be followed by a section which explains the syntax of Pi.}

%%% Local Variables:
%%% mode: latex
%%% TeX-master: "main"
%%% fill-column: 120
%%% End:

\begin{figure}[t]
  {\scalebox{\scalef}{$%
        %%\noindent\begin{minipage}{.7\linewidth}
        \begin{array}{rrcll}
          \idc :     & A                     & \isoone & A                            & : \idc      \\
          \\
          \identlp : & \zerot + A            & \isoone & A                            & : \identrp  \\
          \swapp :   & A + B                 & \isoone & B + A                        & : \swapp    \\
          \assoclp : & A + (B + C)           & \isoone & (A + B) + C                  & : \assocrp  \\ [1.5ex]
          \identlt : & \onet \times A        & \isoone & A                            & : \identrt  \\
          \swapt :   & A \times B            & \isoone & B \times A                   & : \swapt    \\
          \assoclt : & A \times (B \times C) & \isoone & (A \times B) \times C        & : \assocrt  \\ [1.5ex]
          \absorbr : & ~ \zerot \times A     & \isoone & \zerot ~                     & : \factorzl \\
          \dist :    & ~ (A + B) \times C    & \isoone & (A \times C) + (B \times C)~ & : \factor
        \end{array}$}}

  \medskip

  {\scalebox{\scalef}{%
      \Rule{}
      {\jdg{}{}{c_1 : A \isoone B} \quad \vdash c_2 : B \isoone C}
      {\jdg{}{}{c_1 \fatsemi c_2 : A \isoone C}}
      {}

      \Rule{}
      {\jdg{}{}{c_1 : A \isoone B} \quad \vdash c_2 : C \isoone D}
      {\jdg{}{}{c_1 \oplus c_2 : A + C \isoone B + D}}
      {}

      \Rule{}
      {\jdg{}{}{c_1 : A \isoone B} \quad \vdash c_2 : C \isoone D}
      {\jdg{}{}{c_1 \otimes c_2 : A \times C \isoone B \times D}}
      {}
    }}
  \caption{$\Pi$-terms, combinators, and their types.}
  \label{fig:pi-terms}
\end{figure}

\section{A Reversible Programming Language}
%% ~3 pages
\label{sec:pi}
\label{sec:reversibleone}
\label{sec:reversibletwo}
\label{langeqeq}
\label{sec:informal}

The circuit model of reversible computation discussed in the introduction is a useful abstraction close to the hardware
platform. However, since its main data abstraction is a \emph{sequence of wires}, it only provides an ``assembly-level''
programming abstraction (e.g., \verb|qasm|). As motivated by \citet{LAFONT2003257}, a mathematical model based on
permutations of finite sets provides a richer algebraic structure which we review in this section.

%%%%%%%%%%%%%%%%%
\subsection{The $\Pi$ Family of Languages}
\label{sec:langRev-examples}
\label{examples}

In a reversible boolean circuit, the number of input bits matches the number of output bits. Circuits can be composed
sequentially, vertically, or horizontally, respectively preserving the number of bits, adding up the number of bits or
multiplying them. So, to program with reversible circuits, all we need to do is make sure that each primitive operation
preserves the number of bits, which is just a natural number. Natural numbers are the free commutative semiring (or,
commutative rig) on one generator, with $(0,+)$ for addition, and $(1,\times)$ for multiplication, so the commutative
semirig identities can be used to design a logic for reversible programming. This inspired the syntax for various
first-order reversible logics and reversible programming
languages~\cite*{sparksSuperstructuralReversibleLogic2014,jamesInformationEffects2012}.

Natural number identities have no computational content. To obtain a computational interpretation of the commutative rig
structure, the logic needs to be equipped with values and types, and a notion of operational semantics and contextual
equivalence~\cite{jamesInformationEffects2012}. This is a programming language which embodies the computational content
of isomorphisms of finite types, or permutations. On the semantic side, we need to consider the groupoidification of a
commutative rig, that is, a symmetric rig groupoid. A sound semantics for $\PiLang$ in weak rig groupoids was
established in~\cite{caretteComputingSemiringsWeak2016}, and conjectured to be complete.

% In programming parlance, as shown in~\cref{fig:pi-terms}, each primitive algebraic identity of commutative rigs becomes
% a reversible combinator in the programming language and the algebraic closure operators become composition operators in
% the programming language. Once equipped with a notion of values and types, we get the following syntactic presentation
% of a programming language for permutations~\cite{James:2012:IE:2103656.2103667,Carette2016}:

% A specialised language for programming with permutations on finite sets can be built using ideas going back to Kelly,
% Laplaza, and Mac Lane~\cite{laplaza72,kelly74,KELLY197197} that the commutative semiring (also known as the commutative
% rig) of natural numbers exactly characterizes permutations on finite sets. In programming parlance, as shown in
% Fig.~\ref{fig:pi-terms}, each primitive algebraic identity of commutative rigs becomes a constant in the programming
% language and the algebraic closure operators become composition operators in the programming language. Once equipped
% with a notion of values and types, we get the following syntactic presentation of a programming language for
% permutations~\cite{James:2012:IE:2103656.2103667,Carette2016}:

% The practice of programming languages is replete with \emph{ad hoc} instances of reversible computations: database
% transactions, mechanisms for data provenance, checkpoints, stack and exception traces, logs, backups, rollback
% recoveries, version control systems, reverse engineering, software transactional memories, continuations, backtracking
% search, and multiple-level undo features in commercial applications. In the early nineties,
% \citet{Baker:1992:LLL,Baker:1992:NFT} argued for a systematic, first-class, treatment of reversibility. But intensive
% research in full-fledged reversible models of computations and reversible programming languages was only sparked by the
% discovery of deep connections between physics and
% computation~\cite{Landauer:1961,PhysRevA.32.3266,Toffoli:1980,bennett1985fundamental,Frank:1999:REC:930275, Hey:1999:FCE:304763,fredkin1982conservative}, and by the
% potential for efficient quantum computation~\cite{springerlink:10.1007/BF02650179}.

% The early developments of reversible programming languages started
% with a conventional programming language, e.g., an extended
% $\lambda$-calculus, and either:
% \begin{enumerate}
%   \item extended the language with a history
%         mechanism~\cite{vanTonder:2004,Kluge:1999:SEMCD,lorenz,danos2004reversible}, or
%   \item imposed constraints on the control flow constructs to make them
%         reversible~\cite{Yokoyama:2007:RPL:1244381.1244404}.
% \end{enumerate}
% More foundational approaches recognize that reversible programming languages require a fresh approach and should be
% designed from first principles without the detour via conventional irreversible
% languages~\cite{Yokoyama:2008:PRP,Mu:2004:ILRC,abramsky2005structural,DiPierro:2006:RCL:1166042.1166047,
%   rc2011, James:2012:IE:2103656.2103667, Carette2016}.

\medskip

{\scalebox{\scalef}{$%
      \begin{array}{lrcl}
        \textit{Value types}   & A,B,C,D & ::= & \zerot \alt \onet \alt A+B \alt A\times B        \\
        \textit{Values}        & v,w,x,y & ::= & \Acon{tt} \alt \inlv{v} \alt \inrv{v} \alt (v,w) \\
        \textit{Program types} &         &     & A \isoone B                                      \\
        \textit{Programs}      & c       & ::= & (\textrm{See Fig.~\ref{fig:pi-terms}})
      \end{array}$}}

\medskip\noindent The language of types is built from the empty type ($\zerot$), the unit type
($\onet$) containing just one value~$\Acon{tt}$, the sum type ($+$) containing values of the form $\inlv{v}$ and
$\inrv{v}$, and the product type ($\times$) containing pairs of values $(v_1,v_2)$.
%
% A natural candidate for a semantic foundation for reversible programming languages is the notion of type
% isomorphism. Indeed, the type isomorphisms among finite types are sound and complete for all permutations on finite
% types~\cite{Fiore:2004,fiore-remarks} and hence they are \emph{complete} for expressing reversible combinational
% circuits~\cite{fredkin1982conservative, James:2012:IE:2103656.2103667,Toffoli:1980} and the extension with recursive
% types and trace operators~\cite{Hasegawa:1997:RCS:645893.671607} is a Turing-complete reversible
% language~\cite{James:2012:IE:2103656.2103667,rc2011}.

To see how this language expresses reversible circuits, we present a few examples. First it is possible to directly
mimic the \verb|qasm|-perspective by defining types that describe sequences of booleans. We use the type
$\mathbb{2} = \onet + \onet$ to represent booleans with $\inlv{\Acon{tt}}$ representing \textsf{true} and
$\inrv{\Acon{tt}}$ representing $\textsf{false}$. Boolean negation (the \verb|x|-gate) is straightforward to define using
the primitive combinator $\swapp$. We can represent $n$-bit words using an n-ary product of boolean values, thus the
type $\mathbb{2} \times (\mathbb{2} \times \mathbb{2})$ (abbreviated $\mathbb{B}~3$) corresponds to a collection of
wires that can transmit three bits.
%
% For example, we can express a 3-bit word reversal operation as follows:
% $\Afun{reverse} : \mathbb{B}~3 \iso \mathbb{B}~3$
% $\Afun{reverse} = \swapt \fatsemi (\swapt  \otimes  \idc)~ \fatsemi \assocrt$
% \noindent The manual trace of $\Afun{reverse}$ below confirms that it indeed reverses the three bits:
% \[\begin{array}{rlr}
%                          & (v_1, (v_2, v_3)) \\
%     \swapt               & ((v_2, v_3), v_1) \\
%     \swapt \otimes  \idc & ((v_3, v_2), v_1) \\
%     \assocrt             & (v_3, (v_2, v_1)) \\
%   \end{array}\]
%subcode source isomorphisms.tex:979
%
To express the \verb|cx| and \verb|ccx| gates, we need to encode a notion of conditional expression. Such conditionals
turn out to be expressible using the distributivity and factoring identities of rigs as shown below:

\medskip

\cif{}

\noindent The input value of type $\mathbb{2} \times A$ is processed by the distribute operator \ensuremath{\dist},
which converts it into a value of type $(\onet \times A) + (\onet \times A)$. In the left branch, which corresponds to
the case when the boolean is \textsf{true}, the combinator~\ensuremath{c_1} is applied to the value of
type~\ensuremath{A}. The right branch which corresponds to the boolean being \textsf{false} passes the value of type $A$
through the combinator \ensuremath{c_2}.  The inverse of \ensuremath{\dist}, namely \ensuremath{\factor} is applied to
get the final result. Using this conditional operator, \verb|cx| is defined as $\Afun{cif}~\verb|x|~\idc$ and
\verb|ccx| is defined as $\Afun{cif}~\verb|cx|~\idc$. With these conventions, the first circuit in the introduction
is transcribed as follows:

\medskip

\adder{}

\noindent where we clearly see the sequences of the three operations \verb|ccx|, \verb|cx|, and \verb|cx| but, instead
of using the indices in the sequence of wires to identify the relevant parameters, here we use structural isomorphisms
(elided above) to re-shuffle the types. For the second circuit, instead of transcribing it directly, we express it using
a slightly more abstract notation:

\medskip

\resettwo{}

\noindent Like the original circuit, we examine the bit at index 1 (corresponding to the component $B$ in a tuple
$(A,(B,C))$): if the bit is true, we perform an \verb|x| operation on component $A$, and otherwise we perform a
\verb|cx| operation on $(C,A)$. The two uses of \verb|x| gates in the circuit are now unnecessary as they were only needed
to encode a two-way conditional expression using a sequence of one-way conditional expressions (the only ones available in
the linear circuit model).

All of this is only half the story however as the correspondence between the syntactic commutative rig (the syntax of $\PiLang$)
and permutations on finite sets includes \emph{coherence conditions} that identify different syntactic representations
of the same permutation~\cite{laplaza72,Carette2016}. These coherence conditions are collected in a second level of $\PiLang$
syntax as level-2 combinators whose types are of the form $c_1 \isotwo c_2$ for appropriate $c_1$ and $c_2$ of the same
level-1 type $A \isoone B$. For example, we have:

\medskip

\combtwo{}

\noindent where the first level-2 combinator states the obvious fact that compositions with the identity can be
optimized and the second level-2 combinator states the more interesting observation that swapping the two sides of a sum
type and then applying $c_1$ to the left and $c_2$ to the right is equivalent to applying $c_2$ to the left and $c_1$ to
the right and then swapping. The full set of level-2 combinators is large and is only listed in the accompanying code.

% and that impose various coherence conditions on the level-1 combinators. In the remainder of this section, we show how
% to use a subset of these level-2 combinators to show the equivalence of \Afun{reversibleOr1} and \Afun{reversibleOr2}.

% \medskip

% \orequiv{}

% \note{This section should explain the main technical parts of the paper
%   informally, without using any technology. Use an example, such as, a
%   reversible language with $\leq 5$ bits, and examples of permutations and
%   transpositions, and when they're equal.}

%%%%%%%%%
\subsection{Denotational Semantics}

\noindent The accompanying code includes an operational semantics for $\Pi$. A denotational semantics can be specified
by mapping each type to a finite set and each combinator to a permutation between finite sets. In this section, we
outline a simple denotational semantics expressed in the metalanguage of sets and functions that collapses the level-2
combinators. This semantics will be extended in the next two sections to the HoTT metalanguage to properly model the
level-2 combinators.

The denotation of types is straightforward:

\begin{center}
  {\scalebox{\scalef}{$%
        \begin{array}{rcl}
          \denot{\zerot}     & = & \bot                       \\
          \denot{\onet}      & = & \top                       \\
          \denot{A + B}      & = & \denot{A} \sqcup \denot{B} \\
          \denot{A \times B} & = & \denot{A} \times \denot{B}
        \end{array}$}}
\end{center}

\noindent where $\bot$ is the empty set, $\top$ is a singleton set, $\sqcup$ is the disjoint union of sets, and $\times$
is the cartesian product of sets. The denotation of a combinator $c : A \isoone B$ is a (bijective) function mapping
$\denot{A}$ to $\denot{B}$:

\begin{multicols}{2}
  %%  \begin{longtable}{>{$}r<{$} >{$}l<{$} >{$}c<{$} >{$}l<{$}}
  {\scalebox{\scalef}{$%
        \begin{array}[t]{rlcl}
          \denot{\identlp} & (\inl{v})         & = & v               \\
          \denot{\identrp} & v                 & = & \inl{v}         \\
          \denot{\swapp}   & (\inl{v})         & = & \inr{v}         \\
          \denot{\swapp}   & (\inr{v})         & = & \inl{v}         \\
          \denot{\assoclp} & (\inl{v})         & = & \inl{(\inl{v})} \\
          \denot{\assoclp} & (\inr{(\inl{v})}) & = & \inl{(\inr{v})} \\
          \denot{\assoclp} & (\inr{(\inr{v})}) & = & \inr{v}         \\
          \denot{\assocrp} & (\inl{(\inl{v})}) & = & \inl{v}         \\
          \denot{\assocrp} & (\inl{(\inr{v})}) & = & \inr{(\inl{v})} \\
          \denot{\assocrp} & (\inr{v})         & = & \inr{(\inr{v})}
        \end{array}$}}

  {\scalebox{\scalef}{$%
        \begin{array}[t]{rlcl}
          \denot{\identlt} & (\ttt , v)          & = & v                   \\
          \denot{\identrt} & v                   & = & (\ttt , v)          \\
          \denot{\swapt}   & (v_1 , v_2)         & = & (v_2 , v_1)         \\
          \denot{\assoclt} & (v_1 , (v_2 , v_3)) & = & ((v_1 , v_2) , v_3) \\
          \denot{\assocrt} & ((v_1 , v_2) , v_3) & = & (v_1 , (v_2 , v_3)) \\
          \denot{\dist}    & (\inl{v_1} , v_3)   & = & \inl{(v_1 , v_3)}   \\
          \denot{\dist}    & (\inr{v_2 , v_3})   & = & \inr{(v_2 , v_3)}   \\
          \denot{\factor}  & (\inl{(v_1 , v_3)}) & = & (\inl{v_1} , v_3)   \\
          \denot{\factor}  & (\inr{(v_2 , v_3)}) & = & (\inr{v_2} , v_3)
        \end{array}$}}

  %%\end{longtable}
\end{multicols}

\begin{center}
  {\scalebox{\scalef}{$%
        \begin{array}{rlcl}
          \denot{\idc}               & v           & = & v                                   \\
          \denot{(c_1 \fatsemi c_2)} & v           & = & (\denot{c_2} \circ \denot{c_1}) v   \\
          \denot{(c_1 \oplus c_2)}   & (\inl{v})   & = & \inl{(\denot{c_1}~v)}               \\
          \denot{(c_1 \oplus c_2)}   & (\inr{v})   & = & \inr{(\denot{c_2}~v)}               \\
          \denot{(c_1 \otimes c_2)}  & (v_1 , v_2) & = & (\denot{c_1} v_1 , \denot{c_2} v_2)
        \end{array}
      $}}
\end{center}

% \begingroup
% \allowdisplaybreaks
% \begin{align*}
% \end{align*}
% \endgroup

% \begin{center}
% \begin{tikzpicture}[scale=0.7,every node/.style={scale=0.8}]
%   \draw[>=latex,<->,double,red,thick] (2.25,-1.2) -- (2.25,-2.9) ;
%   \draw[fill] (-2,-1.5) circle [radius=0.025];
%   \node[below] at (-2.1,-1.5) {$A$};
%   \node[below] at (-2.1,-1.9) {$+$};
%   \draw[fill] (-2,-2.5) circle [radius=0.025];
%   \node[below] at (-2.1,-2.5) {$B$};

%   \draw[fill] (6.5,-1.5) circle [radius=0.025];
%   \node[below] at (6.7,-1.5) {$C$};
%   \node[below] at (6.7,-1.9) {$+$};
%   \draw[fill] (6.5,-2.5) circle [radius=0.025];
%   \node[below] at (6.7,-2.5) {$D$};

%   \draw[<-] (-2,-1.5) to[bend left] (1,0.5) ;
%   \draw[<-] (-2,-2.5) to[bend left] (1,-0.5) ;
%   \draw[->] (3.5,0.5) to[bend left] (6.5,-1.45) ;
%   \draw[->] (3.5,-0.5) to[bend left] (6.5,-2.45) ;

%   \draw[<-] (-2,-1.5) to[bend right] (1,-3.5) ;
%   \draw[<-] (-2,-2.5) to[bend right] (1,-4.5) ;
%   \draw[->] (3.5,-3.5) to[bend right] (6.5,-1.55) ;
%   \draw[->] (3.5,-4.5) to[bend right] (6.5,-2.55) ;


%   \draw     (2,0.5)  -- (2.5,0.5)  ;
%   \draw     (2,-0.5) -- (2.5,-0.5) ;

%   \draw     (2.5,0.5)  -- (3.5,-0.5)  ;
%   \draw     (2.5,-0.5) -- (3.5,0.5) ;

%   \draw     (1,-3.5)  -- (2,-4.5)    ;
%   \draw     (1,-4.5) -- (2,-3.5)   ;

%   \draw     (2,-3.5)  -- (2.5,-3.5)    ;
%   \draw     (2,-4.5) -- (2.5,-4.5)   ;

%   \path (1.5,0.5) node (tc1) [func] {$c_1$};
%   \path (1.5,-0.5) node (tc2) [func] {$c_2$};
%   \path (3,-4.5) node (bc1) [func] {$c_1$};
%   \path (3,-3.5) node (bc2) [func] {$c_2$};
% \end{tikzpicture}
% \end{center}
% The top path is the $\Pi$ program $(c_1~\oplus~c_2)~\odot~\swapp$ which acts on the type $A$ by $c_1$, acts on the type
% $B$ by $c_2$, and acts on the resulting value by a twist that exchanges the two injections into the sum type. The bottom
% path performs the twist first and then acts on the type $A$ by $c_1$ and on the type $B$ by $c_2$ as before. One could
% imagine the paths are physical \emph{elastic} wires in $3$ space, where the programs $c_1$ and $c_2$ as arbitrary
% deformations on these wires, and the twists do not touch but are in fact well-separated. Then, holding the points $A$,
% $B$, $C$, and $D$ fixed, it is possible to imagine sliding $c_1$ and $c_2$ from the top wire rightward past the twist,
% and then using the elasticity of the wires, pull the twist back to line up with that of the bottom --- thus making both
% parts of the diagram identical.  Each of these moves can be undone (reversed), and doing so would take the bottom part
% of the diagram into the top part.  In other words, there exists an equivalence of the program
% $(c_1~\oplus~c_2)~\odot~\swapp$ to the program $\swapp \odot (c_2~\oplus~c_1)$. We can also show that this means that,
% as permutations, $(c_1~\oplus~c_2)~\odot~\swapp$ and $\swapp \odot (c_2~\oplus~c_1)$ are equal. And, of course, not all
% programs between the same types can be deformed into one another. The simplest example of inequivalent deformations are
% the two automorphisms of $1+1$, namely $\idc$ and $\swapp$.

% The denotational semantics can be used to evaluate programs and to check equivalence of programs. For example, it is
% straightforward to verify that $\denot{(c_1~\oplus~c_2)~\fatsemi~\swapp} = \denot{\swapp \fatsemi (c_2~\oplus~c_1)}$. A
% slightly more involved example are the following two programs:

% \rotate{}

% \noindent The first program performs the following sequence of transformations:
% \[
%   \Tree [ {\small a} [ {\small b} {\small c} ] ] ~\to~
%   \Tree [ [ {\small a} {\small b} ] {\small c} ] ~\to~
%   \Tree [ {\small c} [ {\small a} {\small b} ] ] ~\to~
%   \Tree [ {\small c} [ {\small b} {\small a} ] ] ~.
% \]
% \noindent
% while the second evaluates as follows:
% \[
%   \Tree [ {\small a} [ {\small b} {\small c} ] ] ~\to~
%   \Tree [ {\small a} [ {\small c} {\small b} ] ] ~\to~
%   \Tree [ [ {\small a} {\small c} ] {\small b} ] ~\to~
%   \Tree [ [ {\small c} {\small a} ] {\small b} ] ~\to~
%   \Tree [ {\small c} [ {\small a} {\small b} ] ] ~\to~
%   \Tree [ {\small c} [ {\small b} {\small a} ] ] ~.
% \]

% The semantics above can be packaged in the category of finite sets and functions, $\SetFin$, which is the category
% freely generated by finite coproduct completion of the terminal category. Objects of $\SetFin$ can be identified with
% sets of fixed cardinality, that is, $\Fin[n] \defeq \Set{0,1,\ldots,n-1}$. $\SetFin$ has finite coproducts and products,
% which lets us interpret the types of $\PiLang$. Combinators are interpreted as morphisms in $\SetFin$, but we have to
% restrict to invertible morphisms, that is, isomorphisms. This gives the \emph{groupoid} of finite sets and bijections,
% $\BFin \defeq \mathsf{core}(\SetFin)$. The isomorphisms satisfied by coproducts and products in $\SetFin$ lift to
% $\BFin$, but they're no longer categorical coproducts and products. They give two symmetric monoidal tensor products on
% $\BFin$, the additive and multiplicative ones, with the multiplicative tensor distributing over the additive tensor.

% \note{change thm:
%   1. Each function is a bijection. 2. If there is a 2-combinator, the denotations of the 1-combinators are equal
% }

\begin{theorem}\label{thm:semone}
  The denotational semantics is sound in the following sense:
  \begin{itemize}
    \item For every level-1 combinator $c : A \isoone B$, we have that $\denot{c}$ is a bijection between $\denot{A}$ and $\denot{B}$.
    \item For every pair of combinators $c_1$ and $c_2$ of the same type $A \isoone B$, if there exists a level-2
          combinator $\alpha$ such that $\alpha : c_1 \isotwo c_2$, then $\denot{c_1} = \denot{c_2}$ using
          extensional equivalence of functions.
  \end{itemize}
\end{theorem}
\begin{proof}
  For every primitive combinator $c$ listed on one side of Fig.~\ref{fig:pi-terms}, let $!c$ be the combinator listed on
  the other side. Thus $! \assoclp$ is $\assocrp$ and $! \swapp$ is $\swapp$ itself. Then we have that $\denot{c}$ and
  $\denot{!c}$ form an equivalence. For the level-2 combinator \Afun{idr◎l}, we check
  $\denot{\AgdaBound{c}~\AgdaOperator{\AgdaInductiveConstructor{◎}}~\AgdaInductiveConstructor{id⟷₁}}
    = \mathit{id} \circ \denot{c} = \denot{c}$. The other cases are only slightly more involved calculations.
\end{proof}


%   Let $\sim$ be the following equivalence relation on combinators $c_1 \sim c_2$ iff $\denot{c_1} = \denot{c_2}$
%   identifying combinators with the same denotation.  The notation $[c]_{\sim}$ refers to a representative combinator
%   from a $\sim$-equivalence class. Using this relation, we define the
%   category $\mathcal{C}$ as follows:
%   \begin{itemize}
%     \item $\mathit{Obj}(\mathcal{C})$ is the set of $\Pi$-types, and
%     \item $\mathit{Hom}(A,B) = [c : A \isot B]_{\sim}$
%   \end{itemize}
%   The category $\mathcal{C}$ is a groupoid.
% \end{theorem}
% \begin{proof}
%   The relation $\sim$ identifies $c$, $\idc \circ c$, and $c \circ \idc$, and is associative. Furthermore, the denotation
%   of each combinator is a reversible function thus making every morphism into an isomorphism.
% \end{proof}

% \note{Then say that $\mathcal{C}$ is ``the same'' as $\BFin$ We can say here that the relation $\sim$ also identifies
%   $(c_1~\oplus~c_2)~\fatsemi~\swapp$ and $\swapp \fatsemi (c_2~\oplus~c_1)$ and other properties. ?? }

% \[\begin{array}{l}
%     \Tree[ {\small a} [ {\small b} {\small c} ] ] \\
%     2 \quad 1 \quad 0 \\
%     0 \quad 2 \quad 1 \\
%     \Tree[ [ {\small b}  {\small c} ] {\small a} ]
% \end{array}\]

% Such functions are finitely supported, that is, their outputs can be tabulated using the canonical ordering on
% $\Fin[n]$. For bijections, this gives a listed permutation. By observing the action of the combinators on the values of
% the finite set, we can define a denotational semantics which constructs the bijection.
% \note{\(\Fin[n] \to A \eqv <\type{Vec_{n}}(A)\)}

% In the previous section, we examined equivalences between conventional data structures, i.e., structured trees of
% values. We now consider a richer but foundational notion of data: programs themselves. Indeed, universal computation
% models crucially rely on the fact that \emph{programs are (or can be encoded as) data}, e.g., a Turing machine can be
% encoded as a string that another Turing machine (or even the same machine) can manipulate. Similarly, first-class
% functions are the \emph{only} values in the $\lambda$-calculus.  In our setting, we ask whether the programs developed
% in the previous section can themselves be subject to (higher-level) equivalences?

Using categorical language, the denotational semantics for $\PiLang$ is using the category of finite sets and functions
$\SetFin$. However, we only use the bijective functions for the semantics, which means, we use the groupoid $\BFin =
  \term{core}(\SetFin)$ of finite sets and bijections. $\SetFin$ has finite coproducts $(\emptyt, \sqcup)$ and finite
products $(\unit, \times)$, which we use to interpret our types. In $\BFin$, the coproducts and products restrict to
additive and multiplicative symmetric monoidal structures, respectively, making $\BFin$ a symmetric rig groupoid.

%%% Local Variables:
%%% mode: latex
%%% TeX-master: "main"
%%% fill-column: 120
%%% End:

\section{The Groupoid of Finite Types}
\label{sec:ufin}

% First - recap the previous section
% Second - say what was wrong

In categorical language, the setting for the semantics in the previous section is the category of finite sets and
functions $\SetFin$. However, as $\PiLang$ only refers to bijective functions, a more precise setting is the groupoid
$\BFin = \term{core}(\SetFin)$ of finite sets and bijections. $\SetFin$ has finite coproducts $(\emptyt, \sqcup)$ and
finite products $(\unit, \times)$ and in $\BFin$ these restrict to additive and multiplicative symmetric monoidal
structures, respectively, making $\BFin$ a symmetric rig groupoid -- the \emph{vertical categorification} of the
commutative rig of natural numbers $\Nat$~\cite{baezHIGHERDIMENSIONALALGEBRA2010}.

The semantics interprets types of $\PiLang$ as objects in $\BFin$, 1-combinators as isomorphisms, and for every pair of 1-combinators
related by a 2-combinator, their interpretations in~$\BFin$ are equal.  The groupoid $\BFin$ is strict, since the
collection of isomorphisms is a set, that is, a discrete category. There is no explicit witness for the equality of two
isomorphisms, since we can decide by evaluating two bijections whether they are equal. To be able to establish
completeness for $\PiLang$, we want a witness for this equality, so that we can quote back to the syntax and produce a
2-combinator witnessing the equality of the corresponding 1-combinators.

% The denotational semantics for $\PiLang$ in the previous section uses the symmetric rig groupoid structure of $\BFin$,
% whose objects are finite sets, whose morphisms as bijections between finite sets, and whose symmetric monoidal
% structures are induced by the operations $\sqcup$ and $\times$. The

% two equal permutations are extensionally identified, without an explicit witness for the equality. In order to
% establish completeness, we want to quote back from the semantics to the syntax. Specifically, given a morphism in
% $\BFin$, we want to produce a 1-combinator in $\PiLang$, and given two extensionally equal morphisms in $\BFin$, we
% want to produce a 2-combinator witnessing the equality of the quoted 1-combinators.

% Third - say how are we fixing it

We observe that the implicit equalities between the isomorphisms are pointwise equalities of functions, that is,
homotopies. We therefore \emph{weaken} the groupoid $\BFin$, exposing these homotopies, by using higher invertible cells.
We work in HoTT (Univalent Foundations) as it provides a proof-relevant, constructive metatheory to get a handle
on these equalities and provides a rich internal language for describing weak groupoids, using the ``types are weak
$\infty$-groupoids'' correspondence.

% To get a handle on these equalities, we will work in a proof-relevant metatheory, that is, HoTT, and then we will
% be able to write functions out of them back into the syntax.

% HoTT provides a rich internal language for describing weak groupoids.

% The idea is that we move from (discrete) sets to proof-relevant $\hSet$s, where the type of equalities between two
% points in a set is a proposition, that is, a prop-enriched groupoid.

% Our first step is therefore to expose the implicit equalities by \emph{weakening} the groupoid $\BFin$, i.e., by
% replacing the equational axioms of the category by higher invertible cells. In a weak category, the identity and
% associativity equational conditions are replaced by higher cells for the left and right units of composition and for
% associativity; these higher cells come with their own coherence data as well, such as vertical composition, and
% horizontal composition, or whiskering as detailed in appendix~\ref{app:twocat}.

% Fourth - explain the tool we use for fixing it
Every type in HoTT is a weak $\infty$-groupoid whose points are the terms of the type, and the (iterated) identity type
gives the (higher) morphisms. The groupoid we are interested in has types as points, type equivalences for 1-cells, and
higher homotopies for higher cells. (See~\cref{app:grpdexample} for an example on a 3-element set.) This is the
groupoid structure for the universe type $\UU$, since the identity type on types can be characterised as type
equivalences (by \emph{univalence}). But, we only want to carve out a \emph{subuniverse} of \emph{finite types}, still
satisfying univalence, to get the groupoid structure. In this section, we formally define \emph{univalent subuniverses},
and proceed to construct the particular instance for finite types, $\UFin$ (\cref{def:ufin}).

% The way to describe such groupoid in HoTT is to construct a type which has the types themselves as points, i.e. a
% universe. Then, equivalences between those types can be characterised using elements of the equality type. For this to
% work, the universe in question has be a \emph{univalent subuniverse}.

% In HoTT, every type carries a groupoid structure, with terms for points, and the (iterated) identity type for (higher)
% morphisms. We're interested in the groupoid structure of the universe, which has types for points, equivalences for
% identities between them (by univalence), and higher homotopies for higher identifications. We want a similar
% universe, whose only points are the finite types, that is, a subuniverse, and, equalities between them should
% correspond to equivalences of finite types -- the subuniverse should be univalent. To construct a univalent
% subuniverse starting from a univalent universe, we will introduce the concept of univalent fibrations.

% However, although weakening the groupoid exposes some implicit equalities, it does not, by itself, give us a handle to
% characterize such equalities. One elegant way to acquire such an ability is to ensure that the groupoid is
% ``univalent,'' as this provides a way to characterize the semantic equivalences using the elements of the identity
% type, i.e., using objects whose equality is equipped with an induction principle. We will therefore work within a HoTT
% metatheory which provides an ambient univalent universe of types and introduce the novel idea of a \emph{univalent
% subuniverse} to carve out $\UFin$: a weakened, univalent, version of the groupoid $\BFin$ which is then extended with
% the appropriate commutative rig structure. The main technical result of this section that is exploited in the next
% section is that, for each finite type $T$, we have transferred the structure of the space of equivalences $T \eqv T$
% to the structure of paths in $\UFin$. Since the space equivalences forms the automorphism group of $T$, we can now use
% the tools of computational group theory together with the induction principle of paths and term rewriting theory to
% describe the paths in $\UFin$.

% HoTT provides a rich internal language for working with weak higher groupoids, using the ``types are weak
% $\infty$-groupoids'' correspondence. In this section, we show that the weakened version of the groupoid $\BFin$ can be
% written in HoTT as a type $\UFin$, which is a 1-groupoid that has points for 0-cells, 1-paths for 1-cells, 2-paths for
% 2-cells with at most one 2-cell between compatible 1-cells.

%% In this section, we develop the tools necessary to construct $\UFin$, starting with a quick review of HoTT. % Using
%the identity type for morphisms, a 1-groupoid in HoTT

% IS THAT FOOTNOTE NECESSARY ??? We should avoid using footnotes as much as possible.\footnote{This is a locally-strict
%   $(2,0)$-category, since every cell above 0 is invertible, and the hom-categories are posets, that is, truth-value
%   enriched.}

\begin{toappendix}
  \label{app:grpdexample}
  We give an example of the groupoid structure on a 3-element set.

  \[
    \begin{tikzcd}
      \Fin[3]
      \arrow[""{name=0, anchor=center, inner sep=0}, "{f_{3}}", no head, loop, distance=4em, in=115, out=65]
      \arrow[""{name=0, anchor=center, inner sep=0}, "{f_{2}}", no head, loop, distance=8em, in=125, out=55]
      \arrow[""{name=1, anchor=center, inner sep=0}, "{f_{1}}"', no head, loop, distance=12em, in=135, out=45]
      \arrow["", "{h}", shorten <=3pt, shorten >=3pt, Rightarrow, no head, from=0, to=1]
    \end{tikzcd}
  \]

  \noindent We have $\Fin[3] = \Set{0,1,2} \eqv \unit \sqcup (\unit \sqcup \unit)$ which fixes a particular enumeration of the
  elements. Suppose we have a set $X = (\unit \sqcup \unit) \sqcup \unit$, it has the same cardinality as $\Fin[3]$, so it
  is represented by the same 0-cell. But, $X$ can be made equivalent to $\Fin[3]$ in many different ways since there are
  many bijections between them. One bijection is
  $\Set{\inl(\inl(\ttt)) \mapsto 0, \inl(\inr(\ttt)) \mapsto 1, \inr(\ttt) \mapsto 2}$ which can be written in two
  different ways by composing more primitive operations, $f_{1} = \assocrp$, or
  $f_{2} = \swapp \compc \assoclp \compc \swapp$. Another bijection is
  $\Set{\inl(\inl(\ttt)) \mapsto 1, \inl(\inr(\ttt)) \mapsto 2, \inr(\ttt) \mapsto 0}$ which is given by $f_{3} = \ldots$.
  Since $f_{1}$ and $f_{2}$ produce the same enumeration of the elements of $X$, they are identified by a homotopy $h$
  which is encoded in the 2-cell between them.

  At level 0, all we know is that if $X : \UFin[3]$, then X is merely equal to $\Fin[3]$, that is
  $\Trunc[-1]{X \id \Fin[3]}$, and we don't have access to the bijection. At level 1, if we know that both $X$ and $Y$ are
  \emph{equal} in $\UFin[3]$, then we can extract an equivalence between them, that is, $(X \id Y) \to (X \eqv Y)$.
  $\UFin[3]$ being a univalent subuniverse asserts that there are as many elements (upto higher homotopy) in $X \id Y$ as
  there are $X \eqv Y$.
\end{toappendix}

% \begin{toappendix}

%   \subsection{To Put Somewhere or Delete}

%   This approach is sufficient to prove the semantics forms a 1-category but ignores the rich structure at the next
%   level~\cite{carette2016}.
%   As explained in the previous section, a $\Pi$-type $A$ has $\sizet{A}$-elements and for all combinators $c : A \iso B$
%   we have that $\sizet{A} = \sizet{B}$. Hence, the denotation $\denot{A}$ of a type $A$ with $n$-elements can be the finite
%   set $\Fin[n] = \{ 0, 1, \cdots, n-1\}$; the denotation of a value $v : A$ such that $\sizet{A}=n$ will be an index in
%   the range $[0,n-1]$, and the denotation $\denot{c}$ of a combinator $c : A \iso B$ such that
%   $\sizet{A} = \sizet{B} = n$ will be a function from $\Fin[n]$ to $\Fin[n]$ that permutes the elements. Thus, all types
%   with 3 elements will denote $\Fin[3]$ and combinators between them will denote permutations on $\Fin[3]$, e.g.:
%   \[\begin{array}{rcl}
%       \denot{\onet + (\onet + \onet)}                                         & = & \{ 0,1,2 \} \\
%       \denot{(\onet + \onet) + \onet}                                         & = & \{ 0,1,2 \} \\
%       \\
%       \denot{\assoclp : \onet + (\onet + \onet) \iso (\onet + \onet) + \onet} & = & (0~1~2)     \\
%       \denot{\swapp : \onet + (\onet + \onet) \iso (\onet + \onet) + \onet}   & = & (2~0~1)
%     \end{array}\]
%   where we have used the one-line notation for permutations with $(a~b~c)$ representing the
%   permutation that maps 0 to $a$, 1 to $b$, and 2 to $c$. To make the denotation of values precise, we compute a canonical
%   enumeration of the elements of each type:
%   \[\begin{array}{rcl}
%       \mathit{enum}(\zerot)     & = & [~]                                                                                                          \\
%       \mathit{enum}(\onet)      & = & [ ~\Acon{tt}~ ]                                                                                              \\
%       \mathit{enum}(A + B)      & = & \mathit{map}~\Acon{inj₁}~\mathit{enum}(A) ~\textsf{+\!+}~ \mathit{map}~\Acon{inj₂}~\mathit{enum}(B)          \\
%       \mathit{enum}(A \times B) & = & \mathit{concat}~(\mathit{map}~(\lambda v.\mathit{map}~(\lambda w. (v,w))~\mathit{enum}(B))~\mathit{enum}(A))
%     \end{array}\]
%   \noindent The specification uses a Haskell-like notation for sequences with $\mathit{map}$ as the operation that applies
%   a function to each element of a sequence, \textsf{+\!+} as the binary append operation, and $\mathit{concat}$ the
%   operation that appends all the subsequences in a sequence.

%   Using the definition, we have:
%   \[\begin{array}{rcl}
%       \mathit{enum}(\onet + (\onet + \onet)) & = & [ \inlv{\Acon{tt}},~\inrv{(\inlv{\Acon{tt}})},~\inrv{(\inrv{\Acon{tt}})} ] \\
%       \mathit{enum}((\onet + \onet) + \onet) & = & [ \inlv{(\inlv{\Acon{tt}})},~\inlv{(\inrv{\Acon{tt}})},~\inrv{\Acon{tt}} ]
%     \end{array}\]
%   Thus, as shown in the diagrams below, $\assoclp~(\inlv{\Acon{tt}})$ applies the permutation $(0~1~2)$ to the index of
%   $\inlv{\Acon{tt}}$ which is 0 and produces index 0 in the $(\onet + \onet) + \onet$ type corresponding to value
%   $\inlv{(\inlv{\Acon{tt}})}$. Similarly, $\swapp~(\inlv{\Acon{tt}})$ applies the permutation $(2~0~1)$ to the index of
%   $\inlv{\Acon{tt}}$ which is 0 and produces index 2 in the $(\onet + \onet) + \onet)$ type corresponding to value
%   $\inrv{\Acon{tt}}$.

%   \begin{center}
%     \input{assoc-perm.tikz}
%     \qquad
%     \begin{tikzpicture}[scale=0.8,every node/.style={scale=0.8}]
\node at (-3,1.3) {$\mathit{enum}$};
\node at (-1,1.3) {$\swapp$};
\node at (0.7,1.3) {$\mathit{enum}$};
	\begin{pgfonlayer}{nodelayer}
		\node (4) at (-4, 1) {$\inlv{\Acon{tt}}$};
		\node (0) at (-4, 0) {$\inrv{(\inlv{\Acon{tt}})}$};
		\node (5) at (-4, -1) {$\inrv{(\inrv{\Acon{tt}})}$};

		\node (6) at (-2, 1) {0};
		\node (1) at (-2, 0) {1};
		\node (7) at (-2, -1) {2};

		\node (8) at (0, 1) {0};
		\node (2) at (0, 0) {1};
		\node (9) at (0, -1) {2};

		\node (10) at (2, 1) {$\inlv{(\inlv{\Acon{tt}})}$};
		\node (11) at (2, 0) {$\inlv{(\inrv{\Acon{tt}})}$};
		\node (12) at (2, -1) {$\inrv{\Acon{tt}}$};
	\end{pgfonlayer}
	\begin{pgfonlayer}{edgelayer}
		\draw[->] (4) to (6);
		\draw[->] (0) to (1);
		\draw[->] (5) to (7);
		\draw (6) to (9);
		\draw (1) to (8);
		\draw (7) to (2);
		\draw[<-] (8) to (10);
		\draw[<-] (2) to (11);
		\draw[<-] (9) to (12);
	\end{pgfonlayer}
\end{tikzpicture}

%   \end{center}


%   We choose a canonical set of size $n$, called $\mathsf{Fin}~n$, whose elements are natural numbers less than $n$. To
%   compute the denotation of a type $A$, we first calculate its size $n = \sizet{A}$. We then construct the canonical set
%   $\mathsf{Fin}~n$ and provide the (trivial) evidence that this set is identical to $(\mathsf{Fin}~n)$:

%   \[\begin{array}{rcll}
%       \sem{A} & = & (\mathsf{Fin}~n, [ n , \mathsf{refl} ]) & \mbox{where}~\sizet{A} = n
%     \end{array}\]

%   \noindent The denotation $\sem{c}$ of a combinator $c : A \isoone B$ is a path between $\sem{A}$ and $\sem{B}$. If the
%   size of $A$ is $m$ and the size of $B$ is $n$, the desired path is between $(\mathsf{Fin}~m, [ m , \mathsf{refl} ])$ and
%   $(\mathsf{Fin}~n, [ n , \mathsf{refl} ])$. This path is directly constructed using $\mathit{ap}$ and the fact that $m=n$
%   since combinators are always between types of the same size.

%   \noindent Finally, given two combinators $p , q : A \isoone B$ and a 2-combinator $\alpha : p \isotwo q$, the denotation
%   $\sem{\alpha}$ of $\alpha$ is a path between $\sem{p}$ and $\sem{q}$.

%   \note{We use the rig structure of $\UFin$ in~\Cref{subsec:rig} to interpret $\PiLang$.}


%   We need a formal definition of normal form (canonical form)

%   Recalling that the $\lambda$-calculus arises as the internal language of Cartesian Closed Categories
%   (Elliott~\cite{Elliott-2017} gives a particularly readable account of this), we can think of $\Pi$ in similar terms, but
%   for symmetric Rig Groupoids instead. For example, we can ask what does the equivalence above represent? It is actually a
%   ``linear'' representation of a 2-categorial commutative diagram! In fact, it is a painfully verbose version thereof, as
%   it includes many \emph{refocusing} steps because our language does not build associativity into its syntax. Categorical
%   diagrams usually do.  Thus if we rewrite the example in diagrammatic form, eliding all uses of associativity, but
%   keeping explicit uses of identity transformations, we get that \AgdaFunction{swap{-}fl2⇔swap{-}fl1} represents

%   \vspace*{3mm}
%   \begin{tikzcd}[column sep=normal, row sep=normal]
%     && (a+c)+b \arrow [r, "\swapp \oplus\idd", ""{name=U, below}] & (c+a)+b \arrow [dr, "\assocrp"] && \\
%     & a+(c+b) \arrow [ur, "\assoclp"] & & & c+(a+b) \arrow [dr, "\idd\oplus\swapp"] &  \\
%     a+(b+c) \arrow [ur, "\idd\oplus\swapp"] \arrow [r, "\assoclp"]
%     \arrow [dr, "\assoclp"]
%     \arrow [ddr, swap, "\assoclp"]
%     & (a+b)+c \arrow [r, "\swapp"] &
%     c+(a+b) \arrow [r, swap, "\assoclp", ""{name=D, above}]
%     & |[alias=Z]| (c+a)+b \arrow [r, "\assocrp"] &c+(a+b) \arrow [r, "\idd\oplus\swapp"] & c+(b+a) \\
%     & (a+b)+c \arrow [dr, "\swapp"] &&&& \\
%     & (a+b)+c \arrow [dr, swap, "\swapp"] & c+(a+b) \arrow [rr, swap, "\idd", ""{name=DD, above}]
%     \arrow [d, Rightarrow, "\idf\, \mathit{idl}\odot{l}"] &&
%     c+(a+b) \arrow [ruu, "\idd\oplus\swapp"] & \\
%     && c+(a+b) \arrow [rrruuu, bend right = 40, swap, "\idd\oplus\swapp"] && \\
%     \arrow[Rightarrow, from=U, to=D, "\mathit{hexagon}\oplus{r}\, \boxdot\, \idf"]
%     \arrow[Rightarrow, from=Z, to=DD, swap, "\idf\boxdot\mathit{linv}\odot{l}\,\boxdot\,\idf"]
%   \end{tikzcd}

% \end{toappendix}

%%%
\subsection{The Type Theory}
\label{subsec:type-theory}
% \subsection{Univalent Subuniverses}
% \label{sec:univalent}

% In order to define the weak groupoid we're after, we seek a mathematical structure satisfying the following properties:
% (i) it contains structures corresponding to all the finite types and nothing but the finite types, and (ii) it is robust
% enough to ensure that equivalent encodings of finite types are identified.

We use the type theory of the HoTT book~\cite{univalentfoundationsprogramHomotopyTypeTheory2013}, that is, we use
intensional Martin-L\"{o}f Type Theory, with a hierarchy of (univalent) universes $\UU_0 : \UU_1 : \ldots$ (though we
will just write $\UU$ since we mostly use one universe), and a few Higher Inductive Types (HITs) for truncations and
quotients. We write $\dfun*{x:A}{B(x)}$ or simply $(x:A) \to B(x)$ for dependent function types, and $\dsum*{x:A}{B(x)}$
or simply $(x:A) \times B(x)$ for dependent sum types.

All arguments will hold in a Cubical Type
Theory~\cite*{cohenCubicalTypeTheory2018,angiuliComputationalSemanticsCartesianCubical2019,vezzosiCubicalAgdaDependently2019}
as well. We review the basics of identity types, homotopy types, and some HITs that we use, and refer the reader to the
book for more details.

% To recap, types in HoTT have a weak groupoid structure given by the identity type, where points are given by terms of
% the type, and higher morphisms are given by the iterated identity type. Functions between types are functors between
% groupoids, and type families, or functions to the universe, are indexed families of groupoids.

\subsection{Identity Types}
\label{app:identitytypes}

Given two terms $x:A$ and $y:A$, we write $x \id_{A} y$, or simply $x \id y$, for the identity type, which is the type
of equalities or identifications between them. The identity type is generated by reflexivity $\refl_{x} : x \id_{A}
  x$, and the eliminator for the identity type is given by path induction or the $J$-rule (see~\cref{def:path-induction}
in the appendix). This construction can be iterated, giving the identity type between two terms of an identity type,
repeating ad infinitum. Using the iterated identity type for morphisms, each type is equipped with the structure of a
weak $\infty$-groupoid, where each morphism satisfies groupoid laws only upto a higher one.

\begin{toappendix}
  Given an arbitrary type (or groupoid) $A$, we list some laws that are provable using path induction.

  \begin{definition}[Path Induction]
    \label{def:path-induction}
    Given a type family $C : \dfun{x,y:A}{(x \id_A y)} \to \UU$, and a function $c : \dfun{x:A}{C(x,x,\refl_x)}$, there is
    a function $f : \dfun{x,y:A}{\dfun{p:x \id_A y}{C(x,y,p)}}$ such that $f(x,x,\refl_x) \defeq c(x)$.
  \end{definition}

  \begin{gather*}
    \begin{aligned}
      \term{\inv{\blank}}      & : (x \id_{A} y) \to (y \id_{A} x)                   \\
      \term{\blank\comp\blank} & : (x \id_{A} y) \to (y \id_{A} z) \to (x \id_{A} z)
    \end{aligned}
    \qquad
    \begin{aligned}
      \term{assoc} & : (p : x \id_{A} y)  (q : y \id_{A} z) (r : z \id_{A} w) \\
                   & \to (p \comp q) \comp r \id p \comp (q \comp r)          \\
      \term{invr}  & : (p : x \id_{A} y) \to p \comp \inv{p} \id \refl_{x}
    \end{aligned}
  \end{gather*}
\end{toappendix}

A homotopy between functions $f, g : A \to B$, written $f \htpy g$, is given by pointwise equality between them,
$\dfun*{x:A}{f(x) \id_{B} g(x)}$. The identity type for functions is equivalent to homotopies between them ${(f \id_{A
        \to B} g)} \eqv {(f \htpy g)}$, by $\term{funext}$ and $\term{happly}$. An equivalence between types $A \eqv B$ is given
by a pair of functions between them which compose to the identity, $f \comp g \htpy \idfunc_{B}$ and $g \comp f \htpy
  \idfunc_{A}$ (also see~\cref{prop:three-chars-equiv} in the appendix), and this is equivalent to the identity type for
the universe, $(A \id_{\UU} B) \eqv (A \eqv B)$, which is the univalence principle.

Functions between types are functors between groupoids. Given a function $f : A \to B$, there is a functorial action on
the paths given by $\term{ap}$. Type families, that is, types indexed by terms, are simply functions from a type to the
universe, such as $A \to \UU$, which is an $A$-indexed family of groupoids. For a type family $P : A \to \UU$ and a
point $x : A$, the type $P(x)$ is the fiber over $x$.  The $\term{transport}$ operation, named
$\term{transport}/\term{tr}$, lifts paths in the indexing type to functions between fibers.
\begin{gather*}
  \term{ap}_{f} : \dfun*{x,y:A}{x \id_{A} y \to f(x) \id_{B} f(y)}
  \qquad
  \term{transport}_{P} : \dfun*{x,y:A}{x \id_{A} y \to P(x) \to P(y)}
\end{gather*}

The type $\dsum*{x:A}{P(x)}$ is the collection of all the fibers and is the total space of $P$. The first projection
${\pi_1 : \dsum*{x:A}{P(x)} \to A}$ from the total space to the base space $A$ has the structure of a fibration, that
is, there is a lifting operation (\cref{fig:lift} in the appendix) which lifts paths in the base space to paths in the
total space.

\begin{toappendix}
  Given a path $p : x \id_{A} y$ in the base space, and $u : P(x)$ a point in the fiber over $x$, we have:
  \[
    \term{lift}(u,p) : (x , u) \id_{\dsum*{x:A}{P(x)}} (y , \transport{P}{p}{u})
  \]

  % where $\tr{p}{u}$ is shorthand for $$.

  \begin{figure}
    \begin{center}
      \begin{tikzpicture}[yscale=.5,xscale=2]
        \draw (0,0) arc (-90:170:8ex) node[anchor=south east] {$A$} arc (170:270:8ex);
        \draw (0,6) arc (-90:170:10ex) node[anchor=south east] {$\dsum{x:A}{P(x)}$} arc (170:270:10ex);
        \draw[->] (0,5.8) -- node[auto] {$\fst$} (0,3.2);
        \node[circle,fill,inner sep=1pt,label=left:{$x$}] (b1) at (-.5,1.4) {};
        \node[circle,fill,inner sep=1pt,label=right:{$y$}] (b2) at (.5,1.4) {};
        \draw[decorate,decoration={snake,amplitude=1}] (b1) -- node[auto,swap] {$p$} (b2);
        \node[circle,fill,inner sep=1pt,label=left:{$\pair{x,u}$}] (b1) at (-.5,7.2) {};
        \node[circle,fill,inner sep=1pt,label=right:{$\pair{y,\transport*{P}{p}{u}}$}] (b2) at (.5,7.2) {};
        \draw[decorate,decoration={snake,amplitude=1}] (b1) -- node[auto] {$\term{lift}(u,p)$} (b2);
      \end{tikzpicture}
    \end{center}
    \caption{Lifting operation in $P$}
    \label{fig:lift}
  \end{figure}
\end{toappendix}

\subsection{Univalent Fibrations}

Using the groupoid structure of $A$, for any $x, y : A$ and a path $p : x \id_{A} y$, $\term{transport}(p)$ and
$\term{transport}(\inv{p})$ form an equivalence, lifting paths in the base space to equivalences between fibers.
\[
  \tptEqv{P} : (x \id_{A} y) \to (P(x) \eqv P(y))
\]

\noindent The type families (or fibrations) we are interested in are the ones where paths in the base space completely
determine the equivalences in the fibers -- these are called univalent
fibrations~\cite*{kapulkinUnivalenceSimplicialSets2012,kapulkinSimplicialModelUnivalent2021,christensenCharacterizationUnivalentFibrations2015}.

\begin{definition}[Univalent Fibration]
  $P$ is a univalent type family (or, $\fst : {\dsum*{x:A}{P(x)}} \to A$ is a univalent fibration) if $\tptEqv{P}$ is an
  equivalence.
\end{definition}

Univalent fibrations were introduced by~\citet*{kapulkinUnivalenceSimplicialSets2012}, to build a model of Voevodsky's
\emph{univalence} principle in simplicial sets. Indeed, univalence characterises paths in the universe as equivalences
between types, which follows from the canonical fibration $\idfunc : \UU \to \UU$ being univalent.

\subsection{Homotopy Types}
\label{app:homotopytypes}

A type is \emph{contractible} (-2-type) if it has a unique element, that is, there is a center of contraction and every
other point is equal to it. A type is a \emph{proposition} (-1-type) if its equality types are contractible, that is, it
has at most one inhabitant. Iterating this, we can define sets or 0-types (whose equality types are propositions) and
1-groupoids or 1-types (whose equality types are sets), and similarly, higher homotopy $n$-types.
\begin{gather*}
  \begin{aligned}
    \isContr{A} & \defeq \dsum*{x:A}{\dfun*{y:A}{y \id x}} \\
    \isProp{A}  & \defeq \dfun*{x,y:A}{\isContr{x \id y}}
  \end{aligned}
  \qquad
  \begin{aligned}
    \isSet{A} & \defeq \dfun*{x,y:A}{\isProp{x \id y}} \\
    \isGpd{A} & \defeq \dfun*{x,y:A}{\isSet{x \id y}}
  \end{aligned}
\end{gather*}

\subsection{Higher Inductive Types}
\label{app:hits}

Higher Inductive Types generalise Inductive Types, by allowing path constructors besides point constructors. While point
constructors generate the elements of the type, path constructors generate equalities between points in the type. We
describe a few basic HITs that we use.

Given a type $A$, the propositional truncation $\Trunc[-1]{A}$, squashes the elements of $A$ turning it into a
proposition. It is given by a HIT with a point constructor $\trunc{\blank} : A \to \Trunc{A}$, and a path constructor
$\term{trunc}(x,y) : x \id_{\Trunc{A}} y$, which equates every pair of points in the truncation
(see~\cref{def:prop-trunc}).

\begin{toappendix}
  \begin{definition}[Propositional Truncation]
    \label{def:prop-trunc}
    Given a type $A$, the propositional truncation $\Trunc[-1]{A}$, or simply $\Trunc{A}$, is a higher inductive type
    generated by the following constructors,
    \begin{itemize}
      \item a function $\trunc{\blank} : A \to \Trunc{A}$,
      \item for each $x, y : \Trunc{A}$, a path $\term{trunc}(x,y) : x \id_{\Trunc{A}} y$,
    \end{itemize}
    such that, given any type $B$ with
    \begin{itemize}
      \item a function $g : A \to B$,
      \item for each $x, y : B$, a path $\term{trunc*}(x,y) : x \id_{B} y$,
    \end{itemize}
    there is a unique function $f : \Trunc{A} \to B$ such that,
    \begin{itemize}
      \item $f(\trunc{a}) \equiv g(a)$
      \item for each $x, y : \Trunc{A}$, $\ap{f}{\term{trunc}(x,y)} \id_{B} \term{trunc*}(f(x),f(y))$.
    \end{itemize}
  \end{definition}
\end{toappendix}

\begin{toappendix}
  \begin{definition}[${\fib_{f}} : B \to \UU$]
    \label{def:fib}
    The fiber of $f : A \to B$ at $b : B$ is
    \[
      \fib_{f}(b) \defeq \dsum*{a:A}{f(a) \id_{B} b}.
    \]
  \end{definition}

  \begin{definition}[${\term{im}} : (f : A \to B) \to \UU$]
    \label{def:im}
    The image of $f$ is the (-1)-truncation of its fiber.
    \[
      \im{f} \defeq \dsum*{b:B}{\Trunc[-1]{\fib_{f}(b)}}
    \]
  \end{definition}

  \begin{proposition}
    \label{prop:three-chars-equiv}
    The following are equivalent.
    \begin{enumerate}
      \item $f : A \to B$ is an equivalence.
      \item $f$ has a left and right inverse.
      \item $f$ has contractible fibers.
    \end{enumerate}
  \end{proposition}
\end{toappendix}

Another HIT that we use is the set-quotient $\quot{A}{R}$ which takes a set $A$ and a binary relation $R : {A \to A \to
  \UU}$. It has a mapping of points $\quotinc : A \to \quot{A}{R}$, and a constructor that adds paths between related
pairs of elements $\quotrel : R(x,y) \to \quotinc(x) \id_{\quot{A}{R}} \quotinc(y)$ (see~\cref{def:set-quot}).  We
recall that the quotient is \emph{effective} if $R$ is a prop-valued equivalence relation, that is, $R(x,y)$ holds iff
$(q(x) \id_{\quot{A}{R}} q(y))$.

\begin{toappendix}
  \begin{definition}[Set Quotient]
    \label{def:set-quot}
    Given a type $A$ which is a set, and a relation $R : A \to A \to \UU$, the set-quotient $\quot{A}{R}$ is the higher
    inductive type generated by
    \begin{itemize}
      \item an inclusion function $\quotinc : A \to \quot{A}{R}$,
      \item for each $x, y : A$ such that $R(x,y)$, a path $\quotinc(x) \id_{\quot{A}{R}} \quotinc(y)$,
      \item a set truncation, for each $x, y : \quot{A}{R}$ and $r, s : x \id_{\quot{A}{R}} y$, we have $r \id s$,
    \end{itemize}
    with an appropriate induction principle.
  \end{definition}
\end{toappendix}

\subsection{Univalent Subuniverses}

Starting from a univalent universe which classifies all types, we want to define a subuniverse which classifies only
certain types, for example, types that satisfy some desired property. We use a prop-valued type family, that is, a
predicate on the universe, which picks out only those types, and collect them into a univalent subuniverse. Being
univalent ensures that the equality type of the ambient universe is reflected in the subuniverse.

\begin{definition}[Universe]
  A universe \`{a} la Tarski is given by the following pieces of data,
  \begin{itemize}
    \item a code $U : \UU$,
    \item a decoding type family $\El : U \to \UU$.
  \end{itemize}
  If $\El$ is univalent, we call $(U,\El)$ a \emph{univalent} universe.
\end{definition}

\begin{propositionrep}[Univalent Subuniverse]
  \label{prop:univsub}
  A universe predicate is a type family $P : \UU \to \UU$ whose fibers are propositions, that is, $P(X)$ is a
  proposition for every $X$. Given such a predicate $P$, the fibration ${\fst : \dsum*{X:\UU}{P(X)} \to \UU}$ is
  univalent and generates a univalent subuniverse ${\UU_{P} \defeq (\dsum*{X:\UU}{P(X)}, \fst)}$.~\footnote{Univalent
    typoids~\cite{petrakisUnivalentTypoids2019} are a different presentation of univalent subuniverses.}
\end{propositionrep}

\begin{proof}
  Suppose $(U, \El) \defeq (\dsum*{X:\UU}{P(X)}, \fst)$ is a subuniverse generated by a subtype $P : \UU \to \UU$. For
  any $X, Y : \UU$ such that $\phi : P(X)$ and $\psi : P(Y)$, we want to show that $\tptEqv{\fst} : (X,\phi) \id
    (Y,\psi) \to X \eqv Y$ is an equivalence. We construct $X \eqv Y \to (X,\phi) \id (Y,\psi)$ by $\ua$ and using the
  fact that $P(\blank)$ is a proposition. That it is an inverse follows by calculation using the appropriate computation
  rules.
\end{proof}

The types we are interested in are the finite types. In constructive mathematics, the notion of finiteness is
subtle~\cite{spiwackConstructivelyFinite2010}. We use the notion of Bishop-finiteness: a type is finite if it is merely
equivalent to a finite set (\cref*{def:finite-set,def:isfin}).

\begin{definition}[$\Fin$]
  \label{def:finite-set}
  The type family $\Fin : \Nat \to \UU$ is the type of finite sets indexed by their cardinality. It is defined
  equivalently in two different ways,
  \begin{gather*}
    \Fin[n] \defeq \dsum*{k:\Nat}{k < n}
    \qquad\qquad \text{or} \qquad\qquad
    \begin{aligned}
       & \Fin[0] \defeq \bot                      \\
       & \Fin[\suc[n]] \defeq \top \sqcup \Fin[n]
    \end{aligned}
  \end{gather*}
  Note that $\Fin[n]$ is a set, and we use both definitions interchangeably.
\end{definition}

\begin{definition}[$\isFin$]
  \label{def:isfin}
  We say that a type is finite if it is merely equal to $\Fin[n]$ for some $n : \Nat$.
  \[
    \isFin[X] \defeq \dsum*{n:\Nat}{\SubP{X}{\Fin[n]}}
  \]
  Note that the natural number $n$ need not be truncated, as justified below.
\end{definition}

\begin{lemmarep}
  \label{prop:isFin}
  For any type $X$, $\isFin[X]$ is a proposition.
\end{lemmarep}

\begin{proof}
  Suppose we have $(n,\phi) : \isFin[X]$ and $(m,\psi) : \isFin[X]$, we need to show that $(n,\phi) \id (m,\psi)$. It is
  enough to show that $n \id m$. Since $\Nat$ is a set, this is a proposition, so we can use the induction principle of
  propositional truncation to eliminate to $n \id m$, applying it on $\phi$ and $\psi$ respectively. This gives us the
  equalities $X \id \Fin[n]$ and $X \id \Fin[m]$, which gives us $\Fin[n] \id \Fin[m]$, from which $n \id m$ follows by
  applying the first projection.
\end{proof}

Since $\isFin$ is a predicate on the universe $\UU$, we easily get our univalent subuniverse $\UFin$.

\begin{definition}
  \label{def:ufin}
  The univalent subuniverse of \emph{all finite types} is given by
  $
    \UFin \defeq \dsum*{X:\UU}{\isFin[X]}.
  $
  We write $F_{n} \defeq (\Fin[n], n, \trunc{\refl})$, for the image of the inclusion of $\Fin[n]$.
\end{definition}

This definition of the groupoid of finite types has also been considered
in~\cite{yorgeyCombinatorialSpeciesLabelled2014}. While $\UFin$ has \emph{all} the finite types, we are also interested
in constructing a subuniverse of finite types of a specified cardinality. To do so, we will start with the subuniverse
$\BAut[T]$, for any type $T : \UU$. \footnote{Characterisations of univalent fibrations using the $\BAut$ construction
  have been studied by~\citet{christensenCharacterizationUnivalentFibrations2015}.}

\begin{definition}[$\BAut$]
  The predicate $P(X) \defeq \Trunc[-1]{X \id T}$ picks out exactly those types that are merely equal to $T$, and this
  generates the subuniverse
  \[
    \BAut[T] \defeq \Sub{T}.
  \]
  We write $T_0 \defeq (T, \trunc{\refl_{T}})$ for the image of the inclusion of $T$ in $\BAut[T]$.
\end{definition}

Using $\BAut$, we can talk about types that are equivalent to a finite set of specified cardinality, for example, the
subuniverse of 2-element sets is given by $\BAut[\Bool]$. This has been used to construct the real projective spaces in
HoTT~\cite{buchholtzRealProjectiveSpaces2017}, and also to give the denotational semantics for a 1-bit reversible
programming language~\cite{caretteReversibleProgramsUnivalent2018}.

\begin{definition}[${\UFin[n]}$]
  For any $n : \Nat$, we define $\UFin[n] \defeq \BAut[\Fin[n]]$ to be the univalent subuniverse of $n$-element sets.
  Note that, $\UFin$ can be equivalently seen as the collection of all types of finite cardinality, that is,
  $\UFin \eqv \dsum*{n:\Nat}{\UFin[n]}$.
\end{definition}

Since $\BAut[T]$ is a univalent subuniverse, we can characterise its path space. The intuition is that $\BAut[T]$ only
has one point $T_0$, and 1-paths $T_0 \id T_0$, that is, loops, and higher paths between these loops. The
type of loops on $T_0$, $\loopspace[\BAut[T],T_{0}]$, is shown to be equivalent to $\Aut[T] \defeq T \eqv T$, which is
the group of automorphisms of $T$.

\begin{lemmarep}
  \leavevmode
  \begin{enumerate}
    \item If $T$ is an $n$-type, $\BAut[T]$ is an $(n+1)$-type.
    \item For any $T : \UU$, $\BAut[T]$ is 0-connected.
    \item For any $T : \UU$, \( \loopspace[\BAut[T],T_{0}] \eqv \Aut[T] \). \label{lem:loop-deloop}
  \end{enumerate}
\end{lemmarep}

\begin{proof}
  We need to show that the equality type of $\BAut[T]$ is an $n$-type. Assume $X, Y : \BAut[T]$. Since $\BAut[T]$ is a
  univalent subuniverse, we have $(X \id Y) \eqv (\fst(X) \eqv \fst(Y))$. Note that being an $n$-type is a proposition.
  Since $T$ is an $n$-type, and $\fst(X)$ and $\fst(Y)$ are merely equal to $T$, they're also $n$-types. It follows that
  $\fst(X) \eqv \fst(Y)$ is an $n$-type, and hence $X \id Y$ is an $n$-type.

  Since $\BAut[T]$ is a univalent universe, it follows that
  \[
    (T_{0} \id_{\BAut[T]} T_{0}) \eqv (\fst(T_{0}) \eqv \fst(T_{0})) \equiv (T \eqv T) \equiv \Aut[T].
  \]
\end{proof}

\begin{theorem}
  $\UFin[n]$ is a pointed, connected, 1-groupoid for every $n:\Nat$, and \( \loopspace[\UFin[n],F_{n}] \eqv
  \Aut[\Fin[n]] \). $\UFin$ is a 1-groupoid with connected components for every $n:\Nat$.
\end{theorem}

We have shown that loops in $\UFin$ exactly encode the automorphism group $\Aut[\Fin[n]]$ for every~$n$. This is a
general technique called \emph{delooping}, where a group can be identified with a 1-object groupoid, internally in HoTT.
This technique also allows defining higher groups~\cite{buchholtzHigherGroupsHomotopy2018}. The loopspace of a pointed
type automatically has the structure of a group, with $\refl_{\pt}$ for the neutral element, path composition for the
group multiplication, and path inverse for the group inverse operation. The group axioms are given by the higher paths
corresponding to groupoid laws.

% \[
%   \begin{tikzcd}
%     F_{0}
%     \arrow[""{anchor=center, inner sep=0}, no head, loop, distance=4em, in=115, out=65]
%     & F_{1}
%     \arrow[""{anchor=center, inner sep=0}, no head, loop, distance=4em, in=115, out=65]
%     & F_{2}
%     \arrow[""{anchor=center, inner sep=0}, no head, loop, distance=4em, in=115, out=65]
%     \arrow[""{anchor=center, inner sep=0}, no head, loop, distance=8em, in=125, out=55]
%     & \ldots
%     & F_{n}
%     \arrow[""{anchor=center, inner sep=0}, no head, loop, distance=4em, in=115, out=65]
%     \arrow[""{anchor=center, inner sep=0}, no head, loop, distance=8em, in=125, out=55]
%     \arrow[""{anchor=center, inner sep=0}, no head, loop, distance=12em, in=135, out=45]
%     & \ldots
%   \end{tikzcd}
% \]

\subsection{Rig structure}
\label{subsec:rig}

Similar to $\BFin$, the groupoid $\UFin$ has two symmetric monoidal structures, the additive and the multiplicative ones,
and the multiplicative tensor product distributes over the additive one. To construct these, we first state and prove
some equivalences on $\Fin$, and some general type isomorphisms. Then we simply lift these equivalences to $\UFin$, by
the univalence principle.

\begin{proposition}
  For any $n, m : \Nat$, and for any types $X, Y, Z$,
  \begin{gather*}
    \begin{aligned}
      \Fin[0]                & \eqv \bot                  \\
      \Fin[n] \sqcup \Fin[m] & \eqv \Fin[n + m]           \\
      \\
      \bot \sqcup X          & \eqv X                     \\
      X \sqcup \bot          & \eqv X                     \\
      (X \sqcup Y) \sqcup Z  & \eqv X \sqcup (Y \sqcup Z) \\
      X \sqcup Y             & \eqv Y \sqcup X            \\
      X \times \bot          & \eqv \bot
    \end{aligned}
    \qquad
    \begin{aligned}
      \Fin[1]                & \eqv \top                             \\
      \Fin[n] \times \Fin[m] & \eqv \Fin[n * m]                      \\
      \\
      \top \times X          & \eqv X                                \\
      X \times \top          & \eqv X                                \\
      (X \times Y) \times Z  & \eqv X \times (Y \times Z)            \\
      X \times Y             & \eqv Y \times X                       \\
      X \times (Y \sqcup Z)  & \eqv (X \times Y) \sqcup (X \times Z)
    \end{aligned}
  \end{gather*}
\end{proposition}

\begin{theorem}
  $\UFin$ has two symmetric monoidal structures, the additive and multiplicative ones, given by $(F_0, \sqcup)$ and
  $(F_1, \times)$, with corresponding natural isomorphisms $\lambda_{X}$, $\rho_{X}$, $\alpha_{X,Y,Z}$, and the braiding
  isomorphism $\mathcal{B}_{X,Y}$ upto 1-paths in $\UFin$. These isomorphisms satisfy the Mac Lane coherence conditions
  for symmetric monoidal categories~\cite{maclaneNaturalAssociativityCommutativity1963}, that is, the triangle,
  pentagon, and hexagon identities, and the symmetry of the braiding, upto 2-paths in $\UFin$. The multiplicative
  structure distributes over the additive structure and satisfies the Laplaza coherence conditions for rig
  categories~\cite{laplaza72}.
\end{theorem}

\begin{toappendix}
  \begin{definition}[Additive symmetric monoidal structure]
    \label{def:additive}
    \begin{align*}
      O                 & \defeq F_{0}                                       \\
      X \oplus Y        & \defeq X \sqcup Y                                  \\
      \lambda_{X}       & : O \oplus X \eqv X                                \\
      \rho_{X}          & : X \oplus O \eqv X                                \\
      \alpha_{X,Y,Z}    & : (X \oplus Y) \oplus Z \eqv X \oplus (Y \oplus Z) \\
      \mathcal{B}_{X,Y} & : X \oplus Y \eqv Y \oplus X
    \end{align*}
  \end{definition}
  \begin{proposition}
    \label{prop:additive}
    % https://q.uiver.app/?q=WzAsNCxbMCwwLCIoWCBcXG9wbHVzIEkpIFxcb3BsdXMgWSJdLFsyLDAsIlggXFxvcGx1cyAoSSBcXG9wbHVzIFkpIl0sWzEsMSwiWCBcXG9wbHVzIFkiXSxbMCwxXSxbMCwxLCJcXGFscGhhX3tYLEksWX0iXSxbMCwyLCJcXHJob197WH0gXFxvcGx1cyAxX3tZfSIsMl0sWzEsMiwiMV97WH0gXFxvcGx1cyBcXGxhbWJkYV97WX0iXSxbNSw2LCJcXGlkIiwwLHsic2hvcnRlbiI6eyJzb3VyY2UiOjIwLCJ0YXJnZXQiOjIwfSwic3R5bGUiOnsiYm9keSI6eyJuYW1lIjoibm9uZSJ9LCJoZWFkIjp7Im5hbWUiOiJub25lIn19fV1d
    \[\begin{tikzcd}
        {(X \oplus I) \oplus Y} && {X \oplus (I \oplus Y)} \\
        {} & {X \oplus Y}
        \arrow["{\alpha_{X,I,Y}}", from=1-1, to=1-3]
        \arrow[""{name=0, anchor=center, inner sep=0}, "{\rho_{X} \oplus 1_{Y}}"', from=1-1, to=2-2]
        \arrow[""{name=1, anchor=center, inner sep=0}, "{1_{X} \oplus \lambda_{Y}}", from=1-3, to=2-2]
        \arrow["\id", Rightarrow, draw=none, from=0, to=1]
      \end{tikzcd}\]
    % https://q.uiver.app/?q=WzAsNSxbMCwxLCIoKFcgXFxvcGx1cyBYKSBcXG9wbHVzIFkpIFxcb3BsdXMgWiJdLFsxLDAsIihXIFxcb3BsdXMgWCkgXFxvcGx1cyAoWSBcXG9wbHVzIFopIl0sWzIsMSwiVyBcXG9wbHVzIChYIFxcb3BsdXMgKFkgXFxvcGx1cyBaKSkiXSxbMiwzLCJXIFxcb3BsdXMgKChYIFxcb3BsdXMgWSkgXFxvcGx1cyBaKSJdLFswLDMsIihXIFxcb3BsdXMgKFggXFxvcGx1cyBZKSkgXFxvcGx1cyBaIl0sWzAsMSwiXFxhbHBoYV97VyBcXG9wbHVzIFgsIFksIFp9Il0sWzEsMiwiXFxhbHBoYV97VyxYLFkgXFxvcGx1cyBafSJdLFszLDIsIjFfe1d9IFxcb3BsdXMgXFxhbHBoYV97WCxZLFp9IiwyXSxbMCw0LCJcXGFscGhhX3tXLFgsWX0gXFxvcGx1cyAxX3tafSIsMl0sWzQsMywiXFxhbHBoYV97VyxYIFxcb3BsdXMgWSxafSIsMl0sWzAsMiwiXFxpZCIsMSx7Im9mZnNldCI6NSwic3R5bGUiOnsiYm9keSI6eyJuYW1lIjoibm9uZSJ9LCJoZWFkIjp7Im5hbWUiOiJub25lIn19fV1d
    \[\begin{tikzcd}
        & {(W \oplus X) \oplus (Y \oplus Z)} \\
        {((W \oplus X) \oplus Y) \oplus Z} && {W \oplus (X \oplus (Y \oplus Z))} \\
        \\
        {(W \oplus (X \oplus Y)) \oplus Z} && {W \oplus ((X \oplus Y) \oplus Z)}
        \arrow["{\alpha_{W \oplus X, Y, Z}}", from=2-1, to=1-2]
        \arrow["{\alpha_{W,X,Y \oplus Z}}", from=1-2, to=2-3]
        \arrow["{1_{W} \oplus \alpha_{X,Y,Z}}"', from=4-3, to=2-3]
        \arrow["{\alpha_{W,X,Y} \oplus 1_{Z}}"', from=2-1, to=4-1]
        \arrow["{\alpha_{W,X \oplus Y,Z}}"', from=4-1, to=4-3]
        \arrow["\id"{description}, shift right=5, draw=none, from=2-1, to=2-3]
      \end{tikzcd}\]
    % https://q.uiver.app/?q=WzAsNixbMSwwLCJYIFxcb3BsdXMgKFkgXFxvcGx1cyBaKSJdLFswLDEsIihYIFxcb3BsdXMgWSkgXFxvcGx1cyBaIl0sWzAsMiwiKFkgXFxvcGx1cyBYKSBcXG9wbHVzIFoiXSxbMSwzLCJZIFxcb3BsdXMgKFggXFxvcGx1cyBaKSJdLFsyLDIsIlkgXFxvcGx1cyAoWiBcXG9wbHVzIFgpIl0sWzIsMSwiKFkgXFxvcGx1cyBaKSBcXG9wbHVzIFgiXSxbMSwwLCJcXGFscGhhX3tYLFksWn0iXSxbMSwyLCJcXG1hdGhjYWx7Qn1fe1gsWX0gXFxvcGx1cyAxX3tafSIsMl0sWzIsMywiXFxhbHBoYV97WSxYLFp9IiwyXSxbMyw0LCIxX3tZfSBcXG9wbHVzIFxcbWF0aGNhbHtCfV97WCxafSIsMl0sWzUsNCwiXFxhbHBoYV97WSxaLFh9Il0sWzAsNSwiXFxtYXRoY2Fse0J9X3tYLFkgXFxvcGx1cyBafSJdLFs3LDEwLCJcXGlkIiwwLHsic2hvcnRlbiI6eyJzb3VyY2UiOjIwLCJ0YXJnZXQiOjIwfSwic3R5bGUiOnsiYm9keSI6eyJuYW1lIjoibm9uZSJ9LCJoZWFkIjp7Im5hbWUiOiJub25lIn19fV1d
    \[\begin{tikzcd}
        & {X \oplus (Y \oplus Z)} \\
        {(X \oplus Y) \oplus Z} && {(Y \oplus Z) \oplus X} \\
        {(Y \oplus X) \oplus Z} && {Y \oplus (Z \oplus X)} \\
        & {Y \oplus (X \oplus Z)}
        \arrow["{\alpha_{X,Y,Z}}", from=2-1, to=1-2]
        \arrow[""{name=0, anchor=center, inner sep=0}, "{\mathcal{B}_{X,Y} \oplus 1_{Z}}"', from=2-1, to=3-1]
        \arrow["{\alpha_{Y,X,Z}}"', from=3-1, to=4-2]
        \arrow["{1_{Y} \oplus \mathcal{B}_{X,Z}}"', from=4-2, to=3-3]
        \arrow[""{name=1, anchor=center, inner sep=0}, "{\alpha_{Y,Z,X}}", from=2-3, to=3-3]
        \arrow["{\mathcal{B}_{X,Y \oplus Z}}", from=1-2, to=2-3]
        \arrow["\id", Rightarrow, draw=none, from=0, to=1]
      \end{tikzcd}\]
    % https://q.uiver.app/?q=WzAsMyxbMCwwLCJYIFxcb3BsdXMgWSJdLFsyLDAsIlggXFxvcGx1cyBZIl0sWzEsMSwiWSBcXG9wbHVzIFgiXSxbMCwxLCIxX3tYIFxcb3BsdXMgWX0iLDAseyJsZXZlbCI6Miwic3R5bGUiOnsiaGVhZCI6eyJuYW1lIjoibm9uZSJ9fX1dLFswLDIsIlxcbWF0aGNhbHtCfV97WCxZfSIsMl0sWzIsMSwiXFxtYXRoY2Fse0J9X3tZLFh9IiwyXSxbNCw1LCJcXGlkIiwwLHsic2hvcnRlbiI6eyJzb3VyY2UiOjIwLCJ0YXJnZXQiOjIwfSwic3R5bGUiOnsiYm9keSI6eyJuYW1lIjoibm9uZSJ9LCJoZWFkIjp7Im5hbWUiOiJub25lIn19fV1d
    \[\begin{tikzcd}
        {X \oplus Y} && {X \oplus Y} \\
        & {Y \oplus X}
        \arrow["{1_{X \oplus Y}}", Rightarrow, no head, from=1-1, to=1-3]
        \arrow[""{name=0, anchor=center, inner sep=0}, "{\mathcal{B}_{X,Y}}"', from=1-1, to=2-2]
        \arrow[""{name=1, anchor=center, inner sep=0}, "{\mathcal{B}_{Y,X}}"', from=2-2, to=1-3]
        \arrow["\id", Rightarrow, draw=none, from=0, to=1]
      \end{tikzcd}\]
  \end{proposition}
\end{toappendix}

\begin{toappendix}
  \begin{definition}[Multiplicative symmetric monoidal structure]
    \label{def:multiplicative}
    \begin{align*}
      I                 & \defeq F_{1}                                           \\
      X \otimes Y       & \defeq X \times Y                                      \\
      \lambda_{X}       & : I \times X \eqv X                                    \\
      \rho_{X}          & : X \times I \eqv X                                    \\
      \alpha_{X,Y,Z}    & : (X \otimes Y) \otimes Z \eqv X \otimes (Y \otimes Z) \\
      \mathcal{B}_{X,Y} & : X \otimes Y \eqv Y \otimes X
    \end{align*}
  \end{definition}
  \begin{proposition}
    \label{prop:multiplicative}
    % https://q.uiver.app/?q=WzAsNCxbMCwwLCIoWCBcXG9wbHVzIEkpIFxcb3BsdXMgWSJdLFsyLDAsIlggXFxvcGx1cyAoSSBcXG9wbHVzIFkpIl0sWzEsMSwiWCBcXG9wbHVzIFkiXSxbMCwxXSxbMCwxLCJcXGFscGhhX3tYLEksWX0iXSxbMCwyLCJcXHJob197WH0gXFxvcGx1cyAxX3tZfSIsMl0sWzEsMiwiMV97WH0gXFxvcGx1cyBcXGxhbWJkYV97WX0iXSxbNSw2LCJcXGlkIiwwLHsic2hvcnRlbiI6eyJzb3VyY2UiOjIwLCJ0YXJnZXQiOjIwfSwic3R5bGUiOnsiYm9keSI6eyJuYW1lIjoibm9uZSJ9LCJoZWFkIjp7Im5hbWUiOiJub25lIn19fV1d
    \[\begin{tikzcd}
        {(X \otimes I) \otimes Y} && {X \otimes (I \otimes Y)} \\
        {} & {X \otimes Y}
        \arrow["{\alpha_{X,I,Y}}", from=1-1, to=1-3]
        \arrow[""{name=0, anchor=center, inner sep=0}, "{\rho_{X} \otimes 1_{Y}}"', from=1-1, to=2-2]
        \arrow[""{name=1, anchor=center, inner sep=0}, "{1_{X} \otimes \lambda_{Y}}", from=1-3, to=2-2]
        \arrow["\id", Rightarrow, draw=none, from=0, to=1]
      \end{tikzcd}\]
    % https://q.uiver.app/?q=WzAsNSxbMCwxLCIoKFcgXFxvcGx1cyBYKSBcXG9wbHVzIFkpIFxcb3BsdXMgWiJdLFsxLDAsIihXIFxcb3BsdXMgWCkgXFxvcGx1cyAoWSBcXG9wbHVzIFopIl0sWzIsMSwiVyBcXG9wbHVzIChYIFxcb3BsdXMgKFkgXFxvcGx1cyBaKSkiXSxbMiwzLCJXIFxcb3BsdXMgKChYIFxcb3BsdXMgWSkgXFxvcGx1cyBaKSJdLFswLDMsIihXIFxcb3BsdXMgKFggXFxvcGx1cyBZKSkgXFxvcGx1cyBaIl0sWzAsMSwiXFxhbHBoYV97VyBcXG9wbHVzIFgsIFksIFp9Il0sWzEsMiwiXFxhbHBoYV97VyxYLFkgXFxvcGx1cyBafSJdLFszLDIsIjFfe1d9IFxcb3BsdXMgXFxhbHBoYV97WCxZLFp9IiwyXSxbMCw0LCJcXGFscGhhX3tXLFgsWX0gXFxvcGx1cyAxX3tafSIsMl0sWzQsMywiXFxhbHBoYV97VyxYIFxcb3BsdXMgWSxafSIsMl0sWzAsMiwiXFxpZCIsMSx7Im9mZnNldCI6NSwic3R5bGUiOnsiYm9keSI6eyJuYW1lIjoibm9uZSJ9LCJoZWFkIjp7Im5hbWUiOiJub25lIn19fV1d
    \[\begin{tikzcd}
        & {(W \otimes X) \otimes (Y \otimes Z)} \\
        {((W \otimes X) \otimes Y) \otimes Z} && {W \otimes (X \otimes (Y \otimes Z))} \\
        \\
        {(W \otimes (X \otimes Y)) \otimes Z} && {W \otimes ((X \otimes Y) \otimes Z)}
        \arrow["{\alpha_{W \otimes X, Y, Z}}", from=2-1, to=1-2]
        \arrow["{\alpha_{W,X,Y \otimes Z}}", from=1-2, to=2-3]
        \arrow["{1_{W} \otimes \alpha_{X,Y,Z}}"', from=4-3, to=2-3]
        \arrow["{\alpha_{W,X,Y} \otimes 1_{Z}}"', from=2-1, to=4-1]
        \arrow["{\alpha_{W,X \otimes Y,Z}}"', from=4-1, to=4-3]
        \arrow["\id"{description}, shift right=5, draw=none, from=2-1, to=2-3]
      \end{tikzcd}\]
    % https://q.uiver.app/?q=WzAsNixbMSwwLCJYIFxcb3BsdXMgKFkgXFxvcGx1cyBaKSJdLFswLDEsIihYIFxcb3BsdXMgWSkgXFxvcGx1cyBaIl0sWzAsMiwiKFkgXFxvcGx1cyBYKSBcXG9wbHVzIFoiXSxbMSwzLCJZIFxcb3BsdXMgKFggXFxvcGx1cyBaKSJdLFsyLDIsIlkgXFxvcGx1cyAoWiBcXG9wbHVzIFgpIl0sWzIsMSwiKFkgXFxvcGx1cyBaKSBcXG9wbHVzIFgiXSxbMSwwLCJcXGFscGhhX3tYLFksWn0iXSxbMSwyLCJcXG1hdGhjYWx7Qn1fe1gsWX0gXFxvcGx1cyAxX3tafSIsMl0sWzIsMywiXFxhbHBoYV97WSxYLFp9IiwyXSxbMyw0LCIxX3tZfSBcXG9wbHVzIFxcbWF0aGNhbHtCfV97WCxafSIsMl0sWzUsNCwiXFxhbHBoYV97WSxaLFh9Il0sWzAsNSwiXFxtYXRoY2Fse0J9X3tYLFkgXFxvcGx1cyBafSJdLFs3LDEwLCJcXGlkIiwwLHsic2hvcnRlbiI6eyJzb3VyY2UiOjIwLCJ0YXJnZXQiOjIwfSwic3R5bGUiOnsiYm9keSI6eyJuYW1lIjoibm9uZSJ9LCJoZWFkIjp7Im5hbWUiOiJub25lIn19fV1d
    \[\begin{tikzcd}
        & {X \otimes (Y \otimes Z)} \\
        {(X \otimes Y) \otimes Z} && {(Y \otimes Z) \otimes X} \\
        {(Y \otimes X) \otimes Z} && {Y \otimes (Z \otimes X)} \\
        & {Y \otimes (X \otimes Z)}
        \arrow["{\alpha_{X,Y,Z}}", from=2-1, to=1-2]
        \arrow[""{name=0, anchor=center, inner sep=0}, "{\mathcal{B}_{X,Y} \otimes 1_{Z}}"', from=2-1, to=3-1]
        \arrow["{\alpha_{Y,X,Z}}"', from=3-1, to=4-2]
        \arrow["{1_{Y} \otimes \mathcal{B}_{X,Z}}"', from=4-2, to=3-3]
        \arrow[""{name=1, anchor=center, inner sep=0}, "{\alpha_{Y,Z,X}}", from=2-3, to=3-3]
        \arrow["{\mathcal{B}_{X,Y \otimes Z}}", from=1-2, to=2-3]
        \arrow["\id", Rightarrow, draw=none, from=0, to=1]
      \end{tikzcd}\]
    % https://q.uiver.app/?q=WzAsMyxbMCwwLCJYIFxcb3BsdXMgWSJdLFsyLDAsIlggXFxvcGx1cyBZIl0sWzEsMSwiWSBcXG9wbHVzIFgiXSxbMCwxLCIxX3tYIFxcb3BsdXMgWX0iLDAseyJsZXZlbCI6Miwic3R5bGUiOnsiaGVhZCI6eyJuYW1lIjoibm9uZSJ9fX1dLFswLDIsIlxcbWF0aGNhbHtCfV97WCxZfSIsMl0sWzIsMSwiXFxtYXRoY2Fse0J9X3tZLFh9IiwyXSxbNCw1LCJcXGlkIiwwLHsic2hvcnRlbiI6eyJzb3VyY2UiOjIwLCJ0YXJnZXQiOjIwfSwic3R5bGUiOnsiYm9keSI6eyJuYW1lIjoibm9uZSJ9LCJoZWFkIjp7Im5hbWUiOiJub25lIn19fV1d
    \[\begin{tikzcd}
        {X \otimes Y} && {X \otimes Y} \\
        & {Y \otimes X}
        \arrow["{1_{X \otimes Y}}", Rightarrow, no head, from=1-1, to=1-3]
        \arrow[""{name=0, anchor=center, inner sep=0}, "{\mathcal{B}_{X,Y}}"', from=1-1, to=2-2]
        \arrow[""{name=1, anchor=center, inner sep=0}, "{\mathcal{B}_{Y,X}}"', from=2-2, to=1-3]
        \arrow["\id", Rightarrow, draw=none, from=0, to=1]
      \end{tikzcd}\]
  \end{proposition}
\end{toappendix}

\begin{toappendix}
  \begin{proposition}[Distributivity]
    \label{prop:distributivity}
    \begin{gather*}
      \begin{aligned}
        \delta_{l} : X \otimes (Y \oplus Z) & \eqv (X \otimes Y) \oplus (X \otimes Z) \\
        \delta_{r} : (X \oplus Y) \otimes Z & \eqv (X \otimes Z) \oplus (Y \otimes Z)
      \end{aligned}
      \qquad
      \begin{aligned}
        a_{l} : X \otimes O \eqv O \\
        a_{r} : O \otimes X \eqv O
      \end{aligned}
    \end{gather*}
  \end{proposition}
\end{toappendix}

%%% Local Variables:
%%% mode: latex
%%% TeX-master: "main"
%%% fill-column: 120
%%% End:

\section{A Reversible Programming Language: Groupoid Semantics}
\label{sec:reversibletwo}
\label{langeqeq}

In the previous section, we examined equivalences between conventional data structures, i.e., structured trees of
values. We now consider a richer but foundational notion of data: programs themselves. Indeed, universal computation
models crucially rely on the fact that \emph{programs are (or can be encoded as) data}, e.g., a Turing machine can be
encoded as a string that another Turing machine (or even the same machine) can manipulate. Similarly, first-class
functions are the \emph{only} values in the $\lambda$-calculus.  In our setting, we ask whether the programs developed
in the previous section can themselves be subject to (higher-level) equivalences?

We will explain the ideas using two small exaamples. Consider the following two programs mapping between the types
$A + B$ and $C+D$:

\begin{center}
\begin{tikzpicture}[scale=0.7,every node/.style={scale=0.8}]
  \draw[>=latex,<->,double,red,thick] (2.25,-1.2) -- (2.25,-2.9) ;
  \draw[fill] (-2,-1.5) circle [radius=0.025];
  \node[below] at (-2.1,-1.5) {$A$};
  \node[below] at (-2.1,-1.9) {$+$};
  \draw[fill] (-2,-2.5) circle [radius=0.025];
  \node[below] at (-2.1,-2.5) {$B$};

  \draw[fill] (6.5,-1.5) circle [radius=0.025];
  \node[below] at (6.7,-1.5) {$C$};
  \node[below] at (6.7,-1.9) {$+$};
  \draw[fill] (6.5,-2.5) circle [radius=0.025];
  \node[below] at (6.7,-2.5) {$D$};

  \draw[<-] (-2,-1.5) to[bend left] (1,0.5) ;
  \draw[<-] (-2,-2.5) to[bend left] (1,-0.5) ;
  \draw[->] (3.5,0.5) to[bend left] (6.5,-1.45) ;
  \draw[->] (3.5,-0.5) to[bend left] (6.5,-2.45) ;

  \draw[<-] (-2,-1.5) to[bend right] (1,-3.5) ;
  \draw[<-] (-2,-2.5) to[bend right] (1,-4.5) ;
  \draw[->] (3.5,-3.5) to[bend right] (6.5,-1.55) ;
  \draw[->] (3.5,-4.5) to[bend right] (6.5,-2.55) ;


  \draw     (2,0.5)  -- (2.5,0.5)  ;
  \draw     (2,-0.5) -- (2.5,-0.5) ;

  \draw     (2.5,0.5)  -- (3.5,-0.5)  ;
  \draw     (2.5,-0.5) -- (3.5,0.5) ;

  \draw     (1,-3.5)  -- (2,-4.5)    ;
  \draw     (1,-4.5) -- (2,-3.5)   ;

  \draw     (2,-3.5)  -- (2.5,-3.5)    ;
  \draw     (2,-4.5) -- (2.5,-4.5)   ;

  \path (1.5,0.5) node (tc1) [func] {$c_1$};
  \path (1.5,-0.5) node (tc2) [func] {$c_2$};
  \path (3,-4.5) node (bc1) [func] {$c_1$};
  \path (3,-3.5) node (bc2) [func] {$c_2$};
\end{tikzpicture}
\end{center}
The top path is the $\Pi$ program $(c_1~\oplus~c_2)~\odot~\swapp$ which acts on the type $A$ by $c_1$, acts on the type
$B$ by $c_2$, and acts on the resulting value by a twist that exchanges the two injections into the sum type. The bottom
path performs the twist first and then acts on the type $A$ by $c_1$ and on the type $B$ by $c_2$ as before. One could
imagine the paths are physical \emph{elastic} wires in $3$ space, where the programs $c_1$ and $c_2$ as arbitrary
deformations on these wires, and the twists do not touch but are in fact well-separated. Then, holding the points $A$,
$B$, $C$, and $D$ fixed, it is possible to imagine sliding $c_1$ and $c_2$ from the top wire rightward past the twist,
and then using the elasticity of the wires, pull the twist back to line up with that of the bottom --- thus making both
parts of the diagram identical.  Each of these moves can be undone (reversed), and doing so would take the bottom part
of the diagram into the top part.  In other words, there exists an equivalence of the program
$(c_1~\oplus~c_2)~\odot~\swapp$ to the program $\swapp \odot (c_2~\oplus~c_1)$. We can also show that this means that,
as permutations, $(c_1~\oplus~c_2)~\odot~\swapp$ and $\swapp \odot (c_2~\oplus~c_1)$ are equal. And, of course, not all
programs between the same types can be deformed into one another. The simplest example of inequivalent deformations are
the two automorphisms of $1+1$, namely $\idc$ and $\swapp$.

As another example, consider consider a circuit that takes an input type consisting of three values \Tree [ {\small a} [
{\small b} {\small c} ] ]~ and swaps the leftmost value with the rightmost value to produce \Tree [ {\small c} [ {\small
  b} {\small a} ] ]~.  We can implement two such circuits using our Agda library for $\Pi$:

\begin{code}%
\>[0]\AgdaFunction{swap{-}fl1}\AgdaSpace{}%
\AgdaFunction{swap{-}fl2}\AgdaSpace{}%
\AgdaSymbol{:}\AgdaSpace{}%
\AgdaSymbol{\{}\AgdaBound{a}\AgdaSpace{}%
\AgdaBound{b}\AgdaSpace{}%
\AgdaBound{c}\AgdaSpace{}%
\AgdaSymbol{:}\AgdaSpace{}%
\AgdaDatatype{U}\AgdaSymbol{\}}\AgdaSpace{}%
\AgdaSymbol{→}\AgdaSpace{}%
\AgdaInductiveConstructor{PLUS}\AgdaSpace{}%
\AgdaBound{a}\AgdaSpace{}%
\AgdaSymbol{(}\AgdaInductiveConstructor{PLUS}\AgdaSpace{}%
\AgdaBound{b}\AgdaSpace{}%
\AgdaBound{c}\AgdaSymbol{)}\AgdaSpace{}%
\AgdaDatatype{⟷}\AgdaSpace{}%
\AgdaInductiveConstructor{PLUS}\AgdaSpace{}%
\AgdaBound{c}\AgdaSpace{}%
\AgdaSymbol{(}\AgdaInductiveConstructor{PLUS}\AgdaSpace{}%
\AgdaBound{b}\AgdaSpace{}%
\AgdaBound{a}\AgdaSymbol{)}\<%
\\
\>[0]\AgdaFunction{swap{-}fl1}\AgdaSpace{}%
\AgdaSymbol{=}\AgdaSpace{}%
\AgdaInductiveConstructor{assocl₊}\AgdaSpace{}%
\AgdaInductiveConstructor{◎}\AgdaSpace{}%
\AgdaInductiveConstructor{swap₊}\AgdaSpace{}%
\AgdaInductiveConstructor{◎}\AgdaSpace{}%
\AgdaSymbol{(}\AgdaInductiveConstructor{id⟷}\AgdaSpace{}%
\AgdaInductiveConstructor{⊕}\AgdaSpace{}%
\AgdaInductiveConstructor{swap₊}\AgdaSymbol{)}\<%
\\
%
\\[\AgdaEmptyExtraSkip]%
\>[0]\AgdaFunction{swap{-}fl2}\AgdaSpace{}%
\AgdaSymbol{=}%
\>[52I]\AgdaSymbol{(}\AgdaInductiveConstructor{id⟷}\AgdaSpace{}%
\AgdaInductiveConstructor{⊕}\AgdaSpace{}%
\AgdaInductiveConstructor{swap₊}\AgdaSymbol{)}\AgdaSpace{}%
\AgdaInductiveConstructor{◎}\<%
\\
\>[.]\<[52I]%
\>[11]\AgdaInductiveConstructor{assocl₊}\AgdaSpace{}%
\AgdaInductiveConstructor{◎}\<%
\\
%
\>[11]\AgdaSymbol{(}\AgdaInductiveConstructor{swap₊}\AgdaSpace{}%
\AgdaInductiveConstructor{⊕}\AgdaSpace{}%
\AgdaInductiveConstructor{id⟷}\AgdaSymbol{)}\AgdaSpace{}%
\AgdaInductiveConstructor{◎}\<%
\\
%
\>[11]\AgdaInductiveConstructor{assocr₊}\AgdaSpace{}%
\AgdaInductiveConstructor{◎}\<%
\\
%
\>[11]\AgdaSymbol{(}\AgdaInductiveConstructor{id⟷}\AgdaSpace{}%
\AgdaInductiveConstructor{⊕}\AgdaSpace{}%
\AgdaInductiveConstructor{swap₊}\AgdaSymbol{)}\<%
\end{code}

\noindent The first implementation rewrites the incoming values as follows:
\[
\Tree [ {\small a} [ {\small b} {\small c} ] ] ~\to~
\Tree [ [ {\small a} {\small b} ] {\small c} ] ~\to~
\Tree [ {\small c} [ {\small a} {\small b} ] ] ~\to~
\Tree [ {\small c} [ {\small b} {\small a} ] ] ~.
\]
\noindent
The second implementation rewrites the incoming values as follows:
\[
\Tree [ {\small a} [ {\small b} {\small c} ] ] ~\to~
\Tree [ {\small a} [ {\small c} {\small b} ] ] ~\to~
\Tree [ [ {\small a} {\small c} ] {\small b} ] ~\to~
\Tree [ [ {\small c} {\small a} ] {\small b} ] ~\to~
\Tree [ {\small c} [ {\small a} {\small b} ] ] ~\to~
\Tree [ {\small c} [ {\small b} {\small a} ] ] ~.
\]
\noindent The two circuits are extensionally equal. Using the level-2
isomorphisms we can \emph{explicitly} construct a sequence of
rewriting steps that transforms the second circuit to the first.

Recalling that the $\lambda$-calculus arises as the internal language of Cartesian Closed Categories
(Elliott~\cite{Elliott-2017} gives a particularly readable account of this), we can think of $\Pi$ in similar terms, but
for symmetric Rig Groupoids instead. For example, we can ask what does the equivalence above represent? It is actually a
``linear'' representation of a 2-categorial commutative diagram! In fact, it is a painfully verbose version thereof, as
it includes many \emph{refocusing} steps because our language does not build associativity into its syntax. Categorical
diagrams usually do.  Thus if we rewrite the example in diagrammatic form, eliding all uses of associativity, but
keeping explicit uses of identity transformations, we get that \AgdaFunction{swap{-}fl2⇔swap{-}fl1} represents

\vspace*{3mm}
\begin{tikzcd}[column sep=normal, row sep=normal]
 && (a+c)+b \arrow [r, "\swapp \oplus\idd", ""{name=U, below}] & (c+a)+b \arrow [dr, "\assocrp"] && \\
 & a+(c+b) \arrow [ur, "\assoclp"] & & & c+(a+b) \arrow [dr, "\idd\oplus\swapp"] &  \\
a+(b+c) \arrow [ur, "\idd\oplus\swapp"] \arrow [r, "\assoclp"]
  \arrow [dr, "\assoclp"]
  \arrow [ddr, swap, "\assoclp"]
    & (a+b)+c \arrow [r, "\swapp"] &
    c+(a+b) \arrow [r, swap, "\assoclp", ""{name=D, above}]
    & |[alias=Z]| (c+a)+b \arrow [r, "\assocrp"] &c+(a+b) \arrow [r, "\idd\oplus\swapp"] & c+(b+a) \\
 & (a+b)+c \arrow [dr, "\swapp"] &&&& \\
 & (a+b)+c \arrow [dr, swap, "\swapp"] & c+(a+b) \arrow [rr, swap, "\idd", ""{name=DD, above}]
             \arrow [d, Rightarrow, "\idf\, \mathit{idl}\odot{l}"] &&
    c+(a+b) \arrow [ruu, "\idd\oplus\swapp"] & \\
 && c+(a+b) \arrow [rrruuu, bend right = 40, swap, "\idd\oplus\swapp"] && \\
 \arrow[Rightarrow, from=U, to=D, "\mathit{hexagon}\oplus{r}\, \boxdot\, \idf"]
 \arrow[Rightarrow, from=Z, to=DD, swap, "\idf\boxdot\mathit{linv}\odot{l}\,\boxdot\,\idf"]
\end{tikzcd}


%%%%%%%%%
\subsection{Denotational Semantics}

An alternative point of view for the semantics in Sec.~\ref{reversibleone} is to consider it  as expressed in the
category of finite sets and functions, $\SetFin$, which is the category freely generated by finite coproduct completion
of the terminal category. Objects of $\SetFin$ can be identified with sets of fixed cardinality, that is,
$\Fin[n] \defeq \Set{0,1,\ldots,n-1}$. $\SetFin$ has finite coproducts and products, which lets us interpret the types
of $\PiLang$. Combinators are intepreted as morphisms in $\SetFin$, but we have to restrict to invertible morphisms,
that is, isomorphisms. This gives the \emph{groupoid} of finite sets and bijections,
$\BFin \defeq \mathsf{core}(\SetFin)$. The isomorphisms satisfied by coproducts and products in $\SetFin$ lift to
$\BFin$, but they're no longer categorical coproducts and products. They give two symmetric monoidal tensor products on
$\BFin$, the additive and multiplicative ones, with the multiplicative tensor distributing over the additive tensor.

We choose a canonical set of size $n$, called $\mathsf{Fin}~n$, whose elements are natural numbers less than $n$. To
compute the denotation of a type $A$, we first calculate its size $n = \sizet{A}$. We then construct the canonical set
$\mathsf{Fin}~n$ and provide the (trivial) evidence that this set is identical to $(\mathsf{Fin}~n)$:

\[\begin{array}{rcll}
\sem{A} &=& (\mathsf{Fin}~n, [ n , \mathsf{refl} ]) & \mbox{where}~\sizet{A} = n
\end{array}\]

\noindent The denotation $\sem{c}$ of a combinator $c : A \isot B$ is a path between $\sem{A}$ and $\sem{B}$. If the
size of $A$ is $m$ and the size of $B$ is $n$, the desired path is between $(\mathsf{Fin}~m, [ m , \mathsf{refl} ])$ and
$(\mathsf{Fin}~n, [ n , \mathsf{refl} ])$. This path is directly constructed using $\mathit{ap}$ and the fact that $m=n$
since combinators are always between types of the same size.

\noindent Finally, given two combinators $p , q : A \isot_1 B$ and a 2-combinator $\alpha : p \iso_2 q$, the denotation
$\sem{\alpha}$ of $\alpha$ is a path between $\sem{p}$ and $\sem{q}$.

%%% Local Variables:
%%% mode: latex
%%% TeX-master: "main"
%%% fill-column: 120
%%% End:

\section{The groupoid of finite types}~\label{sec:finite}

In this section, we describe \review{the algebraic structure} of the groupoid of
finite types, and give \review{a computable presentation} for it.

\vc{The groupoid of finite types is the free symmetric monoidal groupoid on one
  generator. This can be presented as an algebraic 2-theory, which is our syntax
  for $\PiHatLang$. Vertical categorification of natural numbers as a free
  commutative monoid. See groupoidification.}

\todo{Check Brent Yorgey's thesis?}

To do so, we will characterise the automorphisms on finite sets of cardinality
$n$, and show them to be equivalent to the symmetric group $\Sn$, via the
Coxeter presentation. We will do that in two steps,
in~\cref*{subsec:permutations,subsec:lehmer,subsec:symmetric}.

%% Coxeter presentation of $\Sn$ $\eqv$ $\Lehmer[n]$ $\eqv$ $\Aut[\Fin[n]]$

\subsection{Permutations}~\label{subsec:permutations}

In the previous~\cref{sec:univalent}, we established that paths in $\UFin$ are
equivalent to families of automorphisms of $\Fin{n}$ for every $n:\Nat$, that
is, bijections on finite sets of size $n$. This is the extensional view of
permutations. In the following sections, we will characterise these
permutations, going through two intermediate steps.

\vc{This is obvious, maybe add something more here.}

\subsection{Lehmer codes}~\label{subsec:lehmer}

From grade school combinatorics, we know that there are $\fac{n}$ permutations
on a finite set with $n$ elements. The factorial function is defined by
recursion on natural numbers. However, now, for every $n$, we want to produce a
type, which is a finite set, with cardinality $\fac{n}$. And, to characterise
$\Aut[\Fin[n]]$, we further need to construct a bijection between this type and
$\Aut[\Fin[n]]$.

First, let's define this type with $\fac{n}$ elements, we name this type family
$\Lehmer : \Nat \to \UU$, which is defined by recursion on $\Nat$ as follows.
This is the obvious definition of factorials by recursion, but categorified from
natural numbers to sets.

\begin{definition}
  \begin{align*}
    \Lehmer[0]       & \defeq \unit                           \\
    \Lehmer[\suc[n]] & \defeq \Fin[\suc[n]] \times \Lehmer[n]
  \end{align*}
\end{definition}

\todo{Subexcedant sequences and factorial definitions are equivalent, explain
  this!}

The name Lehmer comes from Lehmer
codes~\cite{lehmerTeachingCombinatorialTricks1960a} which are known in
Combinatorial Analysis~\cite{bellmanCombinatorialAnalysis1960}. There are many
ways to represent permutations, e.g. inversions, or cycles, or matrices. Lehmer
codes are a particularly convenient way to represent permutations on a
computer,~\review{they are compact and have exactly the right cardinality.
  $\Lehmer[n]$ is a $n+1$-element tuple, where the position $k \leq n$ has an
  element of $\Fin[k]$. The 0-th position is trivial, so we ignore it, and in
  both the example below and the Agda proof, consider only the remaining
  $n$-element tuple.}

\vc{This is just the classical algorithm to explain the example, not the actual
  type-theoretic proof.}

Suppose we have a permutation $p$ on an $n$-element set
$\{\el{0}, \el{1}, \el{2}, \el{3}, \el{4}\}$, we encode it as follows.
$\Lehmer[n]$ is a $n$-element tuple. At position $k$, we put the number of
inversions of the element $\el{k}$ in $p$, i.e. the number of elements smaller
than $\el{k}$ occurring after $\el{k}$.

As an example, consider the following tabulated presentation of the permutation:

\todo{fix this figure}

\[
  p =
  \begin{array}{ccccccccccccccc}
    | & 0      & | & 1      & | & 2      & | & 3      & | & 4      & | \\
    \hline                                                             \\
    | & \el{2} & | & \el{0} & | & \el{1} & | & \el{4} & | & \el{3} & | \\
    \hline                                                             \\
  \end{array}
\]

%  0 1 2 3 4
% -----------
% |2|0|1|4|3|
% -----------

The element $\el{0}$ has 0 inversions, because there are no elements smaller
than $\el{0}$ occurring after it. In fact, there can be no elements smaller than
$\el{0}$ at all, so the type at the first position of the Lehmer code tuple is
$\unit$.

The element $\el{1}$ has 0 inversions as well, since elements occurring after it
in the permutation are $\el{4}$ and $\el{2}$. There is only one different case,
if $\el{1}$ appeared before $\el{0}$, it would have 1 inversion. This is why the
type of the second component of the Lehmer code is $\Fin[2]$.

The element $\el{2}$ has 2 inversions, because both $\el{0}$ and $\el{1}$ occur
after it in the permutation. The element $\el{3}$ occurs as the last one, so it
has 0 inversions. The element $\el{4}$ has 1 inversion, with the element
$\el{3}$.

Thus, the Lehmer code for the permutation $p$ is the 5-tuple
$l = (0, 0, 2, 0, 1)$.

To reconstruct the tabulated presentation of the permutation from the Lehmer
code, we perform an algorithm similar to \emph{insertion sort}. Starting from
the left-most position of the tuple $l$, we'll read the value $v$, insert the
new element at the end of the newly created list, and shift it backward $v$
places.

\begin{center}
  \begin{tabular}{c|p{0.75\linewidth}}
    (0, 0, 2, 0, 1)               & We start from an empty list $[]$                                                                 \\
    (\highlight{{0}}, 0, 2, 0, 1) & We read 0 as the left-most value from $l$. Thus, we append the element $\el{0}$ to our
                                    list, getting $[\el{0}]$. The element is shifted $0$ places, so it remains in the
                                    same place.                                                                                                                      \\
    (0, \highlight{{0}}, 2, 0, 1) & Then, similarly, we read another 0 for the element $\el{1}$, append it to the
                                    list getting $[\el{0}, \el{1}]$, and don't shift it either.                                                                      \\
    (0, 0, \highlight{{2}}, 0, 1) & We read 2 for the next the element $\el{2}$ - we append $\el{2}$ to our list, getting
                                    $[\el{0}, \el{1}, \el{2}]$, and shift it 2 places right, which results in a list $[\el{0}, \el{2}, \el{1}]$
                                    \todo{Typeset it nicely, with arrows showing the shifting}.                                                                      \\
    (0, 0, 2, \highlight{{0}}, 1) & Then we read 0 - appending $\el{3}$ and not shifting, getting $[\el{0}, \el{2}, \el{1}, \el{3}]$ \\
    (0, 0, 2, 0, \highlight{{1}}) & Finally, reading 1 for element $\el{4}$ - appending $\el{4}$ to the list and shifting it
                                    one place right results in the final list $[\el{0}, \el{2}, \el{1}, \el{4}, \el{3}]$                                             \\
  \end{tabular}
\end{center}
\todo{figure}

Using this Lehmer encoding algorithm, we can now construct the equivalence
between these types.

\begin{proposition}
  For all $n:\Nat$,
  \[
    \Lehmer[n] \eqv \Aut[\Fin[n]]
  \]
\end{proposition}

\begin{proof}
  We have to turn this algorithm into a constructive proof in type theory where
  we only use functions and recursion. \todo{Describe the agda code.}
\end{proof}

\subsection{Symmetric groups}~\label{subsec:symmetric}

There is an obvious group structure on $\Aut[\Fin[n]]$ given by identity,
composition, and inverse. This is the symmetric group $S_n$ on $n$ symbols. In
the rest of the section we will construct a convenient presentation of this
group.

\todo{Reference T-algebra presentations as coequalisers (Mac Lane 6.7)}

First, we formally define a presentation of a group.

\begin{definition}
  A presentation of a group $G$ is a type $FR$ such that \ldots is a coequaliser.
\end{definition}

\vc{this is just a rough draft for now}

\begin{definition}
  Here we'll define a type family $FT n$ to be a free group generated by a type $T n$ with the following constructors
  \begin{align*}
     & cancel : i : \Fin[\suc[n]] \to T                                         \\
     & swap : (i : \Fin[\suc[n]]) \to (j : \Fin[\suc[n]]) \to (i + 1 < j) \to T \\
     & braid : (i : \Fin[n]) \to T                                              \\
  \end{align*}
\end{definition}

\begin{definition}[Adjacent transposition]
  \begin{align*}
    transpose : (n : \Nat) \to (k : \Fin[\suc[n]]) \to \Aut[\Fin[\suc[n]]]
  \end{align*}
  by double induction on $n$ and $k$, where
  \begin{align*}
    transpose (n) (0)             & = \lambda
    0 \to 1;
    1 \to 0;
    m \to m
    \\
    transpose (\suc[n]) (\suc[k]) & = \lambda
    0 \to 0;
    S m \to S ((transpose (n)(k)) m)
  \end{align*}
\end{definition}

\begin{proposition}
  $FR$ is a presentation of the symmetrc group, with $f$ and $g$ as follows:
  \begin{align*}
    f (inr (cancel (i))) & = inr (i :: i :: [])                     \\
    f (inl (cancel (i))) & = inl (i :: i :: [])                     \\
    f (inr (swap (i,j))) & = i :: j :: i :: j :: []                 \\
    f (braid (i))        & = i :: S i :: i :: S i :: i :: S i :: []
  \end{align*}
  and
  \begin{align*}
    g (inl (\_)) & = inl ([]) \\
    g (inr (\_)) & = inr ([])
  \end{align*}
\end{proposition}

We now introduce a presentation of the symmetric group.


\begin{theorem}
  For all $n : \Nat$,
  \[
    Sn(n) \eqv \Lehmer[n]
  \]
\end{theorem}

\subsection{Symmetric Monoidal structure}

$\UFin$ is a symmetric monoidal groupoid with unit and tensor given by

\begin{definition}
  \begin{align*}
    I           & \defeq F_{0}    \\
    X \otimes Y & \defeq X \sqcup Y
  \end{align*}
\end{definition}

with unitors, associator, and braiding satisfying Mac Lane's triangle, pentagon, and hexagon coherence laws, and
symmetry. It is the free symmetric monoidal groupoid on one generator.

%%% Local Variables:
%%% mode: latex
%%% TeX-master: "main"
%%% fill-column: 120
%%% End:

\section{Equivalence between \texorpdfstring{$\PiPlusLang$}{Pi} and \texorpdfstring{$\UFin$}{UFin}}~\label{sec:equivalence}

In this section, we use the semantics developed in the previous sections define a fragment $\PiPlusLang$ of $\PiLang$
and a normalised fragment of it $\PiHatLang$, interpret to $\UFin$ and back, and translate between the languages.

We only present the types and 1-combinators in the language, and the full set of 2-combinators is listed in the appendix
and the accompanying Agda code.

\vc{We can translate types to types, terms to terms, theorem: preserves types, preserves equations.}

\vc{Define maps quote/eval on 0,1,2-cells, and show commuting diagrams.}

\subsection{$\PiHatLang$}

We first present a language $\PiHatLang$ based on the theory of symmetric groups developed in~\cref{sec:finite}. The
types in the language are unary natural numbers, and combinators are allowed to perform adjacent transpositions.

\begin{figure}[t]
  {\scalebox{\scalef}{$
        \begin{array}{rrcll}
          \idc :   & n             & \isoh & n             & : \idc   \\
          \swapc : & \suc[\suc[n]] & \isoh & \suc[\suc[n]] & : \swapc \\
        \end{array}
      $}}

  {\scalebox{\scalef}{
      \Rule{}
      {\jdg{}{}{c_1 : n \isoh m} \quad \vdash c_2 : m \isoh o}
      {\jdg{}{}{c_1 \fatsemi c_2 : n \isoh o}}
      {}

      \Rule{}
      {\jdg{}{}{c : n \isoh m}}
      {\jdg{}{}{\oplus(c) : \suc[n] \isoh \suc[m]}}
      {}
    }}
  \caption{$\PiHatLang$ syntax}
  \label{fig:pihat}
\end{figure}

\begin{proposition}
  We can form a weak 2-category $\PiHatCat$ with
  \begin{itemize}
    \item natural numbers for 0-cells,
    \item for $n, m : \Nat$, a collection of 1-cells $n \isoh m$,
    \item for $p, q : n \isoh m$, a collection of 2-cells $p \Isoh q$.
  \end{itemize}
\end{proposition}

\begin{proposition}
  There is a symmetric monoidal structure on $\PiHatCat$, with $0$ for the unit, and addition for the tensor.
\end{proposition}

Then we establish the completeness of $\PiHatLang$ with respect to $\UFin$. We define $\evalt$ and $\quotet$ for 0, 1,
and 2-cells.

\begin{definition}
  \begin{align*}
    \evalt_{0} & : \UHat \to \UFin                                                          \\
    \evalt_{1} & : (c : t_{1} \isoh t_{2}) \to \evalh_{0}(t_{1}) \id \evalh_{0}(t_{2})      \\
    \evalt_{2} & : (\alpha : c_{1} \Isoh c_{2}) \to \evalt_{1}(c_{1}) \id \evalt_{1}(c_{2}) \\
  \end{align*}
\end{definition}

\begin{definition}
  \begin{align*}
    \quotet_{0} & : \UFin \to \UHat                                                            \\
    \quotet_{1} & : (p : X_{1} \id X_{2}) \to \quoteh_{0}(X_{1}) \isoh \quoteh_{0}(X_{2})      \\
    \quotet_{2} & : (\alpha : p_{1} \id p_{2}) \to \quoteh_{1}(p_{1}) \Isoh \quotet_{1}(p_{2}) \\
  \end{align*}
\end{definition}

\begin{proposition}
  $\evalt/\quotet$ give a symmetric monoidal biequivalence between $\PiHatCat$ and $\UFin$.
\end{proposition}

\subsection{$\PiPlusLang$}

Now, we present the additive fragment of $\PiLang$, called $\PiPlusLang$.

\begin{figure}[t]
  {\scalebox{\scalef}{$
        \begin{array}{rrcll}
          \idc :     & A           & \iso & A           & : \idc     \\
          \identlp : & \zerot + A  & \iso & A           & : \identrp \\
          \swapp :   & A + B       & \iso & B + A       & : \swapp   \\
          \assoclp : & A + (B + C) & \iso & (A + B) + C & : \assocrp \\ [1.5ex]
        \end{array}$}}

  {\scalebox{\scalef}{
      \Rule{}
      {\jdg{}{}{c_1 : A \iso B} \quad \vdash c_2 : B \iso C}
      {\jdg{}{}{c_1 \fatsemi c_2 : A \iso C}}
      {}

      \Rule{}
      {\jdg{}{}{c_1 : A \iso B} \quad \vdash c_2 : C \iso D}
      {\jdg{}{}{c_1 \oplus c_2 : A + C \iso B + D}}
      {}
    }}
  \caption{$\PiPlusLang$ syntax}
  \label{fig:piplus}
\end{figure}

\begin{proposition}
  We can form a weak 2-category $\PiPlusCat$ with
  \begin{itemize}
    \item $\PiPlusLang$ types for 0-cells,
    \item for $X, Y : \UPlus$, a collection of 1-cells $X \iso Y$,
    \item for $p, q : X \isoh Y$, a collection of 2-cells $p \Iso q$.
  \end{itemize}
\end{proposition}

\begin{proposition}
  There is a symmetric monoidal structure on $\PiPlusCat$, with $\zerot$ for the unit, and $+$ for the tensor.
\end{proposition}

We normalise $\PiPlusLang$ to $\PiHatLang$ and back, by giving $\evalh/\quoteh$ maps for 0, 1, and 2-cells.

\begin{definition}
  \begin{align*}
    \evalh_{0} & : \UPlus \to \UHat                                                          \\
    \evalh_{1} & : (c : t_{1} \iso t_{2}) \to \evalh_{0}(t_{1}) \isoh \evalh_{0}(t_{2})      \\
    \evalh_{2} & : (\alpha : c_{1} \Iso c_{2}) \to \evalh_{1}(c_{1}) \Isoh \evalh_{1}(c_{2}) \\
  \end{align*}
\end{definition}

\begin{definition}
  \begin{align*}
    \quoteh_{0} & : \UHat \to \UPlus                                                            \\
    \quoteh_{1} & : (p : X_{1} \isoh X_{2}) \to \quoteh_{0}(X_{1}) \iso \quoteh_{0}(X_{2})      \\
    \quoteh_{2} & : (\alpha : p_{1} \Isoh p_{2}) \to \quoteh_{1}(p_{1}) \Iso \quoteh_{1}(p_{2}) \\
  \end{align*}
\end{definition}

\begin{proposition}
  $\evalh/\quoteh$ give a symmetric monoidal biequivalence between $\PiPlusCat$ and $\PiHatCat$.
\end{proposition}

\subsection{$\PiLang$}

Finally, We show how to translate $\PiLang$ programs to $\PiHatLang$ programs.

\begin{definition}
  \begin{align*}
    \evalt_{0} & : U \to \Nat                                               \\
    \evalt_{1} & : (c : X \iso Y) \to \evalt_{0}(X) \iso \evalt_{0}(Y)      \\
    \evalt_{2} & : (\alpha : p \Iso q) \to \evalt_{1}(p) \Iso \evalt_{1}(q) \\
  \end{align*}
\end{definition}

To normalise a $\PiLang$ circuit, we translate it to $\PiHatLang$ and quote it to $\PiPlusLang$.

\begin{definition}
  \begin{align*}
    \normt_{0} & : U \to \UPlus                                        \\
    \normt_{0} & = \quoteh_{0} \comp \evalt_{0}                        \\
    \\
    \normt_{1} & : (c : X \iso Y) \to \normt_{0}(X) \iso \normt_{0}(Y) \\
    \normt_{1} & = \quoteh_{1} \comp \evalt_{1}                        \\
  \end{align*}
\end{definition}

%%% Local Variables:
%%% mode: latex
%%% TeX-master: "main"
%%% fill-column: 120
%%% End:

\section{Discussion \& Related Work}~\label{sec:discussion}

In this paper, we \ldots


Pi types -- Natural number -- Finite sets
1-combinator -- Generators of Sn -- 1-paths
2-combinator -- Relations of Sn -- 2-paths

In HoTT, univalence characterises the path type in the universe as equivalences of types. The map $\term{idtoeqv} : A
\id_{\UU} B \to A \eqv B$ can be easily constructed using path induction. The term $\term{ua} : A \eqv B \to A \id_{\UU}
B$, its computation rule $\term{ua-\beta} : (e : A \eqv B) \to \term{idtoeqv}(\term{ua}(e)) \id e$, and its
extensionality rule $\term{ua-\eta} : (p : A \id_{\UU} B) \to p == \term{ua}(\term{idtoeqv(p)})$ are generally added as
postulates when formalising in Agda. Together, $\term{ua}$ and $\term{ua}-\beta$ give the full univalence axiom $(A \eqv
B) \eqv (A \id_{\UU} B)$.

% Let's think of the $\PiLang$ combinators as describing the inhabitants of the dentity type of finite types. 

By giving a computable presentation for a univalent subuniverse, we are able to describe the path space of it
syntactically, by giving a complete equational axiomatisation of the equivalences between types in the subuniverse. 
% By the property of being univalent, this subuniverse gives a model of the univalence axiom. 
The $\term{idtoeqv}$ corresponds to giving a denotation for a program (1-combinator), which is easily done by induction.
The $\term{ua}$ map corresponds to synthesing a program from an equivalence (which, in general, is of course
undecidable~\cite{krogmeierDecidableSynthesisPrograms2020}). In case of reversible boolean circuits, it is decidable, as
we have shown, but still far from trivial, which matches the need to assert the existence of $\term{ua}$ without giving
a constructive argument. Then, the computation rule $\term{ua-\beta}$ expresses the fact that program synthesis is
sound, while $\term{ua-\eta}$ corresponds to the soundess of the equational theory ($\PiLang$ 2-combinators). Thus, we
this suggests a new computational interpretation of the univalence principle, which provides an intution on why certain
constructions are hard (or impossible in general case).

We could present a dependent type theory for the topos $\SetFin$, with an identity type for terms (generated by
$\refl$), and one for types (generated by $\PiLang$ combinators). We can't talk about universes in $\SetFin$ since it
doesn't have one, but we could show externally that it satisfies univalence.

Our work lies at the intersection of programming language theory, category theory, group theory, rewriting theory, and
formalised mathematics. We review related work in the literature for each topic.

\paragraph{Algebraic Theories} In universal algebra, algebraic theories are used to describe algebraic structures, such
as groups or rings. A specific group or ring is a model of the appropriate algebraic theory. Algebraic theories are
usually \emph{presented} in terms of logical syntax, that is, as first-order theories whose signatures allow only
functional symbols, and whose axioms are universally quantified equations. In his seminal
thesis~\cite{lawvereFUNCTORIALSEMANTICSALGEBRAIC1963}, Lawvere defined a presentation-free categorical notion of
universal algebraic structure, called a Lawvere theory.

Programming Languages, such as the $\lambda$-calculus, can be viewed as algebraic structures with variable-binding
operators, which can be formalised using second-order algebraic theories~\cite{fioreSecondOrderAlgebraicTheories2010},
or algebraic theories with closed structure~\cite{hylandClassicalLambdaCalculus2017}, called $\lambda$-theories, making
the $\lambda$-calculus the presentation of the initial $\lambda$-theory $\Lambda$.

Our family of reversible languages have been presented as first-order algebraic
2-theories~\cite{cohenCoherenceRewriting2theories2009,bekeCategorificationTermRewriting2011,yanofskySyntaxCoherence2000},
which are a categorification of algebraic theories. The types $\zerot$ and $\onet$ are nullary function symbols, the
type formers $+$ and $\times$ are binary function symbols, the 1-combinators are invertible reduction rules, and the
2-combinators are equations or coherence diagrams of compositions of reduction rules. Just like models of Lawvere
theories are given by algebras of (finitary) monads on $\SetCat$, models of 2-theories are given by algebras of 2-monads
on $\CatCat$. The particular one we're interested in here is the free symmetric monoidal completion 2-monad.

\paragraph{Free Symmetric Monoidal Category} The forgetful functor from $\SymMonCat$, the 2-category of (small)
symmetric monoidal categories, strong symmetric monoidal functors, and monoidal natural transformations, to the
2-category $\CatCat$, has a left adjoint giving the free symmetric monoidal category $\FSM[\CCat{C}]$ on a category
$\CCat{C}$. This is a 2-monad on $\CatCat$~\cite{blackwellTwodimensionalMonadTheory1989}, whose algebras are (strict)
symmetric monoidal categories. Its construction is known in the literature~\cite{abramskyAbstractScalarsLoops2005}.
Concretely, the objects of $\FSM[\CCat{C}]$ are given by lists of objects of $\CCat{C}$, that is, a pair $(n:\Nat, A:[n]
\to \CCat{C}_{0})$. An morphism between $(n,A)$ and $(n,B)$ is given by a pair $(\pi,\lambda)$ where $\pi$ is a
permutation of $[n]$, and $\lambda_{i} : A_{i} \to B_{\pi(i)}$ for $1 \leq i \leq n$. Abstractly, this is given by the
Grothendieck construction $\int F$ of the functor $F : \BFin \to \CatCat$ from the groupoid of finite sets and
bijections to $\CatCat$, assigning each natural number $n$ to the $n$-power $C^{n}$ of $C$, and each permutation on
$[n]$ inducing an endofunctor on $C^{n}$ by action. $\BFin$ is the free symmetric monoidal category (groupoid) on one
generator, $\FSM[\unit]$. The free symmetric monoidal category has been used to study
concurrency~\cite{hylandSymmetricMonoidalSketches2004}, petri nets~\cite{baezCategoriesNets2021}, combinatorial
structures~\cite{fioreCartesianClosedBicategory2008}, quantum mechanics~\cite{abramskyAbstractScalarsLoops2005},
bicategorical models of (differential) linear logic~\cite{melliesTemplateGamesDifferential2019}. 

Coherence and normalisation problems for monoids in constructive type theory using coherence for monoidal categories was
studied in~\cite{beylinExtractingProofCoherence1996}. In HoTT, coherence for the free monoidal groupoid over a groupoid
and the proof of its universal property has been considered in~\cite{piceghelloCoherenceMonoidalGroupoids2020}. 

Free commutative monoids in type theory have been studied in~\cite{gylterudMultisetsTypeTheory2020}, and using HoTT
in~\cite{choudhuryFinitemultisetConstructionHoTT2019}. The free symmetric monoidal groupoid $\FSM[A]$ over a groupoid
$A$ can be given by $\dsum{X:\UFin}{A^{X}}$, or it can be presented as an algebraic 2-theory using 1-HITs. These HITs
and the proof of their universal property have been considered
in~\cite*{piceghelloCoherenceSymmetricMonoidal2019,choudhuryFinitemultisetConstructionHoTT2019}.

The proof of the universal property of the $\FSM$ is asserted by appealing to Mac Lane's coherence theorem for symmetric
monoidal categories, and using the fact that the finite symmetric group $\Sn$ encodes the permutation group on a finite
set. The existence of the proof is folklore, and we have produced a new proof of it while working in constructive type
theory.

\paragraph{Curry-Howard-Lambek correspondence} In~\citet{curryCurryEssaysCombinatory1980}, Lambek extended the
Curry-Howard correspondence to cartesian-closed categories. In this work, we have established a correspondence between a
fragment of the $\PiLang$ family of reversible programming languages and symmetric monoidal groupoids. The Curry-Howard
part of this correspondence with reversible logic was established in~\cite{sparksSuperstructuralReversibleLogic2014}.

\paragraph{Rewriting} We presented a rewriting system for the Coxeter relations for $\Sn$ to solve its word problem.

There exists an algorithm, due to~\citet{knuthSimpleWordProblems1970}, that, when succeeds, constructs a well-behaved
rewriting system for an arbitrary finite set of (undirected) equations. It did not work for us, producing too many
equations, and proving correctness and termination was intractable.

We chose to encode permutations as adjacent transpositions corresponding to words in $\Sn$, since that closely
corresponds to the syntax of $\PiLang$ combinators. Other representations of permutations would give a different
denotational semantics for $\PiLang$ combinators. Permutations can be encoded as listed vectors or matrices, inductively
generated trees (Motzkin trees), Young diagrams, or String diagrams, but the difficulty of formalising them in type
theory varies depending on the encoding.

Rewriting systems and word problems have a long history of being formalised in proof assistants. In the recent past,
higher order rewriting systems have been formalised in proof assistants like {homotopy.io} and Lean and Coq. The use of
HoTT to study rewriting has been considered in~\cite{krausCoherenceWellFoundednessTaming2020}.

\paragraph{Computational group theory} Coxeter relations are used in computational group theory to study XXX problems.

\paragraph{Univalent Fibrations} Univalent Fibrations were introduced by~\citet*{kapulkinUnivalenceSimplicialSets2018},
to build a model of Voevodsky's \emph{univalence} principle in simplicial sets.
\citet{christensenCharacterizationUnivalentFibrations2015} studied characterisations of univalent fibrations using the
$\BAut$ construction. Univalent typoids~\cite{petrakisUnivalentTypoids2019a} are a different presentation of univalent
subuniverses.

Coherence problems in type theory, coherence via Well-Foundedness.

Formalised proofs of Mac Lane's coherence theorem.

Applications of FSMG and history of the coherence theorems.

Other proofs of coherence theorems, Joyal-Street.

Pi has other extensions (fractional/negative/recursive types).
What are the free X monoidal structures they're describing?

Using our presentation of $S_{n}$, we can construct the Eilenberg-Maclane space (using a HIT) $K(S_{n},1)$. Then, it
should be true that $\UFin \eqv \sqcup_{n:\Nat} K(S_{n},1)$. This is future work.

Other applications of symmetric groups.

Actions of symmetric groups, permutation groupoids.

%%% Local Variables:
%%% mode: latex
%%% TeX-master: "main"
%%% fill-column: 120
%%% End:

\input{extra}

%% Acks: Chao-Hong and Jacques

\bibliographystyle{ACM-Reference-Format}
\bibliography{2dtypesZot,survey} %% ,chen}

\end{document}

%%% Local Variables:
%%% mode: latex
%%% TeX-master: t
%%% fill-column: 120
%%% End:
