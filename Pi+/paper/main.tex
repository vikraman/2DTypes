\documentclass[acmsmall,review,anonymous]{acmart}
\settopmatter{printfolios=true,printccs=false,printacmref=false}

\usepackage{cleveref}
\usepackage{macros}
\usepackage{hott}
\usepackage{quiver}

%% Some recommended packages.
%% \usepackage{booktabs}   %% For formal tables:
                        %% http://ctan.org/pkg/booktabs
%% \usepackage{subcaption} %% For complex figures with subfigures/subcaptions
                        %% http://ctan.org/pkg/subcaption
%% \usepackage{verbatim}
%% \usepackage{amsmath,amsbsy}
%% \usepackage{alltt}
%% \usepackage{fdsymbol}
%% \usepackage{amsthm,proof}
%% \usepackage{bbold,stmaryrd,bbm}
\usepackage{bbold}
%% \usepackage{ucs}
%% \usepackage{wrapfig}
%% \usepackage[utf8x]{inputenc}
%% \usepackage{newunicodechar}
%% \usepackage{microtype}
%% \usepackage{subcaption}
%% \usepackage{agda}
\usepackage{tikz}
\usetikzlibrary{cd}
\usetikzlibrary{quotes}
\usetikzlibrary{decorations.markings}
\usetikzlibrary{knots}
% \usepackage{tikzit}
% \input{tikzit.tikzstyles}

\newtheorem{theorem}{Theorem}
\newtheorem{corollary}[theorem]{Corollary}
\newtheorem{lemma}[theorem]{Lemma}
\newtheorem{definition}[theorem]{Definition}
\newtheorem*{remark}{Remark}

\newcommand{\zerot}{\mathbb{0}}
\newcommand{\onet}{\mathbb{1}}
\newcommand{\sumt}{\mathbin{\mathsf{+}}}
\newcommand{\prodt}{\mathbin{\mathsf{\times}}}
\newcommand{\isot}{\leftrightarrow}

%%
%% \BibTeX command to typeset BibTeX logo in the docs
\AtBeginDocument{%
  \providecommand\BibTeX{{%
    \normalfont B\kern-0.5em{\scshape i\kern-0.25em b}\kern-0.8em\TeX}}}

%% Rights management information.  This information is sent to you
%% when you complete the rights form.  These commands have SAMPLE
%% values in them; it is your responsibility as an author to replace
%% the commands and values with those provided to you when you
%% complete the rights form.
\setcopyright{acmcopyright}
\copyrightyear{2018}
\acmYear{2018}
\acmDOI{10.1145/1122445.1122456}

%% These commands are for a PROCEEDINGS abstract or paper.
\acmConference[Woodstock '18]{Woodstock '18: ACM Symposium on Neural
  Gaze Detection}{June 03--05, 2018}{Woodstock, NY}
\acmBooktitle{Woodstock '18: ACM Symposium on Neural Gaze Detection,
  June 03--05, 2018, Woodstock, NY}
\acmPrice{15.00}
\acmISBN{978-1-4503-XXXX-X/18/06}


%%
%% Submission ID.
%% Use this when submitting an article to a sponsored event. You'll
%% receive a unique submission ID from the organizers
%% of the event, and this ID should be used as the parameter to this command.
%%\acmSubmissionID{123-A56-BU3}

%%
%% The majority of ACM publications use numbered citations and
%% references.  The command \citestyle{authoryear} switches to the
%% "author year" style.
%%
%% If you are preparing content for an event
%% sponsored by ACM SIGGRAPH, you must use the "author year" style of
%% citations and references.
%% Uncommenting
%% the next command will enable that style.
\citestyle{acmauthoryear}

%%
%% end of the preamble, start of the body of the document source.
\begin{document}

%%
%% The "title" command has an optional parameter,
%% allowing the author to define a "short title" to be used in page headers.
\title{Reversible Programming with Univalent Finite Types}

%%
%% The "author" command and its associated commands are used to define
%% the authors and their affiliations.
%% Of note is the shared affiliation of the first two authors, and the
%% "authornote" and "authornotemark" commands
%% used to denote shared contribution to the research.
\author{Anonymous}

%%
%% By default, the full list of authors will be used in the page
%% headers. Often, this list is too long, and will overlap
%% other information printed in the page headers. This command allows
%% the author to define a more concise list
%% of authors' names for this purpose.
\renewcommand{\shortauthors}{Anonymous}

%%
%% The abstract is a short summary of the work to be presented in the
%% article.
\begin{abstract}
  We establish a close connection between a reversible programming language
  based on isomorphisms of finite types and a formally presented univalent
  universe.

  The correspondence relates combinators witnessing type isomorphisms in the
  programming language to paths in the univalent universe; and combinator
  optimisations in the programming language to 2-paths in the univalent
  universe.

  The main result is a \ldots
\end{abstract}

%%
%% The code below is generated by the tool at http://dl.acm.org/ccs.cfm.
%% Please copy and paste the code instead of the example below.
%%
\begin{CCSXML}
  <ccs2012>
   <concept>
       <concept_id>10003752.10003790.10011740</concept_id>
       <concept_desc>Theory of computation~Type theory</concept_desc>
       <concept_significance>500</concept_significance>
       </concept>
   <concept>
       <concept_id>10003752.10010124.10010131.10010137</concept_id>
       <concept_desc>Theory of computation~Categorical semantics</concept_desc>
       <concept_significance>500</concept_significance>
       </concept>
   <concept>
       <concept_id>10003752.10010124.10010131.10010133</concept_id>
       <concept_desc>Theory of computation~Denotational semantics</concept_desc>
       <concept_significance>500</concept_significance>
       </concept>
   <concept>
       <concept_id>10011007.10011006.10011008.10011009.10011012</concept_id>
       <concept_desc>Software and its engineering~Functional languages</concept_desc>
       <concept_significance>500</concept_significance>
       </concept>
   <concept>
       <concept_id>10011007.10011006.10011039.10011040</concept_id>
       <concept_desc>Software and its engineering~Syntax</concept_desc>
       <concept_significance>500</concept_significance>
       </concept>
   <concept>
       <concept_id>10011007.10011006.10011039.10011311</concept_id>
       <concept_desc>Software and its engineering~Semantics</concept_desc>
       <concept_significance>500</concept_significance>
       </concept>
 </ccs2012>
\end{CCSXML}

\ccsdesc[500]{Theory of computation~Type theory}
\ccsdesc[500]{Theory of computation~Categorical semantics}
\ccsdesc[500]{Theory of computation~Denotational semantics}
\ccsdesc[500]{Software and its engineering~Functional languages}
\ccsdesc[500]{Software and its engineering~Syntax}
\ccsdesc[500]{Software and its engineering~Semantics}

%%
%% Keywords. The author(s) should pick words that accurately describe
%% the work being presented. Separate the keywords with commas.
\keywords{reversible computing}

%%
%% This command processes the author and affiliation and title
%% information and builds the first part of the formatted document.
\maketitle

%%%%%%%%%%%%%%%%%%%%%%%%%%%%%%%%%%%%%%%%%%%%%%%%%%%%%%

\section{Introduction}~\label{sec:introduction}

%%% Local Variables:
%%% mode: latex
%%% TeX-master: "main"
%%% End:

\section{Reversible Programming Languages}~\label{sec:reversible}

\todo{Not the right title.}

\note{This section should explain the main technical parts of the paper
  informally, without using any technology. Use an example, such as, a
  reversible language with $\leq 5$ bits, and examples of permutations and
  transpositions, and when they're equal.}

\note{Motivation: There are two reversible circuits which describe the following permutation. They can be shown to be
  equal using the 2-combinators.}

\[
  \begin{tikzpicture}
    \begin{knot}[clip width=5]
      \filldraw (0,5) circle (2pt) node[above] {0};
      \filldraw (1,5) circle (2pt) node[above] {1};
      \filldraw (2,5) circle (2pt) node[above] {2};
      \filldraw (3,5) circle (2pt) node[above] {3};
      \filldraw (4,5) circle (2pt) node[above] {4};
      \filldraw (0,0) circle (2pt) node[below] {1};
      \filldraw (1,0) circle (2pt) node[below] {4};
      \filldraw (2,0) circle (2pt) node[below] {0};
      \filldraw (3,0) circle (2pt) node[below] {3};
      \filldraw (4,0) circle (2pt) node[below] {2};
      \strand (0,5) .. controls (0.5,0.5) and (1.5,3.5) .. (2,0);
      \strand (1,5) .. controls (0.75,0.5) and (0.25,3.5) .. (0,0);
      \strand (2,5) .. controls (2.5,2.5) and (3.5,1.5) .. (4,0);
      \strand (3,5) .. controls (4.5,2.5) and (4,1.5) .. (3,0);
      \strand (4,5) .. controls (3.5,2.5) and (1.5,2.5) .. (1,0);
      \flipcrossings{4,5};
    \end{knot}
  \end{tikzpicture}
\]

\note{Example: We reduce $\mathsf{swap} : 2 + 2 \leftrightarrow 2 + 2$ to a sequence of adjacent swaps. This is an
  example of a translation from $\PiPlusLang$ to $\PiHatLang$.}

\[
  \begin{tikzpicture}
    \begin{knot}[clip width=4]
      \filldraw (0,4) circle (2pt) node[above] {0};
      \filldraw (1,4) circle (2pt) node[above] {1};
      \filldraw (2,4) circle (2pt) node[above] {2};
      \filldraw (3,4) circle (2pt) node[above] {3};
      \filldraw (0,0) circle (2pt) node[below] {2};
      \filldraw (1,0) circle (2pt) node[below] {3};
      \filldraw (2,0) circle (2pt) node[below] {0};
      \filldraw (3,0) circle (2pt) node[below] {1};
      \strand (0,4) .. controls (0.5,1.5) and (1.5,2.5) .. (2,0);
      \strand (1,4) .. controls (1.5,1.5) and (2.5,2.5) .. (3,0);
      \strand (2,4) .. controls (1.5,1.5) and (1.5,2.5) .. (0,0);
      \strand (3,4) .. controls (2.5,1.5) and (2.5,2.5) .. (1,0);
    \end{knot}
  \end{tikzpicture}
\]

\begin{align*}
  \begin{tikzpicture}
    \begin{knot}[clip width=4]
      \filldraw (0,4) circle (2pt) node[above] {0};
      \filldraw (1,4) circle (2pt) node[above] {1};
      \filldraw (2,4) circle (2pt) node[above] {2};
      \filldraw (3,4) circle (2pt) node[above] {3};
      \filldraw (0,0) circle (2pt) node[below] {0};
      \filldraw (1,0) circle (2pt) node[below] {2};
      \filldraw (2,0) circle (2pt) node[below] {1};
      \filldraw (3,0) circle (2pt) node[below] {3};
      \strand (0,4) to (0,0);
      \strand (1,4) .. controls (0.5,2) and (2.5,2) .. (2,0);
      \strand (2,4) .. controls (2.5,2) and (0.5,2) .. (1,0);
      \strand (3,4) to (3,0);
    \end{knot}
  \end{tikzpicture}
  &&
    \begin{tikzpicture}
      \begin{knot}[clip width=4]
        \filldraw (0,4) circle (2pt) node[above] {0};
        \filldraw (1,4) circle (2pt) node[above] {2};
        \filldraw (2,4) circle (2pt) node[above] {1};
        \filldraw (3,4) circle (2pt) node[above] {3};
        \filldraw (0,0) circle (2pt) node[below] {2};
        \filldraw (1,0) circle (2pt) node[below] {0};
        \filldraw (2,0) circle (2pt) node[below] {1};
        \filldraw (3,0) circle (2pt) node[below] {3};
        \strand (0,4) .. controls (-0.5,2) and (1.5,2) .. (1,0);
        \strand (1,4) .. controls (1.5,2) and (-0.5,2) .. (0,0);
        \strand (2,4) to (2,0);
        \strand (3,4) to (3,0);
      \end{knot}
    \end{tikzpicture}
  \\
  \begin{tikzpicture}
    \begin{knot}[clip width=4]
      \filldraw (0,4) circle (2pt) node[above] {2};
      \filldraw (1,4) circle (2pt) node[above] {0};
      \filldraw (2,4) circle (2pt) node[above] {1};
      \filldraw (3,4) circle (2pt) node[above] {3};
      \filldraw (0,0) circle (2pt) node[below] {2};
      \filldraw (1,0) circle (2pt) node[below] {0};
      \filldraw (2,0) circle (2pt) node[below] {3};
      \filldraw (3,0) circle (2pt) node[below] {1};
      \strand (0,4) to (0,0);
      \strand (1,4) to (1,0);
      \strand (2,4) .. controls (1.5,2) and (3.5,2) .. (3,0);
      \strand (3,4) .. controls (3.5,2) and (1.5,2) .. (2,0);
    \end{knot}
  \end{tikzpicture}
  &&
    \begin{tikzpicture}
      \begin{knot}[clip width=4]
        \filldraw (0,4) circle (2pt) node[above] {2};
        \filldraw (1,4) circle (2pt) node[above] {0};
        \filldraw (2,4) circle (2pt) node[above] {3};
        \filldraw (3,4) circle (2pt) node[above] {1};
        \filldraw (0,0) circle (2pt) node[below] {2};
        \filldraw (1,0) circle (2pt) node[below] {3};
        \filldraw (2,0) circle (2pt) node[below] {0};
        \filldraw (3,0) circle (2pt) node[below] {1};
        \strand (0,4) to (0,0);
        \strand (1,4) .. controls (0.5,2) and (2.5,2) .. (2,0);
        \strand (2,4) .. controls (2.5,2) and (0.5,2) .. (1,0);
        \strand (3,4) to (3,0);
      \end{knot}
    \end{tikzpicture}
\end{align*}

\note{This might be followed by a section which explains the syntax of Pi.}

%%% Local Variables:
%%% mode: latex
%%% TeX-master: "main"
%%% fill-column: 120
%%% End:

\section{Univalent Subuniverses}~\label{sec:univalent}

In this section, we introduce some basic concepts and notation that we use in Homotopy Type Theory. Then, we define
univalent subuniverses and discuss some specific examples.

\subsection{The Type Theory}~\label{subsec:type-theory}

We work in Homotopy Type Theory, in particular, intensional Martin-L\"{o}f Type Theory, with a univalent universe, and
Higher Inductive Types (HITs) for propositional truncations and set quotients. We recall a few basic facts.

\subsubsection{Identity Types}

\subsubsection{Univalence}

\subsubsection{Higher Inductive Types}

\subsection{Univalent Subuniverses}

\begin{definition}[Type families are fibrations]
  A type family $P : A \to \UU$ gives a fibration $\fst : \dsum{x:A}{P(x)} \to A$ with total space $\dsum{x:A}{P(x)}$
  and base space $A$ with a lifting property.
\end{definition}

\begin{definition}[$\term{transport-equiv}$]
  Given a type family $P : A \to \UU$, we define
  \[
    \term{transport-equiv} : x \id_{A} y \to P(x) \eqv P(y)
  \]
\end{definition}

\begin{definition}[Univalent Fibration]
  $P$ is a univalent fibration if $\term{transport-equiv}$ is an equivalence.
\end{definition}

\begin{definition}[Universe]
  A universe \`{a} la Tarski is given by a code $U : \UU$ and a decoding function $\El : U \to \UU$. If $\El$ is a
  univalent fibration, $U$ is a univalent universe.
\end{definition}

\begin{definition}[Subuniverse]
  A subtype is a type family $P : \UU \to \UU$ whose fibers are prop-valued, that is, $\forall x, \isProp{P(x)}$. A
  subuniverse generated by a subtype has $U \defeq \dsum{X:\UU}{P(X)}$ and $\El \defeq \fst$.
\end{definition}

\begin{proposition}[Univalent Subuniverse]
  Subuniverses generated by subtypes are univalent.
\end{proposition}

\begin{example}[$\Aut$]
  \[
    \Aut[T] \defeq T \eqv T
  \]
\end{example}

\begin{example}[$\BAut$]
  \[
    \BAut[T] \defeq \Sub{T}
  \]
\end{example}

\begin{example}
  \[
    P(X) \defeq \SubP{X}{\Bool}
  \]
\end{example}

\begin{example}
  For any $n : \Nat$, we have
  \[
    P(X) \defeq \SubP{X}{\Fin[n]}
  \]
  \[
    \UFin[n] \defeq \Sub{\Fin[n]}
  \]
  is the univalent subuniverse of types with cardinality $n$.
\end{example}

\begin{definition}[$\isFin$]
  \[
    \isFin[X] \defeq \dsum{n:\Nat}{\SubP{X}{\Fin[n]}}
  \]
\end{definition}

\begin{proposition}
  $\isFin$ is a proposition.
\end{proposition}

\begin{example}
  \[
    \UFin \defeq \dsum{X:\UU}{\isFin[X]}
  \]
  is the univalent subuniverse of all finite types.
\end{example}

We characterise the path space of subunivalent universes.

\begin{proposition}
  If $T$ is an $n$-type, $\BAut[T]$ is an $(n+1)$-type.
\end{proposition}

\begin{proposition}
  $\UFin$ is a 1-type, that is, a 1-groupoid, or has h-level 3.
\end{proposition}

\begin{proposition}
  \[
    \Omega\BAut[T] \eqv \Aut[T]
  \]
\end{proposition}

\begin{proposition}
  For every $n:\Nat$,
  \[
    \Omega\UFin[n] \eqv \Aut[\Fin[n]]
  \]
\end{proposition}

%%% Local Variables:
%%% mode: latex
%%% TeX-master: "main"
%%% End:

\section{The groupoid of finite types}~\label{sec:finite}

In this section, we describe \review{the algebraic structure} of the groupoid of
finite types, and give \review{a computable presentation} for it.

\vc{The groupoid of finite types is the free symmetric monoidal groupoid on one
  generator. This can be presented as an algebraic 2-theory, which is our syntax
  for $\PiHatLang$. Vertical categorification of natural numbers as a free
  commutative monoid. See groupoidification.}

\todo{Check Brent Yorgey's thesis?}

To do so, we will characterise the automorphisms on finite sets of cardinality
$n$, and show them to be equivalent to the symmetric group $\Sn$, via the
Coxeter presentation. We will do that in two steps,
in~\cref*{subsec:permutations,subsec:lehmer,subsec:symmetric}.

%% Coxeter presentation of $\Sn$ $\eqv$ $\Lehmer[n]$ $\eqv$ $\Aut[\Fin[n]]$

\subsection{Permutations}~\label{subsec:permutations}

In the previous~\cref{sec:univalent}, we established that paths in $\UFin$ are
equivalent to families of automorphisms of $\Fin{n}$ for every $n:\Nat$, that
is, bijections on finite sets of size $n$. This is the extensional view of
permutations. In the following sections, we will characterise these
permutations, going through two intermediate steps.

\vc{This is obvious, maybe add something more here.}

\subsection{Lehmer codes}~\label{subsec:lehmer}

From grade school combinatorics, we know that there are $\fac{n}$ permutations
on a finite set with $n$ elements. The factorial function is defined by
recursion on natural numbers. However, now, for every $n$, we want to produce a
type, which is a finite set, with cardinality $\fac{n}$. And, to characterise
$\Aut[\Fin[n]]$, we further need to construct a bijection between this type and
$\Aut[\Fin[n]]$.

First, let's define this type with $\fac{n}$ elements, we name this type family
$\Lehmer : \Nat \to \UU$, which is defined by recursion on $\Nat$ as follows.
This is the obvious definition of factorials by recursion, but categorified from
natural numbers to sets.

\begin{definition}
  \begin{align*}
    \Lehmer[0]       & \defeq \unit                           \\
    \Lehmer[\suc[n]] & \defeq \Fin[\suc[n]] \times \Lehmer[n]
  \end{align*}
\end{definition}

\todo{Subexcedant sequences and factorial definitions are equivalent, explain
  this!}

The name Lehmer comes from Lehmer
codes~\cite{lehmerTeachingCombinatorialTricks1960a} which are known in
Combinatorial Analysis~\cite{bellmanCombinatorialAnalysis1960}. There are many
ways to represent permutations, e.g. inversions, or cycles, or matrices. Lehmer
codes are a particularly convenient way to represent permutations on a
computer,~\review{they are compact and have exactly the right cardinality.
  $\Lehmer[n]$ is a $n+1$-element tuple, where the position $k \leq n$ has an
  element of $\Fin[k]$. The 0-th position is trivial, so we ignore it, and in
  both the example below and the Agda proof, consider only the remaining
  $n$-element tuple.}

\vc{This is just the classical algorithm to explain the example, not the actual
  type-theoretic proof.}

Suppose we have a permutation $p$ on an $n$-element set
$\{\el{0}, \el{1}, \el{2}, \el{3}, \el{4}\}$, we encode it as follows.
$\Lehmer[n]$ is a $n$-element tuple. At position $k$, we put the number of
inversions of the element $\el{k}$ in $p$, i.e. the number of elements smaller
than $\el{k}$ occurring after $\el{k}$.

As an example, consider the following tabulated presentation of the permutation:

\todo{fix this figure}

\[
  p =
  \begin{array}{ccccccccccccccc}
    | & 0      & | & 1      & | & 2      & | & 3      & | & 4      & | \\
    \hline                                                             \\
    | & \el{2} & | & \el{0} & | & \el{1} & | & \el{4} & | & \el{3} & | \\
    \hline                                                             \\
  \end{array}
\]

%  0 1 2 3 4
% -----------
% |2|0|1|4|3|
% -----------

The element $\el{0}$ has 0 inversions, because there are no elements smaller
than $\el{0}$ occurring after it. In fact, there can be no elements smaller than
$\el{0}$ at all, so the type at the first position of the Lehmer code tuple is
$\unit$.

The element $\el{1}$ has 0 inversions as well, since elements occurring after it
in the permutation are $\el{4}$ and $\el{2}$. There is only one different case,
if $\el{1}$ appeared before $\el{0}$, it would have 1 inversion. This is why the
type of the second component of the Lehmer code is $\Fin[2]$.

The element $\el{2}$ has 2 inversions, because both $\el{0}$ and $\el{1}$ occur
after it in the permutation. The element $\el{3}$ occurs as the last one, so it
has 0 inversions. The element $\el{4}$ has 1 inversion, with the element
$\el{3}$.

Thus, the Lehmer code for the permutation $p$ is the 5-tuple
$l = (0, 0, 2, 0, 1)$.

To reconstruct the tabulated presentation of the permutation from the Lehmer
code, we perform an algorithm similar to \emph{insertion sort}. Starting from
the left-most position of the tuple $l$, we'll read the value $v$, insert the
new element at the end of the newly created list, and shift it backward $v$
places.

\begin{center}
  \begin{tabular}{c|p{0.75\linewidth}}
    (0, 0, 2, 0, 1)               & We start from an empty list $[]$                                                                 \\
    (\highlight{{0}}, 0, 2, 0, 1) & We read 0 as the left-most value from $l$. Thus, we append the element $\el{0}$ to our
                                    list, getting $[\el{0}]$. The element is shifted $0$ places, so it remains in the
                                    same place.                                                                                                                      \\
    (0, \highlight{{0}}, 2, 0, 1) & Then, similarly, we read another 0 for the element $\el{1}$, append it to the
                                    list getting $[\el{0}, \el{1}]$, and don't shift it either.                                                                      \\
    (0, 0, \highlight{{2}}, 0, 1) & We read 2 for the next the element $\el{2}$ - we append $\el{2}$ to our list, getting
                                    $[\el{0}, \el{1}, \el{2}]$, and shift it 2 places right, which results in a list $[\el{0}, \el{2}, \el{1}]$
                                    \todo{Typeset it nicely, with arrows showing the shifting}.                                                                      \\
    (0, 0, 2, \highlight{{0}}, 1) & Then we read 0 - appending $\el{3}$ and not shifting, getting $[\el{0}, \el{2}, \el{1}, \el{3}]$ \\
    (0, 0, 2, 0, \highlight{{1}}) & Finally, reading 1 for element $\el{4}$ - appending $\el{4}$ to the list and shifting it
                                    one place right results in the final list $[\el{0}, \el{2}, \el{1}, \el{4}, \el{3}]$                                             \\
  \end{tabular}
\end{center}
\todo{figure}

Using this Lehmer encoding algorithm, we can now construct the equivalence
between these types.

\begin{proposition}
  For all $n:\Nat$,
  \[
    \Lehmer[n] \eqv \Aut[\Fin[n]]
  \]
\end{proposition}

\begin{proof}
  We have to turn this algorithm into a constructive proof in type theory where
  we only use functions and recursion. \todo{Describe the agda code.}
\end{proof}

\subsection{Symmetric groups}~\label{subsec:symmetric}

There is an obvious group structure on $\Aut[\Fin[n]]$ given by identity,
composition, and inverse. This is the symmetric group $S_n$ on $n$ symbols. In
the rest of the section we will construct a convenient presentation of this
group.

\todo{Reference T-algebra presentations as coequalisers (Mac Lane 6.7)}

First, we formally define a presentation of a group.

\begin{definition}
  A presentation of a group $G$ is a type $FR$ such that \ldots is a coequaliser.
\end{definition}

\vc{this is just a rough draft for now}

\begin{definition}
  Here we'll define a type family $FT n$ to be a free group generated by a type $T n$ with the following constructors
  \begin{align*}
     & cancel : i : \Fin[\suc[n]] \to T                                         \\
     & swap : (i : \Fin[\suc[n]]) \to (j : \Fin[\suc[n]]) \to (i + 1 < j) \to T \\
     & braid : (i : \Fin[n]) \to T                                              \\
  \end{align*}
\end{definition}

\begin{definition}[Adjacent transposition]
  \begin{align*}
    transpose : (n : \Nat) \to (k : \Fin[\suc[n]]) \to \Aut[\Fin[\suc[n]]]
  \end{align*}
  by double induction on $n$ and $k$, where
  \begin{align*}
    transpose (n) (0)             & = \lambda
    0 \to 1;
    1 \to 0;
    m \to m
    \\
    transpose (\suc[n]) (\suc[k]) & = \lambda
    0 \to 0;
    S m \to S ((transpose (n)(k)) m)
  \end{align*}
\end{definition}

\begin{proposition}
  $FR$ is a presentation of the symmetrc group, with $f$ and $g$ as follows:
  \begin{align*}
    f (inr (cancel (i))) & = inr (i :: i :: [])                     \\
    f (inl (cancel (i))) & = inl (i :: i :: [])                     \\
    f (inr (swap (i,j))) & = i :: j :: i :: j :: []                 \\
    f (braid (i))        & = i :: S i :: i :: S i :: i :: S i :: []
  \end{align*}
  and
  \begin{align*}
    g (inl (\_)) & = inl ([]) \\
    g (inr (\_)) & = inr ([])
  \end{align*}
\end{proposition}

We now introduce a presentation of the symmetric group.


\begin{theorem}
  For all $n : \Nat$,
  \[
    Sn(n) \eqv \Lehmer[n]
  \]
\end{theorem}

\subsection{Symmetric Monoidal structure}

$\UFin$ is a symmetric monoidal groupoid with unit and tensor given by

\begin{definition}
  \begin{align*}
    I           & \defeq F_{0}    \\
    X \otimes Y & \defeq X \sqcup Y
  \end{align*}
\end{definition}

with unitors, associator, and braiding satisfying Mac Lane's triangle, pentagon, and hexagon coherence laws, and
symmetry. It is the free symmetric monoidal groupoid on one generator.

%%% Local Variables:
%%% mode: latex
%%% TeX-master: "main"
%%% fill-column: 120
%%% End:

\section{Equivalence between \texorpdfstring{$\PiPlusLang$}{Pi} and \texorpdfstring{$\UFin$}{UFin}}~\label{sec:equivalence}

In this section, we use the semantics developed in the previous sections define a fragment $\PiPlusLang$ of $\PiLang$
and a normalised fragment of it $\PiHatLang$, interpret to $\UFin$ and back, and translate between the languages.

We only present the types and 1-combinators in the language, and the full set of 2-combinators is listed in the appendix
and the accompanying Agda code.

\vc{We can translate types to types, terms to terms, theorem: preserves types, preserves equations.}

\vc{Define maps quote/eval on 0,1,2-cells, and show commuting diagrams.}

\subsection{$\PiHatLang$}

We first present a language $\PiHatLang$ based on the theory of symmetric groups developed in~\cref{sec:finite}. The
types in the language are unary natural numbers, and combinators are allowed to perform adjacent transpositions.

\begin{figure}[t]
  {\scalebox{\scalef}{$
        \begin{array}{rrcll}
          \idc :   & n             & \isoh & n             & : \idc   \\
          \swapc : & \suc[\suc[n]] & \isoh & \suc[\suc[n]] & : \swapc \\
        \end{array}
      $}}

  {\scalebox{\scalef}{
      \Rule{}
      {\jdg{}{}{c_1 : n \isoh m} \quad \vdash c_2 : m \isoh o}
      {\jdg{}{}{c_1 \fatsemi c_2 : n \isoh o}}
      {}

      \Rule{}
      {\jdg{}{}{c : n \isoh m}}
      {\jdg{}{}{\oplus(c) : \suc[n] \isoh \suc[m]}}
      {}
    }}
  \caption{$\PiHatLang$ syntax}
  \label{fig:pihat}
\end{figure}

\begin{proposition}
  We can form a weak 2-category $\PiHatCat$ with
  \begin{itemize}
    \item natural numbers for 0-cells,
    \item for $n, m : \Nat$, a collection of 1-cells $n \isoh m$,
    \item for $p, q : n \isoh m$, a collection of 2-cells $p \Isoh q$.
  \end{itemize}
\end{proposition}

\begin{proposition}
  There is a symmetric monoidal structure on $\PiHatCat$, with $0$ for the unit, and addition for the tensor.
\end{proposition}

Then we establish the completeness of $\PiHatLang$ with respect to $\UFin$. We define $\evalt$ and $\quotet$ for 0, 1,
and 2-cells.

\begin{definition}
  \begin{align*}
    \evalt_{0} & : \UHat \to \UFin                                                          \\
    \evalt_{1} & : (c : t_{1} \isoh t_{2}) \to \evalh_{0}(t_{1}) \id \evalh_{0}(t_{2})      \\
    \evalt_{2} & : (\alpha : c_{1} \Isoh c_{2}) \to \evalt_{1}(c_{1}) \id \evalt_{1}(c_{2}) \\
  \end{align*}
\end{definition}

\begin{definition}
  \begin{align*}
    \quotet_{0} & : \UFin \to \UHat                                                            \\
    \quotet_{1} & : (p : X_{1} \id X_{2}) \to \quoteh_{0}(X_{1}) \isoh \quoteh_{0}(X_{2})      \\
    \quotet_{2} & : (\alpha : p_{1} \id p_{2}) \to \quoteh_{1}(p_{1}) \Isoh \quotet_{1}(p_{2}) \\
  \end{align*}
\end{definition}

\begin{proposition}
  $\evalt/\quotet$ give a symmetric monoidal biequivalence between $\PiHatCat$ and $\UFin$.
\end{proposition}

\subsection{$\PiPlusLang$}

Now, we present the additive fragment of $\PiLang$, called $\PiPlusLang$.

\begin{figure}[t]
  {\scalebox{\scalef}{$
        \begin{array}{rrcll}
          \idc :     & A           & \iso & A           & : \idc     \\
          \identlp : & \zerot + A  & \iso & A           & : \identrp \\
          \swapp :   & A + B       & \iso & B + A       & : \swapp   \\
          \assoclp : & A + (B + C) & \iso & (A + B) + C & : \assocrp \\ [1.5ex]
        \end{array}$}}

  {\scalebox{\scalef}{
      \Rule{}
      {\jdg{}{}{c_1 : A \iso B} \quad \vdash c_2 : B \iso C}
      {\jdg{}{}{c_1 \fatsemi c_2 : A \iso C}}
      {}

      \Rule{}
      {\jdg{}{}{c_1 : A \iso B} \quad \vdash c_2 : C \iso D}
      {\jdg{}{}{c_1 \oplus c_2 : A + C \iso B + D}}
      {}
    }}
  \caption{$\PiPlusLang$ syntax}
  \label{fig:piplus}
\end{figure}

\begin{proposition}
  We can form a weak 2-category $\PiPlusCat$ with
  \begin{itemize}
    \item $\PiPlusLang$ types for 0-cells,
    \item for $X, Y : \UPlus$, a collection of 1-cells $X \iso Y$,
    \item for $p, q : X \isoh Y$, a collection of 2-cells $p \Iso q$.
  \end{itemize}
\end{proposition}

\begin{proposition}
  There is a symmetric monoidal structure on $\PiPlusCat$, with $\zerot$ for the unit, and $+$ for the tensor.
\end{proposition}

We normalise $\PiPlusLang$ to $\PiHatLang$ and back, by giving $\evalh/\quoteh$ maps for 0, 1, and 2-cells.

\begin{definition}
  \begin{align*}
    \evalh_{0} & : \UPlus \to \UHat                                                          \\
    \evalh_{1} & : (c : t_{1} \iso t_{2}) \to \evalh_{0}(t_{1}) \isoh \evalh_{0}(t_{2})      \\
    \evalh_{2} & : (\alpha : c_{1} \Iso c_{2}) \to \evalh_{1}(c_{1}) \Isoh \evalh_{1}(c_{2}) \\
  \end{align*}
\end{definition}

\begin{definition}
  \begin{align*}
    \quoteh_{0} & : \UHat \to \UPlus                                                            \\
    \quoteh_{1} & : (p : X_{1} \isoh X_{2}) \to \quoteh_{0}(X_{1}) \iso \quoteh_{0}(X_{2})      \\
    \quoteh_{2} & : (\alpha : p_{1} \Isoh p_{2}) \to \quoteh_{1}(p_{1}) \Iso \quoteh_{1}(p_{2}) \\
  \end{align*}
\end{definition}

\begin{proposition}
  $\evalh/\quoteh$ give a symmetric monoidal biequivalence between $\PiPlusCat$ and $\PiHatCat$.
\end{proposition}

\subsection{$\PiLang$}

Finally, We show how to translate $\PiLang$ programs to $\PiHatLang$ programs.

\begin{definition}
  \begin{align*}
    \evalt_{0} & : U \to \Nat                                               \\
    \evalt_{1} & : (c : X \iso Y) \to \evalt_{0}(X) \iso \evalt_{0}(Y)      \\
    \evalt_{2} & : (\alpha : p \Iso q) \to \evalt_{1}(p) \Iso \evalt_{1}(q) \\
  \end{align*}
\end{definition}

To normalise a $\PiLang$ circuit, we translate it to $\PiHatLang$ and quote it to $\PiPlusLang$.

\begin{definition}
  \begin{align*}
    \normt_{0} & : U \to \UPlus                                        \\
    \normt_{0} & = \quoteh_{0} \comp \evalt_{0}                        \\
    \\
    \normt_{1} & : (c : X \iso Y) \to \normt_{0}(X) \iso \normt_{0}(Y) \\
    \normt_{1} & = \quoteh_{1} \comp \evalt_{1}                        \\
  \end{align*}
\end{definition}

%%% Local Variables:
%%% mode: latex
%%% TeX-master: "main"
%%% fill-column: 120
%%% End:

\section{Discussion \& Related Work}~\label{sec:discussion}

In this paper, we \ldots


Pi types -- Natural number -- Finite sets
1-combinator -- Generators of Sn -- 1-paths
2-combinator -- Relations of Sn -- 2-paths

In HoTT, univalence characterises the path type in the universe as equivalences of types. The map $\term{idtoeqv} : A
\id_{\UU} B \to A \eqv B$ can be easily constructed using path induction. The term $\term{ua} : A \eqv B \to A \id_{\UU}
B$, its computation rule $\term{ua-\beta} : (e : A \eqv B) \to \term{idtoeqv}(\term{ua}(e)) \id e$, and its
extensionality rule $\term{ua-\eta} : (p : A \id_{\UU} B) \to p == \term{ua}(\term{idtoeqv(p)})$ are generally added as
postulates when formalising in Agda. Together, $\term{ua}$ and $\term{ua}-\beta$ give the full univalence axiom $(A \eqv
B) \eqv (A \id_{\UU} B)$.

% Let's think of the $\PiLang$ combinators as describing the inhabitants of the dentity type of finite types. 

By giving a computable presentation for a univalent subuniverse, we are able to describe the path space of it
syntactically, by giving a complete equational axiomatisation of the equivalences between types in the subuniverse. 
% By the property of being univalent, this subuniverse gives a model of the univalence axiom. 
The $\term{idtoeqv}$ corresponds to giving a denotation for a program (1-combinator), which is easily done by induction.
The $\term{ua}$ map corresponds to synthesing a program from an equivalence (which, in general, is of course
undecidable~\cite{krogmeierDecidableSynthesisPrograms2020}). In case of reversible boolean circuits, it is decidable, as
we have shown, but still far from trivial, which matches the need to assert the existence of $\term{ua}$ without giving
a constructive argument. Then, the computation rule $\term{ua-\beta}$ expresses the fact that program synthesis is
sound, while $\term{ua-\eta}$ corresponds to the soundess of the equational theory ($\PiLang$ 2-combinators). Thus, we
this suggests a new computational interpretation of the univalence principle, which provides an intution on why certain
constructions are hard (or impossible in general case).

We could present a dependent type theory for the topos $\SetFin$, with an identity type for terms (generated by
$\refl$), and one for types (generated by $\PiLang$ combinators). We can't talk about universes in $\SetFin$ since it
doesn't have one, but we could show externally that it satisfies univalence.

Our work lies at the intersection of programming language theory, category theory, group theory, rewriting theory, and
formalised mathematics. We review related work in the literature for each topic.

\paragraph{Algebraic Theories} In universal algebra, algebraic theories are used to describe algebraic structures, such
as groups or rings. A specific group or ring is a model of the appropriate algebraic theory. Algebraic theories are
usually \emph{presented} in terms of logical syntax, that is, as first-order theories whose signatures allow only
functional symbols, and whose axioms are universally quantified equations. In his seminal
thesis~\cite{lawvereFUNCTORIALSEMANTICSALGEBRAIC1963}, Lawvere defined a presentation-free categorical notion of
universal algebraic structure, called a Lawvere theory.

Programming Languages, such as the $\lambda$-calculus, can be viewed as algebraic structures with variable-binding
operators, which can be formalised using second-order algebraic theories~\cite{fioreSecondOrderAlgebraicTheories2010},
or algebraic theories with closed structure~\cite{hylandClassicalLambdaCalculus2017}, called $\lambda$-theories, making
the $\lambda$-calculus the presentation of the initial $\lambda$-theory $\Lambda$.

Our family of reversible languages have been presented as first-order algebraic
2-theories~\cite{cohenCoherenceRewriting2theories2009,bekeCategorificationTermRewriting2011,yanofskySyntaxCoherence2000},
which are a categorification of algebraic theories. The types $\zerot$ and $\onet$ are nullary function symbols, the
type formers $+$ and $\times$ are binary function symbols, the 1-combinators are invertible reduction rules, and the
2-combinators are equations or coherence diagrams of compositions of reduction rules. Just like models of Lawvere
theories are given by algebras of (finitary) monads on $\SetCat$, models of 2-theories are given by algebras of 2-monads
on $\CatCat$. The particular one we're interested in here is the free symmetric monoidal completion 2-monad.

\paragraph{Free Symmetric Monoidal Category} The forgetful functor from $\SymMonCat$, the 2-category of (small)
symmetric monoidal categories, strong symmetric monoidal functors, and monoidal natural transformations, to the
2-category $\CatCat$, has a left adjoint giving the free symmetric monoidal category $\FSM[\CCat{C}]$ on a category
$\CCat{C}$. This is a 2-monad on $\CatCat$~\cite{blackwellTwodimensionalMonadTheory1989}, whose algebras are (strict)
symmetric monoidal categories. Its construction is known in the literature~\cite{abramskyAbstractScalarsLoops2005}.
Concretely, the objects of $\FSM[\CCat{C}]$ are given by lists of objects of $\CCat{C}$, that is, a pair $(n:\Nat, A:[n]
\to \CCat{C}_{0})$. An morphism between $(n,A)$ and $(n,B)$ is given by a pair $(\pi,\lambda)$ where $\pi$ is a
permutation of $[n]$, and $\lambda_{i} : A_{i} \to B_{\pi(i)}$ for $1 \leq i \leq n$. Abstractly, this is given by the
Grothendieck construction $\int F$ of the functor $F : \BFin \to \CatCat$ from the groupoid of finite sets and
bijections to $\CatCat$, assigning each natural number $n$ to the $n$-power $C^{n}$ of $C$, and each permutation on
$[n]$ inducing an endofunctor on $C^{n}$ by action. $\BFin$ is the free symmetric monoidal category (groupoid) on one
generator, $\FSM[\unit]$. The free symmetric monoidal category has been used to study
concurrency~\cite{hylandSymmetricMonoidalSketches2004}, petri nets~\cite{baezCategoriesNets2021}, combinatorial
structures~\cite{fioreCartesianClosedBicategory2008}, quantum mechanics~\cite{abramskyAbstractScalarsLoops2005},
bicategorical models of (differential) linear logic~\cite{melliesTemplateGamesDifferential2019}. 

Coherence and normalisation problems for monoids in constructive type theory using coherence for monoidal categories was
studied in~\cite{beylinExtractingProofCoherence1996}. In HoTT, coherence for the free monoidal groupoid over a groupoid
and the proof of its universal property has been considered in~\cite{piceghelloCoherenceMonoidalGroupoids2020}. 

Free commutative monoids in type theory have been studied in~\cite{gylterudMultisetsTypeTheory2020}, and using HoTT
in~\cite{choudhuryFinitemultisetConstructionHoTT2019}. The free symmetric monoidal groupoid $\FSM[A]$ over a groupoid
$A$ can be given by $\dsum{X:\UFin}{A^{X}}$, or it can be presented as an algebraic 2-theory using 1-HITs. These HITs
and the proof of their universal property have been considered
in~\cite*{piceghelloCoherenceSymmetricMonoidal2019,choudhuryFinitemultisetConstructionHoTT2019}.

The proof of the universal property of the $\FSM$ is asserted by appealing to Mac Lane's coherence theorem for symmetric
monoidal categories, and using the fact that the finite symmetric group $\Sn$ encodes the permutation group on a finite
set. The existence of the proof is folklore, and we have produced a new proof of it while working in constructive type
theory.

\paragraph{Curry-Howard-Lambek correspondence} In~\citet{curryCurryEssaysCombinatory1980}, Lambek extended the
Curry-Howard correspondence to cartesian-closed categories. In this work, we have established a correspondence between a
fragment of the $\PiLang$ family of reversible programming languages and symmetric monoidal groupoids. The Curry-Howard
part of this correspondence with reversible logic was established in~\cite{sparksSuperstructuralReversibleLogic2014}.

\paragraph{Rewriting} We presented a rewriting system for the Coxeter relations for $\Sn$ to solve its word problem.

There exists an algorithm, due to~\citet{knuthSimpleWordProblems1970}, that, when succeeds, constructs a well-behaved
rewriting system for an arbitrary finite set of (undirected) equations. It did not work for us, producing too many
equations, and proving correctness and termination was intractable.

We chose to encode permutations as adjacent transpositions corresponding to words in $\Sn$, since that closely
corresponds to the syntax of $\PiLang$ combinators. Other representations of permutations would give a different
denotational semantics for $\PiLang$ combinators. Permutations can be encoded as listed vectors or matrices, inductively
generated trees (Motzkin trees), Young diagrams, or String diagrams, but the difficulty of formalising them in type
theory varies depending on the encoding.

Rewriting systems and word problems have a long history of being formalised in proof assistants. In the recent past,
higher order rewriting systems have been formalised in proof assistants like {homotopy.io} and Lean and Coq. The use of
HoTT to study rewriting has been considered in~\cite{krausCoherenceWellFoundednessTaming2020}.

\paragraph{Computational group theory} Coxeter relations are used in computational group theory to study XXX problems.

\paragraph{Univalent Fibrations} Univalent Fibrations were introduced by~\citet*{kapulkinUnivalenceSimplicialSets2018},
to build a model of Voevodsky's \emph{univalence} principle in simplicial sets.
\citet{christensenCharacterizationUnivalentFibrations2015} studied characterisations of univalent fibrations using the
$\BAut$ construction. Univalent typoids~\cite{petrakisUnivalentTypoids2019a} are a different presentation of univalent
subuniverses.

Coherence problems in type theory, coherence via Well-Foundedness.

Formalised proofs of Mac Lane's coherence theorem.

Applications of FSMG and history of the coherence theorems.

Other proofs of coherence theorems, Joyal-Street.

Pi has other extensions (fractional/negative/recursive types).
What are the free X monoidal structures they're describing?

Using our presentation of $S_{n}$, we can construct the Eilenberg-Maclane space (using a HIT) $K(S_{n},1)$. Then, it
should be true that $\UFin \eqv \sqcup_{n:\Nat} K(S_{n},1)$. This is future work.

Other applications of symmetric groups.

Actions of symmetric groups, permutation groupoids.

%%% Local Variables:
%%% mode: latex
%%% TeX-master: "main"
%%% fill-column: 120
%%% End:


%% Acks: Chao-Hong and Jacques

\bibliographystyle{ACM-Reference-Format}
\bibliography{2dtypesZot}

\end{document}

%%% Local Variables:
%%% mode: latex
%%% TeX-master: t
%%% fill-column: 120
%%% End:
