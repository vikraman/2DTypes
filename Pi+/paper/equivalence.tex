\section{Equivalence between \texorpdfstring{$\PiPlusLang$}{Pi} and \texorpdfstring{$\UFin$}{UFin}}~\label{sec:equivalence}

In this section, we use the semantics developed in the previous sections define a fragment $\PiPlusLang$ of $\PiLang$
and a normalised fragment of it $\PiHatLang$, interpret to $\UFin$ and back, and translate between the languages.

We only present the types and 1-combinators in the language, and the full set of 2-combinators is listed in the appendix
and the accompanying Agda code.

\vc{We can translate types to types, terms to terms, theorem: preserves types, preserves equations.}

\vc{Define maps quote/eval on 0,1,2-cells, and show commuting diagrams.}

\subsection{$\PiHatLang$}

We first present a language $\PiHatLang$ based on the theory of symmetric groups developed in~\cref{sec:finite}. The
types in the language are unary natural numbers, and combinators are allowed to perform adjacent transpositions.

\begin{figure}[t]
  {\scalebox{\scalef}{$
        \begin{array}{rrcll}
          \idc :   & n             & \isoh & n             & : \idc   \\
          \swapc : & \suc[\suc[n]] & \isoh & \suc[\suc[n]] & : \swapc \\
        \end{array}
      $}}

  {\scalebox{\scalef}{
      \Rule{}
      {\jdg{}{}{c_1 : n \isoh m} \quad \vdash c_2 : m \isoh o}
      {\jdg{}{}{c_1 \fatsemi c_2 : n \isoh o}}
      {}

      \Rule{}
      {\jdg{}{}{c : n \isoh m}}
      {\jdg{}{}{\oplus(c) : \suc[n] \isoh \suc[m]}}
      {}
    }}
  \caption{$\PiHatLang$ syntax}
  \label{fig:pihat}
\end{figure}

\begin{proposition}
  We can form a weak 2-category $\PiHatCat$ with
  \begin{itemize}
    \item natural numbers for 0-cells,
    \item for $n, m : \Nat$, a collection of 1-cells $n \isoh m$,
    \item for $p, q : n \isoh m$, a collection of 2-cells $p \Isoh q$.
  \end{itemize}
\end{proposition}

\begin{proposition}
  There is a symmetric monoidal structure on $\PiHatCat$, with $0$ for the unit, and addition for the tensor.
\end{proposition}

Then we establish the completeness of $\PiHatLang$ with respect to $\UFin$. We define $\evalt$ and $\quotet$ for 0, 1,
and 2-cells.

\begin{definition}
  \begin{align*}
    \evalt_{0} & : \UHat \to \UFin                                                          \\
    \evalt_{1} & : (c : t_{1} \isoh t_{2}) \to \evalh_{0}(t_{1}) \id \evalh_{0}(t_{2})      \\
    \evalt_{2} & : (\alpha : c_{1} \Isoh c_{2}) \to \evalt_{1}(c_{1}) \id \evalt_{1}(c_{2}) \\
  \end{align*}
\end{definition}

\begin{definition}
  \begin{align*}
    \quotet_{0} & : \UFin \to \UHat                                                            \\
    \quotet_{1} & : (p : X_{1} \id X_{2}) \to \quoteh_{0}(X_{1}) \isoh \quoteh_{0}(X_{2})      \\
    \quotet_{2} & : (\alpha : p_{1} \id p_{2}) \to \quoteh_{1}(p_{1}) \Isoh \quotet_{1}(p_{2}) \\
  \end{align*}
\end{definition}

\begin{proposition}
  $\evalt/\quotet$ give a symmetric monoidal biequivalence between $\PiHatCat$ and $\UFin$.
\end{proposition}

\subsection{$\PiPlusLang$}

Now, we present the additive fragment of $\PiLang$, called $\PiPlusLang$.

\begin{figure}[t]
  {\scalebox{\scalef}{$
        \begin{array}{rrcll}
          \idc :     & A           & \iso & A           & : \idc     \\
          \identlp : & \zerot + A  & \iso & A           & : \identrp \\
          \swapp :   & A + B       & \iso & B + A       & : \swapp   \\
          \assoclp : & A + (B + C) & \iso & (A + B) + C & : \assocrp \\ [1.5ex]
        \end{array}$}}

  {\scalebox{\scalef}{
      \Rule{}
      {\jdg{}{}{c_1 : A \iso B} \quad \vdash c_2 : B \iso C}
      {\jdg{}{}{c_1 \fatsemi c_2 : A \iso C}}
      {}

      \Rule{}
      {\jdg{}{}{c_1 : A \iso B} \quad \vdash c_2 : C \iso D}
      {\jdg{}{}{c_1 \oplus c_2 : A + C \iso B + D}}
      {}
    }}
  \caption{$\PiPlusLang$ syntax}
  \label{fig:piplus}
\end{figure}

\begin{proposition}
  We can form a weak 2-category $\PiPlusCat$ with
  \begin{itemize}
    \item $\PiPlusLang$ types for 0-cells,
    \item for $X, Y : \UPlus$, a collection of 1-cells $X \iso Y$,
    \item for $p, q : X \isoh Y$, a collection of 2-cells $p \Iso q$.
  \end{itemize}
\end{proposition}

\begin{proposition}
  There is a symmetric monoidal structure on $\PiPlusCat$, with $\zerot$ for the unit, and $+$ for the tensor.
\end{proposition}

We normalise $\PiPlusLang$ to $\PiHatLang$ and back, by giving $\evalh/\quoteh$ maps for 0, 1, and 2-cells.

\begin{definition}
  \begin{align*}
    \evalh_{0} & : \UPlus \to \UHat                                                          \\
    \evalh_{1} & : (c : t_{1} \iso t_{2}) \to \evalh_{0}(t_{1}) \isoh \evalh_{0}(t_{2})      \\
    \evalh_{2} & : (\alpha : c_{1} \Iso c_{2}) \to \evalh_{1}(c_{1}) \Isoh \evalh_{1}(c_{2}) \\
  \end{align*}
\end{definition}

\begin{definition}
  \begin{align*}
    \quoteh_{0} & : \UHat \to \UPlus                                                            \\
    \quoteh_{1} & : (p : X_{1} \isoh X_{2}) \to \quoteh_{0}(X_{1}) \iso \quoteh_{0}(X_{2})      \\
    \quoteh_{2} & : (\alpha : p_{1} \Isoh p_{2}) \to \quoteh_{1}(p_{1}) \Iso \quoteh_{1}(p_{2}) \\
  \end{align*}
\end{definition}

\begin{proposition}
  $\evalh/\quoteh$ give a symmetric monoidal biequivalence between $\PiPlusCat$ and $\PiHatCat$.
\end{proposition}

\subsection{$\PiLang$}

Finally, we show how to translate $\PiLang$ programs to $\PiPlusLang$ programs. $\PiLang$ has two symmetric monoidal
structure, the additive and the multiplicative one. In $\PiPlusLang$, the additive structure can be used to build the
multiplicative structure, by defining multiplication as repeated addition.

\begin{definition}
  \begin{align*}
    O \times Y           & \defeq O                       \\
    I \times Y           & \defeq Y                       \\
    (X_1 + X_2) \times Y & \defeq X_1 \times Y + X_2 \times Y
  \end{align*}
\end{definition}

\begin{proposition}
  $\PiPlusCat$ has a symmetric monoidal structure with $I$ for the unit and $\times$ for the tensor product. With the
  two tensor products and the distributivity combinator, $\PiPlusCat$ has a rig structure.
\end{proposition}

Using this rig structure, we translate $\PiLang$ to $\PiPlusLang$.

\begin{definition}
  \begin{align*}
    \evalt_{0} & : U \to \UPlus                                             \\
    \evalt_{1} & : (c : X \iso Y) \to \evalt_{0}(X) \iso \evalt_{0}(Y)      \\
    \evalt_{2} & : (\alpha : p \Iso q) \to \evalt_{1}(p) \Iso \evalt_{1}(q) \\
  \end{align*}
\end{definition}

To normalise a $\PiLang$ circuit, we translate it to $\PiHatLang$, compute a permutation, and quote back it to
$\PiPlusLang$.

\begin{definition}
  \begin{align*}
    \normt_{0} & : U \to \UPlus                                        \\
    \normt_{0} & = \quoteh_{0} \comp \evalt_{0}                        \\
    \\
    \normt_{1} & : (c : X \iso Y) \to \normt_{0}(X) \iso \normt_{0}(Y) \\
    \normt_{1} & = \quoteh_{1} \comp \evalt_{1}                        \\
  \end{align*}
\end{definition}

%%% Local Variables:
%%% mode: latex
%%% TeX-master: "main"
%%% fill-column: 120
%%% End:
