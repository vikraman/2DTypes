\section{Equivalence between \texorpdfstring{$\PiPlusLang$}{Pi} and \texorpdfstring{$\UFin$}{UFin}}~\label{sec:equivalence}

\subsection{$\PiLang$ to $\PiPlusLang$}

We show how to translate $\PiLang$ programs to $\PiPlusLang$ programs.

\subsection{$\PiPlusLang$ to $\PiHatLang$}

We normalise $\PiPlusLang$ to $\PiHatLang$, and show that they are equivalent.

\begin{definition}
  \begin{align*}
    \evalh_{0} & : \UPlus n \to \UHat n                                                      \\
    \evalh_{1} & : (c : t_{1} \iso t_{2}) \to \evalh_{0}(t_{1}) \isoh \evalh_{0}(t_{2})      \\
    \evalh_{2} & : (\alpha : c_{1} \Iso c_{2}) \to \evalh_{1}(c_{1}) \Isoh \evalh_{1}(c_{2}) \\
  \end{align*}
\end{definition}

\begin{definition}
  \begin{align*}
    \quoteh_{0} & : \UHat n \to \UPlus n                                                        \\
    \quoteh_{1} & : (p : X_{1} \isoh X_{2}) \to \quoteh_{0}(X_{1}) \iso \quoteh_{0}(X_{2})      \\
    \quoteh_{2} & : (\alpha : p_{1} \Isoh p_{2}) \to \quoteh_{1}(p_{1}) \Iso \quoteh_{1}(p_{2}) \\
  \end{align*}
\end{definition}

\subsection{$\PiHatLang$ to $\UFin$}

Then we establish the completeness of $\PiHatLang$ with respect to $\UFin$.

\begin{definition}
  \begin{align*}
    \evalt_{0} & : \UHat n \to \UFin[n]                                                     \\
    \evalt_{1} & : (c : t_{1} \isoh t_{2}) \to \evalh_{0}(t_{1}) \id \evalh_{0}(t_{2})      \\
    \evalt_{2} & : (\alpha : c_{1} \Isoh c_{2}) \to \evalt_{1}(c_{1}) \id \evalt_{1}(c_{2}) \\
  \end{align*}
\end{definition}

\begin{definition}
  \begin{align*}
    \quotet_{0} & : \UFin[n] \to \UHat n                                                       \\
    \quotet_{1} & : (p : X_{1} \id X_{2}) \to \quoteh_{0}(X_{1}) \isoh \quoteh_{0}(X_{2})      \\
    \quotet_{2} & : (\alpha : p_{1} \id p_{2}) \to \quoteh_{1}(p_{1}) \Isoh \quotet_{1}(p_{2}) \\
  \end{align*}
\end{definition}

We use this to establish completeness for $\PiPlusLang$.

\begin{definition}
  \begin{align*}
    \evalt_{0} & : \UPlus n \to \UFin[n]                                                   \\
    \evalt_{1} & : (c : t_{1} \iso t_{2}) \to \evalt_{0}(t_{1}) \id \evalt_{0}(t_{2})      \\
    \evalt_{2} & : (\alpha : c_{1} \Iso c_{2}) \to \evalt_{1}(c_{1}) \id \evalt_{1}(c_{2}) \\
  \end{align*}
\end{definition}

\begin{definition}
  \begin{align*}
    \quotet_{0} & : \UFin[n] \to U n                                                          \\
    \quotet_{1} & : (p : X_{1} \id X_{2}) \to \quotet_{0}(X_{1}) \iso \quotet_{0}(X_{2})      \\
    \quotet_{2} & : (\alpha : p_{1} \id p_{2}) \to \quotet_{1}(p_{1}) \Iso \quotet_{1}(p_{2}) \\
  \end{align*}
\end{definition}

\begin{theorem}
  ${\quotet}/{\evalt}$ gives a symmetric monoidal equivalence between $\UPlus$ and $\UFin$.
\end{theorem}

%%% Local Variables:
%%% mode: latex
%%% TeX-master: "main"
%%% fill-column: 120
%%% End:
