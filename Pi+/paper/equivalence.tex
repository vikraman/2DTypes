\section{Going from \texorpdfstring{$\PiLang$}{Pi} to \texorpdfstring{$\UFin$}{UFin} and back}~\label{sec:equivalence}

% https://q.uiver.app/?q=WzAsNCxbMCwwLCJcXFBpTGFuZyJdLFsyLDAsIlxcUGlQbHVzTGFuZyJdLFs0LDAsIlxcUGlIYXRMYW5nIl0sWzYsMCwiXFxVRmluIl0sWzAsMSwiXFxldmFsdCJdLFsxLDIsIlxcZXZhbGgiLDAseyJjdXJ2ZSI6LTR9XSxbMiwzLCJcXGV2YWx0IiwwLHsiY3VydmUiOi00fV0sWzIsMSwiXFxxdW90ZWgiLDAseyJjdXJ2ZSI6LTR9XSxbMywyLCJcXHF1b3RldCIsMCx7ImN1cnZlIjotNH1dXQ==
\[\begin{tikzcd}
    \PiLang && \PiPlusLang && \PiHatLang && \UFin
    \arrow["\evalt", from=1-1, to=1-3]
    \arrow["\evalp", curve={height=-24pt}, from=1-3, to=1-5]
    \arrow["\evalh", curve={height=-24pt}, from=1-5, to=1-7]
    \arrow["\quotep", curve={height=-24pt}, from=1-5, to=1-3]
    \arrow["\quoteh", curve={height=-24pt}, from=1-7, to=1-5]
  \end{tikzcd}\]

In this section, we consider the additive fragment $\PiPlusLang$ of $\PiLang$. This is the language that we interpret to
$\UFin$ and back, using the tools developed in the previous sections. Further, we go through an intermediate step of the
language $\PiHatLang$, which is a simplified variant of $\PiPlusLang$ that uses adjacent transpositions for combinators,
but still preserving all the required structure.

We present the types and 1-combinators of $\PiPlusLang$ and $\PiHatLang$ in~\cref*{fig:piplus,fig:pihat} respectively,
eliding the 2-combinators for brevity. \note{We enforce that there is a unique 2-combinator between compatible
  1-combinators, by introducing a level 3, which is trivial in the same way as in the case of $\PiLang$, as described
  before.} They can be found in the appendix and in the accompanying Agda code.

The translations between the languages are defined separately on types, 1-combinators, and 2-combinators. Following the
terminology of Normalisation by Evaluation, the translations from the left to the right, going from the syntax towards
the semantics, are called $\evalt$ and the translations the other way are called $\quotet$.

To state our results formally, we organise the syntax for each language using a technical device, called a syntactic
category. Since the language have higher-dimensional structure, the syntactic categories here are actually
(locally-strict) weak 2-categories. We define them formally in the appendix, and only state our results here.

For each of the $\PiLang$, $\PiPlusLang$ and $\PiHatLang$ languages, their syntactic categories, respectively $\PiCat$,
$\PiPlusCat$ and $\PiHatCat$, have 0-cells for types, 1-cells for 1-combinators, and 2-cells for 2-combinators. We can
show that these are $(2,0)$-categories, since all the 1-cells and 2-cells are invertible. They are also locally-strict,
or locally-posetal, because there is at most one 2-cell between compatible 1-cells.

\begin{toappendix}
  \todo{What is U?}
  \begin{proposition}
    We can form a weak 2-category $\PiCat$ with
    \begin{itemize}
      \item $\PiLang$ types for 0-cells,
      \item for $X, Y : U$, a collection of 1-cells $X \iso Y$,
      \item for $p, q : X \iso Y$, a collection of 2-cells $p \Iso q$.
    \end{itemize}
  \end{proposition}

  \begin{proposition}
    We can form a weak 2-category $\PiPlusCat$ with
    \begin{itemize}
      \item $\PiPlusLang$ types for 0-cells,
      \item for $X, Y : \UPlus$, a collection of 1-cells $X \isop Y$,
      \item for $p, q : X \isop Y$, a collection of 2-cells $p \Isop q$.
    \end{itemize}
  \end{proposition}

  \begin{proposition}
    We can form a weak 2-category $\PiHatCat$ with
    \begin{itemize}
      \item $\PiHatCat$ types for 0-cells,
      \item for $X, Y : \UHat$, a collection of 1-cells $X \isoh Y$,
      \item for $p, q : X \isoh Y$, a collection of 2-cells $p \Isoh q$.
    \end{itemize}
  \end{proposition}
\end{toappendix}

We use the $\evalt/\quotet$ translation maps to construct functors between these categories. We only name the maps on
the 0, 1, and 2-cells -- the coherences hold by definition or by calculation, which is shown in our accompanying Agda
code. We use these functors to state our results establishing the equivalences between the languages.

\subsection{$\PiLang$ to $\PiPlusLang$}

First, we show how to translate $\PiLang$ programs to $\PiPlusLang$, which is the additive fragment of $\PiLang$. The
syntax is given in~\cref{fig:piplus}.

\begin{figure}[t]
  {\scalebox{\scalef}{$
        \begin{array}{rrcll}
          \idc :     & A           & \iso & A           & : \idc     \\
          \identlp : & \zerot + A  & \iso & A           & : \identrp \\
          \swapp :   & A + B       & \iso & B + A       & : \swapp   \\
          \assoclp : & A + (B + C) & \iso & (A + B) + C & : \assocrp \\ [1.5ex]
        \end{array}$}}

  {\scalebox{\scalef}{
      \Rule{}
      {\jdg{}{}{c_1 : A \iso B} \quad \vdash c_2 : B \iso C}
      {\jdg{}{}{c_1 \fatsemi c_2 : A \iso C}}
      {}

      \Rule{}
      {\jdg{}{}{c_1 : A \iso B} \quad \vdash c_2 : C \iso D}
      {\jdg{}{}{c_1 \oplus c_2 : A + C \iso B + D}}
      {}
    }}
  \caption{$\PiPlusLang$ syntax}
  \label{fig:piplus}
\end{figure}

$\PiLang$ has two 0-ary type constructors, and two binary type constructors -- the additive tensor product and the
multiplicative one. $\PiPlusLang$ has all the type constructors of $\PiLang$ except multiplication. However, we will
show how to recover the multiplicative structure, by defining multiplication as repeated addition. We encode $\times$ in
terms of $+$ as follows.

\begin{definition}[$\times : \UPlus \to \UPlus \to \UPlus$]
  \begin{align*}
    \zerot \times Y      & \defeq \zerot                      \\
    \onet \times Y       & \defeq Y                           \\
    (X_1 + X_2) \times Y & \defeq X_1 \times Y + X_2 \times Y
  \end{align*}
\end{definition}

\begin{proposition}
  There are two symmetric monoidal structures on $\PiPlusCat$, given by $(\zerot, +)$ and $(\onet, \times)$, with
  $\times$ distributing over $+$, giving it a rig structure, as described in \todo{cite earlier section}.
\end{proposition}
\todo{either prove it or cite something}

Using this rig structure, we translate $\PiLang$ to $\PiPlusLang$, constructing a rig functor from $\PiCat$ to
$\PiPlusCat$.

\begin{definition}
  \begin{align*}
    \evalt_{0} & : U \to \UPlus                                             \\
    \evalt_{1} & : (c : X \iso Y) \to \evalt_{0}(X) \iso \evalt_{0}(Y)      \\
    \evalt_{2} & : (\alpha : p \Iso q) \to \evalt_{1}(p) \Iso \evalt_{1}(q) \\
  \end{align*}
\end{definition}

\subsection{$\PiPlusLang$ to $\PiHatLang$}

Next, we show how to translate $\PiPlusLang$ programs to $\PiHatLang$ and back. $\PiHatLang$ is a simplified variant of
$\PiPlusLang$, with (unary) natural numbers for types, constructed by $\zero$ and $\suc$, 1-combinators generated by
adjacent transpositions, and an appropriate set of 2-combinators. We give the syntax, again omitting 2-combinators,
in~\cref{fig:pihat}. \vc{Something about PROPs.}

\begin{figure}[t]
  {\scalebox{\scalef}{$
        \begin{array}{rrcll}
          \idc :   & n             & \isoh & n             & : \idc   \\
          \swapc : & \suc[\suc[n]] & \isoh & \suc[\suc[n]] & : \swapc \\
        \end{array}
      $}}

  {\scalebox{\scalef}{
      \Rule{}
      {\jdg{}{}{c_1 : n \isoh m} \quad \vdash c_2 : m \isoh o}
      {\jdg{}{}{c_1 \fatsemi c_2 : n \isoh o}}
      {}

      \Rule{}
      {\jdg{}{}{c : n \isoh m}}
      {\jdg{}{}{\oplus(c) : \suc[n] \isoh \suc[m]}}
      {}
    }}
  \caption{$\PiHatLang$ syntax}
  \label{fig:pihat}
\end{figure}

As described, $\PiHatLang$ doesn't have a tensor product, but we can build it simply by adding up natural numbers.
However, we need to verify that this indeed equips $\PiHatCat$ with a symmetric monoidal strucuture.

\begin{proposition}
  $\PiHatCat$ has a symmetric monoidal structure, with the unit given by 0 and the tensor product given by natural
  number addition.
\end{proposition}

Using this symmetric monoidal structure, we translate from $\PiPlusLang$ to $\PiHatLang$.

\begin{definition}
  \begin{align*}
    \evalh_{0} & : \UPlus \to \UHat                                                          \\
    \evalh_{1} & : (c : t_{1} \iso t_{2}) \to \evalh_{0}(t_{1}) \isoh \evalh_{0}(t_{2})      \\
    \evalh_{2} & : (\alpha : c_{1} \Iso c_{2}) \to \evalh_{1}(c_{1}) \Isoh \evalh_{1}(c_{2}) \\
  \end{align*}
\end{definition}

To go back from $\PiHatLang$ to $\PiPlusLang$, we turn a natural number into a $\PiPlusLang$ type, using right-biased
addition, that is, the natural number $n$ gets mapped to the type $\onet + (\onet + (\onet + \ldots + \zerot))$. The
non-trivial part is coherently generating a $\PiPlusLang$ combinator from $\PiHatLang$ combinators, using adjacent swaps
to generate big swaps.

\begin{definition}
  \begin{align*}
    \quoteh_{0} & : \UHat \to \UPlus                                                            \\
    \quoteh_{1} & : (p : X_{1} \isoh X_{2}) \to \quoteh_{0}(X_{1}) \iso \quoteh_{0}(X_{2})      \\
    \quoteh_{2} & : (\alpha : p_{1} \Isoh p_{2}) \to \quoteh_{1}(p_{1}) \Iso \quoteh_{1}(p_{2}) \\
  \end{align*}
\end{definition}

\begin{proposition}
  $\evalh/\quoteh$ give a symmetric monoidal biequivalence between $\PiPlusCat$ and $\PiHatCat$.
\end{proposition}

\subsection{$\PiHatLang$ to $\UFin$}

Finally, we show how to interpret $\PiHatLang$ to $\UFin$, and go back from $\UFin$ to $\PiHatLang$.

\todo{Sn/Lehmer/Aut(Fin n) diagram here?}

Types in $\PiHatLang$ are interpreted as 0-cells in $\UFin$, that is, a natural number $n$ is mapped to $\Fin[n]$. The
1-combinators in $\PiHatLang$ are mapped to 1-paths in $\UFin$, that is, 1-loops $\Aut[\Fin[n]]$, or permutations, in
each connected component. Finally, 2-combinators are mapped to 2-paths between loops in $\UFin$.

\begin{definition}
  \begin{align*}
    \evalt_{0} & : \UHat \to \UFin                                                          \\
    \evalt_{1} & : (c : t_{1} \isoh t_{2}) \to \evalh_{0}(t_{1}) \id \evalh_{0}(t_{2})      \\
    \evalt_{2} & : (\alpha : c_{1} \Isoh c_{2}) \to \evalt_{1}(c_{1}) \id \evalt_{1}(c_{2}) \\
  \end{align*}
\end{definition}

0-cells in $\UFin$ are mapped to their cardinalities in $\PiHatLang$, 1-loops are decoded to words in $\Sn$ to generate
a sequence of adjacent transpositions, producing a 1-combinator in $\PiHatLang$. 2-paths are quoted back to
2-combinators in $\PiHatLang$.

\begin{definition}
  \begin{align*}
    \quotet_{0} & : \UFin \to \UHat                                                            \\
    \quotet_{1} & : (p : X_{1} \id X_{2}) \to \quoteh_{0}(X_{1}) \isoh \quoteh_{0}(X_{2})      \\
    \quotet_{2} & : (\alpha : p_{1} \id p_{2}) \to \quoteh_{1}(p_{1}) \Isoh \quotet_{1}(p_{2}) \\
  \end{align*}
\end{definition}

$\UFin$ as a 1-groupoid, is a (locally strict) weak $(2,0)$-category.

\begin{proposition}
  $\evalt/\quotet$ give a symmetric monoidal biequivalence between $\PiHatCat$ and $\UFin$.
\end{proposition}

\subsection{$\PiLang$ circuit normalization}

To normalise a $\PiLang$ circuit, we translate it to $\PiHatLang$, compute a permutation, and quote it back to
$\PiPlusLang$. Note that the normalisation happens in going from $\PiHatLang$ to $\UFin$ to $\PiHatLang$.

\begin{definition}
  \begin{align*}
    \normt_{0} & : U \to \UPlus                                        \\
    \normt_{0} & = \quoteh_{0} \comp \evalt_{0}                        \\
    \\
    \normt_{1} & : (c : X \iso Y) \to \normt_{0}(X) \iso \normt_{0}(Y) \\
    \normt_{1} & = \quoteh_{1} \comp \evalt_{1}                        \\
  \end{align*}
\end{definition}


%%% Local Variables:
%%% mode: latex
%%% TeX-master: "main"
%%% fill-column: 120
%%% End:
