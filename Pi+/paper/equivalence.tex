\section{Equivalence between \texorpdfstring{$\PiPlusLang$}{Pi} and \texorpdfstring{$\UFin$}{UFin}}~\label{sec:equivalence}

First, we normalise $\PiPlusLang$ to $\PiHatLang$, and show that they are equivalent.

Then we establish the completeness of $\PiHatLang$ with respect to $\UFin$.

We use this to establish completeness for $\PiPlusLang$.

Finally, we show how to translate $\PiLang$ programs to $\PiPlusLang$ programs.

\begin{definition}
  \begin{align*}
    \evalt_{0} & : U n \to \UFin[n] \\
    \evalt_{1} & : (c : t_{1} \iso t_{2}) \to \evalt_{0}(t_{1}) \id \evalt_{0}(t_{2}) \\
    \evalt_{2} & : (\alpha : c_{1} \Iso c_{2}) \to \evalt_{1}(c_{1}) \id \evalt_{1}(c_{2}) \\
  \end{align*}
\end{definition}

\begin{definition}
  \begin{align*}
    \quotet_{0} & : \UFin[n] \to U n \\
    \quotet_{1} & : (p : X_{1} \id X_{2}) \to \quotet_{0}(X_{1}) \iso \quotet_{0}(X_{2}) \\
    \quotet_{2} & : (\alpha : p_{1} \id p_{2}) \to \quotet_{1}(p_{1}) \Iso \quotet_{1}(p_{2}) \\
  \end{align*}
\end{definition}

\begin{theorem}
  ${\quotet}/{\evalt}$ gives a symmetric monoidal equivalence between $U$ and
  $\UFin$.
\end{theorem}

%%% Local Variables:
%%% mode: latex
%%% TeX-master: "main"
%%% End:
