\documentclass[a4paper]{article}

\usepackage{amssymb,mathrsfs,amsmath,amscd,amsthm}
\usepackage[mathcal]{euscript}
\usepackage{stmaryrd}
\usepackage[T1]{fontenc}
\usepackage[utf8]{inputenc}
\usepackage{graphics}
\usepackage{amsmath}

\newcommand{\N}{\mathbb{N}}
\newcommand{\la}{\langle}
\newcommand{\ra}{\rangle}

%%%%%%% makra do notacji

\newtheorem{theorem}{Theorem}[section]
\newtheorem{corollary}{Corollary}[theorem]
\newtheorem{lemma}[theorem]{Lemma}
\theoremstyle{definition}
\newtheorem{definition}{Definition}[section]
\theoremstyle{remark}
\newtheorem*{remark}{Remark}
\theoremstyle{example}
\newtheorem{example}{Example}


\begin{document}
\section{Permutations}
\subsection{Lehmer codes}

\begin{definition}
Let $n \in \N$ and a permutation $p$ be on a set $X = \{1, 2, ... n\}$. A Lehmer code $l(p)$ is an $n$-tuple $(r_0, r_1, ... r_{n-1})\in \N^n$ where the $r_k$-th number satisfies $r_k \leq k$. 
\end{definition}

Thus, for any given $n$, there are $n!$ possible Lehmer codes of length $n$. A the same time, there are also $n!$ permutations on a set of $n$ elements. Itmeans there is a bijection $f$ between those two sets, allowing for an encoding a permutation by a corresponting Lehmer code.

We are interested in a particular choice for $f$. To define it, we need to first introduce a notion of an inversion. 
...

The interpretation we have arrived is this: the number $r_k$ in a Lehmer code $c$ denotes the number of inversions of the element $k$ (right inversion count).

\subsection{Symmetric group}

\begin{definition}
A presentation of a group, denoted by $<S | R>$ consists of a set of generators $S$ and a set of equivalence relations $R$.
Relations are defined on the set of words $(S \cup S^{-1})^*$, where $S^{-1}$ denotes a set of formal inverses of elements in $S$.
\end{definition}

\begin{definition}
For a group $G = (G, \dot)$, we say that $\la S | R \ra$ is a presentation of $G$, if:
\begin{itemize}
	\item There is a function $f$ mapping elements of $S$ to elements of $G$.
	\item There is a surjective extension $f^*$ of $f$ to $(S \cup S^{-1})^*$ satisfying
		\begin{equation}
		\begin{cases}
		f^*(s) = f(h) & \text{ for } s \in S \\
		f^*(s^{-1}) = f(s)^{-1} & \text{ for } s \in S^{-1} \\
		f^*(w w') = f^*(w) \dot f^*(w') & \text{ for } w, w' \in (S \cup S^{-1})^*
		\end{cases}
		\end{equation}
	\item Function $f^*$ preserves the congruence closure of the set of relations $R \cup \{s = s^{-1}\}$.
\end{itemize}

For a number $n \in \N$, $n>0$, by $S_n$ we denote the symmetric group on the set $\{1, 2, ... n\}$ - elements of the group are permutations on this set. We are interested in a particular presentation of $S_n$, called Coxeter presentation.
\end{definition}

\begin{definition}
Coxeter presentation of $S_n$ is defined as $\la S|R \ra$, where $S = \{\tau_i : i \in \{1, 2, ... n - 1\}\}$ and
	\begin{equation*}
		R =
		    \begin{cases}
		      \tau_i\tau_j = \tau_j\tau_i & \text{for } i < j + 1\\
		      (\tau_i)^2 = 1 & \text{for } i \in \{1, 2, ... n - 1\}\\
		      (\tau_i\tau_{i+1})^3 = 1 & \text{for} i \in \{1, 2, ... n - 2\}\\
		    \end{cases}
	\end{equation*}
The third equation can be also written in a form $\tau_i\tau_{i+1}\tau_i = \tau_{i+1}\tau_i\tau_{i+1}$.
\end{definition}
\end{document}
