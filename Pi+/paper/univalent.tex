\section{Univalent Subuniverses}~\label{sec:univalent}

In this section, we introduce some basic concepts and notation that we use in Homotopy Type Theory. Then, we define
univalent subuniverses and discuss some specific examples.

\subsection{The Type Theory}~\label{subsec:type-theory}

We work in Homotopy Type Theory, in particular, intensional Martin-L\"{o}f Type Theory, with a univalent universe, and
Higher Inductive Types (HITs) for propositional truncations and set quotients. We recall a few basic facts.

\subsubsection{Identity Types}

\subsubsection{Univalence}

\subsubsection{Higher Inductive Types}

\subsection{Univalent Subuniverses}

\begin{definition}[Type families are fibrations]
  A type family $P : A \to \UU$ gives a fibration $\fst : \dsum{x:A}{P(x)} \to A$ with total space $\dsum{x:A}{P(x)}$
  and base space $A$ with a lifting property.
\end{definition}

\begin{definition}[$\term{transport-equiv}$]
  Given a type family $P : A \to \UU$, we define
  \[
    \term{transport-equiv} : x \id_{A} y \to P(x) \eqv P(y)
  \]
\end{definition}

\begin{definition}[Univalent Fibration]
  $P$ is a univalent fibration if $\term{transport-equiv}$ is an equivalence.
\end{definition}

\begin{definition}[Universe]
  A universe \`{a} la Tarski is given by a code $U : \UU$ and a decoding function $\El : U \to \UU$. If $\El$ is a
  univalent fibration, $U$ is a univalent universe.
\end{definition}

\begin{definition}[Subuniverse]
  A subtype is a type family $P : \UU \to \UU$ whose fibers are prop-valued, that is, $\forall x, \isProp{P(x)}$. A
  subuniverse generated by a subtype has $U \defeq \dsum{X:\UU}{P(X)}$ and $\El \defeq \fst$.
\end{definition}

\begin{proposition}[Univalent Subuniverse]
  Subuniverses generated by subtypes are univalent.
\end{proposition}

\begin{example}[$\Aut$]
  \[
    \Aut[T] \defeq T \eqv T
  \]
\end{example}

\begin{example}[$\BAut$]
  \[
    \BAut[T] \defeq \Sub{T}
  \]
\end{example}

\begin{example}
  \[
    P(X) \defeq \SubP{X}{\Bool}
  \]
\end{example}

\begin{example}
  For any $n : \Nat$, we have
  \[
    P(X) \defeq \SubP{X}{\Fin[n]}
  \]
  \[
    \UFin[n] \defeq \Sub{\Fin[n]}
  \]
  is the univalent subuniverse of types with cardinality $n$.
\end{example}

\begin{definition}[$\isFin$]
  \[
    \isFin[X] \defeq \dsum{n:\Nat}{\SubP{X}{\Fin[n]}}
  \]
\end{definition}

\begin{proposition}
  $\isFin$ is a proposition.
\end{proposition}

\begin{example}
  \[
    \UFin \defeq \dsum{X:\UU}{\isFin[X]}
  \]
  is the univalent subuniverse of all finite types.
\end{example}

We characterise the path space of subunivalent universes.

\begin{proposition}
  If $T$ is an $n$-type, $\BAut[T]$ is an $(n+1)$-type.
\end{proposition}

\begin{proposition}
  $\UFin$ is a 1-type, that is, a 1-groupoid, or has h-level 3.
\end{proposition}

\begin{proposition}
  \[
    \Omega\BAut[T] \eqv \Aut[T]
  \]
\end{proposition}

\begin{proposition}
  For every $n:\Nat$,
  \[
    \Omega\UFin[n] \eqv \Aut[\Fin[n]]
  \]
\end{proposition}

%%% Local Variables:
%%% mode: latex
%%% TeX-master: "main"
%%% End:
