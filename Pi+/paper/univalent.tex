\section{Univalent Subuniverses}~\label{sec:univalent}

In this section, we introduce some basic concepts and notation that we use in
Homotopy Type Theory. Then, we define univalent subuniverses and discuss some
specific examples.

\subsection{The Type Theory}~\label{subsec:type-theory}

We work in Homotopy Type Theory, in particular, intensional Martin-L\"{o}f Type
Theory, with a univalent universe, and Higher Inductive Types (HITs) for
propositional truncations and set quotients. We recall a few basic facts.

\subsubsection{Identity Types}

\subsubsection{Univalence}

\subsubsection{Higher Inductive Types}

\subsection{Univalent Subuniverses}



%%% Local Variables:
%%% mode: latex
%%% TeX-master: "main"
%%% End:
