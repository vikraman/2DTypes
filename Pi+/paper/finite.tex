\section{The groupoid of finite types}~\label{sec:finite}

In this section, we describe the algebraic structure of the groupoid of finite types by characterising its paths.

\subsection{Permutations}

In the previous~\cref{sec:univalent}, we established that paths in $\UFin$ are equivalent to families of automorphisms of $\Fin{n}$ for every $n:\Nat$.
These are permutations on finite sets of size $n$. In the following sections, we will characterise these permutations, going through a number of intermediate step. \jk{I don't like the wording}


\subsubsection{Lehmer codes}

There are $\fac{n}$ permutations on a set of $n$ elements. Thus, if we construct a type with $\fac{n}$ inhabitants, there should be a bijection of that type with the type of permutations $\Aut[\Fin[n]]$. To construct this bijection, we will borrow the idea of Lehmer codes~\cite{lehmerTeachingCombinatorialTricks1960a} from Combinatorial Analysis~\cite{bellmanCombinatorialAnalysis1960}. The type family $\Lehmer$ over $\Nat$ is defined as follows.

\todo{Understand the proof first and then revisit this paragraph}

\begin{definition}
    \begin{align*}
         & \Lehmer : \Nat \to \UU                             \\
         & \Lehmer[0] = \unit                                 \\
         & \Lehmer[\suc[n]] = \Fin[\suc[n]] \times \Lehmer[n]
    \end{align*}
\end{definition}

Now we construct the equivalence.

\begin{proposition}
    For all $n:\Nat$,
    \[
        \Lehmer[n] \eqv \Aut[\Fin[n]]
    \]
\end{proposition}

\subsubsection{Symmetric groups}

There is an obvious group structure on $\Aut[\Fin[n]]$ given by identity, composition, and inverse. This is the symmetric group $S_n$ on $n$ symbols. In the rest of the section we will construct a convenient presentation of this group.

\todo{Find the reference for group presentations as coequalisers (possibly Mac Lane)}

First, we formally define a presentation of a group. 

\begin{definition}
    A presentation of a group $G$ is a type $FR$ where
    $FR === Free n \to \Aut[\Fin[n]]$
    is a coequaliser.
\end{definition}

\begin{definition}
    Here we'll define a type family $FT n$ to be a free group generated by a type $T n$ with the following constructors
    \begin{align*}
        & cancel : i : Fin (S n) -> T \\ 
        & swap : (i : Fin (S n)) -> (j : Fin (S n)) -> (i + 1 < j) -> T \\
        & braid : (i : Fin n) -> T \\
    \end{align*}
\end{definition}


\begin{definition}[Adjacent transposition]
    \begin{align*}
        transpose : (n : \Nat) \to (k : \Fin[S n]) \to \Aut[\Fin[S n]]
    \end{align*}
    by double induction on $n$ and $k$, where
    \begin{align*}
        transpose n 0 = \lambda 
            0 -> 1
            1 -> 0
            m -> m
        transpose (S n) (S k) = \lambda
            0 -> 0
            S m -> S ((transpose n k) m)
    \end{align*}
\end{definition}

\begin{proposition}
    $FR$ is a presentation of the symmetrc group, with $f$ and $g$ as follows:
    \begin{align*}
        f inr (cancel i) = inr (i :: i :: [])
        f inl (cancel i) = inl (i :: i :: [])
        f inr (swap i j) = i :: j :: i :: j :: []
        f braid i = i :: S i :: i :: S i :: i :: S i :: []
    \end{align*}
    and
    \begin{align*}
        g inl _ = inl [] 
        g inr _ = inr [] 
    \end{align*}
\end{proposition}

We have : Aut(Fin n) ~ Sn
It is required to prove that 
    FR is presentation of Aut(Fin n)

% \begin{definition}
% A presentation of a group, denoted by $<S | R>$ consists of a set of generators $S$ and a set of equivalence relations $R$.
% Relations are defined on the set of words $(S \cup S^{-1})^*$, where $S^{-1}$ denotes a set of formal inverses of elements in $S$.
% \end{definition}

% \begin{definition}
% For a group $G = (G, \dot)$, we say that $\langle S | R \rangle$ is a presentation of $G$, if:
% \begin{itemize}
% 	\item There is a function $f$ mapping elements of $S$ to elements of $G$.
% 	\item There is a surjective extension $f^*$ of $f$ to $(S \cup S^{-1})^*$ satisfying
% 		\begin{equation}
% 		\begin{cases}
% 		f^*(s) = f(h) & \text{ for } s \in S \\
% 		f^*(s^{-1}) = f(s)^{-1} & \text{ for } s \in S^{-1} \\
% 		f^*(w w') = f^*(w) \dot f^*(w') & \text{ for } w, w' \in (S \cup S^{-1})^*
% 		\end{cases}
% 		\end{equation}
% 	\item Function $f^*$ preserves the congruence closure of the set of relations $R \cup \{s = s^{-1}\}$.
% \end{itemize}

% For a number $n \in \Nat$, $n>0$, by $S_n$ we denote the symmetric group on the set $\{1, 2, ... n\}$ - elements of the group are permutations on this set. We are interested in a particular presentation of $S_n$, called Coxeter presentation.
% \end{definition}

% \begin{definition}
% Coxeter presentation of $S_n$ is defined as $\langle S|R \rangle$, where $S = \{\tau_i : i \in \{1, 2, ... n - 1\}\}$ and
% 	\begin{equation*}
% 		R =
% 		    \begin{cases}
% 		      \tau_i\tau_j = \tau_j\tau_i & \text{for } i < j + 1\\
% 		      (\tau_i)^2 = 1 & \text{for } i \in \{1, 2, ... n - 1\}\\
% 		      (\tau_i\tau_{i+1})^3 = 1 & \text{for} i \in \{1, 2, ... n - 2\}\\
% 		    \end{cases}
% 	\end{equation*}
% The third equation can be also written in a form $\tau_i\tau_{i+1}\tau_i = \tau_{i+1}\tau_i\tau_{i+1}$.
% \end{definition}

We introduce a presentation of the symmetric group

%%% Local Variables:
%%% mode: latex
%%% TeX-master: "main"
%%% End:
