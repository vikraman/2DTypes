\section{The groupoid of finite types}~\label{sec:finite}

In this section, we describe \review{the algebraic structure} of the groupoid of
finite types, and give \review{a computable presentation} for it.

\vc{The groupoid of finite types is the free symmetric monoidal groupoid on one
  generator. This can be presented as an algebraic 2-theory, which is our syntax
  for $\PiHatLang$. Vertical categorification of natural numbers as a free
  commutative monoid. See groupoidification.}

\todo{Check Brent Yorgey's thesis?}

To do so, we will characterise the automorphisms on finite sets of cardinality
$n$, and show them to be equivalent to the symmetric group $\Sn$, via the
Coxeter presentation. We will do that in two steps,
in~\cref*{subsec:permutations,subsec:lehmer,subsec:symmetric}.

\todo{Big example: Start from a listed permutation, show Lehmer code, then adjacent swaps.}
\todo{Justification for why we need group theory.}

%% Coxeter presentation of $\Sn$ $\eqv$ $\Lehmer[n]$ $\eqv$ $\Aut[\Fin[n]]$

\subsection{Groups}

From universal algebra, a group is simply a set with a 0-ary constant, the neutral element, a binary operation for group
multiplication, and a unary inverse operation. A simple example is $\mathbb{Z}/n\mathbb{Z}$, where the neutral element
is 0, the inverse of $k$ is $-k$, and the group multiplication is given by addition modulo $n$. The neutral element has
to satisfy unit and inverse laws, and the multiplication has to be associative.

In type theory, a group $G$ can be defined as an $\hSet$ $S$ with the following pieces of data:

\begin{enumerate}
  \item a unit or neutral element $e : S$
  \item a multiplication function $m : S \times S \to S$ written as $(g_{1}, g_{2}) \mapsto g_{1} \mult g_{2}$, that satisfies
        \begin{enumerate}
          \item the unit laws, for all $g : S$, that \( g \mult e \id g \) and \( e \mult g \id g \)
          \item the associativity law, for all $g_{1}, g_{2}, g_{3} : S$, that \( g_{1} \mult (g_{2} \mult g_{3}) \id (g_{1} \mult g_{2}) \mult g_{3} \)
        \end{enumerate}
  \item an inversion function $i : S \to S$ written as $g \mapsto \inv{g}$, that satisfies
        \begin{enumerate}
          \item the inverse laws, for all $g : S$, that \( g \mult \inv{g} \id e \) and \( \inv{g} \mult g \id e \)
        \end{enumerate}
\end{enumerate}

However, more conveniently, in HoTT, we can instead use groupoids to talk about groups. A group can be identified with a
1-object groupoid, using a technique called delooping. The delooping of a group $G$ is a groupoid $\B{G}$ given by a
unique object $\pt$ with self-loops that are 1-paths $\pt \id_{\B{G}} \pt$ corresponding to the elements of $G$. Note
that the group operations are automatically given by operations on the identity type, with $\refl_{\pt}$ for the neutral
element, path composition for the group multiplication, and path inverse for the group inverse. These satisfy the group
laws as well, up to the identity type, using the groupoid coherence laws. Moreover, for 1-groups which are supposed to
be sets, these 2-paths should be propositions, so we have to restrict $\B{G}$ to be a 1-groupoid. Hence, a group is
simply given by a pointed, connected 1-type~\cite*{buchholtzHigherGroupsHomotopy2018,symmetryBook2021}.

For example, given a pointed type $(A:\UU, a:A)$, the automorphism group structure at $a$ is given by $a \id_{A} a$. Of
course, for 1-groups we will require that $a \id_{A} a$ is an $\hSet$, which is enforced by having $A$ be a groupoid. In
our running example for the permutation group on finite sets, we have that $\Fin[n]$ is an $\hSet$, and hence,
$\UFin[n] \defeq \BAut[\Fin[n]]$ is a pointed, connected 1-type, whose loopspace $\loopspace[\BAut[\Fin[n]],F_{n}]$ is
equivalent to $\Aut[\Fin[n]] \defeq (\Fin[n] \eqv \Fin[n]) \eqv (\Fin[n] \id_{\UU} \Fin[n])$, which has the
corresponding automorphism group structure.

\subsection{Free groups}

Given a generating set $A$, we can naively construct a \emph{free group on $A$},
whose elements are drawn from the alphabet $A$, and closed under the group
operations of multiplication and inverse, identified by the group axioms.
For example, the singleton set generates the additive group of integers,
$\mathbb{Z}$.

Usually, there are are many equations, besides the group axioms, that hold for
the elements of the group. For example, in the group $\mathbb{Z}/3\mathbb{Z}$,
we have an equation $1 + 1 + 1 = 0$. A free group has the property that has no
other relations than the ones directly implied by the group axioms.

\todo{What's the best notation for HITs?}
\begin{definition}
  Given an $\hSet$ $A$, the free group $F(A)$ on it is given by a higher
  inductive type with the following point and path constructors. Notice the
  similarity with the definition of a group structure in~\ref{subsec:groups},
  but note that each operation here is a generator for the type $F(A)$\dots.
  \begin{itemize}
    \item An inclusion function $\eta_{A} : A \to F(A)$
    \item A multiplication function $m : F(A) \times F(A) \to F(A)$
    \item An element $e : F(A)$
    \item An inverse function $i : F(A) \to F(A)$
  \end{itemize}
  \smallskip
  \begin{itemize}
    \item For every $x, y, z : F(A)$, a path $\term{assoc} : m(x, m(y, z)) \id m(m(x, y), z)$
    \item For every $x : F(A)$, paths $\term{unitr} : m(x, e) \id x$ and $\term{unitl} : m(e, x) \id x$
    \item For every $x : F(A)$, paths $\term{invr} : m(x, i(x)) \id e$ and $\term{invl} : m(i(x), x) \id e$
    \item A 0-truncation, for every $x, y : F(A)$ and $p, q : x \id y$, a 2-path $\term{trunc} : p \id q$
  \end{itemize}
\end{definition}

A group homomorphism between groups is a function between the underlying sets
that preserves the group structure, which we write as $G_1 \to^G G_2$. With
this, we can state the universal property of free groups.

\begin{proposition}[Universal Property of $F(A)$]~\label{prop:free-groups}
  Given a group $G$ and a map $f : A \to G$, there is a unique group
  homomorphism $\extend{f} : F(A) \to^G G$ such that $\extend{f} \comp \eta_A
    \htpy f$. Equivalently, composition with $\eta_A$ gives an equivalence $F(A)
    \to^G G \eqv A \to G$. Alternatively, the type of group homomorphisms $h :
    F(A) \to^G G$ satisfying $h \comp \eta_A \htpy f$ is contractible.

  % https://q.uiver.app/?q=WzAsMyxbMCwyLCJBIl0sWzAsMCwiRihBKSJdLFsyLDAsIkciXSxbMCwxLCJcXGV0YV9BIl0sWzAsMiwiZiIsMl0sWzEsMiwiXFxleHRlbmR7Zn0iLDAseyJzdHlsZSI6eyJib2R5Ijp7Im5hbWUiOiJkYXNoZWQifX19XSxbMyw0LCJcXGlkIiwwLHsic2hvcnRlbiI6eyJzb3VyY2UiOjIwLCJ0YXJnZXQiOjIwfSwic3R5bGUiOnsiYm9keSI6eyJuYW1lIjoibm9uZSJ9LCJoZWFkIjp7Im5hbWUiOiJub25lIn19fV1d
  \[\begin{tikzcd}
      {F(A)} && G \\
      \\
      A
      \arrow[""{name=0, anchor=center, inner sep=0}, "{\eta_A}", from=3-1, to=1-1]
      \arrow[""{name=1, anchor=center, inner sep=0}, "f"', from=3-1, to=1-3]
      \arrow["{\extend{f}}", dashed, from=1-1, to=1-3]
      \arrow["\id", Rightarrow, draw=none, from=0, to=1]
    \end{tikzcd}\]
\end{proposition}

This definition of $F(A)$ however has lots of path constructors corresponding to
each group axiom. Alternatively, the free group construction can be thought of
as drawing letters from the generating set, while adding formal inverses, and
constructing words which are sequences of these letters.

Each element in the sequence can either be an element drawn from the generating
set $a$, or an inverse of such an element $\inv{a}$. If we take the disjoint
union of $A$ with itself, that is, $A + A$ as the underlying set, we can use
$\inl/\inr$ to mark the elements. Hence, we can encode the free group using the
free monoid, that is, lists.

\begin{definition}
  Let $A$ be an $\hSet$, and $\List[\blank]$ the free monoid. The free group
  $F(A)$ on $A$ is the set-quotient of $\List[A + A]$ by the congruence closure
  of the relation $a \cons \inv{a} \cons \nil \sim \nil$.
\end{definition}

\begin{proposition}
  $F(A) \defeq \quot{\List[A + A]}{\sim^{\ast}}$ has a group structure, with the
  empty list $\nil$ for the neutral element, multiplication given by list append
  $\append$, and inverse given by list reversal. Further, $F(A)$ with $\eta_A :
    A \to F(A) \defeq \inl(a) \cons \nil$ satisfies the universal property of free
  groups.~\cref{prop:free-groups}
\end{proposition}

\subsubsection{Group presentations}

One useful way of defining groups is by their presentations. A presentation of a
group builds it by starting from the free group $F(A)$ and introducing a
collection of equations that have to be satisfied in the resulting group. For
example, if we take $F(\unit) \defeq \mathbb{Z}$ and add an equation $1 + 1 + 1
  = 0$, the resulting group would be $\mathbb{Z}/3\mathbb{Z}$.

The generators can be thought of as primitive operations in a (reversible)
programming language, group structure gives the way of composing these
operations and inverting them, and relations describe how these primitive
operations interact with each other.

Instead of opereating directly in the semantics of the group operation, here the
focus is on the syntax. While before, the equality of elements (such as the
result of multiplication of two elements) had to be computed externally, now it is
reduced to a \emph{word problem}, i.e., deciding one word can be reduced to another
using group's equations.

However, because equations are not directed, is not always possible to construct a
well-behaving rewriting system. In general, the word problem is proven to be undecidable.

\jk{Example?}

\begin{definition}
  Let $A$ be an $\hSet$ and $R : (A + A) \to (A + A) \to \UU$ a binary relation
  on $A + A$. The group $G$ presented by $\langle A ; R \rangle$ is given by the
  set-quotient of the free group $F(A)$ by the closure of $R$, or equivalently,
  as the coequaliser
  \[\begin{tikzcd}
      FR && FA && G
      \arrow[shift right=2, from=1-1, to=1-3]
      \arrow[shift left=2, from=1-1, to=1-3]
      \arrow[two heads, from=1-3, to=1-5]
    \end{tikzcd}\]
\end{definition}

\vc{do we need this level of detail?}
\todo{Universal property}
\todo{Examples: empty relation, van Kampen of $\pi_{1}$}

The above definition is correct because the universal property of the free group allows
for properly extending the relation on the generating set to the whole group.
\jk{Turn that into a propostion?}

\subsection{Permutation groups}

Because $\PiLang$ is used for reasoning about reversible functions on finite
sets, the group that we are interested in is the group of permutations on a
fininte set, which is classically know as the symmetric group on $n$ words,
$\Sn$. We've already seen that we can define it formally by taking the
automorphism group given by the type $\Aut[\Fin[n]]$, and showing that it has a
group structure satisfying the group axioms. Alternatively, we can write a group
presentation for $\Sn$ -- by defining a set of generators and a set of relations.

The technical contribution of this section is a proof that these two
descriptions are equivalent. By doing that, we bridge the gap between the
syntactic and semantic notions in our completeness proof -- by establishing a
correspondence between the meaning of a program (a bijection) and its syntax (a
word).

However, in giving a group presentation, there is an element of choice. A group
can be presented in many different ways. For example, we could generate the
permutation group on $\Fin[n]$ by using generators that:

\begin{itemize}
  \item swap the $i$-the element with the $(i+1)$-th element, that is, adjacent swaps, or
  \item swap the $i$-th element with the $j$-th element, for arbitrary $i$-s and $j$-s, or
  \item swap the $i$-th element with an element at a fixed position, or
  \item flip a prefix $\Fin[k]$ of $\Fin[n]$ for $k \leq n$, or
  \item cyclically shift any subset of $\Fin[n]$.
\end{itemize}

One way of thinking about these presentations is via sorting algorithms, which
use different primitive operations. A sorting algorithm has to calculate a
permutation of a list or a finite set, which satisfies the invariant of being a
sorted sequence, which means, the primitive operations of a sorting algorithm
should be able to generate all the permutations on a given list. If using a
chosen set of reversible operations, one can write a sorting algorithm, then
those operations generate the permutation group.

For example, bubble sort uses the primitive operation of adjacent swaps,
insertion sort and selection sort use the primitive operation of swapping the
$i$-th element with the $j$-th element, cycle sort uses cyclical shifts of
subsequences, pancake sort uses flips of prefixes of the list, et cetera.
\todo{check!}

\subsection{Coxeter Presentation}

We use a presentation based on adjacent transpositions, which is called the
Coxeter presentation. The primitive operations are adjacent swaps. Further,
there are three laws that these operations have to satisfy -- it is easiest to
visualise them as braid diagrams.

Swapping the same two elements two times in a row is the same as doing nothing:

\[
  \begin{tabular}{m{0.3\linewidth}m{0.1\linewidth}m{0.3\linewidth}}
    \begin{center}
      \begin{tikzpicture}
        \pic[local bounding box=my braid,braid/.cd,
          number of strands = 2,
          thick]
        {braid={ s_1, s_1}};
      \end{tikzpicture}
    \end{center}
     &
    \(\xlongrightarrow[]{\cancel}\)
     &
    \begin{center}
      \begin{tikzpicture}
        \pic[local bounding box=my braid,braid/.cd,
          number of strands = 2,
          thick]
        {braid={1, 1}};
      \end{tikzpicture}
    \end{center}
  \end{tabular}
\]

When swapping two pairs of elements in positions "far-away", it does not matter
in which order the swapping happens, that is, we can slide the wires freely.

\[
  \begin{tabular}{m{0.3\linewidth}m{0.1\linewidth}m{0.3\linewidth}}
    \begin{center}
      \begin{tabular}{m{0.3\linewidth}m{0.1\linewidth}m{0.3\linewidth}}
        \begin{tikzpicture}
          \pic[local bounding box=my braid,braid/.cd,
            number of strands = 2,
            thick]
          {braid={ 1, s_1}};
        \end{tikzpicture}
         &
        \(\cdots\)
         &
        \begin{tikzpicture}
          \pic[local bounding box=my braid,braid/.cd,
            number of strands = 2,
            thick]
          {braid={ s_1, 1 }};
        \end{tikzpicture}
      \end{tabular}
    \end{center}
     &
    \(\xlongrightarrow[]{\swap}\)
     &
    \begin{center}
      \begin{tabular}{m{0.3\linewidth}m{0.1\linewidth}m{0.3\linewidth}}
        \begin{tikzpicture}
          \pic[local bounding box=my braid,braid/.cd,
            number of strands = 2,
            thick]
          {braid={ s_1, 1}};
        \end{tikzpicture}
         &
        \(\cdots\)
         &
        \begin{tikzpicture}
          \pic[local bounding box=my braid,braid/.cd,
            number of strands = 2,
            thick]
          {braid={ 1, s_1}};
        \end{tikzpicture}
      \end{tabular}
    \end{center}
  \end{tabular}
\]

\todo{explain braid in words, give some intution on why it holds}

\[
  \begin{tabular}{m{0.3\linewidth}m{0.1\linewidth}m{0.3\linewidth}}
    \begin{center}
      \begin{tikzpicture}
        \pic[local bounding box=my braid,braid/.cd,
          number of strands = 3,
          thick]
        {braid={ s_2, s_1, s_2}};
      \end{tikzpicture}
    \end{center}
     &
    \(\xlongrightarrow[]{\braid}\)
     &
    \begin{center}
      \begin{tikzpicture}
        \pic[local bounding box=my braid,braid/.cd,
          number of strands = 3,
          thick]
        {braid={ s_1, s_2, s_1}};
      \end{tikzpicture}
    \end{center}
  \end{tabular}
\]

\todo{an example/motivation from book.pdf}

This construction is called a Coxeter presentation. Writing it formally, we get
the following definition for $\Sn$. We define a binary relation $\cox$ on
$\List[\Fin[n]]$.

\begin{definition}[$\cox$]
  \begin{align*}
    \cancel
     & : \forall n \to (n \cons n \cons \nil) \cox \nil                                                     \\
    \swap
     & : \forall k, n \to (\suc[k] < n) \to (n \cons k \cons \nil) \cox (k \cons n \cons \nil)              \\
    \braid
     & : \forall n \to (\suc[n] \cons n \cons \suc[n] \cons \nil) \cox (n \cons \suc[n] \cons n \cons \nil) \\
  \end{align*}
\end{definition}

We define $\cox*$ as the congruence closure of $\cox$.

\begin{definition}[$\cox*$]
  \begin{align*}
    \reflr{\cox}
     & : \forall w \to w \cox* w                                                                                                           \\
    \symr{\cox}
     & : \forall w_{1}, w_{2} \to w_{1} \cox* w_{2} \to w_{2} \cox* w_{1}                                                                  \\
    \transr{\cox}
     & : \forall w_{1}, w_{2}, w_{3} \to  w_{1} \cox* w_{2} \to w_{2} \cox* w_{3} \to w_{1} \cox* w_{3}                                    \\
    \congrf{\cox}{\append}
     & : \forall w_{1}, w_{2}, w_{3}, w_{4} \to  w_{1} \cox* w_{2} \to w_{3} \cox* w_{4} \to w_{1} \append w_{3} \cox* w_{2} \append w_{4} \\
    \relr{\cox}
     & : \forall w_{1}, w_{2} \to w_{1} \cox w_{2} \to w_{1} \cox* w_{2}                                                                   \\
  \end{align*}
\end{definition}

The choice of generators for this presentation is important. \todo{Think about this.}

\begin{itemize}
  \item It affects the difficulty of solving the word problem in $\Sn$ and
        formalising the proof of its correctness.
  \item It affects the proof of strong normalisation of reduction for the relations.
  \item It dictates which words are normal forms in this presentation of $\Sn$.
  \item It has to closely match the $\PiLang$ comobinators so that we can prove
        completeness.
  \item It dictates the choice of reversible gates in the synthesis and
        normalisation of boolean circuits which are applications that we show in
        the later sections.
\end{itemize}

To consider solving the word problem for $\Sn$, we need to formalise the generators for the Coxeter relations on
$\List[\Fin[n]]$ as a rewriting system $(\List[\Fin[n]],\cox*)$. The $\cox*$ relation as defined, does not produce a
well-behaved rewriting system due to the symmetric closure. Throwing out reflexivity and symmetry, we observe that the
right hand sides of the relations are strictly smaller than the left hand sides, in terms of the lexicographical
ordering on words in $\Fin[n]$. By directing the relation from left to right, we would be able to prove a termination
property out of the box.

A well-behaved rewriting system should also be confluent, however, it is not at all obvious from this definition. Due to
overlapping reduction rules, there are many possible critical pairs, such as

\[
  \begin{array}{lcr}
    \gspan[\braid][\braid]{\tau_2\tau_1\tau_2\tau_1\tau_2}{\tau_1\tau_2\tau_1\tau_1\tau_2}{\tau_2\tau_1\tau_1\tau_2\tau_1}
     &
    \text{or}
     &
    \gspan[\braid][\cancel]{\tau_2\tau_1\tau_2\tau_2}{\tau_1\tau_2\tau_1\tau_2}{\tau_2\tau_1}
  \end{array}
\]

For confluence, all critical pairs should converge. The pairs above can be resolved, but there are many examples where
they do not. The following is a simple example.

\[
  \gspan[\braid][\swap]{\tau_3\tau_2\tau_3\tau_1}{\tau_2\tau_3\tau_2\tau_1}{\tau_3\tau_2\tau_1\tau_3}
\]

Because of this counterexample, the relations have to be changed. We propose a long form of the relations as a new
presentation, where we allow $\cancel$ and $\swap$ reductions anywhere inside the list, called $\longcancel$ and
$\longswap$, but replace $\braid$ with a more general $\longbraid$ relation.

\[
  \begin{tabular}{m{0.4\linewidth}m{0.1\linewidth}m{0.4\linewidth}}
    \begin{center}
      \begin{tikzpicture}
        \pic[local bounding box=my braid,braid/.cd,
          number of strands = 6,
          thick]
        {braid={s_5, s_4, s_3, s_2, s_1, s_5}};
      \end{tikzpicture}
    \end{center}
     &
    \(\xlongrightarrow[]{\longbraid}\)
     &
    \begin{center}
      \begin{tikzpicture}
        \pic[local bounding box=my braid,braid/.cd,
          number of strands = 6,
          thick]
        {braid={s_4, s_5, s_4, s_3, s_2, s_1}};
      \end{tikzpicture}
    \end{center}
  \end{tabular}
\]

The directed $\longcox$ relation is defined formally with the following generators.

\begin{definition}[$\longcox$]
  \begin{align*}
    \longcancel
     & : \forall n, l, r \to (l \append n \cons n \cons r) \longcox (l \append r)                                                                           \\
    \longswap
     & : \forall k, n, l, r \to (\suc[k] < n) \to (l \append n \cons k \cons r) \longcox (l \append k \cons n \append r)                                    \\
    \longbraid
     & : \forall n, l, r \to (l \append (n \downarrow 2 + k) \append (1 + k + n) \cons r) \longcox (l \append (k + n) \cons (n \downarrow 2 + k) \append r) \\
  \end{align*}
\end{definition}

Like before, we define the $\longcox*$ to be the closure of $\longcox$, but only under reflexivity and transitivity. The
transitive extension is only performed on the right.

\begin{definition}[$\longcox*$]
  \begin{align*}
    \reflr{\longcox}
     & : \forall w \to w \longcox* w                                                                               \\
    \transr{\longcox}
     & : \forall w_{1}, w_{2}, w_{3} \to  w_{1} \longcox w_{2} \to w_{2} \longcox* w_{3} \to w_{1} \longcox* w_{3} \\
  \end{align*}
\end{definition}

Despite the increased complexity of the relations, the rewriting system $(\List[\Fin[n]],\longcox*)$ can be shown to
satisfy confluence, that is, the Church-Rosser (diamond) property. Further, we can show that this system is strongly
normalising, by producing a unique normal form that doesn't reduce any further. We follow the terminology
of~\cite{krausCoherenceWellFoundednessTaming2020} to state our results.

\begin{proposition}
  \leavevmode
  \begin{enumerate}
    \item $\longcox$ is (locally) confluent. For every span $\coxspan{w_{1}}{w_{2}}{w_{3}}$, there is a matching
          extended cospan $\coxcospan*{w_{2}}{w_{3}}{w}$.
    \item $\longcox*$ is confluent. For every extended span $\coxspan*{w_{1}}{w_{2}}{w_{3}}$, there is a matching
          extended cospan $\coxcospan*{w_{2}}{w_{3}}{w}$.
    \item $\longcox*$ is strongly normalising. For every $w$, there exists a unique $v$ such that $w \longcox* v$.
  \end{enumerate}
\end{proposition}

Another property we're interested in is well-foundedness of the relation $\longcox*$.
\todo{lexicographic order, accessibility predicate and well-foundedness}

In the theory of group presentations, the Coxeter relations are generally presented in their short form~\todo{citation},
but computation is usually done by using Coxeter matrices~\cite{davisGeometryTopologyCoxeter2008}. The long form of the
Coxeter relations are unwieldy and difficult to prove properties about by induction, and have little connection to the
broader scope of computational group theory. However, we can prove that the closures of the two forms of the relations
are equivalent, and work up to this equivalence!

\begin{proposition}
  $\cox*$ and $\longcox*$ are equivalent. For every $w$ and $v$, $w \cox* v$ iff $w \longcox* v$.
\end{proposition}

18. We define the equivalence classes on List (Fin n) to be precisely these
elements that have the same normal form.

20. Now, to prove the equivalence of 9, on one side we have Aut(Fin n), and
on the other, equivalence classes of List (Fin n) wrt to the usual Coxeter
group, we would define two functions from and back. The easiest way to
define a function out of the group presentation is to define them on the
equivalence classes' representatives. But to do that, we have to know how
do these representatives look like exactly.

The type $\Sn$ is defined as the set-quotient of $\List[\Fin[n]]$ by $\cox*$.

\begin{definition}[$\Sn$]
  \(\Sn \defeq \quot{\List(\Fin[n])}{\cox*}\)
\end{definition}

We will now prove the group structure of $\Sn$.

\begin{proposition}
  There is a group structure on $\Sn$, where the identity element is $\nil$, multiplication is given by list append, and
  inverse is given by list reversal.
\end{proposition}

\begin{proposition}
  $\Sn$ is equivalent to the generated group given by the normal closure of $\cox*$ extended to
  $\List(\Fin[n] + \Fin[n])$ along the codiagonal map $[\term{id},\term{id}] : A + A \to A$.
\end{proposition}

\subsection{Lehmer Codes}

21. Enter: Lehmer codes.
TODO diagram with List (Fin n) being divided into equivalence classes, and
Lehmer codes being an image of immersion, being the normal form
(representative) in each class.

\subsection{Permutations}~\label{subsec:permutations}

In the previous~\cref{sec:univalent}, we established that paths in $\UFin$ are
equivalent to families of automorphisms of $\Fin{n}$ for every $n:\Nat$, that
is, bijections on finite sets of size $n$. This is the extensional view of
permutations. In the following sections, we will characterise these
permutations, going through two intermediate steps.

\vc{This is obvious, maybe add something more here.}

\subsection{Lehmer codes}~\label{subsec:lehmer}

From grade school combinatorics, we know that there are $\fac{n}$ permutations
on a finite set with $n$ elements. The factorial function is defined by
recursion on natural numbers. However, now, for every $n$, we want to produce a
type, which is a finite set, with cardinality $\fac{n}$. And, to characterise
$\Aut[\Fin[n]]$, we further need to construct a bijection between this type and
$\Aut[\Fin[n]]$.

First, let's define this type with $\fac{n}$ elements, we name this type family
$\Lehmer : \Nat \to \UU$, which is defined by recursion on $\Nat$ as follows.
This is the obvious definition of factorials by recursion, but categorified from
natural numbers to sets.

\begin{definition}
  \begin{align*}
    \Lehmer[0]       & \defeq \unit                           \\
    \Lehmer[\suc[n]] & \defeq \Fin[\suc[n]] \times \Lehmer[n]
  \end{align*}
\end{definition}

\todo{Subexcedant sequences and factorial definitions are equivalent, explain
  this!}

The name Lehmer comes from Lehmer
codes~\cite{lehmerTeachingCombinatorialTricks1960} which are known in
Combinatorial Analysis~\cite{bellmanCombinatorialAnalysis1960}. There are many
ways to represent permutations, e.g. inversions, or cycles, or matrices. Lehmer
codes are a particularly convenient way to represent permutations on a
computer,~\review{they are compact and have exactly the right cardinality.
  $\Lehmer[n]$ is a $n+1$-element tuple, where the position $k \leq n$ has an
  element of $\Fin[k]$. The 0-th position is trivial, so we ignore it, and in
  both the example below and the Agda proof, consider only the remaining
  $n$-element tuple.}

\vc{This is just the classical algorithm to explain the example, not the actual
  type-theoretic proof.}

Suppose we have a permutation $p$ on an $n$-element set
$\{\el{0}, \el{1}, \el{2}, \el{3}, \el{4}\}$, we encode it as follows.
$\Lehmer[n]$ is a $n$-element tuple. At position $k$, we put the number of
inversions of the element $\el{k}$ in $p$, i.e. the number of elements smaller
than $\el{k}$ occurring after $\el{k}$.

As an example, consider the following tabulated presentation of the permutation:

\todo{fix this figure}

\[
  p =
  \begin{array}{ccccccccccccccc}
    | & 0      & | & 1      & | & 2      & | & 3      & | & 4      & | \\
    \hline                                                             \\
    | & \el{2} & | & \el{0} & | & \el{1} & | & \el{4} & | & \el{3} & | \\
    \hline                                                             \\
  \end{array}
\]

%  0 1 2 3 4
% -----------
% |2|0|1|4|3|
% -----------

The element $\el{0}$ has 0 inversions, because there are no elements smaller
than $\el{0}$ occurring after it. In fact, there can be no elements smaller than
$\el{0}$ at all, so the type at the first position of the Lehmer code tuple is
$\unit$.

The element $\el{1}$ has 0 inversions as well, since elements occurring after it
in the permutation are $\el{4}$ and $\el{2}$. There is only one different case,
if $\el{1}$ appeared before $\el{0}$, it would have 1 inversion. This is why the
type of the second component of the Lehmer code is $\Fin[2]$.

The element $\el{2}$ has 2 inversions, because both $\el{0}$ and $\el{1}$ occur
after it in the permutation. The element $\el{3}$ occurs as the last one, so it
has 0 inversions. The element $\el{4}$ has 1 inversion, with the element
$\el{3}$.

Thus, the Lehmer code for the permutation $p$ is the 5-tuple
$l = (0, 0, 2, 0, 1)$.

To reconstruct the tabulated presentation of the permutation from the Lehmer
code, we perform an algorithm similar to \emph{insertion sort}. Starting from
the left-most position of the tuple $l$, we'll read the value $v$, insert the
new element at the end of the newly created list, and shift it backward $v$
places.

\begin{center}
  \begin{tabular}{c|p{0.75\linewidth}}
    (0, 0, 2, 0, 1)               & We start from an empty list $[]$                                                                 \\
    (\highlight{{0}}, 0, 2, 0, 1) & We read 0 as the left-most value from $l$. Thus, we append the element $\el{0}$ to our
    list, getting $[\el{0}]$. The element is shifted $0$ places, so it remains in the
    same place.                                                                                                                      \\
    (0, \highlight{{0}}, 2, 0, 1) & Then, similarly, we read another 0 for the element $\el{1}$, append it to the
    list getting $[\el{0}, \el{1}]$, and don't shift it either.                                                                      \\
    (0, 0, \highlight{{2}}, 0, 1) & We read 2 for the next the element $\el{2}$ - we append $\el{2}$ to our list, getting
    $[\el{0}, \el{1}, \el{2}]$, and shift it 2 places right, which results in a list $[\el{0}, \el{2}, \el{1}]$
    \todo{Typeset it nicely, with arrows showing the shifting}.                                                                      \\
    (0, 0, 2, \highlight{{0}}, 1) & Then we read 0 - appending $\el{3}$ and not shifting, getting $[\el{0}, \el{2}, \el{1}, \el{3}]$ \\
    (0, 0, 2, 0, \highlight{{1}}) & Finally, reading 1 for element $\el{4}$ - appending $\el{4}$ to the list and shifting it
    one place right results in the final list $[\el{0}, \el{2}, \el{1}, \el{4}, \el{3}]$                                             \\
  \end{tabular}
\end{center}
\todo{figure}

Using this Lehmer encoding algorithm, we can now construct the equivalence between these types.

We define a type family $\FinExcept{n} : \Fin[n] \to \UU$ which picks out all elements in $\Fin[n]$ except the one
provided. Note that $\FinExcept{n}[i]$ for $i : \Fin[n]$ is a subtype of $\Fin[n]$ and is hence an $\hSet$.

\begin{definition}
  \( \FinExcept{n}[i] \defeq \dsum{j : \Fin[n]}{i \neq j} \).
\end{definition}

\begin{proposition}
  For any $k : \Fin[n]$, $\unit \sqcup \FinExcept{n}[k] \eqv \Fin[n]$.
\end{proposition}

\begin{proposition}
  For any $k : \Fin[\suc[n]]$, $\FinExcept{\suc[n]}[k] \eqv \Fin[n]$.
\end{proposition}

\begin{proposition}
  For any $n : \Nat$,
  \( \Aut[\Fin[\suc[n]]] \eqv \dsum{k : \Fin[\suc[n]]}{\FinExcept{\suc[n]}[\fzero] \eqv \FinExcept{\suc[n]}{k}} \).
\end{proposition}

\begin{proposition}
  For all $n:\Nat$, \( \Lehmer[n] \eqv \Aut[\Fin[n]] \).
\end{proposition}

\begin{proof}
  For $n = 0$, note that $\Lehmer[0]$ is contractible, and so is $\Aut[\Fin[0]]$. For $n = \suc[m]$, we have the
  following chain of equivalences.
  \[\arraycolsep=0.5em\def\arraystretch{1.5}
    \begin{array}{rl}
           & \Aut[\Fin[\suc[m]]]                                                               \\
      \eqv & \dsum{k : \Fin[\suc[m]]}{\FinExcept{\suc[m]}[\fzero] \eqv \FinExcept{\suc[m]}[k]} \\
      \eqv & \dsum{k : \Fin[\suc[m]]}{\FinExcept{\suc[m]}[\fzero] \eqv \Fin[m]}                \\
      \eqv & \dsum{k : \Fin[\suc[m]]}{\Fin[m] \eqv \Fin[m]}                                    \\
      \eqv & \Fin[\suc[m]] \times \Aut[\Fin[m]]                                                \\
      \eqv & \Fin[\suc[m]] \times \Lehmer[m]                                                   \\
    \end{array}
  \]
\end{proof}

We will now show how to generate a Lehmer code from a word in $\Sn$ and back.

\begin{definition}
  \[
    \encode{\List} : \List[\Fin[n]] \to \Lehmer[n]
  \]
\end{definition}

\begin{proposition}
  \leavevmode
  \begin{enumerate}
    \item \( l_{1} \cox* l_{2} \to \encode{\List}(l_{1}) \id \encode{\List}(l_{1}) \)
  \end{enumerate}
\end{proposition}

\begin{definition}
  \begin{align*}
    \encodeSn & : \Sn[n] \to \Lehmer[n] \\
    \decodeSn & : \Lehmer[n] \to \Sn[n]
  \end{align*}
\end{definition}

\begin{proposition}
  For all $n : \Nat$, $(\encodeSn, \decodeSn)$ is an equivalence.
\end{proposition}

\begin{corollary}
  For all $n : \Nat$,
  \(
  \Sn \eqv \Lehmer[n] \eqv \Aut[\Fin[n]]
  \).
\end{corollary}

%%% Local Variables:
%%% mode: latex
%%% TeX-master: "main"
%%% fill-column: 120
%%% End:
