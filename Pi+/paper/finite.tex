\section{The groupoid of finite types}~\label{sec:finite}

In the previous~\cref{sec:ufin}, we established that paths in $\UFin$ are equivalent to families of loops on $\Fin{n}$
for every $n:\Nat$, that is, automorphisms of finite sets of size $n$. The loopspace has a group structure -- the
identity function is the neutral element, group multiplication is given by function composition, and the group inverse
is given by inverting the bijection. This is also known to be the finite symmetric group $\Sn[n]$. In the following
sections, we will characterise these permutations syntactically.

In order to study syntactic descriptions of permutations, we need to decide whether two descriptions refer to the same
permutation. This problem is well known in group theory as the \emph{word problem} for $\Sn$. Putting it in this form
also allows us to connect to the broader scope of computational group theory and combinatorics -- we borrow ideas such
as Coxeter relations and Lehmer codes from there.

The goal of this section is to reconcile two different approaches to defining the symmetric group -- as an automorphism
group, and as a group presented using generators and relations.

\subsection{Groups}

From universal algebra, a group is simply a set with a 0-ary constant $e$ (the neutral element), a binary operation
$\blank\mult\blank$ for group multiplication, and a unary inverse operation $\inv{\blank}$. The neutral element has to
satisfy unit and inverse laws, and the multiplication has to be associative (see~\cref{def:group}).

A very simple example of a group is $\mathbb{Z}$, where the neutral element is 0, the inverse of $k$ is $-k$, and the
group multiplication is given by integer addition.

\begin{toappendix}
  \begin{definition}~\label{def:group}
    In type theory, a group $G$ can be defined as a set $S$ with the following pieces of data:

    \begin{enumerate}
      \item a unit or neutral element $e : S$
      \item a multiplication function $m : S \times S \to S$ written as $(g_{1}, g_{2}) \mapsto g_{1} \mult g_{2}$, that satisfies
            \begin{enumerate}
              \item the unit laws, for all $g : S$, that \( g \mult e \id g \) and \( e \mult g \id g \)
              \item the associativity law, for all $g_{1}, g_{2}, g_{3} : S$, that \( g_{1} \mult (g_{2} \mult g_{3}) \id (g_{1} \mult g_{2}) \mult g_{3} \)
            \end{enumerate}
      \item an inversion function $i : S \to S$ written as $g \mapsto \inv{g}$, that satisfies
            \begin{enumerate}
              \item the inverse laws, for all $g : S$, that \( g \mult \inv{g} \id e \) and \( \inv{g} \mult g \id e \)
            \end{enumerate}
    \end{enumerate}
  \end{definition}
\end{toappendix}

However, more conveniently, in HoTT, we can instead use groupoids to talk about groups internally. A group can be
identified with a 1-object groupoid, using a technique called delooping. The delooping of a group $G$ is a groupoid
$\B{G}$, with a unique object $\pt$, where the group elements are encoded as self-loops $\pt \id_{\B{G}} \pt$.

The group operations are automatically given by operations on the identity type, with $\refl_{\pt}$ for the neutral
element, path composition for the group multiplication, and path inverse for the group inverse. These satisfy the group
laws, up to the identity type, using the groupoid coherence laws.

Since groups are sets, these 2-paths are propositions, making $\B{G}$ a 1-groupoid. Hence, a group is simply given by a
pointed, connected 1-type~\cite*{buchholtzHigherGroupsHomotopy2018,symmetryBook2021}. \footnote{This technique also
  allows defining higher groups in HoTT, as done in~\cite{buchholtzHigherGroupsHomotopy2018}.}

% For example, given a pointed type $(A:\UU, a:A)$, the automorphism group structure at $a$ is given by $a \id_{A} a$. Of
% course, for 1-groups we will require that $a \id_{A} a$ is a set, which is enforced by having $A$ be a groupoid. In our
% running example for the permutation group on finite sets, we have that $\Fin[n]$ is a set, and hence, $\UFin[n] \defeq
% \BAut[\Fin[n]]$ is a pointed, connected 1-type. Its loopspace $\loopspace[\BAut[\Fin[n]],F_{n}]$ is equivalent to
% $\Aut[\Fin[n]] \defeq (\Fin[n] \eqv \Fin[n]) \eqv (\Fin[n] \id_{\UU} \Fin[n])$, which has the corresponding automorphism
% group structure.

\subsection{Free groups}

Usually, there are are many equations, besides the group axioms, that hold for the elements of a group. For example, in
the group $\Aut[\Bool]$, or $\mathbb{Z}_2$, we have an equation $1 + 1 = 0$, which is not a consequence of the group
axioms, but is specific to this particular group. A free group has the property that no other equations hold except the
ones directly implied by the group axioms. To be able to give an equational presentation of a group, we start by
constructing the free group, then show how to add new equations to it.

Given any set $A$, we can describe the free group $F(A)$ -- we call $A$ the generating set of $F(A)$. We draw the
elements from $A$, close them under the group operations, and identify them by the group axioms. For example, the
additive group of integers $\mathbb{Z}$ is the free group on the singleton. In HoTT, we can use a higher inductive type
to define the free group, by adding path constructors for the axioms we want (see~\cref{def:free-group}).

\begin{toappendix}
  \begin{definition}
    \label{def:free-group}
    Given a set $A$, the free group $F(A)$ on it is given by a higher inductive type with the following point and path
    constructors. Notice the similarity with the definition of a group structure (\cref{def:group}), but note that each
    operation here is a generator for $F(A)$.
    \begin{itemize}
      \item An inclusion function $\eta_{A} : A \to F(A)$
      \item A multiplication function $m : F(A) \times F(A) \to F(A)$
      \item An element $e : F(A)$
      \item An inverse function $i : F(A) \to F(A)$
    \end{itemize}
    \smallskip
    \begin{itemize}
      \item For every $x, y, z : F(A)$, a path $\term{assoc} : m(x, m(y, z)) \id m(m(x, y), z)$
      \item For every $x : F(A)$, paths $\term{unitr} : m(x, e) \id x$ and $\term{unitl} : m(e, x) \id x$
      \item For every $x : F(A)$, paths $\term{invr} : m(x, i(x)) \id e$ and $\term{invl} : m(i(x), x) \id e$
      \item A 0-truncation, for every $x, y : F(A)$ and $p, q : x \id y$, a 2-path $\term{trunc} : p \id q$
    \end{itemize}
  \end{definition}
\end{toappendix}

A group homomorphism between groups $G$ and $H$, which we write as $G \to^G H$, is a function between the underlying
sets that preserves the group structure. The universal property of free groups, stemming from the free-forgetful
adjunction between the category of groups and sets, follows from the induction principle for $F(A)$. Informally, it
states that if we know how a map $f$ acts on the generating set $A$, we know exactly how it acts on every element of the
group $F(A)$.

\begin{proposition}[Universal Property of $F(A)$]~\label{prop:free-groups}
  Given a group $G$ and a map $f : A \to G$, there is a unique group
  homomorphism $\extend{f} : F(A) \to^G G$ such that $\extend{f} \comp \eta_A
    \htpy f$. Equivalently, composition with $\eta_A$ gives an equivalence $F(A)
    \to^G G \eqv A \to G$. Alternatively, the type of group homomorphisms $h :
    F(A) \to^G G$ satisfying $h \comp \eta_A \htpy f$ is contractible.

  % https://q.uiver.app/?q=WzAsMyxbMCwyLCJBIl0sWzAsMCwiRihBKSJdLFsyLDAsIkciXSxbMCwxLCJcXGV0YV9BIl0sWzAsMiwiZiIsMl0sWzEsMiwiXFxleHRlbmR7Zn0iLDAseyJzdHlsZSI6eyJib2R5Ijp7Im5hbWUiOiJkYXNoZWQifX19XSxbMyw0LCJcXGlkIiwwLHsic2hvcnRlbiI6eyJzb3VyY2UiOjIwLCJ0YXJnZXQiOjIwfSwic3R5bGUiOnsiYm9keSI6eyJuYW1lIjoibm9uZSJ9LCJoZWFkIjp7Im5hbWUiOiJub25lIn19fV1d
  \[\begin{tikzcd}
      {F(A)} && G \\
      \\
      A
      \arrow[""{name=0, anchor=center, inner sep=0}, "{\eta_A}", from=3-1, to=1-1]
      \arrow[""{name=1, anchor=center, inner sep=0}, "f"', from=3-1, to=1-3]
      \arrow["{\extend{f}}", dashed, from=1-1, to=1-3]
      \arrow["\id", Rightarrow, draw=none, from=0, to=1]
    \end{tikzcd}\]
\end{proposition}

Although the description of $F(A)$ discussed above is close to the intuitive universal-algebraic definition, it is not
the most convenient one to work with. Besides the elements of the group, we also want to characterise its the path
space. This definition of $F(A)$ has lots of path constructors corresponding to each group axiom, making it difficult to
describe its path space.

Instead, we will think about elements of the free group as words over an alphabet of letters drawn from the generating
set \emph{and} the set of their formal inverses. If we take the disjoint union of $A$ with itself, that is, $A + A$ as
the group's underlying set, we can use $\inl/\inr$ to mark the elements -- $\inl{a}$ means $a$ and $\inr{a}$ means
$\inv{a}$. Then, we can encode the free group using the free monoid, that is, lists of $A + A$. Additionally, we need to
ensure that the inverse laws hold, so we have to coalesce adjacent occurences of $a$ and $\inv{a}$.

\begin{definition}
  \label{def:presentation}
  Let $A$ be a set, and $\List[\blank]$ the free monoid. The free group $F(A)$ on $A$ is the set-quotient of $\List[A +
  A]$ by the congruence closure of the relation $a \cons \inv{a} \cons \nil \sim \nil$ and $\inv{a} \cons a \cons \nil
  \sim \nil$.
\end{definition}

\begin{proposition}
  $F(A) \defeq \quot{\List[A + A]}{\sim^{\ast}}$ has a group structure, with the empty list $\nil$ for the neutral
  element, multiplication given by list append $\append$, and inverse given by flipping $\inl$ and $\inr$, followed by
  reversing the list. Further, $F(A)$ with $\eta_A(a) \defeq \inl(a) \cons \nil$ satisfies the universal property of
  free groups, as stated in~\cref{prop:free-groups}.
\end{proposition}

\subsection{Group presentations}

A presentation of a group builds it by starting from the free group $F(A)$ and introducing a collection of equations
that have to be satisfied in the resulting group. For example, if we take $F(\unit) \defeq \mathbb{Z}$ and add an
equation $1 + 1 = 0$, the resulting group would be $\mathbb{Z}_2 \eqv \Aut[\Bool]$. Note that not all groups have finite
(or computable) presentations, and, a group can have any number of different presentations.

\begin{definition}
  Let $A$ be a set and $R : \List[A + A] \to \List[A + A] \to \UU$ a binary relation on $\List[A + A]$. The group
  $F(\langle A ; R \rangle)$ presented by $A$ and $R$, is given by the set-quotient of the free group $F(A)$ by the
  closure of $R$.
\end{definition}

The universal property of the above definition allows for properly extending the relation on the generating set to the
whole group.\vc{check!}

\begin{proposition}[Universal property of $F(\langle A ; R \rangle)$]
  Given a group $G$ and a map $f : A \to G$ that respects $R$, there is a unique group homomorphism 
  $\extend{f} : F(\langle A ; R \rangle) \to^G G$ such that $\extend{f} \comp \eta_A \htpy f$.
\end{proposition}

To relate this to our original problem, the generators of the group can be thought of as the primitive combinators in a
(reversible) programming language, the group structure gives the composition and inversion operations, and the relations
describe how these primitive operations interact with each other.

As we mentioned, instead of operating directly with a particular group, we focus on the syntactic presentation. While
before, the only way to decide the equality of two elements in the group was to compute and check them on the nose, now
it is reduced to a \emph{word problem}, that is, deciding whether one word -- a representative of the group elements'
equivalence class, can be reduced to another word, using the group's relations. This can be thought of as a rewriting
system. However, these equations are not directed, so it is not always possible to construct a well-behaved rewriting
system. In general, the word problem for groups is proven to be undecidable.

\jk{Example?}
\todo{Examples: empty relation, full relation, van Kampen of $\pi_{1}$}

\subsection{Presenting the permutation group}

We have already seen that the group of permutations, or the symmetric group $\Sn$, can be defined by taking the type
$\Aut[\Fin[n]]$ as the underlying set, and showing that it has the appropriate group structure. Now we're going to give
a presentation for it, by defining a set of generators and relations.

There is an element of choice here -- as we already mentioned, a group can be presented in many different ways. For
example, we could generate the permutation group on $\Fin[n]$ by using generators that:

\begin{itemize}
  \item swap the $i$-the element with the $(i+1)$-th element, that is, adjacent swaps, or
  \item swap the $i$-th element with the $j$-th element, for arbitrary $i$-s and $j$-s, or
  \item swap the $i$-th element with an element at a fixed position, or
  \item flip a prefix $\Fin[k]$ of $\Fin[n]$ for $k \leq n$, or
  \item cyclically shift any subset of $\Fin[n]$.
\end{itemize}

One way of thinking about these presentations is via sorting algorithms, which use different primitive operations. A
sorting algorithm has to calculate a permutation of a list or a finite set, which satisfies the invariant of being a
sorted sequence, which means, the primitive operations of a sorting algorithm are able to generate all the permutations
on a given list. So, a chosen set of reversible operations in a sorting algorithm can be a good candidate for the
generators of a permutation group.

For example, bubble sort uses the primitive operation of adjacent swaps, insertion sort and selection sort use the
primitive operation of swapping the $i$-th element with the $j$-th element, cycle sort uses cyclical shifts of
subsequences, pancake sort uses flips of prefixes of the list, et cetera~\todo{check!}. The choice of generators for our
presentation is important for the following reasons.

\begin{itemize}
  \item It affects the difficulty of solving the word problem in $\Sn$ and formalising the proof of its correctness. In
        the following subsections, we build a rewriting system using the generators and relations, and we need to prove
        and formalise strong normalisation for this system.
  \item The choice of generators dictates which words become normal forms in this presentation of $\Sn$. These normal
        forms dictate the shape of the synthesised and normalised boolean circuits, which is the application we have in
        mind.
  \item Finally, the generators have to closely match the $\PiLang$ combinators so that we can quote back a permutation
        to a $\PiLang$ program, for the proof of completeness.
\end{itemize}

We use a presentation based on adjacent transpositions for generators, called a~\emph{Coxeter presentation}, which we
also use to solve the word problem for $\Sn$. Connecting it back to the language $\PiLang$, we will show that it is
possible to encode all $\PiLang$ combinators using adjacent transpositions (in~\cref{sec:equivalence}).

\subsubsection{Coxeter Presentation}

The primitive operations we use are going to be adjacent swaps. When dealing with permutations on an $n + 1$-element
set, there are $n$ adjacent transpositions, transposition number $k$ swapping elements $\el{k}$ and $\el{k+1}$. Thus,
the generating set would be $\Fin[n]$. There are three relations that we're going to specify for this presentation --
these are the laws that these generators should satisfy. It is easiest to visualise them as braid diagrams.

First, swapping the same two elements two times in a row should be the same as doing nothing:

\[
  \begin{tabular}{m{0.3\linewidth}m{0.1\linewidth}m{0.3\linewidth}}
    \begin{center}
      \begin{tikzpicture}
        \pic[local bounding box=my braid,braid/.cd,
          number of strands = 2,
          thick]
        {braid={ s_1, s_1}};
      \end{tikzpicture}
    \end{center}
     &
    \(\xlongrightarrow[]{\cancel}\)
     &
    \begin{center}
      \begin{tikzpicture}
        \pic[local bounding box=my braid,braid/.cd,
          number of strands = 2,
          thick]
        {braid={1, 1}};
      \end{tikzpicture}
    \end{center}
  \end{tabular}
\]

Second, when swapping two distinct pairs of elements (i.e. when the indices of two transpositions differ by at least 1),
it should not matter in which order the swapping happens, that is, we can slide the wires freely.

\[
  \begin{tabular}{m{0.3\linewidth}m{0.1\linewidth}m{0.3\linewidth}}
    \begin{center}
      \begin{tabular}{m{0.3\linewidth}m{0.1\linewidth}m{0.3\linewidth}}
        \begin{tikzpicture}
          \pic[local bounding box=my braid,braid/.cd,
            number of strands = 2,
            thick]
          {braid={ 1, s_1}};
        \end{tikzpicture}
         &
        \(\cdots\)
         &
        \begin{tikzpicture}
          \pic[local bounding box=my braid,braid/.cd,
            number of strands = 2,
            thick]
          {braid={ s_1, 1 }};
        \end{tikzpicture}
      \end{tabular}
    \end{center}
     &
    \(\xlongrightarrow[]{\swap}\)
     &
    \begin{center}
      \begin{tabular}{m{0.3\linewidth}m{0.1\linewidth}m{0.3\linewidth}}
        \begin{tikzpicture}
          \pic[local bounding box=my braid,braid/.cd,
            number of strands = 2,
            thick]
          {braid={ s_1, 1}};
        \end{tikzpicture}
         &
        \(\cdots\)
         &
        \begin{tikzpicture}
          \pic[local bounding box=my braid,braid/.cd,
            number of strands = 2,
            thick]
          {braid={ 1, s_1}};
        \end{tikzpicture}
      \end{tabular}
    \end{center}
  \end{tabular}
\]

The previous two cases were for the transpositions that either completely overlap, or not overlap at all. The third case
says what happens if we perform transpositions that move the same elements. If the next transposition in the sequence is
equal to the the first one, we endorce a third law -- the braiding relation.

\[
  \begin{tabular}{m{0.3\linewidth}m{0.1\linewidth}m{0.3\linewidth}}
    \begin{center}
      \begin{tikzpicture}
        \pic[local bounding box=my braid,braid/.cd,
          number of strands = 3,
          thick]
        {braid={ s_2, s_1, s_2}};
      \end{tikzpicture}
    \end{center}
     &
    \(\xlongrightarrow[]{\braid}\)
     &
    \begin{center}
      \begin{tikzpicture}
        \pic[local bounding box=my braid,braid/.cd,
          number of strands = 3,
          thick]
        {braid={ s_1, s_2, s_1}};
      \end{tikzpicture}
    \end{center}
  \end{tabular}
\]

\todo{an example/motivation from book.pdf}

This construction is called a Coxeter presentation. Writing it formally, we take the generating set to be $\Fin[n]$,
where the element $k$ corresponds to an adjacent transpositions $\tau_k$, which swaps elements $\el{k}$ and $\el{k+1}$.
Then, we define a binary relation $\cox$ on $\List[\Fin[n]]$, which encodes the laws discussed above.

\begin{definition}[$\cox$]
  \begin{align*}
    \cancel
     & : \forall n \to (n \cons n \cons \nil) \cox \nil                                                     \\
    \swap
     & : \forall k, n \to (\suc[k] < n) \to (n \cons k \cons \nil) \cox (k \cons n \cons \nil)              \\
    \braid
     & : \forall n \to (\suc[n] \cons n \cons \suc[n] \cons \nil) \cox (n \cons \suc[n] \cons n \cons \nil) \\
  \end{align*}
\end{definition}

We define $\cox*$ as the congruence closure of $\cox$.

\begin{definition}[$\cox*$]
  \begin{align*}
    \reflr{\cox}
     & : \forall w \to w \cox* w                                                                                                           \\
    \symr{\cox}
     & : \forall w_{1}, w_{2} \to w_{1} \cox* w_{2} \to w_{2} \cox* w_{1}                                                                  \\
    \transr{\cox}
     & : \forall w_{1}, w_{2}, w_{3} \to  w_{1} \cox* w_{2} \to w_{2} \cox* w_{3} \to w_{1} \cox* w_{3}                                    \\
    \congrf{\cox}{\append}
     & : \forall w_{1}, w_{2}, w_{3}, w_{4} \to  w_{1} \cox* w_{2} \to w_{3} \cox* w_{4} \to w_{1} \append w_{3} \cox* w_{2} \append w_{4} \\
    \relr{\cox}
     & : \forall w_{1}, w_{2} \to w_{1} \cox w_{2} \to w_{1} \cox* w_{2}                                                                   \\
  \end{align*}
\end{definition}

To solve the word problem for $\Sn$, we turn the relations into a rewriting system $(\List[\Fin[n]],\cox*)$. A normal
form of an element can then be computed by following the rewrite rules, and two words can be compared for
$\cox*$-equality by comparing their normal forms. A well-behaved rewriting system has to be strongly normalising and
confluent. First, we observe that after throwing out reflexivity and symmetry, the right hand sides of the relations
$\cox*$ are strictly smaller than the left hand sides, in terms of the lexicographical ordering on words in $\Fin[n]$.
Thus, by directing the relation from left to right, we would get the termination property out of the box.

The system also has to be confluent, meaning that all critical pairs, that is, terms with overlapping possible reduction
rules, have to converge. For example, the pairs below converge.

\[
  \begin{array}{lcr}
    \gspan[\braid][\braid]{\tau_2\tau_1\tau_2\tau_1\tau_2}{\tau_1\tau_2\tau_1\tau_1\tau_2}{\tau_2\tau_1\tau_1\tau_2\tau_1}
     &
    \text{or}
     &
    \gspan[\braid][\cancel]{\tau_2\tau_1\tau_2\tau_2}{\tau_1\tau_2\tau_1\tau_2}{\tau_2\tau_1}
  \end{array}
\]

\begin{proof}
  \[
    \begin{array}{lr}
      % https://q.uiver.app/?q=WzAsNixbMiwwLCJcXHRhdV8yXFx0YXVfMVxcdGF1XzJcXHRhdV8xXFx0YXVfMiJdLFswLDEsIlxcdGF1XzFcXHRhdV8yXFx0YXVfMVxcdGF1XzFcXHRhdV8yIl0sWzQsMSwiXFx0YXVfMlxcdGF1XzFcXHRhdV8xXFx0YXVfMlxcdGF1XzEiXSxbMCwyLCJcXHRhdV8xXFx0YXVfMlxcdGF1XzIiXSxbMiwzLCJcXHRhdV8xIl0sWzQsMiwiXFx0YXVfMlxcdGF1XzJcXHRhdV8xIl0sWzAsMSwiYnJhaWQiLDIseyJzdHlsZSI6eyJib2R5Ijp7Im5hbWUiOiJzcXVpZ2dseSJ9fX1dLFswLDIsImJyYWlkIiwwLHsic3R5bGUiOnsiYm9keSI6eyJuYW1lIjoic3F1aWdnbHkifX19XSxbMSwzLCJjYW5jZWwiLDIseyJzdHlsZSI6eyJib2R5Ijp7Im5hbWUiOiJzcXVpZ2dseSJ9fX1dLFszLDQsImNhbmNlbCIsMix7InN0eWxlIjp7ImJvZHkiOnsibmFtZSI6InNxdWlnZ2x5In19fV0sWzIsNSwiY2FuY2VsIiwwLHsic3R5bGUiOnsiYm9keSI6eyJuYW1lIjoic3F1aWdnbHkifX19XSxbNSw0LCJjYW5jZWwiLDAseyJzdHlsZSI6eyJib2R5Ijp7Im5hbWUiOiJzcXVpZ2dseSJ9fX1dXQ==
      \begin{tikzcd}
        && {\tau_2\tau_1\tau_2\tau_1\tau_2} \\
        {\tau_1\tau_2\tau_1\tau_1\tau_2} &&&& {\tau_2\tau_1\tau_1\tau_2\tau_1} \\
        {\tau_1\tau_2\tau_2} &&&& {\tau_2\tau_2\tau_1} \\
        && {\tau_1}
        \arrow["braid"', squiggly, from=1-3, to=2-1]
        \arrow["braid", squiggly, from=1-3, to=2-5]
        \arrow["cancel"', squiggly, from=2-1, to=3-1]
        \arrow["cancel"', squiggly, from=3-1, to=4-3]
        \arrow["cancel", squiggly, from=2-5, to=3-5]
        \arrow["cancel", squiggly, from=3-5, to=4-3]
      \end{tikzcd}
       &
      % https://q.uiver.app/?q=WzAsNSxbMiwwLCJcXHRhdV8yXFx0YXVfMVxcdGF1XzJcXHRhdV8yIl0sWzAsMSwiXFx0YXVfMVxcdGF1XzJcXHRhdV8xXFx0YXVfMiJdLFsyLDQsIlxcdGF1XzJcXHRhdV8xIl0sWzAsMiwiXFx0YXVfMVxcdGF1XzJcXHRhdV8xXFx0YXVfMiJdLFsxLDMsIlxcdGF1XzFcXHRhdV8xXFx0YXVfMlxcdGF1XzEiXSxbMCwxLCJicmFpZCIsMix7InN0eWxlIjp7ImJvZHkiOnsibmFtZSI6InNxdWlnZ2x5In19fV0sWzAsMiwiY2FuY2VsIiwwLHsic3R5bGUiOnsiYm9keSI6eyJuYW1lIjoic3F1aWdnbHkifX19XSxbMSwzLCJicmFpZCIsMix7InN0eWxlIjp7ImJvZHkiOnsibmFtZSI6InNxdWlnZ2x5In19fV0sWzMsNCwiYnJhaWQiLDIseyJzdHlsZSI6eyJib2R5Ijp7Im5hbWUiOiJzcXVpZ2dseSJ9fX1dLFs0LDIsImNhbmNlbCIsMix7InN0eWxlIjp7ImJvZHkiOnsibmFtZSI6InNxdWlnZ2x5In19fV1d
      \begin{tikzcd}
        && {\tau_2\tau_1\tau_2\tau_2} \\
        {\tau_1\tau_2\tau_1\tau_2} \\
        {\tau_1\tau_2\tau_1\tau_2} \\
        & {\tau_1\tau_1\tau_2\tau_1} \\
        && {\tau_2\tau_1}
        \arrow["braid"', squiggly, from=1-3, to=2-1]
        \arrow["cancel", squiggly, from=1-3, to=5-3]
        \arrow["braid"', squiggly, from=2-1, to=3-1]
        \arrow["braid"', squiggly, from=3-1, to=4-2]
        \arrow["cancel"', squiggly, from=4-2, to=5-3]
      \end{tikzcd}
    \end{array}
  \]
\end{proof}

Unfortunately, in the system we defined, this is not true for all critical pairs. One example is the following, where
both endpoints are normal with respect to $\cox*$ relation.

\[
  \gspan[\braid][\swap]{\tau_3\tau_2\tau_3\tau_1}{\tau_2\tau_3\tau_2\tau_1}{\tau_3\tau_2\tau_1\tau_3}
\]

\subsection{Rewriting via Coxeter}

Because of the counter-example discussed, the relations have to be changed. In this section, we will formally define a
rewriting system partially based on the Coxeter relations, and prove that it has the desired properties of confluence
and strong normalisation.

The Coxeter relations are a standard notion in the theory of group presentation~\todo{citation} -- even though we change
the relations a bit, we will not lose the connection. In fact, we will prove that the new relations are equivalent, in a
technical sense, to the standard Coxeter relations $\cox*$.

The new reduction system $(\List[\Fin[n]], \longcox*)$ has generators corresponding to $\swap$, $\cancel$ and $\braid$.
We fix the problem of the non-converging critical pairs discussed previously by changing $\braid$ to be a slightly more
general relation $\longbraid$. \review{Further, we also inline the congruence closure in $\longcox$, allowing arbitrary
  reductions inside the list.}

\todo{but computation is usually done by using Coxeter matrices~\cite{davisGeometryTopologyCoxeter2008}.}

First, we need to define a helper function $n \downf k$.

\begin{definition}[$\downf : (n : \Nat) \to (k : \Nat) \to {\List[\Fin[k + n]]}$]
  \begin{align*}
    n \downf \zero   & \defeq \nil                       \\
    n \downf \suc[k] & \defeq (k + n) \cons (n \downf k)
  \end{align*}
\end{definition}

The result of this function is the sequence \([k + n - 1, k + n - 2, k + n - 3, \ldots, n]\). Since we think of the
number $m$ as the transposition $\tau_m$ which swaps elements $\el{m}$ and $\el{m+1}$, the role of this helper function
is to produce a sequence of transpositions -- a permutation -- which moves element $\el{k + n}$ $k$ places left,
shifting all the elements in between one place right. Expressed in terms of the braiding diagram, for $n = 0$ and $k =
  4$, it has the following form:

\[
  \begin{tikzpicture}
    \pic[local bounding box=my braid,braid/.cd,
      number of strands = 5,
      thick]
    {braid={s_4, s_3, s_2, s_1}};
  \end{tikzpicture}
\]

Then, the directed relation $\longcox$ is defined with the following generators.

\begin{definition}[$\longcox$]
  \begin{align*}
    \longcancel
     & : \forall n, l, r \to (l \append n \cons n \cons r) \longcox (l \append r)                                                                   \\
    \longswap
     & : \forall k, n, l, r \to (\suc[k] < n) \to (l \append n \cons k \cons r) \longcox (l \append k \cons n \append r)                            \\
    \longbraid
     & : \forall n, l, r \to (l \append (n \downf 2 + k) \append (1 + k + n) \cons r) \longcox (l \append (k + n) \cons (n \downf 2 + k) \append r) \\
  \end{align*}
\end{definition}

Constructors $\longcancel$ and $\longswap$ correspond directly to the appropriate constructors of $\cox$ and can be
visualised in the same way as before. The remaining constructor uses the helper function to exchange the order of a long
sequence of transpositions and a single transposition afterwards. For example, for $n = 2$ and $k = 3$, it allows for
the reduction $[6, 5, 4, 3, 2, 6] \longcox* [5, 6, 5, 4, 3, 2]$. Visualised on the braid diagram, it looks as follows:

\[
  \begin{tabular}{m{0.4\linewidth}m{0.1\linewidth}m{0.4\linewidth}}
    \begin{center}
      \begin{tikzpicture}
        \pic[local bounding box=my braid,braid/.cd,
          number of strands = 6,
          thick]
        {braid={s_5, s_4, s_3, s_2, s_1, s_5}};
      \end{tikzpicture}
    \end{center}
     &
    \(\xlongrightarrow[]{\longbraid}\)
     &
    \begin{center}
      \begin{tikzpicture}
        \pic[local bounding box=my braid,braid/.cd,
          number of strands = 6,
          thick]
        {braid={s_4, s_5, s_4, s_3, s_2, s_1}};
      \end{tikzpicture}
    \end{center}
  \end{tabular}
\]

Note that the previous $\braid$ rule is a special case of $\longbraid$, with $k = 0$. Also, as before, the left-hand
sides of the relation are strictly larger than the right-hand sides, in terms of the lexicographic order on
$\List[\Fin[n]]$.

We define the relation $\longcox*$ to be the closure of $\longcox$, under reflexivity and (right-step extended)
transitivity.

\begin{definition}[$\longcox*$]
  \begin{align*}
    \reflr{\longcox}
     & : \forall w \to w \longcox* w                                                                               \\
    \transr{\longcox}
     & : \forall w_{1}, w_{2}, w_{3} \to  w_{1} \longcox w_{2} \to w_{2} \longcox* w_{3} \to w_{1} \longcox* w_{3} \\
  \end{align*}
\end{definition}

Despite the increased complexity of the generators, the rewriting system $(\List[\Fin[n]],\longcox*)$ has the properties
we desire. It satisfies confluence, that is, the Church-Rosser (diamond) property, and it is strongly normalising,
because it produces a unique irreducible normal form. Existence of the normal form is implied by the well-foundedness of
the relation -- informally, it does not contain any infinite decreasing sequences. We follow the terminology
of~\cite{krausCoherenceWellFoundednessTaming2020} to state our results formally.~\vc{check!}

\begin{proposition}
  \leavevmode
  \begin{enumerate}
    \item $\longcox$ is (locally) confluent. For every span $\coxspan{w_{1}}{w_{2}}{w_{3}}$, there is a matching
          extended cospan $\coxcospan*{w_{2}}{w_{3}}{w}$.
    \item $\longcox*$ is confluent. For every extended span $\coxspan*{w_{1}}{w_{2}}{w_{3}}$, there is a matching
          extended cospan $\coxcospan*{w_{2}}{w_{3}}{w}$.
    \item $\longcox*$ is well-founded. With $<$ the lexicographic ordering on $\List[\Fin[n]]$, and $w$ being
          $<$-accessible if every $v < w$ is $<$-accessible, we have that every $w$ is $<$-accesible.
    \item $\longcox*$ is strongly normalising. For every $w$, there exists a unique $v$ such that $w \longcox* v$.
  \end{enumerate}
\end{proposition}

The modified form of the Coxeter relations are unwieldy and difficult to prove properties about by induction. However,
we can use the following observation to be able to use the more standard $\cox*$ relation.~\todo{Check Huet's paper for
  this property.}

\begin{proposition}~\label{prop:coxlongcox}
  $\cox*$ and $\longcox*$ are equivalent in the following sense: for every $w$ and $v$, $w \cox* v$ iff there is a $u$
  such that $w \longcox* u$ and $v \longcox* u$.
\end{proposition}

By strong normalisation, we get a unique choice function $\normf : {\List[\Fin[n]]} \to {\List[\Fin[n]]}$ that produces
a normal form for terms of $\List[\Fin[n]]$. We state and prove a couple of important properties enjoyed by $\normf$.

\begin{proposition}
  \leavevmode
  \begin{enumerate}
    \item For all $l : \List[\Fin[n]]$, we have that $l \cox* \normf(l)$.
    \item $\normf$ is idempotent, that is, $\normf \comp \normf \htpy \normf$.
  \end{enumerate}
\end{proposition}

The type $\Sn$ is defined as the set-quotient of $\List[\Fin[n]]$ by $\cox*$,

\begin{definition}[$\Sn$]
  \(\Sn \defeq \quot{\List(\Fin[n])}{\cox*}\)
\end{definition}

Note that this relation $\cox*$ is not $\hProp$-valued, that is, $w \cox* v$ is not a proposition because reductions are
not unique. Because of this, the quotient $\Sn$ is not effective, that is, $\quotrel : w \cox* v \to q(w) \id q(v)$ is
not an equivalence. We could truncate the relation to change this, but then the codomain of $\normf$ also needs to be
truncated.

We could also define equivalence classes on $\List[\Fin[n]]$ to be those terms that have the same normal form. Using the
$\normf$ function, we could define a new relation $(w \approx v) \defeq (\normf(w) \id \normf(v))$ which is
$\hProp$-valued, and quotient $\List[\Fin[n]]$ by $\approx$. Using the properties of the rewriting system we proved, we
make a few observations.

\begin{proposition}
  \leavevmode
  \begin{enumerate}
    \item $\normf$ splits into a section-retraction pair, that is, we have ${\List[\Fin[n]]} \xrightarrow{s} \Sn[n]
            \xrightarrow{r} {\List[\Fin[n]]}$ such that $s \comp r \htpy \normf$ and $r \comp s \htpy \idfunc_{\Sn[n]}$.
    \item \(\im{\quotinc} \eqv \Sn \eqv \im{\normf} \).
  \end{enumerate}
\end{proposition}

Notice however, that a group presentation, as defined in~\cref{def:presentation}, requires the relation to be on the set
of words $A + A$, where the right copy corresponds to the set of formal inverses of the generators. The constructor
$\cancel$ specifies that the inverse of each element is again the same element, using which we can show that our
definition of $\Sn$ is equivalent to the definition of a presented group, by lifting the $\cox*$ relation.

\begin{proposition}
  \leavevmode
  \begin{enumerate}
    \item There is a group structure on $\Sn$, where the identity element is $\nil$, multiplication is given by list
          append, and inverse is given by list reversal.
    \item $\Sn$ is equivalent to the group presented by generators $\Fin[n] + \Fin[n]$ with the relations given by the
          normal closure of $\cox*$ extended to $\List(\Fin[n] + \Fin[n])$ along the codiagonal map $\nabla_{A} : A + A
            \to A$.
  \end{enumerate}
\end{proposition}

We set out to solve the word problem for $\Sn$. To decide if two words in $\List[\Fin[n]]$ are equal, we simply have to
compute their normal forms using $\normf$ and check if they're equal on the nose. If they do, they correspond to the
same permutation.

\subsection{Lehmer Codes}~\label{subsec:lehmer}

To prove the equivalence between $\Aut[\Fin[n]]$ and $\Sn[n]$, we will need to define functions back and forth between
the two types. The terms in $\Sn$ can be identified with equivalence classes of terms in $\List[\Fin[n]]$ with respect
to the Coxeter relation $\cox*$. The easiest way to define a function out of this presentation is to define it on the
representatives. We know that these are the unique normal forms in the set-quotient given by $\quotinc \comp \normf$,
but now we will describe what these representatives exactly look like, using an encoding called Lehmer
codes~\cite{lehmerTeachingCombinatorialTricks1960}.

There are many ways to represent permutations, e.g. inversions, or cycles, or matrices. Lehmer codes are known in
Combinatorial Analysis~\cite{bellmanCombinatorialAnalysis1960} where they are sometimes called "subexcedant sequences",
or "factoriadics", which give the factorial number system. They are a particularly convenient way of representing
permutations on a computer, partly because they are bitwise-optimal: for any $n : \Nat$, the type $\Lehmer[n]$ hass the cardinality $\fac{n}$, and has an easy to consttuct bijection with $\Aut[\Fin[n]]$. .

Formally, we define $\Lehmer[n]$ to be an $n+1$-element tuple, where the position $k \leq n$ stores an element of $\Fin[k]$. Since
the 0-th position is trivial, in practice, it is ignored, and the sequence starts at
1~\cite{duboisTestsProofsCustom2018,vajnovszkiNewEulerMahonian2011}.\todo{There are more examples - listed
  in~\cite{duboisTestsProofsCustom2018}}. We can then define $\Lehmer$ in two equivalent ways, by a simple recursion on
$\Nat$, or as a type family generated by two constructors.

\begin{definition}[$\Lehmer : \Nat \to \UU$]
  \begin{gather*}
    \begin{aligned}
      \Lehmer[\zero]   & \defeq \Fin[\suc[\zero]]                     \\
      \Lehmer[\suc[n]] & \defeq \Fin[\suc[\suc[n]]] \times \Lehmer[n]
    \end{aligned}
    \qquad
    \begin{aligned}
      \lzero & : \Lehmer[\zero]                                                     \\
      \lsuc  & : \forall n, r, (r \leq \suc[n]) \to \Lehmer[n] \to \Lehmer[\suc[n]]
    \end{aligned}
  \end{gather*}
\end{definition}

\todo{Explain why the two definitions are equivalent?}

For a permutation $\sigma : \Aut[\Fin[n]]$, for any element $i: \Fin[n]$, we can define the inversion count of $i$ as
the number of smaller elements appearing after it in the permutation.

\todo{typify}
\begin{definition}
  Given a permutation $\sigma : \Aut[\Fin[n]]$, the inversion count of $k: \Fin[n]$ is given by
  \[ |\Set{j < i | \sigma(j) > \sigma(i)}|. \]
\end{definition}

It turns out that from knowing just the inversion counts for all the elements, one can reconstruct the starting
permutation. Also, observe that the inversion count for element $i$ is guaranteed to be smaller than $i$, thus fitting
in the $i$-th place of a Lehmer code tuple. As an example, consider the following tabulated presentation of a
permutation of $\Fin[5]$.

\[
  \sigma =
  \begin{pmatrix}
    0      & 1      & 2      & 3      & 4      \\
    \el{2} & \el{1} & \el{4} & \el{0} & \el{3} \\
  \end{pmatrix}
\]

The inversion count for $\el{0}$ is 0 (because there are no smaller elements at all), for $\el{1}$ is 1 (because of
$\el{0}$ appearing after), for $\el{2}$ is 2 (because of $\el{0}$ and $\el{1}$), for $\el{3}$ is 0 (because it comes
last in the sequence), and for $\el{4}$ is 2 (because of $\el{3}$ and $\el{0}$). Thus, the Lehmer code for the permutation $\sigma$
is the 5-tuple $l = (0, 1, 2, 0, 2)$.

To decode the permutation back from this Lehmer code, we perform an algorithm similar to \emph{insertion sort}. The
element of the Lehmer code being currently processed is highlighted in the left column of the table below. Starting from
a sorted list, the element at index $k$ has to be given $l[k]$ inversions. Because of the the invariant that all the
elements before newly processed one are smaller than it, the proper number of inversions is created by simply shifting
the element $l[k]$ places left.

\begin{center}
  \begin{tabular}{l|r}
    (\highlight{{0}}, 1, 2, 0, 2) & $[\highlightAlt{\el{0}}, \el{1}, \el{2}, \el{3}, \el{4}]$ \\
    (0, \highlight{{1}}, 2, 0, 2) & $[\highlightAlt{\el{1}}, \el{0}, \el{2}, \el{3}, \el{4}]$ \\
    (0, 1, \highlight{{2}}, 0, 2) & $[\highlightAlt{\el{2}}, \el{1}, \el{0}, \el{3}, \el{4}]$ \\
    (0, 1, 2, \highlight{{0}}, 2) & $[\el{2}, \el{1}, \el{0}, \highlightAlt{\el{3}}, \el{4}]$ \\
    (0, 1, 2, 0, \highlight{{2}}) & $[\el{2}, \el{1}, \highlightAlt{\el{4}}, \el{0}, \el{3}]$ \\
  \end{tabular}
\end{center}

Writing formally, to turn a code into a permutation written in the Coxeter presentation, we define a function
$\immersion$. As described above, the number $r$ at position $k$ in the tuple describes how many inversions the element
$\el{k}$ has. Thus, we need to perform $r$ many adjacent transpositions to get to the desired position, which is given
by $(\suc[n] - r) \downf r$.

\begin{definition}[$\immersion : (n : \Nat) \to {\Lehmer[n]} \to {\List[\Fin[\suc[n]]]}$]
  \begin{align*}
    \immersion(\zero, \zero)    & \defeq \nil                                              \\
    \immersion(\suc[n], (r, l)) & \defeq \immersion(n, l) \append ((\suc[n] - r) \downf r)
  \end{align*}
\end{definition}

As an example, let us look at the code $c = (4, 3, 2, 1, 0)$. The list $\immersion(c)$ is then going to be $[0, 1, 0, 2,
      1, 0, 3, 2, 1, 0, 4, 3, 2, 1, 0]$. It is a concatenation of a sequence of decreasing lists. Where could a reduction
happen? First, it can't happen inside any of the decreasing components: $\longcancel$ requires repeating elements,
$\longswap$ acts when a smaller number precedes a larger one, and $\longbraid$ has a non-monotone sequence on the left.
This leaves the case of a reduction happening on a fragment that borders two subsequences. Again, $\longcancel$ requires
two equal consecutive numbers, which have to be the last one in some decreasing sequence and the first one in the next
one. But the first number in a sequence is always larger than every number in the sequence before it - which also shows
why $\longswap$ cannot happen. The remaining case of $\longbraid$ is proven on similar grounds, since it requires the
number appearing after the decreasing sequence to be equal to the first one in the sequence.

Using this idea, we can show that the function $\immersion$ gives an equivalence betweeen $\Lehmer[n]$ and
$\im{\normf}$.

\begin{proposition}
  \leavevmode
  \begin{enumerate}
    \item For any Lehmer code $c$, $\immersion(c)$ is a normal form with resepct to $\longcox*$, that is, $\immersion(c)$ is
          in $\im{\normf}$.
    \item Any element of $\im{\normf}$ can be constructed from a unique Lehmer code by $\immersion$, that is, the fibers
          of $\immersion: \Lehmer[n] \to {\im{\normf}}$ are contractible.
  \end{enumerate}
  Therefore, there is an equivalence between $\Lehmer[n]$ and $\im{\normf}$.
\end{proposition}

We also established an equivalence between $\im{\normf}$ and $\Sn$, which gives the following.

\begin{proposition}~\label{prop:sn-im-lehmer-equiv}
  For all $n : \Nat$, \( \Sn \eqv \im{\normf} \eqv \Lehmer[n] \).
\end{proposition}

\subsection{Running Lehmer codes}

Finally, it is time to complete our goal of characterising the permutation groups. Having produced a Lehmer code by
normalising words in $\Sn$, we need to run it to produce a concrete bijection of finite sets, and, given a bijection
between finite sets, we need to encode it as a Lehmer code. We will prove that these maps construct an equivalence
between the types $\Lehmer[n]$ and $\Aut[\Fin[\suc[n]]]$.

To do so, we need to construct some equivalences by counting the elements of $\Fin[n]$ using its decidable equality.
First, we define a helper type family $\FinExcept{n} : \Fin[n] \to \UU$ which picks out all elements in $\Fin[n]$ except
the one in the argument. Note that $\FinExcept{n}[i]$ for $i : \Fin[n]$ is a subtype of $\Fin[n]$ and is hence an
$\hSet$.

\begin{definition}
  \( \FinExcept{n}[i] \defeq \dsum{j : \Fin[n]}{i \neq j} \).
\end{definition}

We state and prove a few auxiliary lemmas about how $\FinExcept{n}$ interacts with $\Fin$.

\begin{proposition}~\label{prop:fin-finexcept}
  \leavevmode
  \begin{enumerate}
    % \item For any $k : \Fin[n]$, $\unit \sqcup \FinExcept{n}[k] \eqv \Fin[n]$. \label{prop:fin-finexcept-1}
    \item For any $k : \Fin[\suc[n]]$, $\FinExcept{\suc[n]}[k] \eqv \Fin[n]$. \label{prop:fin-finexcept-2}
    \item For any $n : \Nat$, \( \Aut[\Fin[\suc[n]]] \eqv \dsum{k : \Fin[\suc[n]]}{\FinExcept{\suc[n]}[n] \eqv
            \FinExcept{\suc[n]}[n - k]} \). \label{prop:fin-finexcept-3}
  \end{enumerate}
\end{proposition}

\begin{proof}
  The first and second propositions follow from simply constructing the bijections using the decidable equality of
  $\Fin[n]$, and making sure to punch-in and punch-out the element $k$ at the right place.

  The third proposition performs some combinatorial tricks. On the left, we have the type of automorphisms of
  $\Fin[\suc[n]]$. Assume a particular $\sigma : \Fin[\suc[n]] \xrightarrow{\sim} \Fin[\suc[n]]$. Pick $k$ to be the
  inversion count of $n$, the largest element in $\Fin[\suc[n]]$. Then, the image of $n$ under $\phi$ has to be $n - k$,
  since all other elements in the set are smaller. Removing those two from the domain and codomain of $\phi$, the rest
  of the elements are fixed by $\sigma$, so we compute the bijection between the rest of the elements.

  For the other direction, if we are given a $k$ and a bijection $\pi$ between $\FinExcept{\suc[n]}[n]$ and
  $\FinExcept{\suc[n]}[n - k]$, we can extend $\pi$ to $\sigma : \Fin[\suc[n]] \eqv \Fin[\suc[n]]$ by inserting the
  element $n$ at the position $n - k$, resulting in the element $n$ having inversion count $k$.
\end{proof}

Using these facts, we can now prove the main result of this section.

\begin{proposition}~\label{prop:lehmer-aut-equiv}
  For all $n:\Nat$, \( \Lehmer[n] \eqv \Aut[\Fin[\suc[n]]] \).
\end{proposition}

\begin{proof}
  For $n = \zero$, note that $\Lehmer[\zero]$ is contractible, and so is $\Aut[\Fin[\suc[\zero]]]$. For $n = \suc[m]$,
  we compute a chain of equivalences.
  \begin{gather*}
    \arraycolsep=0.5em\def\arraystretch{1.5}
    \begin{array}{rl}
           & \Lehmer[\zero]          \\
      \eqv & \unit                   \\
      \eqv & \Aut[\Fin[\suc[\zero]]] \\
    \end{array}
    \qquad\qquad
    \begin{array}{rlr}
           & \Lehmer[\suc[m]]                                                                                   &                                                                 \\
      \eqv & \Fin[\suc[\suc[m]]] \times \Lehmer[m]                                                              & \text{by definition}                                            \\
      \eqv & \Fin[\suc[\suc[m]]] \times \Aut[\Fin[\suc[m]]]                                                     & \text{induction hypothesis}                                     \\
      \eqv & \dsum{k : \Fin[\suc[\suc[m]]]}{\Fin[\suc[m]] \eqv \Fin[\suc[m]]}                                   & \text{$\Sigma$ over a constant family}                          \\
      \eqv & \dsum{k : \Fin[\suc[\suc[m]]]}{\FinExcept{\suc[\suc[m]]}[m] \eqv \Fin[\suc[m]]}                    & \text{by~\cref{prop:fin-finexcept}~\cref{prop:fin-finexcept-2}} \\
      \eqv & \dsum{k : \Fin[\suc[\suc[m]]]}{\FinExcept{\suc[\suc[m]]}[m] \eqv \FinExcept{\suc[\suc[m]]}[m - k]} & \text{by~\cref{prop:fin-finexcept}~\cref{prop:fin-finexcept-2}} \\
      \eqv & \Aut[\Fin[\suc[\suc[m]]]]                                                                          & \text{by~\cref{prop:fin-finexcept}~\cref{prop:fin-finexcept-3}} \\
    \end{array}
  \end{gather*}
\end{proof}

By composing~\cref{prop:lehmer-aut-equiv} and~\cref{prop:sn-lehmer-fin-equiv}, we obtain the final equivalence.

\begin{corollary}
  For all $n : \Nat$,
  \(
  \Sn \eqv \Lehmer[n] \eqv \Aut[\Fin[\suc[n]]]
  \).
\end{corollary}

%%% Local Variables:
%%% mode: latex
%%% TeX-master: "main"
%%% fill-column: 120
%%% End:
