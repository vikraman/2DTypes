\section{The groupoid of finite types}~\label{sec:finite}

In this section, we describe the algebraic structure of the groupoid of finite types by characterising its paths.

\subsection{Permutations}

In the previous~\cref{sec:univalent}, we established that paths in $\UFin$ are equivalent to families of automorphisms
of $\Fin{n}$ for every $n:\Nat$. These are permutations on finite sets of size $n$. In the following sections, we will
characterise these permutations, going through a number of intermediate step.

\jk{I don't like the wording}

\subsubsection{Lehmer codes}

There are $\fac{n}$ permutations on a set of $n$ elements. Thus, if we construct a type with $\fac{n}$ inhabitants,
there should be a bijection of that type with the type of permutations $\Aut[\Fin[n]]$. To construct this bijection, we
will borrow the idea of Lehmer codes~\cite{lehmerTeachingCombinatorialTricks1960a} from Combinatorial
Analysis~\cite{bellmanCombinatorialAnalysis1960}. The type family $\Lehmer$ over $\Nat$ is defined as follows.

\todo{Understand the proof first and then revisit this paragraph}

\begin{definition}
    \begin{align*}
         & \Lehmer : \Nat \to \UU                             \\
         & \Lehmer[0] = \unit                                 \\
         & \Lehmer[\suc[n]] = \Fin[\suc[n]] \times \Lehmer[n]
    \end{align*}
\end{definition}

Now we construct the equivalence.

\begin{proposition}
    For all $n:\Nat$,
    \[
        \Lehmer[n] \eqv \Aut[\Fin[n]]
    \]
\end{proposition}

\subsubsection{Symmetric groups}

There is an obvious group structure on $\Aut[\Fin[n]]$ given by identity, composition, and inverse. This is the
symmetric group $S_n$ on $n$ symbols. In the rest of the section we will construct a convenient presentation of this
group.

\todo{Reference T-algebra presentations as coequalisers (Mac Lane 6.7)}

First, we formally define a presentation of a group.

\begin{definition}
  A presentation of a group $G$ is a type $FR$ such that \ldots is a coequaliser.
\end{definition}

\vc{this is just a rough draft for now}

\begin{definition}
  Here we'll define a type family $FT n$ to be a free group generated by a type $T n$ with the following constructors
  \begin{align*}
    & cancel : i : \Fin[\suc[n]] \to T \\
    & swap : (i : \Fin[\suc[n]]) \to (j : \Fin[\suc[n]]) \to (i + 1 < j) \to T \\
    & braid : (i : \Fin[n]) \to T \\
  \end{align*}
\end{definition}

\begin{definition}[Adjacent transposition]
  \begin{align*}
    transpose : (n : \Nat) \to (k : \Fin[\suc[n]]) \to \Aut[\Fin[\suc[n]]]
  \end{align*}
  by double induction on $n$ and $k$, where
  \begin{align*}
    transpose (n) (0) & = \lambda
    0 \to 1;
    1 \to 0;
    m \to m
    \\
    transpose (\suc[n]) (\suc[k]) & = \lambda
    0 \to 0;
    S m \to S ((transpose (n)(k)) m)
  \end{align*}
\end{definition}

\begin{proposition}
  $FR$ is a presentation of the symmetrc group, with $f$ and $g$ as follows:
  \begin{align*}
    f (inr (cancel (i))) & = inr (i :: i :: []) \\
    f (inl (cancel (i))) & = inl (i :: i :: []) \\
    f (inr (swap (i,j))) & = i :: j :: i :: j :: [] \\
    f (braid (i)) & = i :: S i :: i :: S i :: i :: S i :: []
  \end{align*}
  and
  \begin{align*}
    g (inl (\_)) & = inl ([]) \\
    g (inr (\_)) & = inr ([])
  \end{align*}
\end{proposition}

We now introduce a presentation of the symmetric group.

%%% Local Variables:
%%% mode: latex
%%% TeX-master: "main"
%%% End:
