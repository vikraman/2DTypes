\section{The groupoid of finite types}~\label{sec:finite}

In this section, we describe \review{the algebraic structure} of the groupoid of
finite types, and give \review{a computable presentation} for it.

\vc{The groupoid of finite types is the free symmetric monoidal groupoid on one
  generator. This can be presented as an algebraic 2-theory, which is our syntax
  for $\PiHatLang$. Vertical categorification of natural numbers as a free
  commutative monoid. See groupoidification.}

\todo{Check Brent Yorgey's thesis?}

To do so, we will characterise the automorphisms on finite sets of cardinality
$n$, and show them to be equivalent to the symmetric group $\Sn$, via the
Coxeter presentation. We will do that in two steps,
in~\cref*{subsec:permutations,subsec:lehmer,subsec:symmetric}.

\todo{Big example: Start from a listed permutation, show Lehmer code, then adjacent swaps.}
\todo{Justification for why we need group theory.}

%% Coxeter presentation of $\Sn$ $\eqv$ $\Lehmer[n]$ $\eqv$ $\Aut[\Fin[n]]$

\subsection{Groups}

From universal algebra, a group is simply a set with a 0-ary constant, the neutral element, a binary operation for group
multiplication, and a unary inverse operation. A simple example is $\mathbb{Z}/n\mathbb{Z}$, where the neutral element
is 0, the inverse of $k$ is $-k$, and the group multiplication is given by addition modulo $n$. The neutral element has
to satisfy unit and inverse laws, and the multiplication has to be associative.

In type theory, a group $G$ can be defined as an $\hSet$ $S$ with the following pieces of data:

\begin{enumerate}
  \item a unit or neutral element $e : S$
  \item a multiplication function $m : S \times S \to S$ written as $(g_{1}, g_{2}) \mapsto g_{1} \mult g_{2}$, that satisfies
  \begin{enumerate}
    \item the unit laws, for all $g : S$, that \( g \mult e \id g \) and \( e \mult g \id g \)
    \item the associativity law, for all $g_{1}, g_{2}, g_{3} : S$, that \( g_{1} \mult (g_{2} \mult g_{3}) \id (g_{1} \mult g_{2}) \mult g_{3} \)
  \end{enumerate}
  \item an inversion function $i : S \to S$ written as $g \mapsto \inv{g}$, that satisfies
  \begin{enumerate}
    \item the inverse laws, for all $g : S$, that \( g \mult \inv{g} \id e \) and \( \inv{g} \mult g \id e \)
  \end{enumerate}
\end{enumerate}

However, more conveniently, in HoTT, we can instead use groupoids to talk about groups. A group can be identified with a
1-object groupoid, using a technique called delooping. The delooping of a group $G$ is a groupoid $\B{G}$ given by a
unique object $\pt$ with self-loops that are 1-paths $\pt \id_{\B{G}} \pt$ corresponding to the elements of $G$. Note
that the group operations are automatically given by operations on the identity type, with $\refl_{\pt}$ for the neutral
element, path composition for the group multiplication, and path inverse for the group inverse. These satisfy the group
laws as well, up to the identity type, using the groupoid coherence laws. Moreover, for 1-groups which are supposed to
be sets, these 2-paths should be propositions, so we have to restrict $\B{G}$ to be a 1-groupoid. Hence, a group is
simply given by a pointed, connected 1-type~\cite*{buchholtzHigherGroupsHomotopy2018,symmetryBook2021}.

For example, given a pointed type $(A:\UU, a:A)$, the automorphism group structure at $a$ is given by $a \id_{A} a$. Of
course, for 1-groups we will require that $a \id_{A} a$ is an $\hSet$, which is enforced by having $A$ be a groupoid. In
our running example for the permutation group on finite sets, we have that $\Fin[n]$ is an $\hSet$, and hence,
$\UFin[n] \defeq \BAut[\Fin[n]]$ is a pointed, connected 1-type, whose loopspace $\loopspace[\BAut[\Fin[n]],F_{n}]$ is
equivalent to $\Aut[\Fin[n]] \defeq (\Fin[n] \eqv \Fin[n]) \eqv (\Fin[n] \id_{\UU} \Fin[n])$, which has the
corresponding automorphism group structure.

Alternatively, groups can also be presented using generators and relations, or a quotient of the free group \ldots
\subsection{Permutations}~\label{subsec:permutations}

In the previous~\cref{sec:univalent}, we established that paths in $\UFin$ are
equivalent to families of automorphisms of $\Fin{n}$ for every $n:\Nat$, that
is, bijections on finite sets of size $n$. This is the extensional view of
permutations. In the following sections, we will characterise these
permutations, going through two intermediate steps.

\vc{This is obvious, maybe add something more here.}

\subsection{Lehmer codes}~\label{subsec:lehmer}

From grade school combinatorics, we know that there are $\fac{n}$ permutations
on a finite set with $n$ elements. The factorial function is defined by
recursion on natural numbers. However, now, for every $n$, we want to produce a
type, which is a finite set, with cardinality $\fac{n}$. And, to characterise
$\Aut[\Fin[n]]$, we further need to construct a bijection between this type and
$\Aut[\Fin[n]]$.

First, let's define this type with $\fac{n}$ elements, we name this type family
$\Lehmer : \Nat \to \UU$, which is defined by recursion on $\Nat$ as follows.
This is the obvious definition of factorials by recursion, but categorified from
natural numbers to sets.

\begin{definition}
  \begin{align*}
    \Lehmer[0]       & \defeq \unit                           \\
    \Lehmer[\suc[n]] & \defeq \Fin[\suc[n]] \times \Lehmer[n]
  \end{align*}
\end{definition}

\todo{Subexcedant sequences and factorial definitions are equivalent, explain
  this!}

The name Lehmer comes from Lehmer
codes~\cite{lehmerTeachingCombinatorialTricks1960} which are known in
Combinatorial Analysis~\cite{bellmanCombinatorialAnalysis1960}. There are many
ways to represent permutations, e.g. inversions, or cycles, or matrices. Lehmer
codes are a particularly convenient way to represent permutations on a
computer,~\review{they are compact and have exactly the right cardinality.
  $\Lehmer[n]$ is a $n+1$-element tuple, where the position $k \leq n$ has an
  element of $\Fin[k]$. The 0-th position is trivial, so we ignore it, and in
  both the example below and the Agda proof, consider only the remaining
  $n$-element tuple.}

\vc{This is just the classical algorithm to explain the example, not the actual
  type-theoretic proof.}

Suppose we have a permutation $p$ on an $n$-element set
$\{\el{0}, \el{1}, \el{2}, \el{3}, \el{4}\}$, we encode it as follows.
$\Lehmer[n]$ is a $n$-element tuple. At position $k$, we put the number of
inversions of the element $\el{k}$ in $p$, i.e. the number of elements smaller
than $\el{k}$ occurring after $\el{k}$.

As an example, consider the following tabulated presentation of the permutation:

\todo{fix this figure}

\[
  p =
  \begin{array}{ccccccccccccccc}
    | & 0      & | & 1      & | & 2      & | & 3      & | & 4      & | \\
    \hline                                                             \\
    | & \el{2} & | & \el{0} & | & \el{1} & | & \el{4} & | & \el{3} & | \\
    \hline                                                             \\
  \end{array}
\]

%  0 1 2 3 4
% -----------
% |2|0|1|4|3|
% -----------

The element $\el{0}$ has 0 inversions, because there are no elements smaller
than $\el{0}$ occurring after it. In fact, there can be no elements smaller than
$\el{0}$ at all, so the type at the first position of the Lehmer code tuple is
$\unit$.

The element $\el{1}$ has 0 inversions as well, since elements occurring after it
in the permutation are $\el{4}$ and $\el{2}$. There is only one different case,
if $\el{1}$ appeared before $\el{0}$, it would have 1 inversion. This is why the
type of the second component of the Lehmer code is $\Fin[2]$.

The element $\el{2}$ has 2 inversions, because both $\el{0}$ and $\el{1}$ occur
after it in the permutation. The element $\el{3}$ occurs as the last one, so it
has 0 inversions. The element $\el{4}$ has 1 inversion, with the element
$\el{3}$.

Thus, the Lehmer code for the permutation $p$ is the 5-tuple
$l = (0, 0, 2, 0, 1)$.

To reconstruct the tabulated presentation of the permutation from the Lehmer
code, we perform an algorithm similar to \emph{insertion sort}. Starting from
the left-most position of the tuple $l$, we'll read the value $v$, insert the
new element at the end of the newly created list, and shift it backward $v$
places.

\begin{center}
  \begin{tabular}{c|p{0.75\linewidth}}
    (0, 0, 2, 0, 1)               & We start from an empty list $[]$                                                                 \\
    (\highlight{{0}}, 0, 2, 0, 1) & We read 0 as the left-most value from $l$. Thus, we append the element $\el{0}$ to our
                                    list, getting $[\el{0}]$. The element is shifted $0$ places, so it remains in the
                                    same place.                                                                                                                      \\
    (0, \highlight{{0}}, 2, 0, 1) & Then, similarly, we read another 0 for the element $\el{1}$, append it to the
                                    list getting $[\el{0}, \el{1}]$, and don't shift it either.                                                                      \\
    (0, 0, \highlight{{2}}, 0, 1) & We read 2 for the next the element $\el{2}$ - we append $\el{2}$ to our list, getting
                                    $[\el{0}, \el{1}, \el{2}]$, and shift it 2 places right, which results in a list $[\el{0}, \el{2}, \el{1}]$
                                    \todo{Typeset it nicely, with arrows showing the shifting}.                                                                      \\
    (0, 0, 2, \highlight{{0}}, 1) & Then we read 0 - appending $\el{3}$ and not shifting, getting $[\el{0}, \el{2}, \el{1}, \el{3}]$ \\
    (0, 0, 2, 0, \highlight{{1}}) & Finally, reading 1 for element $\el{4}$ - appending $\el{4}$ to the list and shifting it
                                    one place right results in the final list $[\el{0}, \el{2}, \el{1}, \el{4}, \el{3}]$                                             \\
  \end{tabular}
\end{center}
\todo{figure}

Using this Lehmer encoding algorithm, we can now construct the equivalence between these types.

We define a type family $\FinExcept{n} : \Fin[n] \to \UU$ which picks out all elements in $\Fin[n]$ except the one
provided. Note that $\FinExcept{n}[i]$ for $i : \Fin[n]$ is a subtype of $\Fin[n]$ and is hence an $\hSet$.

\begin{definition}
  \( \FinExcept{n}[i] \defeq \dsum{j : \Fin[n]}{i \neq j} \).
\end{definition}

\begin{proposition}
  For any $k : \Fin[n]$, $\unit \sqcup \FinExcept{n}[k] \eqv \Fin[n]$.
\end{proposition}

\begin{proposition}
  For any $k : \Fin[\suc[n]]$, $\FinExcept{\suc[n]}[k] \eqv \Fin[n]$.
\end{proposition}

\begin{proposition}
  For any $n : \Nat$,
  \( \Aut[\Fin[\suc[n]]] \eqv \dsum{k : \Fin[\suc[n]]}{\FinExcept{\suc[n]}[\fzero] \eqv \FinExcept{\suc[n]}{k}} \).
\end{proposition}

\begin{proposition}
  For all $n:\Nat$, \( \Lehmer[n] \eqv \Aut[\Fin[n]] \).
\end{proposition}

\begin{proof}
  For $n = 0$, note that $\Lehmer[0]$ is contractible, and so is $\Aut[\Fin[0]]$. For $n = \suc[m]$, we have the
  following chain of equivalences.
  \[\arraycolsep=0.5em\def\arraystretch{1.5}
    \begin{array}{rl}
      & \Aut[\Fin[\suc[m]]] \\
      \eqv & \dsum{k : \Fin[\suc[m]]}{\FinExcept{\suc[m]}[\fzero] \eqv \FinExcept{\suc[m]}[k]} \\
      \eqv & \dsum{k : \Fin[\suc[m]]}{\FinExcept{\suc[m]}[\fzero] \eqv \Fin[m]} \\
      \eqv & \dsum{k : \Fin[\suc[m]]}{\Fin[m] \eqv \Fin[m]} \\
      \eqv & \Fin[\suc[m]] \times \Aut[\Fin[m]] \\
      \eqv & \Fin[\suc[m]] \times \Lehmer[m] \\
    \end{array}
  \]
\end{proof}

\subsection{Symmetric groups}~\label{subsec:symmetric}

There is an obvious group structure on $\Aut[\Fin[n]]$ given by identity,
composition, and inverse. This is the symmetric group $S_n$ on $n$ symbols. In
the rest of the section we will construct a convenient presentation of this
group.

\vc{this is just a rough draft for now}

\todo{Reference T-algebra presentations as coequalisers (Mac Lane 6.7)}

First, we formally define a presentation of a group.

\todo{Free 1-group, needs to be truncated}

\begin{definition}
  Let $A$ be a type, and $\List[A]$ the free monoid on $A$. The free group $F(A)$ on $A$ is the set-quotient of
  $\List[A + A]$ by the congruence closure of the relation $(\inl(a) \cons \inr(a) \cons \nil) \sim \nil$.
\end{definition}

\begin{proposition}
  Let $\eta_{A} : A \to F(A)$ be given by $a \mapsto \inl(a) \cons \nil$. Given any group $G$ with a map $f : A \to G$,
  there is a canonical lifting of $f$ to a group homomorphism $\extend{f} : F(A) \to G$, such that
  $\extend{f} \comp \eta_{A} \htpy f$, that is, the type of group homomorphisms $h : \Grp(F(A),G)$ satisfying
  $h \comp \eta_{A} \htpy f$ is contractible. Equivalently, composition by $\eta_{A}$ is an equivalence
  $\Grp(F(A),G) \eqv (A \to G)$.
\end{proposition}

\todo{diagram}
\todo{Plus/Minus}

\begin{definition}
  Let $A$ be a type and $R : A \to A \to \UU$ a binary relation on $A$. The group $G$ presented by
  $\langle A ; R \rangle$ is given by the set-quotient of the free group $F(A)$ by the normal closure $N_{R}$ of $R$, or
  equivalently, as the coequaliser
  \[\begin{tikzcd}
      FR && FA && G
      \arrow[shift right=2, from=1-1, to=1-3]
      \arrow[shift left=2, from=1-1, to=1-3]
      \arrow[two heads, from=1-3, to=1-5]
    \end{tikzcd}\]
\end{definition}

\todo{Universal property}
\todo{Examples: empty relation, van Kampen of $\pi_{1}$}

\subsubsection{Coxeter Relations}

We define a binary relation on $\List[\Fin[n]]$.

\begin{definition}[$\cox$]
  \begin{align*}
    \cancel
    & : \forall n \to (n \cons n \cons \nil) \cox \nil \\
    \swap
    & : \forall k, n \to (\suc[k] < n) \to (n \cons k \cons \nil) \cox (k \cons n \cons \nil) \\
    \braid
    & : \forall n \to (\suc[n] \cons n \cons \suc[n] \cons \nil) \cox (n \cons \suc[n] \cons n \cons \nil) \\
  \end{align*}
\end{definition}

We define $\cox*$ as the congruence closure of $\cox$.
\vc{This could be simplified a bit depending on our formalisation. $\cox$ is directed, use a different symbol?}

\begin{definition}[$\cox*$]
  \begin{align*}
    \reflr{\cox}
    & : \forall w \to w \cox* w \\
    \symr{\cox}
    & : \forall w_{1}, w_{2} \to w_{1} \cox* w_{2} \to w_{2} \cox* w_{1} \\
    \transr{\cox}
    & : \forall w_{1}, w_{2}, w_{3} \to  w_{1} \cox* w_{2} \to w_{2} \cox* w_{3} \to w_{1} \cox* w_{3} \\
    \congrf{\cox}{\append}
    & : \forall w_{1}, w_{2}, w_{3}, w_{4} \to  w_{1} \cox* w_{2} \to w_{3} \cox* w_{4} \to w_{1} \append w_{3} \cox* w_{2} \append w_{4} \\
    \relr{\cox}
    & : \forall w_{1}, w_{2} \to w_{1} \cox w_{2} \to w_{1} \cox* w_{2} \\
  \end{align*}
\end{definition}

\todo{Long form of Coxeter relations and their equivalence.}

We state and prove two desirable properties of this relation on $\List[\Fin[n]]$ from the point of view of an abstract
rewriting system. We follow the terminology of~\cite{krausCoherenceWellFoundednessTaming2020}. \vc{Not completely
  formalised. Is this with Coxeter or LongCoxeter?}

\begin{proposition}
  \leavevmode
  \begin{enumerate}
    \item $\cox$ is (locally) confluent. For every span $w_{1} \cox w \cox w_{2}$, there is a matching extended cospan
          $w_{1} \cox* v \cox* w_{2}$.
    \item $\cox*$ is strongly normalising. For every $w$, there exists a unique $v$ such that $w \cox* v$.
          \review{Equivalently, the type $\dsum{v:\List[\Fin[n]]}{w \cox* v}$ is contractible, and $w \cox* v$ is a
          proposition for every $w$ and $v$.}
  \end{enumerate}
\end{proposition}

The type $\Sn$ is defined as the set-quotient of $\List[\Fin[n]]$ by $\cox*$.

\begin{definition}[$\Sn$]
  \(\Sn \defeq \quot{\List(\Fin[n])}{\cox*}\)
\end{definition}

We will now prove the group structure of $\Sn$.

\begin{proposition}
  There is a group structure on $\Sn$, where the identity element is $\nil$, multiplication is given by list append, and
  inverse is given by list reversal.
\end{proposition}

\begin{proposition}
  $\Sn$ is equivalent to the generated group given by the normal closure of $\cox*$ extended to
  $\List(\Fin[n] + \Fin[n])$ along the codiagonal map $[\term{id},\term{id}] : A + A \to A$.
\end{proposition}

We will now show how to generate a Lehmer code from a word in $\Sn$ and back.

\begin{definition}
  \[
    \encode{\List} : \List[\Fin[n]] \to \Lehmer[n]
  \]
\end{definition}

\begin{proposition}
  \leavevmode
  \begin{enumerate}
    \item \( l_{1} \cox* l_{2} \to \encode{\List}(l_{1}) \id \encode{\List}(l_{1}) \)
  \end{enumerate}
\end{proposition}

\begin{definition}
  \begin{align*}
    \encodeSn & : \Sn[n] \to \Lehmer[n] \\
    \decodeSn & : \Lehmer[n] \to \Sn[n]
  \end{align*}
\end{definition}

\begin{proposition}
  For all $n : \Nat$, $(\encodeSn, \decodeSn)$ is an equivalence.
\end{proposition}

\begin{corollary}
  For all $n : \Nat$,
  \(
    \Sn \eqv \Lehmer[n] \eqv \Aut[\Fin[n]]
  \).
\end{corollary}

\subsection{Symmetric Monoidal structure}

We describe the symmetric monoidal structure of the groupoid $\UFin$.

First, we observe a few equivalences.

\begin{proposition}
  For any $n, m : \Nat$,
  \begin{align*}
    \Fin[0]                & \eqv \bot \\
    \Fin[n] \sqcup \Fin[m] & \eqv \Fin[n + m] \\
  \end{align*}
  and for any types $X, Y, Z$,
  \begin{align*}
    \bot \sqcup X          & \eqv X \\
    X \sqcup \bot          & \eqv X \\
    (X \sqcup Y) \sqcup Z  & \eqv X \sqcup (Y \sqcup Z) \\
    X \sqcup Y             & \eqv Y \sqcup Y \\
  \end{align*}
\end{proposition}

We lift these equivalences to $\UFin$ giving it the (additive) symmetric monoidal structure $(I, \oplus)$, with natural
isomorphisms $\lambda_{X}$, $\rho_{X}$, $\alpha_{X,Y,Z}$, and the symmetry isomorphism $\mathcal{B}_{X,Y}$. Note that
types in $\UFin$ are $\hSet$s since they're merely equivalent to $\Fin[n]$ for some $n : \Nat$.

\begin{definition}
  \begin{align*}
    I           & \defeq F_{0}    \\
    X \oplus Y & \defeq X \sqcup Y \\
    \lambda_{X} & : I \oplus X \eqv X \\
    \rho_{X} & : X \oplus I \eqv X \\
    \alpha_{X,Y,Z} & : (X \oplus Y) \oplus Z \eqv X \oplus (Y \oplus Z) \\
    \mathcal{B}_{X,Y} & : X \oplus Y \eqv Y \oplus X
  \end{align*}
\end{definition}

These isomorphisms satisfy the Mac Lane coherence conditions for symmetric monoidal categories, that is, the triangle,
pentagon, and hexagon identities, and the syllepsis of the braiding, upto 2-paths in $\UFin$.

\begin{proposition}
  % https://q.uiver.app/?q=WzAsNCxbMCwwLCIoWCBcXG9wbHVzIEkpIFxcb3BsdXMgWSJdLFsyLDAsIlggXFxvcGx1cyAoSSBcXG9wbHVzIFkpIl0sWzEsMSwiWCBcXG9wbHVzIFkiXSxbMCwxXSxbMCwxLCJcXGFscGhhX3tYLEksWX0iXSxbMCwyLCJcXHJob197WH0gXFxvcGx1cyAxX3tZfSIsMl0sWzEsMiwiMV97WH0gXFxvcGx1cyBcXGxhbWJkYV97WX0iXSxbNSw2LCJcXGlkIiwwLHsic2hvcnRlbiI6eyJzb3VyY2UiOjIwLCJ0YXJnZXQiOjIwfSwic3R5bGUiOnsiYm9keSI6eyJuYW1lIjoibm9uZSJ9LCJoZWFkIjp7Im5hbWUiOiJub25lIn19fV1d
  \[\begin{tikzcd}
      {(X \oplus I) \oplus Y} && {X \oplus (I \oplus Y)} \\
      {} & {X \oplus Y}
      \arrow["{\alpha_{X,I,Y}}", from=1-1, to=1-3]
      \arrow[""{name=0, anchor=center, inner sep=0}, "{\rho_{X} \oplus 1_{Y}}"', from=1-1, to=2-2]
      \arrow[""{name=1, anchor=center, inner sep=0}, "{1_{X} \oplus \lambda_{Y}}", from=1-3, to=2-2]
      \arrow["\id", Rightarrow, draw=none, from=0, to=1]
    \end{tikzcd}\]
  % https://q.uiver.app/?q=WzAsNSxbMCwxLCIoKFcgXFxvcGx1cyBYKSBcXG9wbHVzIFkpIFxcb3BsdXMgWiJdLFsxLDAsIihXIFxcb3BsdXMgWCkgXFxvcGx1cyAoWSBcXG9wbHVzIFopIl0sWzIsMSwiVyBcXG9wbHVzIChYIFxcb3BsdXMgKFkgXFxvcGx1cyBaKSkiXSxbMiwzLCJXIFxcb3BsdXMgKChYIFxcb3BsdXMgWSkgXFxvcGx1cyBaKSJdLFswLDMsIihXIFxcb3BsdXMgKFggXFxvcGx1cyBZKSkgXFxvcGx1cyBaIl0sWzAsMSwiXFxhbHBoYV97VyBcXG9wbHVzIFgsIFksIFp9Il0sWzEsMiwiXFxhbHBoYV97VyxYLFkgXFxvcGx1cyBafSJdLFszLDIsIjFfe1d9IFxcb3BsdXMgXFxhbHBoYV97WCxZLFp9IiwyXSxbMCw0LCJcXGFscGhhX3tXLFgsWX0gXFxvcGx1cyAxX3tafSIsMl0sWzQsMywiXFxhbHBoYV97VyxYIFxcb3BsdXMgWSxafSIsMl0sWzAsMiwiXFxpZCIsMSx7Im9mZnNldCI6NSwic3R5bGUiOnsiYm9keSI6eyJuYW1lIjoibm9uZSJ9LCJoZWFkIjp7Im5hbWUiOiJub25lIn19fV1d
  \[\begin{tikzcd}
      & {(W \oplus X) \oplus (Y \oplus Z)} \\
      {((W \oplus X) \oplus Y) \oplus Z} && {W \oplus (X \oplus (Y \oplus Z))} \\
      \\
      {(W \oplus (X \oplus Y)) \oplus Z} && {W \oplus ((X \oplus Y) \oplus Z)}
      \arrow["{\alpha_{W \oplus X, Y, Z}}", from=2-1, to=1-2]
      \arrow["{\alpha_{W,X,Y \oplus Z}}", from=1-2, to=2-3]
      \arrow["{1_{W} \oplus \alpha_{X,Y,Z}}"', from=4-3, to=2-3]
      \arrow["{\alpha_{W,X,Y} \oplus 1_{Z}}"', from=2-1, to=4-1]
      \arrow["{\alpha_{W,X \oplus Y,Z}}"', from=4-1, to=4-3]
      \arrow["\id"{description}, shift right=5, draw=none, from=2-1, to=2-3]
    \end{tikzcd}\]
  % https://q.uiver.app/?q=WzAsNixbMSwwLCJYIFxcb3BsdXMgKFkgXFxvcGx1cyBaKSJdLFswLDEsIihYIFxcb3BsdXMgWSkgXFxvcGx1cyBaIl0sWzAsMiwiKFkgXFxvcGx1cyBYKSBcXG9wbHVzIFoiXSxbMSwzLCJZIFxcb3BsdXMgKFggXFxvcGx1cyBaKSJdLFsyLDIsIlkgXFxvcGx1cyAoWiBcXG9wbHVzIFgpIl0sWzIsMSwiKFkgXFxvcGx1cyBaKSBcXG9wbHVzIFgiXSxbMSwwLCJcXGFscGhhX3tYLFksWn0iXSxbMSwyLCJcXG1hdGhjYWx7Qn1fe1gsWX0gXFxvcGx1cyAxX3tafSIsMl0sWzIsMywiXFxhbHBoYV97WSxYLFp9IiwyXSxbMyw0LCIxX3tZfSBcXG9wbHVzIFxcbWF0aGNhbHtCfV97WCxafSIsMl0sWzUsNCwiXFxhbHBoYV97WSxaLFh9Il0sWzAsNSwiXFxtYXRoY2Fse0J9X3tYLFkgXFxvcGx1cyBafSJdLFs3LDEwLCJcXGlkIiwwLHsic2hvcnRlbiI6eyJzb3VyY2UiOjIwLCJ0YXJnZXQiOjIwfSwic3R5bGUiOnsiYm9keSI6eyJuYW1lIjoibm9uZSJ9LCJoZWFkIjp7Im5hbWUiOiJub25lIn19fV1d
  \[\begin{tikzcd}
      & {X \oplus (Y \oplus Z)} \\
      {(X \oplus Y) \oplus Z} && {(Y \oplus Z) \oplus X} \\
      {(Y \oplus X) \oplus Z} && {Y \oplus (Z \oplus X)} \\
      & {Y \oplus (X \oplus Z)}
      \arrow["{\alpha_{X,Y,Z}}", from=2-1, to=1-2]
      \arrow[""{name=0, anchor=center, inner sep=0}, "{\mathcal{B}_{X,Y} \oplus 1_{Z}}"', from=2-1, to=3-1]
      \arrow["{\alpha_{Y,X,Z}}"', from=3-1, to=4-2]
      \arrow["{1_{Y} \oplus \mathcal{B}_{X,Z}}"', from=4-2, to=3-3]
      \arrow[""{name=1, anchor=center, inner sep=0}, "{\alpha_{Y,Z,X}}", from=2-3, to=3-3]
      \arrow["{\mathcal{B}_{X,Y \oplus Z}}", from=1-2, to=2-3]
      \arrow["\id", Rightarrow, draw=none, from=0, to=1]
    \end{tikzcd}\]
  % https://q.uiver.app/?q=WzAsMyxbMCwwLCJYIFxcb3BsdXMgWSJdLFsyLDAsIlggXFxvcGx1cyBZIl0sWzEsMSwiWSBcXG9wbHVzIFgiXSxbMCwxLCIxX3tYIFxcb3BsdXMgWX0iLDAseyJsZXZlbCI6Miwic3R5bGUiOnsiaGVhZCI6eyJuYW1lIjoibm9uZSJ9fX1dLFswLDIsIlxcbWF0aGNhbHtCfV97WCxZfSIsMl0sWzIsMSwiXFxtYXRoY2Fse0J9X3tZLFh9IiwyXSxbNCw1LCJcXGlkIiwwLHsic2hvcnRlbiI6eyJzb3VyY2UiOjIwLCJ0YXJnZXQiOjIwfSwic3R5bGUiOnsiYm9keSI6eyJuYW1lIjoibm9uZSJ9LCJoZWFkIjp7Im5hbWUiOiJub25lIn19fV1d
  \[\begin{tikzcd}
      {X \oplus Y} && {X \oplus Y} \\
      & {Y \oplus X}
      \arrow["{1_{X \oplus Y}}", Rightarrow, no head, from=1-1, to=1-3]
      \arrow[""{name=0, anchor=center, inner sep=0}, "{\mathcal{B}_{X,Y}}"', from=1-1, to=2-2]
      \arrow[""{name=1, anchor=center, inner sep=0}, "{\mathcal{B}_{Y,X}}"', from=2-2, to=1-3]
      \arrow["\id", Rightarrow, draw=none, from=0, to=1]
    \end{tikzcd}\]
\end{proposition}

%%% Local Variables:
%%% mode: latex
%%% TeX-master: "main"
%%% fill-column: 120
%%% End:
